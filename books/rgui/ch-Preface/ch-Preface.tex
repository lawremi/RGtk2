


\section*{About this book}
\label{sec:about}


\R\/ has a number of packages that provide a link between the \R\/
user and a graphical toolkit, such as \code{tcltk}, \code{RGtk2} and
\code{rJava} (for Swing). In addition, an \R\/ user can interface with
python or other external languages to provide access to graphical
toolkits within those languages. This book is about writing graphical
user interfaces (GUI) within \R\/ that do not rely on knowing an
external language.  

Currently, there is a range of graphical interfaces for \R\/ that are
programmed within R. For example, several package authors have
provided GUIs for their functions. Examples include \pkg{limmaGUI},
\pkg{caGUI}, \pkg{clustTool}, \pkg{Metabonomic}, \XXX{ade, fill me
  in}. There are a few tools to automatically generate such GUIs, such
as the \pkg{fgui} package and the \function{guiDlgFunction} function
from the \pkg{svDialogs} package. Other authors have provided
graphical interfaces to explore data sets, such as \pkg{playwith},
\pkg{latticist} and \pkg{aplpack}. While others have provided packages
with GUIs aimed at allowing students to perform some simulation, e.g.,
\pkg{teachingDemos}. The \pkg{rattle} package provides an interface
for several data mining operations. The \pkg{Rcmdr} package provides a
menu- and dialog-driven interface to a wide range of \R's
functionality. There are several user-contributed plugins that extend
the \code{Rcmdr}. Additionally, as \R\/ finds wider usage outside of
academia, it is not uncommon for people who work in a team setting to
desire an interface to their \R\/ code that allows non-\R\/ users
access.

Such examples are all within the scope of this book. We set out to
show that for many purposes adding a graphical interface to one's work
is not terribly sophisticated nor time-consuming (atleast each piece
isn't).  This book does not attempt to cover GUIs for \R\/ that
require knowledge of another programming language, although several
such projects provide some of the most sophisticated
interfaces. Examples are the \pkg{JGR} GUI or \pkg{iPlots} GUI written
in Java through \pkg{rJava}, the \pkg{rkWard} GUI written within KDE,
the \pkg{biocep} GUI written using Java and the \pkg{RServe} package,
or even the Windows GUI that comes with \R's Windows package.


The bulk of this text covers three different approaches to writing
GUIs. The \pkg{gWidgets} package is covered first. This provides a
common programming interface to some of the different graphical
toolkits provided with \R. This interface is much simpler (and less
powerful) than the native toolkits, so is useful for a programmer who
does not wish to invest too much time into perfecting their GUI. There
are a few other packages that provide an \R\/ interface to a toolkit
such as \code{rpanel} or \code{svDialogs}, but we focus on this one.


Next, we discuss the \pkg{RGtk2} package which provides a link between
\R\/ and the cross-platform \GTK\/ libraries. These libraries are
feature rich and used by several widely used projects. For the \R\/
user, they provide the most complete and modern-looking interfaces.

\XXX{Insert bit about Qt here}


Finally, we discuss the \pkg{tcltk} package. This package provides the
\R\/ user access to the \TK\/ libraries. Although not as modern as
\GTK, these libraries come pre-installed with the Windows binary and
so there are no installation issues for the average end-user. The bindings
to \TK\/ were the first ones to appear for \R\/ and the several of the
GUI projects above, notably \pkg{Rcmdr}, use this toolkit.

In addition to chapters on each of these three packages, the book has
an introductory chapter on GUIs, \XXX{a chapter on issues and techniques
for programming GUIs within \R}, and a chapter on web GUIs.

The text is written with the belief that much can be learned by
studying examples, and so several examples are given. Some of these
are meant as sketches of what can be done, while a few illustrate how
to code actual useful interfaces.  This text can't expect to cover all
of the features of a graphical toolkit. For the \pkg{tcltk} and
\pkg{RGtk2} packages, both underlying toolkits have well documented
APIs.


This text comes with an accompanying package \pkg{\PACKAGENAME}. This
package includes the complete code for all the examples. In order to
save space, some examples in the text have code that is not shown. The
package provides the functions \code{browsegWidgetsFiles},
\code{browseTclTkFiles} and \code{browseRGtk2Files} for browsing the
examples from the respective chapters. Additionally, this package will
contain vignettes describing aspects that did not make it into the
text.


This text was written with the \pkg{Sweave} package. To suppress
superflous output an assignment to a variable named \code{QT} is made
at times.
 


