\documentclass{article}

\begin{document}

\section*{Review of ``RGtk2: A Graphical User Interface Toolkit for R''}

I am afraid that I have to give this paper a negative evaluation as it stands. The
software as such is mostly fine - an impressive piece of work in fact, but the
paper is far from the academic standards that we require for the JSS. I do think
it is possible to revise it to become an attractive and useful paper, though, so I
shall try to be quite specific in my criticism below.

\subsection*{General notes}

Overall, the paper reads unevenly and has a sloppy, unfinished feel to it. Stylistically,
there is an annoying tendency to write in a clichee-ridden ``brochure
language''. The paper also tends to polemicize unnecessarily, and self-praises a
number of times. This only weakens the paper; a much stronger case can be
made by letting the facts and features stand for themselves.
In general, it would be wise to reconsider the use of adjectives, adverbs, and
other fillers. Often, text is improved by their removal.

\emph{We have taken this advice and reworded the problem phrases}

It does not seem like the authors have made their target audience clear. In many
cases strong implicit assumptions are being made about the readers' knowledge.
It is of course always a problem to decide what is common knowledge, but
I really don't think it can be assumed that everyone is conversant in six or
seven programming languages. We have to realize that the typical reader is
a statistician who just wants to try building some simple interfaces using the
toolkit.

I think it is reasonable to assume that readers will fall in three broad categories.
The structure of the paper should reflect the structure of the audience.

\begin{enumerate}
\item Basic users. These may not know any other computer languages than R.
They will be interested in trying out the package to write some simple
GUIs. Initial descriptions and examples should be for these people.
\item Experienced GUI coders. This group will be interested in the interface as
such, and will need to know at least R and C. I conjecture that in quite
a few cases, that is exactly what they know, so I suggest that references
to other languages should be minimized when describing the interface
implementation.
\item People interested in the interfacing issue as such. At this point it is relevant
to bring in aspects of other languages, but it is still not given that the
reader knows about each of their characteristics, so a few chosen words of
introduction could be in order.
\end{enumerate}

\emph{We have restructured the paper. It now consists of two major parts: 
(1) a tutorial and (2) technical description of the package design. The tutorial
begins with the fundamentals and slowly builds up to a full application. This 
is followed by a treatment of tricks/features that may be of interest to
advanced users. The second part is geared toward those interested in software
interfaces. The paper concludes with a discussion of related work, which was 
reworked to be more accessible to novice users.}

\subsection*{Specific comments}

Notwithstanding the possible need to restructure the paper, I give some comments
section for section in the following.

\subsubsection*{Section 1}

\emph{The introduction, like virtually every other section of the paper, has
been largely rewritten.}

p.2 What exactly does it mean to ``interrupt the flow of a graphical analysis''?

\emph{Rephrased}

The detail level of the history of GTK+ seems a bit high.

\emph{Reduced}

The literature references to Tcl/Tk are unsatisfactory (see References below).

\emph{Hopefully resolved using your suggested references}

\subsubsection*{Section 2}

To have a whole section of a scholarly paper consist entirely of bullet points is
just plain wrong!

\emph{There are no longer any bullet points in the paper.}

Please avoid self-praise like ``built on the succesful RGtk''.
As the documentation for RGtk was never very detailed, I don't think it suffices
just to describe the differences between that and RGtk2; it is necessary to describe
the final product. 

\emph{The paper no longer attempts to compare RGtk2 to RGtk.}

Also, since readers are not necessarily C programmers,
prior understanding of GTK+ can not be assumed. I.e., there needs to be a
ground-level description of the basic concepts involved (Gtk objects, callbacks,
signals, packing of widget components, etc.).

\emph{This now exists.}

\subsubsection*{Section 3}

The reference to ``random microarray data'' is a bit obscure. What is generated
is a sample from a 2-d mixture distribution of correlated variables. If this
is intended to emulate data from microarray experiments, it needs to be said
explicitly. 

\emph{Rephrased.}

The code for generating the data is a bit lengthy and might be moved
out of sight (into an appendix, e.g.)

\emph{It doesn't seem that long and we'd rather not complicate the structure
by adding appendices.}

The phrase ``camel back capitalization'' is amusing, but I believe the proper
term is ``mixed case''. 

\emph{According to wikipedia, the main term is ``camelBack'', although
``mixed case'' is a synonym.}

Do get rid of at least one part of ``compact and easier to type''.

\emph{Done.}

Last para on p. 6 ``the default is acceptable'': What is the default?
``The final steps are. . . '' (p.7, 3rd para). These steps do not appear to be final,
since more steps follow later on. It doesn't really matter whether they are,
though.

\emph{Thanks for catching this, fixed.}

``Rgtk2 supports a convenient Java-like syntax'' (p.7). It is not needed to say
that it is convenient (let the user decide that), nor necessarily at this point to
bring Java into the picture. I'd just say that it supports making methods for an
object available using the \$ operator and maybe afterwards say that this was
inspired by object-oriented languages like Java. As it stands, the sentence also
has the (unintended?) effect of implying that Java syntax is somehow superior
R syntax.

\emph{This has been reworded.}

The example as such is OK, but it is very simplistic. A user approaching this
paper will likely be looking for an impression on how it is like to program in
RGtk2 and will be disappointed over the small amount of information. I don't
think the reference to the demos in the package is quite sufficient. Things that
I would be looking for is how the packing of widget elements works, and how to
communicate in both directions between widget elements (e.g., in a text widget
with a scrollbar). A few more examples, possibly with a less detailed description
would be good.

\emph{More examples have been added, including a hello world GUI, dialog boxes,
various types of buttons, and a spreadsheet application. 
There is a careful introduction to widget packing. There also advanced examples
demonstrating integration with other GObject-based libraries and the
creation of GObject-derived classes.}

\subsubsection*{Section 4}

Putting the bound libraries in an itemized list is OK, but describing the components
of a binding with one sentence per piece doesn't work. A subsection
for each is more like it. This is partly a restructuring matter; some of the information
I'd want in a subsection about function wrapping is in the 2nd para.
on p.9, the following para. would belong with the discussion of Fields, etc.
The statement about ``Converters'' is close to vacuous.

\emph{This section has been restructured along with these points in mind.}

The defs format is described as ``scheme-based''. Should this not be Schemebased
(i.e. the language)? It should be possible to say a little more about this
and about the conversion processes.

\emph{More detail has been added.}

p.8 ``enjoy the benefits''. Come on!
p.9 ``wrapped on the fly'' could use a bit more explanation
The pages 9-10 obviously reflect areas where the authors have spent much of the
hard work, but unfortunately it also comes across as a hard read. It is as if the
text tries to say too much at once. Some structure in the form of subheadings
could help here, as could a more leaned-back writing style where more time is
spent on explaining what the problem is before jumping to the solution. In
particular the two paragraphs on simple C structures look like they could be
written considerably more clearly.

\emph{Again, this section has been completely reworked with these points in mind.}

I am not sure it can be assumed that readers know about opaque and transparent
structures (it is a rather recent feature of the C standards). A few words of
definition could be in order.

\emph{These terms have been avoided.}

\subsubsection*{Section 5}

The literature references are strongly lacking in this section. (Again, see References
below.)

I find it somewhat debatable whether it really is a good design principle to
use fixed-GUI designs like Glade to wrap code for an extensible programming
language.

\emph{Certainly Glade is not appropriate in many cases, but it is still
a feature that users seem to appreciate.}

p.11 ``with Tcl/Tk it is not possible to extend. . . ''. This is untrue, if meant as
written. Megawidgets have existed in Tk for a long time, written in Tcl or in
C. It might however be true is that it is not obvious how to write a new widget
in R.

\emph{We've tried to make a distinction between the compound ``megawidgets'' and
overriding the base behavior of a widget with RGtk2.}

It is not clear what the authors mean when saying that the tcltk package ``aims
to expose complexity'' etc., if anything, it aims to expose functionality. I would
characterize the difference as mainly that Tk was always designed for binding
to a high level language (which, removing the syntactic sugar, is LISP-like like
R is) whereas GTK is a C library so that there is some work to be done.

\emph{Good point, this has been reworded.}

In the penultimate section on p.11, the concept of Java ``interfaces'' is introduced
without definition.

\emph{Java interfaces are no longer mentioned.}

\subsubsection*{Section 6}

Final sentence of 3rd para. seems to have been garbled (``bring them with'').

\emph{Thanks, fixed.}

\subsubsection*{Section 7}

p.13, top: There has been no previous mention of the ``issues with the R event
loops'', and there probably should be. I'm also not sure whether it is warranted
to call such issues ``minor''; event loop issues are usually a major headache.

\emph{A section has been added to the discussion on event loops.}

One issue not discussed is speed. RGtk2 feels considerably slower than similar
Tk code. It would be nice to have some discussion about the reasons for this
and the potential for improvement.

\emph{It would not be surprising if RGtk2 were slower than Tcl/TK, given the
complexity of GTK 2.0, which is slower than GTK 1.2. However, we do
not feel that the performance impact is so significant that it merits any
special discussion (beyond the mention when comparing RGtk2 to tcltk).
}

\subsubsection*{References}

The authors need to get the Tcl/Tk references straight. There are two R News
articles about the tcltk package which could be cited. The website given as the
only reference is the ``Tcl Developer Xchange'' and not what is written, and it
makes no sense as a reference for tcltk2.
There are books on Tcl/Tk. I'd suggest Ousterhout's book from 1994 ``Tcl and
the Tk Toolkit'' for historical reasons and Brent Welch's ``Practical Programming
in Tcl and Tk'', 4th ed., 2003 for a reasonably up to date guide.

Similarly, better references for Swing, SWT, and wxWindows must be available.
There is at least one book on wxWindows, for instance.

\emph{The references have been completely revised.}

The software
I have tested the software superficially, mostly by running the demos. Currently
(R-2.5.1, RGtk2 2.10.11, cairoDevice 2.3), there seems to be quite a few
issues with the demos, many of them failing with errors like VECTOR ELT()
can only be applied to a 'list', not a 'integer'. The potential is impressive,
but the quality control apparently needs tightening. 

\emph{Unfortunately it seems that we were caught at a bad time. These issues
were fixed as of RGtk2 2.10.12, having been introduced with R 2.5.0.}

Also, there
seems to be a problem with system files: demo(images) attempts to load
/usr/share/gtk-2.0/demo/floppybuddy.gif which isn't there on my system,
but in /opt/gnome.

I couldn't help but notice that the source tarball at CRAN has some SVN files
inside, which indicates that some cleanup step was omitted.

\emph{This seems to be a problem with R CMD build not cleaning out subdirectories
of the 'src' directory. A patch was submitted to R to fix this problem. Not sure
if it was incorporated.}

Also, the auxiliary documentation in the package could do with some revision
instead of being left in the current half-finished state.

\emph{Agreed. We are porting the C GTK+ tutorial to R but it is tedious work;
there may be a better way. This paper may be a good interim solution, as it now
has a more substantial tutorial.}

\end{document}
