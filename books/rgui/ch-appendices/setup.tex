%%%%%%%%%%%%%%%%%%%%%%%%%%%%%%%%%%%%%%%%%%%%%%%%%%
%% Load packages
\usepackage{mathptmx}  %% 420 pages          % for math fonts type 1
\usepackage[pdftex]{graphicx}           % for graphics files
\usepackage{floatflt}           % for ``floating boxes''
\usepackage{index}
\usepackage{relsize}            % for relative size fonts
\usepackage{amsmath}            % for amslatex stuff
\usepackage{amsfonts}           % for amsfonts
\usepackage{url}                % for \url,
\usepackage{listings}
\usepackage{booktabs}
\usepackage{fancyvrb}
\usepackage{multicol}          % for making multiple columns
\usepackage{prelim2e}           % put on bottom of each page
\usepackage{lscape}             % landscape tables
\usepackage{natbib}

%%%%%%%%%%%%%%%%%%%%%%%%%%%%%%%%%%%%%%%%%%%%%%%%%%
%%% The page

%%%%%%%%%%
%% page layout
%% assume basic page size for now
\raggedbottom


%%%%%%%%%%
%% fonts
%% Review:
% Upright shape \textup{Upright shape} 
% Italic shape \textit{Italic shape} 
% Slanted shape \textsl{Slanted shape} 
% S MAL L CAP S S HAP E \textsc{Small Caps shape} 
% Series or weight 
% Medium series \textmd{Medium series} 
% Bold series \textbf{Bold series} 
% Family 
% Roman family \textrm{Roman family} 
% Sans serif family \textsf{Sans serif family} 
% Typewriter family \texttt{Typewriter family} 

%% which fonts?
%%\usepackage{avant}
%%\usepackage{palatcm}
\usepackage[T1]{fontenc}
%\usepackage[adobe-utopia]{mathdesign}
%\usepackage{aurical}

\renewcommand{\encodingdefault}{T1}
\usepackage[sc]{mathpazo}
\linespread{1.05}         % Palatino needs more leading (space between lines)




%%%%%%%%%%
%% titles

%%%%%%%%%%
%% divisions

%% chapter styles
%\chapterstyle{ell}


%%% pagestyle
\pagestyle{ruled}
%\pagestyle{companion}

% captions -- The class uses the following to specify the standard LaTeX caption style: 
% \captionnamefont{} 
% \captiontitlefont{} 
\captionstyle[\centering]{\raggedright} 
\captionwidth{\linewidth} 
% \normalcaptionwidth 
% \normalcaption 
\captiondelim{: } 
%\postcaption{\rule{\linewidth}{0.4pt}\par}

%%%%%%%%%%
%% pagination headers

%%%%%%%%%%
%% paragraphs, lists
\tightlists

%%%%%%%%%%
%% content lists

%%%%%%%%%%
%% floats and captions

%%%%%%%%%%
%% rows and columns

%%%%%%%%%%
%% page notes

%%%%%%%%%%
%% decorative text


%%%%%%%%%%
%% Boxes verbatims files

%%%%%%%%%%
%% cross referencing

%%%%%%%%%%
%% back matter


%%
%%
%%%%%%%%%%%%%%%%%%%%%%%%%%%%%%%%%%%%%%%%%%%%%%%%%%


%%%%%%%%%%%%%%%%%%%%%%%%%%%%%%%%%%%%%%%%%%%%%%%%%%
%% Abbreviations (most are Rd-ish)
%% Flag something to look at -- XXX is easy to search for
\newcommand{\XXX}[1]{}%%{XXX-- #1 --XXX\\}


\newcommand{\R}{\textsf{R}}
\newcommand{\code}[1]{\texttt{#1}} % code
\newcommand{\qcode}[1]{\code{"#1"}} % quoted code 

\newcommand{\defn}[1]{\textit{#1}\index{\textit{#1}}}   % add in index
\newcommand{\command}[1]{\code{#1}} % name of command
\newcommand{\function}[1]{\code{#1}} % name of function
\newcommand{\constructor}[1]{\function{#1}\index{#1}}

\newcommand{\args}[1]{\code{#1}} % name of argument only
\newcommand{\argument}[2]{\args{#1}\index{#2|\texttt{#1}}} % name of an argument, plus
                                % function for index
\newcommand{\subcommand}[2]{\textit{#2} \args{#1}\index{#2|\code{#1}}} % name of an tk subcommand plus
                                % function for index
\newcommand{\subcommanda}[3]{\subcommand{#1}{#2} \textit{#3} }
\newcommand{\option}[2]{\args{#1}\index{#2|\code{#1}}} % name of an option plus constructor
\newcommand{\class}[1]{\code{#1}}  % a class
\newcommand{\generic}[1]{\code{#1}} % name of generic method -- no
\newcommand{\meth}[1]{\generic{#1}}     % single arg, no class
\newcommand{\method}[2]{\meth{#1}\index{#2|\code{#1}}} % name of method with
                                % class for index

\newcommand{\signal}[1]{\code{#1}} % name of signal
\newcommand{\dfn}[1]{\textit{#1}} % definition
\newcommand{\dfnref}[1]{\textit{#1}} % refer to a definition
\newcommand{\env}[1]{\texttt{#1}} % environment setting
\newcommand{\file}[1]{\texttt{#1}}
\newcommand{\kbd}[1]{\textmd{#1}}
\newcommand{\pkg}[1]{\texttt{#1}}
\newcommand{\opt}[1]{\texttt{#1}} % R option
\newcommand{\acronym}[1]{\texttt{#1}}

\usepackage{fancyvrb}
\DefineShortVerb{\|}
\SaveVerb{ASSIGN}|<-|
%%\newcommand{\ASSIGN}{\code{\UseVerb{ASSIGN}}} % <- formats funny
\newcommand{\leftBracket}{$<$}
\newcommand{\rightBracket}{$>$}
\newcommand{\ASSIGN}{\code{$<$-}}  %% <-
\newcommand{\backslashn}{\code{$\backslash$n}} %% \n

\newcommand{\GTK}{GTK+}
\newcommand{\TCL}{Tcl}
\newcommand{\Tcl}{\TCL}
\newcommand{\TK}{Tk}
\newcommand{\Tk}{Tk}
\newcommand{\tcltk}{Tcl/Tk}
\newcommand{\wxWidgets}{wxWidgets}
\newcommand{\Java}{Java}
\newcommand{\gWidgets}{gWidgets}

\newcommand{\TITLE}{Programming GUIs within R}
\title{\TITLE}
\newcommand{\PACKAGENAME}{ProgGUIinR}
\newcommand{\WINDOZE}{Windows}
\newcommand{\UNIX}{Unix}
\newcommand{\LINUX}{Linux}
\newcommand{\OSX}{Mac OS X}
%%
%%%%%%%%%%%%%%%%%%%%%%%%%%%%%%%%%%%%%%%%%%%%%%%%%%


\usepackage{color}
%%% Define some colors
\definecolor{gray70}{gray}{.70}
\definecolor{gray60}{gray}{.60}
\definecolor{gray50}{gray}{.50}
\definecolor{gray40}{gray}{.40}
\definecolor{gray25}{gray}{.25}


%% Save space things
%% http://www-h.eng.cam.ac.uk/help/tpl/textprocessing/squeeze.html 
%% \usepackage{float}        
% \usepackage{jvfloatstyle}       % redefine float.sty for my style. Hack
% \floatstyle{jvstyle}            
% \restylefloat{table}
% \restylefloat{figure}


%% In natbib
\bibpunct{(}{)}{;}{a}{,}{,}

