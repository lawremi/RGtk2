\documentclass[showtrims]{memoir}


%%%%%%%%%%%%%%%%%%%%%%%%%%%%%%%%%%%%%%%%%%%%%%%%%%
%% Load packages
\usepackage{mathptmx}  %% 420 pages          % for math fonts type 1
\usepackage[pdftex]{graphicx}           % for graphics files
\usepackage{floatflt}           % for ``floating boxes''
\usepackage{index}
\usepackage{relsize}            % for relative size fonts
\usepackage{amsmath}            % for amslatex stuff
\usepackage{amsfonts}           % for amsfonts
\usepackage{url}                % for \url,
\usepackage{listings}
\usepackage{booktabs}
\usepackage{fancyvrb}
\usepackage{multicol}          % for making multiple columns
\usepackage{prelim2e}           % put on bottom of each page
\usepackage{lscape}             % landscape tables
\usepackage{natbib}

%%%%%%%%%%%%%%%%%%%%%%%%%%%%%%%%%%%%%%%%%%%%%%%%%%
%%% The page

%%%%%%%%%%
%% page layout
%% assume basic page size for now
\raggedbottom


%%%%%%%%%%
%% fonts
%% Review:
% Upright shape \textup{Upright shape} 
% Italic shape \textit{Italic shape} 
% Slanted shape \textsl{Slanted shape} 
% S MAL L CAP S S HAP E \textsc{Small Caps shape} 
% Series or weight 
% Medium series \textmd{Medium series} 
% Bold series \textbf{Bold series} 
% Family 
% Roman family \textrm{Roman family} 
% Sans serif family \textsf{Sans serif family} 
% Typewriter family \texttt{Typewriter family} 

%% which fonts?
%%\usepackage{avant}
%%\usepackage{palatcm}
\usepackage[T1]{fontenc}
%\usepackage[adobe-utopia]{mathdesign}
%\usepackage{aurical}

\renewcommand{\encodingdefault}{T1}
\usepackage[sc]{mathpazo}
\linespread{1.05}         % Palatino needs more leading (space between lines)




%%%%%%%%%%
%% titles

%%%%%%%%%%
%% divisions

%% chapter styles
%\chapterstyle{ell}


%%% pagestyle
\pagestyle{ruled}
%\pagestyle{companion}

% captions -- The class uses the following to specify the standard LaTeX caption style: 
% \captionnamefont{} 
% \captiontitlefont{} 
\captionstyle[\centering]{\raggedright} 
\captionwidth{\linewidth} 
% \normalcaptionwidth 
% \normalcaption 
\captiondelim{: } 
%\postcaption{\rule{\linewidth}{0.4pt}\par}

%%%%%%%%%%
%% pagination headers

%%%%%%%%%%
%% paragraphs, lists
\tightlists

%%%%%%%%%%
%% content lists

%%%%%%%%%%
%% floats and captions

%%%%%%%%%%
%% rows and columns

%%%%%%%%%%
%% page notes

%%%%%%%%%%
%% decorative text


%%%%%%%%%%
%% Boxes verbatims files

%%%%%%%%%%
%% cross referencing

%%%%%%%%%%
%% back matter


%%
%%
%%%%%%%%%%%%%%%%%%%%%%%%%%%%%%%%%%%%%%%%%%%%%%%%%%


%%%%%%%%%%%%%%%%%%%%%%%%%%%%%%%%%%%%%%%%%%%%%%%%%%
%% Abbreviations (most are Rd-ish)
%% Flag something to look at -- XXX is easy to search for
\newcommand{\XXX}[1]{}%%{XXX-- #1 --XXX\\}


\newcommand{\R}{\textsf{R}}
\newcommand{\code}[1]{\texttt{#1}} % code
\newcommand{\qcode}[1]{\code{"#1"}} % quoted code 

\newcommand{\defn}[1]{\textit{#1}\index{\textit{#1}}}   % add in index
\newcommand{\command}[1]{\code{#1}} % name of command
\newcommand{\function}[1]{\code{#1}} % name of function
\newcommand{\constructor}[1]{\function{#1}\index{#1}}

\newcommand{\args}[1]{\code{#1}} % name of argument only
\newcommand{\argument}[2]{\args{#1}\index{#2|\texttt{#1}}} % name of an argument, plus
                                % function for index
\newcommand{\subcommand}[2]{\textit{#2} \args{#1}\index{#2|\code{#1}}} % name of an tk subcommand plus
                                % function for index
\newcommand{\subcommanda}[3]{\subcommand{#1}{#2} \textit{#3} }
\newcommand{\option}[2]{\args{#1}\index{#2|\code{#1}}} % name of an option plus constructor
\newcommand{\class}[1]{\code{#1}}  % a class
\newcommand{\generic}[1]{\code{#1}} % name of generic method -- no
\newcommand{\meth}[1]{\generic{#1}}     % single arg, no class
\newcommand{\method}[2]{\meth{#1}\index{#2|\code{#1}}} % name of method with
                                % class for index

\newcommand{\signal}[1]{\code{#1}} % name of signal
\newcommand{\dfn}[1]{\textit{#1}} % definition
\newcommand{\dfnref}[1]{\textit{#1}} % refer to a definition
\newcommand{\env}[1]{\texttt{#1}} % environment setting
\newcommand{\file}[1]{\texttt{#1}}
\newcommand{\kbd}[1]{\textmd{#1}}
\newcommand{\pkg}[1]{\texttt{#1}}
\newcommand{\opt}[1]{\texttt{#1}} % R option
\newcommand{\acronym}[1]{\texttt{#1}}

\usepackage{fancyvrb}
\DefineShortVerb{\|}
\SaveVerb{ASSIGN}|<-|
%%\newcommand{\ASSIGN}{\code{\UseVerb{ASSIGN}}} % <- formats funny
\newcommand{\leftBracket}{$<$}
\newcommand{\rightBracket}{$>$}
\newcommand{\ASSIGN}{\code{$<$-}}  %% <-
\newcommand{\backslashn}{\code{$\backslash$n}} %% \n

\newcommand{\GTK}{GTK+}
\newcommand{\TCL}{Tcl}
\newcommand{\Tcl}{\TCL}
\newcommand{\TK}{Tk}
\newcommand{\Tk}{Tk}
\newcommand{\tcltk}{Tcl/Tk}
\newcommand{\wxWidgets}{wxWidgets}
\newcommand{\Java}{Java}
\newcommand{\gWidgets}{gWidgets}

\newcommand{\TITLE}{Programming GUIs within R}
\title{\TITLE}
\newcommand{\PACKAGENAME}{ProgGUIinR}
\newcommand{\WINDOZE}{Windows}
\newcommand{\UNIX}{Unix}
\newcommand{\LINUX}{Linux}
\newcommand{\OSX}{Mac OS X}
%%
%%%%%%%%%%%%%%%%%%%%%%%%%%%%%%%%%%%%%%%%%%%%%%%%%%


\usepackage{color}
%%% Define some colors
\definecolor{gray70}{gray}{.70}
\definecolor{gray60}{gray}{.60}
\definecolor{gray50}{gray}{.50}
\definecolor{gray40}{gray}{.40}
\definecolor{gray25}{gray}{.25}


%% Save space things
%% http://www-h.eng.cam.ac.uk/help/tpl/textprocessing/squeeze.html 
%% \usepackage{float}        
% \usepackage{jvfloatstyle}       % redefine float.sty for my style. Hack
% \floatstyle{jvstyle}            
% \restylefloat{table}
% \restylefloat{figure}


%% In natbib
\bibpunct{(}{)}{;}{a}{,}{,}


%% Begin here



%%\usepackage{/Users/verzani/R/R.framework/Resources/share/texmf/Sweave}
\begin{document}
\thispagestyle{empty}
\bibliographystyle{plainnat}
%%% These go inside begin document, so in a different file


%% From http://tolstoy.newcastle.edu.au/R/e6/help/09/04/11790.html
\newenvironment{Schunk}{}{}
%%\usepackage{listings}
\definecolor{gris05}{gray}{0.05}
\definecolor{gris10}{gray}{0.10}
\definecolor{gris40}{gray}{0.40}
\definecolor{gris90}{gray}{0.90}
\definecolor{gris95}{gray}{0.95}
\lstloadlanguages{R}

\lstnewenvironment{Sinput}[1][]{
  \lstset{
    language={R},
    basicstyle=\small,                        % print whole listing small
    %
    % keywordstyle=\color{black}\bfseries,      % style for keyword
    keywordstyle=\ttfamily,      % style for keyword
    %%
    %
    emph={in},                                % define a list of word to emphasis
    % emphstyle=\color{black}\bfseries,         % define the way to    emphase
    emphstyle=\ttfamily,         % define the way to emphase
    % emph={[2]out},                            % define a second list of word to emphasis
    % emphstyle=[2]\color{red}\bfseries,        % define the way to emphase the list 2
    %
    commentstyle=\color{gris40}\small\itshape,%\sffamily, % comments
    %
    stringstyle=\ttfamily,
    identifierstyle=\ttfamily,
    %% 
    frame=leftline,                             % box arround the code
    backgroundcolor=\color{gris95},           % background color
    showspaces=false,                         % show the space in code, or not
    stringstyle=\color{gris05}\ttfamily,                    % style of the string (like "hello word")
    showstringspaces=false,                   % show the space in
                                % string, on not
    aboveskip=\medskipamount,
 #1
  }
}{}

\lstnewenvironment{Soutput}[1][]{
  \lstset{%
    %% No language here, otherwise emphasis is odd
    %% language={R},                           
   basicstyle=\ttfamily\small,                        % print whole listing small
%%    keywordstyle=\color{black}\bfseries,      % style for keyword
    %% emph={in},                                % define a list of word to emphasis
    %% emphstyle=\color{black}\bfseries,         % define the way to emphase
    % emph={[2]out},                            % define a second list of word to emphasis
    % emphstyle=[2]\color{red}\bfseries,        % define the way to emphase the list 2
    %% frame=single,                             % box arround the code
    backgroundcolor=\color{gris95},           % background color
    %% commentstyle=\color{gris10},              % define the style of the comments
    stringstyle=\ttfamily,                    % style of the string (like "hello word")
    showspaces=false,                         % show the space in code, or not
    showstringspaces=false,                   % show the space in  string, on not
   #1
  }
}{} 




% %% Override the verbatim environment for Sinput, Soutput, Scode

% \DefineVerbatimEnvironment{Sinput}{Verbatim}{
% fontfamily=lmss,
%   fontseries=b
%   ,fontsize=\relsize{0}
% %%  ,xleftmargin=2em
% }
% \DefineVerbatimEnvironment{Soutput}{Verbatim}{
%   formatcom=\color{blue},
% %%  fontshape=sl,
%  fontsize=\relsize{0},
%   showspaces=FALSE
% %%  ,xleftmargin=2em
% %%  formatcom=\color{gray40}
% }
% \DefineVerbatimEnvironment{Scode}{Verbatim}{
%   %% xleftmargin=2em
%   %% fontshape=sl
% }

% %% Same as Sinput
% \DefineVerbatimEnvironment{CodeSnippet}{Verbatim}{
% %  formatcom=\color{RInput},
%   fontseries=b, 
%   fontsize=\relsize{-1}
% }


% %%% DUncan Murdoch's solution to tighten up space in Sweave
% \fvset{listparameters={\setlength{\topsep}{0pt}}}
% \renewenvironment{Schunk}{\vspace{\topsep}}{\vspace{\topsep}}

% % %% use   fontshape=it, for italics
% % \DefineVerbatimEnvironment{Sinput}{Verbatim}{
% % %  formatcom=\color{RInput},
% %   fontfamily=courier,
% %   fontseries=b,
% %   fontsize=\relsize{-1},
% %   showspaces=false
% % %%  baselinestretch=0
% % }
% % \DefineVerbatimEnvironment{Soutput}{Verbatim}{
% % %  formatcom=\color{ROutput},
% %   fontfamily=courier,
% %   fontseries=b,
% %   fontsize=\relsize{-2},
% %   showspaces=false
% % %%  baselinestretch=0
% % }


%%% An environment for HTML text
\lstnewenvironment{HTMLinput}[1][]{
  \lstset{%
    language={html},
    basicstyle=\small,                        % print whole listing small
    % keywordstyle=\color{black}\bfseries,      % style for keyword
    keywordstyle=\ttfamily,      % style for keyword
    emph={in},                                % define a list of word to emphasis
    % emphstyle=\color{black}\bfseries,         % define the way to    emphase
    emphstyle=\ttfamily,         % define the way to emphase
    % emph={[2]out},                            % define a second list of word to emphasis
    % emphstyle=[2]\color{red}\bfseries,        % define the way to emphase the list 2
    % commentstyle=\color{gris10}\ttfamily,              % define the
                                % style of the comments
    stringstyle=\ttfamily,
    identifierstyle=\ttfamily,
    %% 
    frame=leftline,                             % box arround the code
    backgroundcolor=\color{gris95},           % background color
    showspaces=false,                         % show the space in code, or not
    stringstyle=\ttfamily,                    % style of the string (like "hello word")
    showstringspaces=false,                   % show the space in
    % string, on not
    aboveskip=\medskipamount,
 #1
  }
}{}

%% Use to call in an HTML file
\newcommand{\HTMLinputlisting}[1]{
  \lstinputlisting[
  language={html},
  basicstyle=\small\ttfamily,
   emph={in},                                % define a list of word to emphasis
   emphstyle=\ttfamily,         % define the way to emphase
   stringstyle=\ttfamily,
   identifierstyle=\ttfamily,
%   %% 
   frame=leftline,                             % box arround the code
   backgroundcolor=\color{gris95},           % background color
   showspaces=false,                         % show the space in code, or not
   stringstyle=\ttfamily,                    % style of the string (like "hello word")
   showstringspaces=false,                   % show the space in
%   % string, on not
  aboveskip=\medskipamount
  ]{#1}
}




%% We use footnote bibliography citations
\footbibliographystyle{plain}
\footbibliography{guis}



%% showtrims


% %%% Local Variables: 
% %%% mode: latex
% %%% TeX-master: t
% %%% End: 
                  % sweave

\frontmatter

\chapter*{Preface}
\label{chap:preface}



\section*{About this book}
\label{sec:about}


\R\/ has a number of packages that provide a link between the \R\/
user and a graphical toolkit, such as \code{tcltk}, \code{RGtk2} and
\code{rJava} (for Swing). In addition, an \R\/ user can interface with
python or other external languages to provide access to graphical
toolkits within those languages. This book is about writing graphical
user interfaces (GUI) within \R\/ that do not rely on knowing an
external language.  

Currently, there is a range of graphical interfaces for \R\/ that are
programmed within R. For example, several package authors have
provided GUIs for their functions. Examples include \pkg{limmaGUI},
\pkg{caGUI}, \pkg{clustTool}, \pkg{Metabonomic}, \XXX{ade, fill me
  in}. There are a few tools to automatically generate such GUIs, such
as the \pkg{fgui} package and the \function{guiDlgFunction} function
from the \pkg{svDialogs} package. Other authors have provided
graphical interfaces to explore data sets, such as \pkg{playwith},
\pkg{latticist} and \pkg{aplpack}. While others have provided packages
with GUIs aimed at allowing students to perform some simulation, e.g.,
\pkg{teachingDemos}. The \pkg{rattle} package provides an interface
for several data mining operations. The \pkg{Rcmdr} package provides a
menu- and dialog-driven interface to a wide range of \R's
functionality. There are several user-contributed plugins that extend
the \code{Rcmdr}. Additionally, as \R\/ finds wider usage outside of
academia, it is not uncommon for people who work in a team setting to
desire an interface to their \R\/ code that allows non-\R\/ users
access.

Such examples are all within the scope of this book. We set out to
show that for many purposes adding a graphical interface to one's work
is not terribly sophisticated nor time-consuming (atleast each piece
isn't).  This book does not attempt to cover GUIs for \R\/ that
require knowledge of another programming language, although several
such projects provide some of the most sophisticated
interfaces. Examples are the \pkg{JGR} GUI or \pkg{iPlots} GUI written
in Java through \pkg{rJava}, the \pkg{rkWard} GUI written within KDE,
the \pkg{biocep} GUI written using Java and the \pkg{RServe} package,
or even the Windows GUI that comes with \R's Windows package.


The bulk of this text covers three different approaches to writing
GUIs. The \pkg{gWidgets} package is covered first. This provides a
common programming interface to some of the different graphical
toolkits provided with \R. This interface is much simpler (and less
powerful) than the native toolkits, so is useful for a programmer who
does not wish to invest too much time into perfecting their GUI. There
are a few other packages that provide an \R\/ interface to a toolkit
such as \code{rpanel} or \code{svDialogs}, but we focus on this one.


Next, we discuss the \pkg{RGtk2} package which provides a link between
\R\/ and the cross-platform \GTK\/ libraries. These libraries are
feature rich and used by several widely used projects. For the \R\/
user, they provide the most complete and modern-looking interfaces.

\XXX{Insert bit about Qt here}


Finally, we discuss the \pkg{tcltk} package. This package provides the
\R\/ user access to the \TK\/ libraries. Although not as modern as
\GTK, these libraries come pre-installed with the Windows binary and
so there are no installation issues for the average end-user. The bindings
to \TK\/ were the first ones to appear for \R\/ and the several of the
GUI projects above, notably \pkg{Rcmdr}, use this toolkit.

In addition to chapters on each of these three packages, the book has
an introductory chapter on GUIs, \XXX{a chapter on issues and techniques
for programming GUIs within \R}, and a chapter on web GUIs.

The text is written with the belief that much can be learned by
studying examples, and so several examples are given. Some of these
are meant as sketches of what can be done, while a few illustrate how
to code actual useful interfaces.  This text can't expect to cover all
of the features of a graphical toolkit. For the \pkg{tcltk} and
\pkg{RGtk2} packages, both underlying toolkits have well documented
APIs.


This text comes with an accompanying package \pkg{\PACKAGENAME}. This
package includes the complete code for all the examples. In order to
save space, some examples in the text have code that is not shown. The
package provides the functions \code{browsegWidgetsFiles},
\code{browseTclTkFiles} and \code{browseRGtk2Files} for browsing the
examples from the respective chapters. Additionally, this package will
contain vignettes describing aspects that did not make it into the
text.


This text was written with the \pkg{Sweave} package. To suppress
superflous output an assignment to a variable named \code{QT} is made
at times.
 



\newpage



%% Put these in front for easier reference.
\setcounter{tocdepth}{3}
\tableofcontents
\newpage



%% Find graphic files in subdirectories
\graphicspath{
  {ch-Preface/}
  {ch-GUIBasics/}
  {ch-ProgrammingPractices/}
  {ch-gWidgets/}
%%  {ch-gWidgetsExamples/}
  {ch-RGtk2/}
%%  {ch-RGtk2Example/}
  {ch-RwxWidgets/}
  {ch-tcltk/}
  {ch-WebGUIs/}
}

%% Begin. Chapters all call in ch-XXX/ch-XXX.tex file

\mainmatter
\chapter{The basic ideas of Graphical User Interfaces}
\label{chap:GUIBasics}



%% Goal is 

% \section{Introduction}
% \label{sec:GUI:introduction}
% \SweaveInput{Introduction}


\XXX{Space}
\XXX{HCI}
\XXX{http://www.useit.com/alertbox}
\XXX{Apple HIG}
\XXX{comment on radio vs. check http://www.useit.com/alertbox/20040927.html}
\section{A simple GUI in R}
\label{sec:GUI:tic-tac-toe}
%% Tic Tac Toe examples

We begin with an example showing how one can use \R's standard
graphics device as a canvas for a ``game'' of
tic-tac-toe against the computer. Although this example
has nothing to do with statistics, it illustrates, in a familiar way,
some of the issues involved in developing GUIs in \R. 


%% ML: the note about the MVC pattern can probably come after the
%% concrete model, view and controller have been presented. We can
%% look back and point out that the design of the example can be
%% generalized to MVC and that we will encounter such designs
%% throughout the book, even if it is not always explicit. It's OK to
%% generalize a little after each step, but I think we want to lead
%% with the concrete.

Many GUIs can be thought of as different views of some data model.  In
this example, the data simply consists of information holding the
state of the game, defined here in a global variable \code{board}.

%% Model
\begin{Schunk}
\begin{Sinput}
 board <- matrix(rep(0,9), nrow=3)      
\end{Sinput}
\end{Schunk}

%% View
A GUI contains one or more views, each of which is tied to an
underlying data model. In our case, the view is the game board that we
display through an \R\/ graphics device. The \function{layoutBoard}
function creates a canvas for this view:
\begin{Schunk}
\begin{Sinput}
 layoutBoard <- function() {
   plot.new()
   plot.window(xlim=c(1,4), ylim=c(1,4))
   abline(v=2:3);  abline(h=2:3)
   mtext("Tic Tac Toe. Click a square:")
 }
\end{Sinput}
\end{Schunk}
%
This example uses a single view; more complex GUIs
will contain multiple coordinated, interactive views. The layout of
the GUI should help the user navigate the interface and is an
important factor in usability. Here we benefit from the universal
familiarity with the board game.


%% Controller
The user typically sends input to a GUI through a mouse or keyboard. 
The underlying toolkit allows the programmer to assign
functions to be called when some specific event occurs, such as user interaction. Typically, the
toolkit \textit{signals} that some action has occurred, and then
invokes \textit{callbacks} or \textit{event handlers} that have been
assigned by the programmer. Each toolkit has a different implementation. 
For our game, we will use the \code{locator} function built
into the base \R\/ graphics system:
\begin{Schunk}
\begin{Sinput}
 doPlay <- function() {
   iloc <- locator(n=1, type="n")
   clickHandler(iloc)
 }
\end{Sinput}
\end{Schunk}
%
The \code{locator} function responds to mouse clicks. One specifies
how many mouse clicks to gather and the \textit{control} of the
program is suspended until the user clicks the sufficient number of
times (or somehow interrupts the loop). Such a GUI that blocks the
flow of a program contingent on user input is known as a
\textit{modal} GUI. This design is common for simple dialogs that
require immediate user attention, although in general a GUI will
listen asynchronously for user input.


In the above function \function{doPlay}, \function{clickHandler} is
an \defn{event handler}. Its job is to process the output of the
\function{locator} function, checking first if the user terminated
\function{locator} using the keyboard. If not it proceeds to draw the
move, and then, if necessary, the computer's move. Afterwards, play is
repeated until there is a winner or a ``cat's'' game.


\begin{Schunk}
\begin{Sinput}
 clickHandler <- function(iloc) {
   if(is.null(iloc)) 
     stop("Game terminated early")
   move <- floor(unlist(iloc))
   drawMove(move,"x")
   board[3*(move[2]-1) + move[1]] <<- 1
   if(!isFinished()) 
     doComputerMove()
   if(!isFinished()) 
     doPlay()
 }
\end{Sinput}
\end{Schunk}

The use of \verb+<<-+ in the handler illustrates a typical issue in
GUI design in \R.  User input changes the state of the application
through callback functions. These callbacks need to modify variables
in some shared scope, which may be application-wide or specific to a
component. The lexical scoping rules of \R, i.e., nesting of closures,
has proven to be a useful strategy for managing GUI state. When this
is inconvenient, direct manipulation of environment objects is a
viable alternative. In the above case, we simply modify the global
environment, which encloses \function{clickHandler}.

%% validation of user input
Validation of user input is an important task for a GUI. In the above,
the \function{clickHandler} function checks to see if the user
terminated the game early and issues a message.

%% Implement logic
At this point, we have a data model, a view of the model and the
logic that connects the two, but we still need to implement some of the
functions to tie it together.


This function draws either an ``x'' or an ``o''. It does so using the
\function{lines} function. The \code{z} argument contains the
coordinates of the square to draw.
\begin{Schunk}
\begin{Sinput}
 drawMove <- function(z,type="x") {
   i <- max(1,min(3,z[1])); j <- max(1,min(3,z[2]))
   if(type == "x") {
     lines(i + c(.1,.9),j + c(.1,.9))
     lines(i + c(.1,.9),j + c(.9,.1))
   } else {
     theta <- seq(0,2*pi,length=100)
     lines(i + 1/2 + .4*cos(theta), j + 1/2 + .4*sin(theta))
   }
 }
\end{Sinput}
\end{Schunk}

%% Resizing
One could use \code{text} to place a text ``x'' or ``o'', but this may
not scale well if the GUI is resized. Most GUI layouts allow for
dynamic resizing. This is necessary to handle the variety of data a
GUI will display. Even the labels, which one generally considers
static, will display different text depending on the language (as long
as translations are available).

%% JV: thanks
This function is used to test if a game is finished:
\begin{Schunk}
\begin{Sinput}
 isFinished <- function() {
   (any(abs(rowSums(board)) == 3) || 
    any(abs(colSums(board)) == 3) || 
    abs(sum(diag(board))) == 3 || 
    abs(sum(diag(apply(board, 2, rev)))) == 3)
 }
\end{Sinput}
\end{Schunk}
%
The matrix \code{m} allows us to easily check all the possible ways
to get three in a row.

This function picks a move for the computer:
\begin{Schunk}
\begin{Sinput}
 doComputerMove <- function() {
   newMove <- sample(which(board == 0),1) # random !
   board[newMove] <<- -1    
   z <- c((newMove-1) %% 3, (newMove-1) %/% 3) + 1
   drawMove(z,"o")
 }
\end{Sinput}
\end{Schunk}
%
The move is converted into coordinates using \code{\%\%} to get the
remainder and \code{\%/\%} to get the quotient when dividing an
integer by an integer. This function just chooses at random from the
left over positions; we leave implementing a better strategy to the
interested reader.

%% main equivalent
Finally, we implement the main entry point for our GUI:
\begin{Schunk}
\begin{Sinput}
 playGame <- function() {
   layoutBoard()
   doPlay()
   mtext("All done\n",1)
 }
\end{Sinput}
\end{Schunk}
%
When the game is launched, we first lay out the board and then call
\function{doPlay}. When \function{doPlay} returns, a message is written
on the screen.

This example adheres to the model-view-controller design pattern that
is implemented by virtually every complex GUI. We will encounter this
pattern throughout this book, although it is not always explicit.

%% endless tweak
For many GUIs there is a necessary trade-off between usability and
complexity. As with any software, there is always the temptation to
continually add features without regard for the long term cost. In
this case, there are many obvious improvements: implementing a better
artificial intelligence, drawing a line connecting three in a row when
there is a win, indicating who won, etc. Adding a feature adds
complexity to the interface, often useful, but sometime it just
increases the burden on the user.

\section{GUI Design Principles}
\label{sec:GUI:design}
% Section to introduce GUI design and principles through a comparison of
% three dialogs and general discussion


%% mac defs:

% Document windows contain file-based user data. They present a view
% into the content that people create and store. If the document is
% larger than the window, the window shows a portion of the document’s
% contents and provides users with the ability to scroll to other areas.

% Application windows are the main windows of applications that are not
% document-based. These windows can use the standard Aqua window look
% and features or (less frequently) the brushed metal look.

% Utility windows float above other windows and provide tools or
% controls that users can work with while documents are open. Utility
% windows (also called palettes) are discussed in more detail
% in “Utility Windows.” (page 202)

% Dialogs and alerts require a response from the user. These are
% discussed in “Dialogs.” (page 207)


The most prevalent pattern of user interface design is denoted WIMP,
which stands for Window, Icon, Menu and Pointer (i.e., mouse). The
WIMP approach was developed at Xerox PARC in the 1970's and later
popularized by the Apple Macintosh in 1984. This is particularly
evident in the separation of the window from the menu bar on the Mac
desktop. Other graphical operating systems, such as Microsoft Windows,
later adapted the WIMP paradigm, and libraries of reusable GUI
components emerged to support development of applications in such
environments. Thus, GUI development in R adheres to a WIMP approach.

The primary WIMP component from our perspective is the window. A
typical interface design consists of a top-level window referred to as
the \dfnref{document window} that shows the current state of a
``document,'' whatever that is taken to be. In \R\/ it could be a data
frame, a command line, a function editor, a graphic or an arbitrarily
complex form containing an assortment of such elements. 

%% JV: control or "action" here?
%% ML: tried to make this clearer

Abstractly, WIMP is a command language, where the user executes
commands, often called actions, on a document by interacting with
graphical controls. Every control in a window belongs to some abstract
menu. Two common ways of organizing controls into menus are the
menu bar and toolbar.

The parameters of an action call, if any, are controlled in
sub-windows. These sub-windows are termed \dfnref{application windows}
by Apple~\footcite{APPLE:HIG}, but we prefer the term \dfnref{dialogs},
or \dfnref{dialog boxes}. These terms may also refer to smaller
sub-windows that are used for alerts or confirmation. The program
often needs to wait for user input before continuing with an action,
in which case the window is modal. We refer to these as \dfnref{modal
  dialog boxes}.

Each window or dialog typically consists of numerous controls laid out
in some manner to facilitate the user interaction. Each window and
control is a type of \textit{widget}, the basic element of a
GUI. Every GUI is constituted by its widgets. Not all widgets are
directly visible by the user; for example, many GUI frameworks employ
invisible widgets to lay out the other widgets in a window.

There is a wide variety of available widget types, and widgets may be
combined in an infinite number of ways. Thus, there are often numerous
means to achieve the same goals. For example,
Figures~\ref{fig:GUI:print-dialogs} and \ref{fig:GUI:print-dialogs-R} show three dialogs that perform
the same task -- collect arguments from the user to customize the
printing of a document. Although all were designed to do the same
thing, there are many differences in implementation.

%% ML: Print dialogs might be problematic, especially with Firefox, which
%% (through GTK+) will use the platform native print dialog. On Linux,
%% there is no native dialog, so GTK+ implements one. This might not
%% be true of Firefox 2.0 specifically, but that is getting old
%% now. What about taking the print dialog from the R Windows GUI?

%% Principles of GUI layout
%% http://www.sylvantech.com/~talin/projects/ui_design.html has a nice list

\begin{figure}
  \centering
  \includegraphics[width=.75\textwidth]{fig-mac-print}
   \\
   
  \includegraphics[width=.75\textwidth]{kde-print}
  \caption{Two print dialogs. One from Mac OS X 10.6 and one from KDE 3.5.}
  \label{fig:GUI:print-dialogs}
\end{figure}


\begin{figure}
  \centering
  \includegraphics[width=.80\textwidth]{r-print-dialog}
  \caption{\R's print dialog under windows XP using XP's native dialog.}
  \label{fig:GUI:print-dialogs-R}
\end{figure}

%% Choice of widget -- familiar metaphors, use of icons, 
In some cases, typical usage suggests one control over another. The
choice of printer for each is specified through a combo box. However,
for other choices a variety of widgets are employed. For example, the
control to indicate the number of copies for the Mac is a simple text
entry window, whereas for the KDE and \R\/ dialog it
is a spinbutton. The latter minimizes user error, say through entering
a non-positive integer. The KDE and Mac dialogs have icons to
compactly represent actions, whereas the \R\/ example has none. The
landscape icon for the Mac is very clear and provides this feature
without having to use a sub dialog.


%% Choice of layout -- positioning, focus, use of spacing, center
%% balance, vs. ...
How the interfaces are laid out also varies.  All
panels are read top to bottom, although the Mac interface also has a very
nice preview feature on the left side. The KDE dialog uses frames to
separate out the printer arguments from the arguments that specify how
the print job is to proceed. The Mac uses a vertical arrangement to
guide the user through this. For the Mac, horizontal separators are
used instead of frames to break up the areas, although a frame is used
towards the bottom. Apple uses a center balance for its controls. They
are not left justified as are the KDE and Windows dialogs. Apple has
strict user-interface guidelines and this center balance is a design
decision.

%% feature exposure, Choice of options -- what to show, what to leave out
The layout also determines how many features and choices are visible to the
user at a given time.  For example, the Mac GUI uses ``disclosure
buttons'' to allow access to printer properties and the PDF settings,
whereas KDE uses a notebook container to show only a subset of the
options at once.

%% state visualization: sensitive/not; focus, not, 
The Mac GUI provides a very nice preview of the current document
indicating to the user clearly what is to be printed and how
much. Adjusting GUIs to the possible state is an important user
interface property.  GUI areas that are not currently sensitive to
user input are grayed out. For example, the ``collate'' feature of the
GUI only makes sense when multiple copies are selected, so the
designers have it grayed out until then. A common element of GUI
design is to only enable controls when their associated action is
possible, given the state of the application.

%% shortcuts -- default button, keyboard accelerators
 
The Mac GUI has the number of pages in focus, whereas Windows places
the printer in focus. This allows the user to interact with the GUI
without the mouse. Typically the \kbd{tab} key is used to step through
the controls. GUI's often have shortcuts that allow power users to
initiate actions or shift the focus directly to a specific widget
through the keyboard.  Most dialogs also have a default button, which
will initiate the dialog action when the \kbd{return} key is
pressed. The KDE dialog, for example, indicates that the ``print''
button is the default button through special shading.

%% help
% For such a common dialog, it is unlikely the user will need help. As
% such the Windows dialog does not provide a link. However, the
% KDE and Mac dialogs do. A dialog should provide assistance for
% complex and unfamiliar tasks.

%% safety -- postion of buttons
%% ML: Do we really want to mention something that is not applicable?
%% JV; Agreed
% The Apple human interface guidelines suggest putting buttons that can
% cause the destruction of data separate from other control buttons. As
% this isn't directly applicable here, we see that Apple does separate
% buttons that are common to many dialogs (cancel, print) from the ones
% specific to the dialog. The KDE buttons have nice icons, but their
% similar, but irregular, sizing is a bit unusual.


Each dialog presents the user with a range of buttons to initiate or
cancel the printing. The Windows ones are set on the right and consist
of the standard ``OK'' and ``Cancel'' buttons. The Mac interface uses
a spring to push some buttons to the left, and some to the right to
keep separate their level of importance. The KDE buttons do so as
well, although one can't tell from this. However, one can see the use
of stock icons on the buttons to guide the user. 


%% JV should we leave this in? I'm wondering. We do it better in the
%% chapters perhaps
\section{Controls}
\label{sec:GUI:basic-components}
%% ML: This section and the next should probably be reorganized so
%% that they do a better job of referring to examples. Otherwise it
%% feels like we're stating a series of groundless generalities.

This section provides an overview of many common controls, i.e.,
widgets that either accept input, display data or provide visual
guides to help the user navigate the interface. If the reader is
already familiar with the conventional types of widgets and how they
are arranged on the screen, this section and the next should be
considered optional.


\subsection{Choice of control}
\label{sec:choice-widget}
%% real estate, type of data

A GUI is comprised of one or more widgets. The appropriate choice
depends on a balance of considerations.  For example, many widgets
offer the user a selection from one or more possible choices.  An
appropriate choice depends on the type and size of the information
being displayed, the constraints on the user input, and on the space
available in the GUI layout. As an example,
Table~\ref{tab:gui-design-widget-type} suggests different types
of widgets used for this purpose depending on the type and size of
data and the number of items to select.
  
\begin{table}
\centering
\label{tab:gui-design-widget-type}
\caption{Table of possible selection widgets by data type and size}
\begin{tabular}{@{}lp{0.35\textwidth}p{0.35\textwidth}@{}}
\toprule

Type of data&Single&Multiple\\
\midrule
Boolean&Checkbox, toggle button&-\\Small list&radio button group\newline combo box\newline list box&checkboxgroup\newline list box\\Moderate list&combo box\newline list box&list box\\Large list&list box, auto complete&list box\\Sequential&slider\ spin button&\\Tabular&table&table\\Hierarchical&tree&tree
\\ \bottomrule
\end{tabular}
\end{table}

Figure~\ref{fig:GUI:spss-11-term-selection} shows several such
controls in a single GUI. A checkbox enables an intercept,
a radio group selects either full factorial or a custom
model, a combo box selects the ``sum of squares'' type, and a
list box allows for multiple selection from the available
variables in the data set. 



\begin{figure}
  \centering
  \includegraphics[width=.65\textwidth]{spss-11-model-selection}
 \caption{A dialog box from SPSS version 11 for specifying terms
    for a linear model. The graphic shows a dialog that allows
    the user to specify individual terms in the model  using
    several types of widgets for selection of values, such as a radio button
    group, a checkbox, combo boxes, and list boxes. }
  \label{fig:GUI:spss-11-term-selection}
\end{figure}

%% Metaphors, user base
For many \R\/ object types there are natural choices of widget. For
example, values from a sequence map naturally to a slider or spin
button; a data frame maps naturally to a table widget; or a list with
similar structure can map naturally to a tree widget. However, certain
\R\/ types have less common metaphors. For instance, a formula object
can be fairly complex. Figure~\ref{fig:GUI:spss-11-term-selection}
shows an SPSS dialog to build terms in a model. \R\/ power users may
be much faster specifying the formula through a text entry box, but
beginning \R\/ users coming to grips with the command line and the
concept of a formula may benefit from the assistance of a well
designed GUI. One might desire an interface that balances the needs of
both types of user, or the SPSS interface may be appropriate. Knowing
the potential user base is important.






\subsection{Presenting options}
\label{sec:GUI:basic-selection}

The widgets that receive user input need to translate that input into
a command that modifies the state of the application. Commands, like
\R{} functions, often have parameters, or options. For many options,
there is a discrete set of possible choices, and the user needs to
select one of them. Examples include selecting a data frame from a list of data
frames, selecting a variable in a data frame, selecting certain cases
in a data frame, selecting a logical value for a function argument,
selecting a numeric value for a confidence level or selecting a string
to specify an alternative hypothesis. Clearly there can be no
one-size-fits-all widget to handle the selection of a value.

% XXX REDO FIGURE
% \begin{figure}
%   \centering
%   \includegraphics[width=.45\textwidth]{spss-11-model-selection}
%   \includegraphics[width=.45\textwidth]{spss-11-one-way-anova}
%   \caption{Two dialog boxes from SPSS version 11 for specifying terms
%     for a linear model. The left graphic shows a dialog that allows
%     the user to specify individual terms in the model. This uses
%     several types of widgets for selection of values, such as a radio
%     group, a checkbox, combo boxes, and list boxes. The right graphic
%     shows a dialog that allows the user to specify response variables
%     and a grouping variable for a one-way ANOVA.}
%   \label{fig:GUI:spss-11-model-selection}
% \end{figure}

\subsubsection{Checkboxes}
\label{sec:GUI:checkboxes}

A \dfn{checkbox} specifies a value for a logical
(boolean) option. Checkboxes have labels to indicate which variable is
being selected. Combining multiple checkboxes into a group allows for
the selection of one or more values at a time.

%% tcltk examples
\begin{figure}
  \centering
  \includegraphics[width=.35\textwidth]{ex-listbox}
  \includegraphics[width=.35\textwidth]{tcltk-tkdensity}
  \caption{
    Two applications of the \pkg{tcltk} package. 
    %% 
    The left graphic is
    produced by \command{chooseCRANmirror} and uses a list box to
    allow selection from a long list of possibilities.
    %% 
    The right graphic is the \code{tkdensity} demo from the
    package. It uses radio buttons and a slider to select the
    parameter values for a density plot.
  }
  \label{fig:GUI:ex-tcltk}
\end{figure}

\subsubsection{Radio buttons}
\label{sec:GUI:radio=button-groups}

A \dfn{radio button group} selects exactly one value from a vector of
possible values. The analogy dates back to old car radios where there
were a handful of buttons to select a preset channel. When a new
button was pushed in, the previously pressed button popped up.  Radio
button groups are useful, provided there are not too many values to
choose from, as all the values are shown. These values can be arranged
in a row, a column or both rows and columns to better fill the
available space. Figure~\ref{fig:GUI:ex-tcltk} uses radio button
groups for choosing the distribution, kernel and sample size for the
density plot.

\subsubsection{Combo boxes}
\label{sec:GUI:combo-boxes}

A \dfn{combo box} is similar to a radio button group, in that it is
used to select one value from several. However, a combo box only
displays the value currently selected, which reduces visual complexity
and saves space, at the cost of an extra click to show the
choices. Toolkits often combine a combo box with a text entry area for
specifying an arbitrary value, possibly one that is not represented in
the set of choices. A combo box is generally desirable over radio
buttons when there are more than four or five choices. However, the
combo box also has its limits. For example, some web forms require
choosing a country from a list of hundreds. In such cases, features
like incremental type ahead search enhance usability.

\subsubsection{List boxes}

A \dfn{list box} displays a list of possible choices in a column.
While the radio button group and combo box select only a single value,
a list box supports multiple selection. Another difference is that the
number of displayed choices depends dynamically on the available
space. If a list box contains too many items to display them
simultaneously, a scrollbar is typically provided for adjusting the
visible range. Unlike the combo box, the choices are immediately visible
to the user.  Figure~\ref{fig:GUI:ex-tcltk} shows a list
box created by \R\/ that is called from the function
\command{chooseCRANmirror}. There are too many mirrors to fit on the
screen, but a combo box would not take advantage of the available
space. The list box is a reasonable compromise.

\subsubsection{Sliders and spinbuttons}
\label{sec:GUI:sliders}

A \dfn{slider} is a widget that selects a value from a sequence of
possible values typically through the manipulation of a knob that
moves or ``slides'' along a line that represents the range of possible
values. 
%%Some toolkits (e.g. Java/Swing) only allow for the sequence to
%%have integer values.  
Some toolkits generalize beyond a numeric
sequence. The slider is a good choice for offering the user a
selection of ordinal or numerical parameter values. For example, the
letters of the alphabet could be a sequence. The \code{tkdensity} demo
of the \pkg{tcltk} package (Figure~\ref{fig:GUI:ex-tcltk}) uses a
slider to dynamically adjust the bandwidth of a density estimate.

A \dfn{spin button} plays a similar role to the slider, in that it
selects a value within a set of bounds. Typically, this widget is
drawn with a text box displaying the current value and two arrows to
increment or decrement the selection. The text box can usually be
edited directly.  A spin button has the advantage of using less screen
space, and directly entering a specific value, if known, is easier
than selecting it with a slider. One disadvantage is that the position
of the selected value within the range is not as obvious compared to
the slider. As a compromise, combining a text box with a slider is
possible and often effective. A spin button is used in the KDE print
dialog of Figure~\ref{fig:GUI:print-dialogs} to adjust the number of
copies.

\subsection{Initiating an action}

After the user has specified the parameters of an action, typically
by interacting with the selection widgets presented above, it comes time to
execute the action. Widgets that execute actions include the familiar
buttons, which are often organized into menubars and toolbars.

\subsubsection{Buttons}
\label{sec:GUI:buttons}

A \dfn{button} issues commands when invoked, usually via a mouse click.
In Figure~\ref{fig:GUI:print-dialogs}, the ``Properties'' button, when
clicked, opens a dialog for setting printer properties. The button
with the wizard icon also opens a dialog.  As buttons execute an
action, they are often labeled with a verb.~\footcite{APPLE:HIG} In
Figure~\ref{fig:GUI:spss-11-term-selection} we see how SPSS uses
buttons in its dialogs: buttons which are not valid in the current
state are disabled; buttons which are designed to open subsequent
dialogs have trailing dots; and the standard actions of resetting the
data, canceling the dialog or requesting help are given their own
buttons on the right edge of the dialog box.

To speed the user through a dialog, a button may be singled out as the
default button, so its action will be called if the user presses the
\kbd{return} key. Actions may be given shortcut bindings, and their
button proxies typically reflect the proper key combination to invoke
the action The KDE print dialog in Figure~\ref{fig:GUI:print-dialogs}
has these bindings indicated through the underlined letter on the
button labels.

%% ML: besides the accelerator comment, the below might be too much detail
%% JV: I'll leave it for the toolkit chapters if appropriate.
%% adjustments
% The look of the button can usually be manipulated.  A button is given
% a relief through its border, shading, and perhaps a color gradient
% along its face. Some toolkits allow these to be optionally drawn,
% thereby making a button look more like a label, as described below.
% The button text may have some markup or an indication of a accelerator
% keyboard binding, such as the \text{\underline{C}ontrasts...} button
% in the dialog shown in the right graphic of
% Figure~\ref{fig:GUI:spss-11-model-selection}.

\subsubsection{Icons}
\label{sec:GUI:icons}

In the WIMP paradigm, an \dfn{icon} is a pictorial representation of a
resource, such as a document or program, or, more generally, a
concept, such as a type of file. An application GUI typically adopts
the more general definition, where an icon is used to augment or
replace a text label on a button, a toolbar, in a list box, etc. When
icons appear on toolbars and buttons, they are associated with
actions, so an icon should be a pictorial representation of an
action. The choice of icons can have a significant impact on usability
and appearance.

% ML: too much detail?
% JV: agreed
% Except for the default installation of \pkg{tcltk}, images and icons
% may be specified in a variety of different formats.  Icons can come in
% several different sizes from 16 by 16 pixels to 128 by 128. For
% toolbars and menu bars, the toolkit takes care of selecting the
% appropriate icon.


\subsubsection{Menu Bars}
\label{sec:GUI:menubars}

Menus play a central role in the \acronym{WIMP} desktop. The \dfn{menu
  bar} contains items for many of the actions supported by the
application.  By convention, menu bars are associated with a top-level
window. This is enforced by some toolkits and operating systems, but
not all. In Mac OS X, the menu bar appears on the top line of the
display, but other platforms place the menu bar at the top of the
top-level window. In a statistics application, the ``document'' may be
viewed, for example, as the active data frame, a report, or a graphic.

The styles used for menu bars are fairly standardized, as this allows
new users to quickly orient themselves with a GUI. The visible menu
names are often in the order \code{File}, \code{Edit}, \code{View},
\code{Tools}, then application specific menus, and finally a
\code{Help} menu. Each visible menu item when clicked opens a menu of
possible actions. The text for these actions conventionally use a
\code{...}  to indicate that a subsequent dialog will open so that
more information can be gathered to complete the action. The text may
also indicate a key-board accelerator, such as \code{Find
  \underline{N}ext F3} indicating that both ``N'' as a keyboard
accelerator and F3 as a shortcut will initiate this same
action. (Shortcuts are not translated, but keyboard accelerators must
be. As such, their use is less so. In particular, keyboard
accelerators are not supported in Mac OS X menus.)

Not all actions will be applicable at any given time. It is
recommended that rather than deleting these menu items, they be
disabled, or grayed out, instead. %%~\ref{KDE:HIG}

Menus may come to contain many items. To help the user navigate, menu
items are usually grouped with either horizontal separators or
hierarchical submenus. %%The latter are indicated with an arrow.

The use of menus has evolved to also allow the user to set properties
or attributes of current state of the GUI. There may be checkboxes
drawn next to the menu item or some icon indicating the current state.

Another use of menus is to bind contextual menus (popup menus) to
certain mouse clicks on GUI elements. Typically right mouse clicks
will pop up a menu that lists often-used commands that are appropriate
for that widget and the current state of the GUI. In Mac OS X
one-button users, these menus are bound to a \kbd{control}-click.

\subsubsection{Toolbars}
\label{sec:GUI:toolbars}

Toolbars are used to give immediate access to the frequently used actions
defined in the menu bar. Toolbars typically have icons representing the
action and perhaps accompanying text. They traditionally appear on the
top of a window, but sometimes are used along the edges. 

%% Might not be the best place... any other options?
\subsubsection{Action Objects}
\label{sec:GUI:actions}

When clicking on a button, the user expects some ``action'' to
occur. For example, some save dialog is summoned, or some page is
printed.  GUI toolkits commonly represent such actions as formal,
invisible objects that are proxied by widgets, usually buttons, on the
screen.  Often, all of the primary commands supported by an
application have a corresponding action object, and the buttons
associated with those actions are organized into menu bars and
toolbars.

An action object is essentially a data model, with each proxy widget
acting as a view. Common components of an action include a textual
label, an icon, perhaps a shortcut, and a handler to call
when the action is selected.
%% JV this repeats the above
% When a particular action is not possible
% due to the state of the GUI, it should be disabled, so that the
% associated widgets are not sensitive to user interaction.


\subsection{Modal dialogs}
\label{sec:GUI:modal-dialogs}

A \dfnref{modal dialog box} is a dialog box that keeps the focus until
the user takes an action to dismiss the box. It prompts a user for
immediate input, for example asking for confirmation when overwriting
a file. Modal dialog boxes can be disruptive to the flow of
interaction, so are used sparingly. As the control flow is
blocked until the window is dismissed, functions that display modal
dialogs can return a value when an event occurs, rather than have a
handler respond to asynchronous input. The \command{file.choose}
function, mentioned below, is a good example. When called during an
interactive \R\/ session, the user is unable to interact with the
command line until a file has been specified and the dialog dismissed.

\subsubsection{Message dialogs}
\label{sec:GUI:message-dialogs}

A \dfnref{message dialog} is a high-level dialog widget for
communicating a message to the user. By convention, there is a small
rectangular box that appears in the middle of the screen with an icon
on the left and a message on the right. At the bottom is a button to
dismiss the dialog, often labeled ``OK.''  Additional
buttons/responses are possible. The \dfnref{confirmation dialog}
variant would add a ``Cancel'' button which invalidates the proposed
action.

\subsubsection{File choosers}
\label{sec:GUI:file-choosers}

A file chooser allows for the selection of files and directories. They
are familiar to any user of a GUI. A typical \R\/ installation has the
functions \command{file.choose} and \command{tkchooseDirectory} (in
the \pkg{tcltk} package) to select files and directories.

Other common choosers are color choosers and font choosers.

\subsection{Displaying data}
\label{sec:GUI:tabular-display}

Table and tree widgets support the display and manipulation of tabular
and hierarchical data, respectively. More arbitrary data
visualization, such as statistical plots, can be drawn within a GUI
window, but such is beyond the scope of this section.

%% JV: Need to include filtering example here

\subsubsection{Tabular display}

A \dfn{table widget} shows tabular data, such as a data frame, where
each column has a specific data type and cell rendering strategy.
Table widgets handle the display, sorting and selection of records
from a dataset. Depending on the configuration of the widget, cells
may be editable.  Figure~\ref{fig:GUI:spotfire} shows a table widget
in a Spotfire web player demonstration. 

\begin{figure}
  \centering
  \includegraphics[width=0.8\textwidth]{fig-spotfire}
  \caption{A screen shot from Tibco's Spotfire web player illustrating
    a table widget (lower left), displaying the cases that are
    summarized in the graphic. The right bar filters the cases in the table. }
  \label{fig:GUI:spotfire}
\end{figure}


% \begin{figure}
%   \centering
% %%  \includegraphics[width=.4\textwidth]{JGR-data-editor}
%   \includegraphics[width=.55\textwidth]{fathom-2-1-xyplot}
%   \caption{
%     Two windows showing the use of table widgets.
%     %%
    
% %     The left graphic shows the data editor from \pkg{JGR} using the
% %     table widget in Java.  
%     %%
%     The right graphic shows a data table and a graph in Fathom 2.1
%     with two views of the same data. One view uses a table widget, the
%     other a graph. Changes to one or the other views cause an update
%     to the underlying model. This model then will notify its various
%     displays to update. This arrangement allows for dynamic linking of
%     the table and the graph.}
%   \label{fig:GUI:table-widgets}
% \end{figure}

\subsubsection{Tree widgets}
\label{sec:GUI:tree-widgets}

So far, we have seen how list boxes display homogeneous vectors of
data, and how table widgets display tabular data, like that in a data
frame. Other widgets support the display of more complex data
structures. If the data has a hierarchical structure, then a \dfn{tree
  widget} may be appropriate for its display. Examples of hierarchical
data in \R\/ are directory structures, the components of a list, or
class hierarchies. The object browser in \pkg{JGR} uses a tree
widget to show the components of the objects in a users session (the
left graphic of Figure~\ref{fig:GUI:R-guis-exs-JGR-Rcmdr}). The root
node of the tree is the ``data'' folder, and each data object in the
global workspace is treated as an offspring of this root node. For the
data frame \code{iraq}, its variables are considered as offspring of
the data frame. In this case these variables have no further
offspring, as indicated by the ``page'' icon.

%%%%%%%%%%%%%%%%%%%%%%%%%%%%%%%%%%%%%%%%%%%%%%%%%% 
\subsection{Displaying and editing text}
\label{sec:GUI:text-widgets}

The letter \acronym{P} in \acronym{WIMP} stands for ``pointer,'' so it
is unsurprising that \acronym{WIMP} GUIs are designed around the
pointing device. The keyboard is generally relegated to a secondary
role, in part because it is difficult to both type and move the mouse
at the same time. For statistical GUIs, especially when integrating
with the command-line interface of \R\/, the flexibility afforded by
arbitrary text entry is essential for any moderately complex
GUI. Toolkits generally provide separate widgets for text entry
depending on whether the editor supports a single line or multiple
lines.

\subsubsection{Single line text}
\label{sec:GUI:single-line-text}

A text entry widget for editing a single line of text is found in the
KDE print dialog (Figure~\ref{fig:GUI:print-dialogs}). It specifies
the page range. Specifying a complex page range, which might include
gaps, would require a complex point-and-click interface. In
order to avoid complicating the GUI for a feature that is rarely
useful, a simple language has been developed for specifying page
ranges. There is overhead involved in the parsing and validation of
such a language, but it is still preferable to the alternative.

\subsubsection{Text edit boxes}
\label{sec:GUI:textboxes}

The right graphic
of Figure~\ref{fig:GUI:R-guis-exs-JGR-Rcmdr} shows three multi-line text entries
in an \pkg{Rcmdr} window. It provides an \R\/ console and status message area. The
``Output Window'' demonstrates the utility of formatting
attributes. In this case, attributes specify the color of
the commands, so that the input can be distinguished from the output.

\begin{figure}
  \centering
  \includegraphics[width=.35\textwidth]{JGR-object-browser}
  \includegraphics[width=.5\textwidth]{Rcmdr-main-window}
  \caption{
    Some windows from \R\/ GUIs.
    %% 
    The left graphic shows the object browser in the \pkg{JGR} GUI
    using a tree widget 
    to display the possibly hierarchical nature of \R\/ objects.
    %%
    The right graphic shows the main Rcmdr (1.3-11) window
    illustrating the use of multi-line text entry areas for a command
    area, an output area and a message area.}
  \label{fig:GUI:R-guis-exs-JGR-Rcmdr}
\end{figure}


\XXX{Not needed here?}
% %% A table showing the values and constructors
% %% Make changes to gnumeric spreadsheet, export
% {\small
% \newcommand{\PARASIZE}{1.25in}
% \newcommand{\LARGEPARASIZE}{1.45in}
% \begin{landscape}
%   \begin{table}[tbp]
%     \centering
%     \begin{minipage}{1.0\textwidth}
%       \begin{tabular}{lp{\PARASIZE}@{\quad}p{\LARGEPARASIZE}@{\quad}p{\PARASIZE}@{\quad}p{\PARASIZE}@{\quad}p{\PARASIZE}@{\quad}c}
%         %%
%         Widget & \code{gWidgets} & \code{RGtk2} & \code{RwxWidgets} &
%         \code{tcltk}~\footnote{Some constructors require add-on
%           libraries, as indicated by parentheses.} & \code{rJava} &\\
%         \hline
%         \SweaveInput{widgets-constructors}
%       \end{tabular}
%     \end{minipage}
%     \caption{A table listing several common widgets with a constructor for
%       different toolkits discussed in the text.}
% \label{tab:GUI:widgets-constructors}
%   \end{table}
% \end{landscape}
% }

%%%%%%%%%%%%%%%%%%%%%%%%%%%%%%%%%%%%%%%%%%%%%%%%%% 
\subsection{Guides and feedback}
\label{sec:GUI:info-display}

Some widgets display information but do not respond to user
input. Their main
purpose is to guide and the user through the GUI and to display
feedback and status messages.

\subsubsection{Labels}
\label{sec:GUI:labels}
%% Static Text
A label is a widget for placing text into a GUI that is typically not
intended for editing, or even selecting with a mouse. The main role of
a label is to describe another component of the GUI. Most toolkits
support rich text in labels. Figure~\ref{fig:GUI:R-guis-exs-JGR-Rcmdr}
shows labels marked in red and blue in \pkg{tcltk}.

\subsubsection{Statusbars}
\label{sec:GUI:statusbars}

A statusbar displays general status messages, as well as feedback on
actions initiated by the user, such as progress or errors. In the
traditional document-oriented GUI, statusbars are placed at the bottom.

Related to status bars are info bars or alert boxes, that allow a
programmer to leave a transient message for the user usually just
below the toolbar.

\subsubsection{Tooltips}
\label{sec:GUI:basic-tooltips}

A tooltip is a small window that is displayed when a user hovers their
mouse over a tooltip-enabled widget. These are an embellishment for
providing extra information about a particular piece of content
displayed by a widget. A common use-case is to guide new users of a
GUI. Many toolkits support the display of interactive hypertext in a
tooltip, which allows the user to request additional details.

\subsubsection{Progress bars}

A progress bar indicates progress on a particular task, which may or
may not be bounded. A bounded progress bar usually reports progress in
terms of percentage completed. Progress bars should be familiar, as
they are often displayed during software installation and while
downloading a file. For long-running statistical procedures they can
give useful feedback to the user that something is happening.

%% combined with modal dialogs
% \subsection{Choosers}
% \label{sec:GUI:choosers}

% Certain standard widgets are used to select values from a range
% defined by the system the user is on.


% \subsubsection{Color choosers}
% \label{sec:GUI:color-pickers}

% A color picker allows the selection of a color. 

% \subsubsection{Font choosers}
% \label{sec:GUI:font-choosers}

% A font chooser allows the selection of a font. 


\section{Containers}
\label{sec:GUI:basic-components-containers}
%% Containers

%%% FIXME: move to \paragraph{} instead of \subsubsection{}?

%% ML: Layout management is orthogonal to the type of container. This
%% is made obvious in the design of Swing and Qt: there is a separate
%% class hierarchy for layout managers, which are assigned to
%% container widgets. The container could be invisible or provide an
%% interface around its children, as with a frame, notebook or top-level
%% window. Thus, in Qt, a frame container could lay out its children
%% as a box, a grid, etc. I suggest discussing container types first,
%% and then discuss layout management policies, explaining that some
%% toolkits have widgets specific to a particular policy (and so
%% require a lot of nesting), while in others any container type is
%% capable of any type of layout.


%% JV: good point. 
% Widgets are arranged in a window to produce a GUI. Container widgets
% manage the layout. The simplest containers are like boxes
% that get packed in left to right or top to bottom. These boxes may be
% decorated with a frame or label, or may have some means of being
% hidden or displayed by the user. The nesting of box containers can
% provide a great deal of flexibility, but usually not
% enough. An example of a more flexible layout strategy is to position
% widgets on a grid.

%% Widget Hierarchy

The KDE print dialog of Figure~\ref{fig:GUI:print-dialogs} contains
many of the widgets we discussed in the previous section. Before we
can create such a dialog, we need to introduce how to position widgets
on the screen. This process is called \textit{widget layout}.

A layout emerges from the organization of the widgets into a
hierarchy, where a parent widget positions its children within its
allocated space.  The top-level window is parentless and forms the
root of the hierarchy. A parent visually contains its children and
thus is usually called a \defn{container}. This design is natural,
because almost every GUI has a hierarchical layout. It is easy to
apply a different layout strategy to each region of a GUI, and when a
parent is added or removed from the GUI, so are its children.

It is sometimes tempting for novices to simply assign a fixed position
and dimensions for every widget in a GUI. However, such static layouts
do not scale well to changes in the state of the application or simply
changes to the window size dictated by the window manager. Thus, it is
strongly encouraged to delegate the responsibility of layout to a
\defn{layout manager} that dynamically calculates the layout as
constraints change. Depending on the toolkit, the layout manager might
be the container itself, or it might be a separate object to which the
container delegates.

Regardless, the type of layout is generally orthogonal to the type of
container. For example, a container might draw a border around its
children, and this would be independent of how its children are laid
out.  The rest of this section is divided into two parts: container
widgets and layout algorithms. We will continually refer back to the
KDE print dialog example as we proceed.

% The Apple guidelines\footcite[Ch. 15]{APPLE:HIG} suggest using ``center
% equalization'' for arranging widgets within a window. This means that
% the visual weight is balanced between the right and left side of the
% content area. This is not the case with the KDE print dialog.

\subsection{Containers}
\label{sec:containers}

\subsubsection{Top level windows}
\label{sec:GUI:top-level-windows}

The top-level window of a GUI is the root of the container
hierarchy. All other widgets are contained within it.  The
conventional main application window will consist of a menu bar, a
tool bar and a status bar. The primary content of the window is
inserted between the tool bar and the status bar, in an area known as
the \dfnref{client area} or \dfnref{content area}. In the case of a
dialog, the content usually appears above a row of buttons, each of
which represent a possible response. The print dialog conforms to the
dialog convention. The print options fill the content area, and there
is a row of buttons at the bottom for issuing a response, such as
``Print''.

A window is typically decorated with a title and buttons to iconify,
maximize, or close. In the case of the print dialog, the top-level
window is entitled ``Print -- KPDF.''. Besides the text of the title,
the decorations are generally the domain of the window manager (often
part of the operating system). The application controls the contents
of the window.

Once a window is shown, its dimensions are managed by the user,
through the window manager. Thus, the programmer must size the window
before it becomes visible. This is often referred to as the
``default'' size of the window. Positioning of a top-level window is
generally left to the window manager.

The top-level window forwards window manager events to the
application. For example, an application might listen to the window
close event in order to prompt a user if there are any unsaved changes
to a document.

% Table~\ref{tab:GUI:containers-constructors} lists them
% together and provides the constructor name for the different toolkits
% discussed in this book.

\subsubsection{Frames}
\label{sec:GUI:frames}

A frame is a simple container that draws a border, possibly with a
label, around its child. The purpose of a frame is to enhance
comprehension of a GUI by visually distinguishing one group of
components from the others. The displayed page of the notebook in
Figure~\ref{fig:GUI:print-dialogs} contains two frames, visually
grouping widgets by their function: either \code{Page Selection}
or \code{Output Settings}.

\subsubsection{Tabbed notebooks}
\label{sec:GUI:notebooks}

A notebook represents each of its children as a page in a notebook. A
page is selected by clicking on a button that appears as a tab. Only a
single child is shown at once. The tabbed notebook is a space
efficient, categorizing container that is most appropriate when a user
is only interested in one page at a time. Modern web browsers take
advantage of it to allow several web pages to be open at once within
the same window. In the KDE print dialog, detailed options are
collapsed into a notebook in order to save space and organize the
multitude of options into simple categories: ``Copies'', ``Advanced
Options'', and ``Additional Tags''.

% %% A table showing the values and constructors
% %% Make changes to gnumeric spreadsheet, export
% {\small
% \newcommand{\PARASIZE}{1.25in}
% \begin{landscape}
%   \begin{table}
%     \centering
%     \caption{A table listing several containers with a constructor for the
%     different toolkits discussed in the text.}
%     \begin{tabular}{lp{\PARASIZE}@{\quad}p{\PARASIZE}@{\quad}p{\PARASIZE}@{\quad}p{\PARASIZE}@{\quad}p{\PARASIZE}@{\quad}c}
%       Widget & \code{gWidgets} & \code{RGtk2} & \code{RwxWidgets} &
%       \code{tcltk} & \code{rJava} &\\
%       \hline
%       \SweaveInput{containers-constructors}
%     \end{tabular}
%     \label{tab:GUI:containers-constructors}
%   \end{table}
% \end{landscape}
% }


\subsubsection{Expanding boxes}
\label{sec:GUI:expanding-boxes}

An expanding container, or box, will show or hide its children, according to the
state of a toggle button. By way of analogy, radio buttons are to
notebooks as check buttons are to expanding containers. An expanding box
allows the user to adapt a GUI to a particular use case or mode of
operation. Often, an expanding box contains so-called ``advanced''
widgets that are only occasionally useful and are only of interest to
a small subset of the users. For example, the \code{Options} button in
Figure~\ref{fig:GUI:print-dialogs} controls an expanding box that
contains the print options, which are usually best left to their
defaults.

\subsubsection{Paned boxes}
\label{sec:GUI:paned-boxes}

Usually, a layout manager allocates screen space to widgets, but
sometimes the user needs to adapt the allocation, according to a
present need. For example, the user may wish to increase the size of
an image to see the fine details. The \dfnref{paned container}
supports this by juxtaposing two panes, either vertically (stacked) or
horizontally. The area separating the two panes, sometimes called a
\dfnref{sash}, can be adjusted by dragging it with the mouse.
   
\subsection{Layout algorithms}
\label{sec:GUI:layout}

\subsubsection{Box layout}
\label{sec:GUI:Box-containers}

The box container is perhaps the simplest and most common type of
layout container. A box will pack its children either horizontally or
vertically. Usually, the widgets are packed from left to right, for
horizontal boxes, or from top to bottom, in the case of a vertical
box.  The upper left figure in Figure~\ref{fig:GUI:box-possibilities}
illustrates this.

The box layout needs to allocate space to its children in both the
vertical and horizontal directions. The typical box layout algorithm
begins by satisfying the minimum size requirements of its
children. The box may need to request more space for itself in order
to meet the requirement. 

Once the minimum requirements are satisfied, it is conventional and
usually desirable for the widgets to fill the space in the direction
orthogonal to the packing. For example, widgets in a horizontal box
will fill all of their vertical space (the upper right graphic in
Figure~\ref{fig:GUI:box-possibilities} shows some fill
possibilities). When this is not desired, most box widgets support
different ways of vertically (or horizontally) aligning the widgets
(the lower left graphic in Figure~\ref{fig:GUI:box-possibilities}).

More complex logic is involved in the allocation of space in the
direction of packing. Any available space after meeting minimum
requirements needs to be either allocated to the children or left
empty. This depends on whether any children are set to expand. The
available space will be distributed evenly to all expanding
children. Each child may fill that space or leave it empty. The
non-expanding children are simply packed against their
side of the container. If there are no expanding children, the
remaining space is left empty in the middle (or end if there are no
widgets packed against the other side). See the lower right
panel in Figure~\ref{fig:GUI:box-possibilities}. One could think
of this space being occupied by an invisible spring. Invisible
expanding widgets also act as springs.


\begin{figure}
  \centering
  \includegraphics[width=.40\textwidth]{fig-basics-hor-ver}
  \includegraphics[width=.40\textwidth]{fig-basics-fill}\\
  \includegraphics[width=.40\textwidth]{fig-basics-alignment}
  \includegraphics[width=.40\textwidth]{fig-basics-spring}
  \caption{
   %% 
    Different possibilities for packing child components within
    a box. 
    %% 
    The upper left shows horizontal and vertical layout.
    %% 
    The upper right shows some possible alignments or anchorings.
    %% 
    The lower left shows that a child could ``expand'' to fill the space
    either horizontally, vertically, or both.
    %% 
    The lower right shows both a fixed amount of space between the
    children and an expanding spring between the child components.  }
  \label{fig:GUI:box-possibilities}
\end{figure}

The button box in the KDE print dialog shows five buttons as child
components. At first glance the sizing appears to show that each
button is drawn to fully show its label with some fixed space placed
between the buttons. If the dialog is expanded, it is seen that there
is a spring between the 3rd and 4th buttons, so that the first 3 are
aligned with the left side of the window and the last two the right
side.

\subsubsection{Grid layout}
\label{sec:GUI:grid-layout}

The box layout algorithm only aligns its children along a single
dimension. The horizontal box, for example, vertically aligns its
children. Nevertheless, nesting permits the construction of complex
layouts using only simple boxes. However, it is sometimes desirable to
align widgets in both dimensions, i.e., to lay them out on a grid. The
most flexible grid layout algorithms allow non-regular sizing of rows
and columns, as well as the ability for a widget to span multiple
cells. Usually, a widget fills the cells allocated to it, but if this
is not possible, it may be anchored at a specific point within their
cell. 

The widgets in the ``Printer'' frame of
Figure~\ref{fig:GUI:box-possibilities} are subject to a grid layout
with five columns and six rows. The first row begins with the
``Name:'' label, and each widget in that row occupy a separate
column. This exposes the size of each column. The first column has
only labels, with text justified to the left.  The labels are aligned
horizontally to each other and vertically with adjacent field.


% \section{End of chapter notes}
% \label{sec:GUI:end-of-chapter}
% \XXX{ fill this in }

% More documentation on GUIs is available in book format or online. 

% For \GTK\/ there is the gtk tutorial (pygtk); GTK API; DTL's notes; example
% code in the \pkg{RGtk2} package; php-gtk cookbook

% For \wxWidgets the book; DTL omegahat pages; wxWidgets API;

% For \tcltk\/ ActiveStates API; wettenhall examples (sciviews);
% Dalgaard's papers; R mailing list; book

% For \Java\/ Sun's website tutorials; API; rJava package page; 

% Event loops




%%%%%%%%%%%%%%%%%%%%%%%%%%%%%%%%%%%%%%%%%%%%%%%%%%





%% R functions, practices for GUI programmng
\chapter{R Programming Practices for GUIs}
\label{chap:programming-practices}
%%%
\section{Why program GUIs in \R?}
\label{sec:GUIsInR}
%% place to discuss why program GUIs in R


\XXX{Cover these}
\begin{itemize}
\item Benefit to user of code -- complexity
\item Convenience
\item Reasonable speed, power
\item many options: RGtk2, tcltk, RPad, rJava, RwxWidgets, DTL stuff
\item Cross platform
\end{itemize}


%%
\section{Object Oriented Programming in \R}
\label{sec:OOP}
%% Object oriented programming in R

\subsection{Principles of object oriented programming}
\label{sec:princ-object-orient}

\subsubsection{Generic functions and methods in R}
\label{sec:gener-funct-meth}



\subsection{S3 Style}
\label{sec:PROG:S3}


 
% S3
% class(), dispatch, ``inheritance''
% use current generics
% define new generics
% not class based -- object based (MMaechler, but what does it mean?)




% -------------
% proto
% environments: \$, [, like a list; but passed through to functions,
%  unlike lists 
% <<>>=
% library(proto)
% f = function(obj) obj[['new']]<-"new"
% p = proto(); p$new = "old"; q = p$proto()
% e = environment(); e$new="old"
% l = list(); l$new = "old"
% f(q); f(e); f(l); print(c(q$new, p$new, e$new, l$new))
% @ 
% parents share properties with children -- not the other way. Why?
% child\$new looks in the enviroment of child for "new", if it is not
% found then it looks in enviroment of parent where it is found. This is
% broken when the child defines a property or method of that name.

% override dollar sign
% get/assign
% ex: ex-UndoRedo support: 

% ex: issue with overriding \$, (nextMethod) -- do that

% %%


\subsection{S4 Style}
\label{sec:PROG:S4}

% ----------------
% S4
% dispath -- multiple arguments
% defining classes, get classes, inheritance
% method signature
% use current methods
% add new methods
% promote  S3 object to S4 object


\section{Programming Issues when programming GUIs}
\label{sec:ProgrammingIssues}

\subsection{Pass by copy}
\label{sec:pass-by-copy}

\XXX{issues with modifying values in callback, double arrow
  assignement, assign, etc.}

\subsubsection{Variable lookup: Scoping}
\label{sec:vari-look-scop}

\subsubsection{Environments in \R}
\label{sec:environements-r}



\subsection{Converting text into \R\/ commands and \R\/ commands to text.}
\label{sec:text-to-R}
\XXX{eval(parse), do.call, ...}

\XXX{deparse, dput, sys.call, serialization}


\subsection{Efficient programming in \R}
\label{sec:efficient-programming}
\XXX{do.call, apply functions,...}
  
  
\subsection{Localization}
\label{sec:localization}

\XXX{gettext, etc.}



* nargs ..1 working with function args

* stringToObject: get, eval(parse(text=...)), do.call, match.fun,
serialize methods? assign, expressions?


\XXX{hash package  for global environment}




%% gWidgets
%%\part{The \pkg{gWidgets} package}
\label{chap:gWidgets-intro}
%% Put these somewhwere:

\XXX{use svalue(g), spacing to give some breathing room in a ggroup}

\XXX{something like ggroup(cont=w,spacing=10); svalue(g)=10}

\XXX{include listing data frame, listing variables by type}

\XXX{relationship with the toolkits

\XXX{getToolkitWidget, add method, ... only with RGtk2 (rJava?), only some widgets}}



%% gWidgets introduction
 
\newcommand{\ONLYIN}[1]{[only in #1]}

\chapter{\pkg{gWidgets}: Overview}
\label{sec:overview}

%% Overview of gWidgets

% ML: do we really want to discuss the unsupported rJava backend?
% I am also confused about gWidgetsWWW - is it an implementation or
% something that just resembles the gWidgets API?

% JV, I will drop rJava, put in gWidgetsQt, and make brief mention of gWidgetsWWW

The \pkg{gWidgets} package provides a toolkit-independent interface
for the \R\/ user to program graphical user interfaces from within
R. Although the package provides much less functionality than using a
native toolkit interface, \pkg{gWidgets} can be used to create
moderately complex GUIs quickly and easily using a programming
interface that is simpler and more familiar to the \R\/ user.

Figure~\ref{fig:gWidgets-three-oses} demonstrates the portability of
\pkg{gWidgets} commands, as it shows realizations on different
operating systems and with different graphical toolkits.

\begin{figure}
  \centering
  \begin{tabular}{ll}
    \includegraphics[width=0.45\textwidth]{ex-33-macosx-rgtk2} &
    \includegraphics[width=0.45\textwidth]{fig-gWidgets-ex-33-tlctk}\\
    \includegraphics[width=0.45\textwidth]{ex-33-linux-qt} &
    \includegraphics[width=0.45\textwidth]{ex-33-gWidgetsWWW}
  \end{tabular}
  \caption{The \pkg{gWidgets} package works with different operating
    systems and different GUI toolkits. This shows, the same code using the
    \pkg{RGtk2}, \pkg{tcltk}, \pkg{qtbase} packages for a toolkit. Additionally,
    the \pkg{gWidgetsWWW} package is used in the lower right figure.}
  \label{fig:gWidgets-three-oses}
\end{figure}

% %% Make figure -- work on layout here
% \XXX{Do mac, windows}
% \begin{figure}
%   \centering
%   \begin{tabular}{lccc}
%     & \pkg{RGtk2} & \pkg{tcltk} & \pkg{rJava} 
%     \\
%     L &
%     \includegraphics[width=0.3\textwidth]{ex-33-linux-rgtk2.png} &
%     \includegraphics[width=0.3\textwidth]{ex-33-linux-tcltk} &
%     \includegraphics[width=0.3\textwidth]{ex-33-linux-rJava} 
%     \\
%     W &
%     \includegraphics[width=0.3\textwidth]{ex-33-linux-rgtk2} &
%     \includegraphics[width=0.3\textwidth]{ex-33-linux-tcltk} &
%     \includegraphics[width=0.3\textwidth]{ex-33-linux-rJava} 
%     \\
%     Mac &
%     \includegraphics[width=0.3\textwidth]{ex-33-linux-rgtk2} &
%     \includegraphics[width=0.3\textwidth]{ex-33-linux-tcltk} &
%     \includegraphics[width=0.3\textwidth]{ex-33-linux-rJava}
%   \end{tabular}
%   \caption{The \pkg{gWidgets} package works with different operating systems and different GUI toolkits. This shows the combination of \code{linux}, \code{Mac OS X (10.5)} and \code{Windows XP} and the packages \pkg{RGtk2}, \pkg{tcltk}, and \pkg{rJava}}
%   \label{fig:three-oses-three-toolkits}
% \end{figure}


% \begin{figure}
%   \centering
%   \includegraphics[width=.45\textwidth]{ex-33-gWidgetsWWW}
%   \caption{A GUI shown using \pkg{gWidgetsWWW}.}
%   \label{fig:gWidgetsWWW-same-gui}
% \end{figure}





\section{Constructors}
\label{sec:constructors}

We begin with some sample \pkg{gWidgets} commands that set up a basic
interface allowing a user to input some test. The first line loads the
package, the others will be described later.

\begin{Schunk}
\begin{Sinput}
 require(gWidgets)
 options(guiToolkit="RGtk2")
 w <- gwindow("Text input example", visible=FALSE)


%% RGtk2
%%\part{The \pkg{RGtk2} package}
\label{chap:RGtk2}

% XXX- discuss GtkBin; which widgets have a window, which do not --image and label do not, and we mightwant them too,
%   modify_bg() only affects widgets that have an associated gtk.gdk.Window.
% > Widgets that do not have an associated window include .... gtk.Arrow,
% > gtk.Bin, gtk.Box, gtk.Button, gtk.CheckButton, gtk.Fixed , gtk.Image ,
% > gtk.Label , gtk.MenuItem , gtk.Notebook , gtk.Paned , gtk.RadioButton ,
% > gtk.Range , gtk.ScrolledWindow , gtk.Separator , gtk.Table , gtk.Toolbar ,
% > gtk.AspectFrame , gtk.Frame , gtk.VBox , gtk.HBox , gtk.VSeparator ,
% > gtk.HSeparator . These widgets can be added to a gtk.EventBox to overcome
% > this limitation.




%% RGtk2 chapter
\chapter{RGtk2: Overview}
%%\section{Introduction}
\label{sec:RGtk2-Introduction}
%% intorduction

%% Technical, but short beginning

As the name implies, the \pkg{RGtk2} package provides a connection, or
bindings, between \GTK\/ and \R\/ allowing nearly the full power of
\GTK\/ to be available to the \R\/ programmer. In addition,
\pkg{RGtk2} provides bindings to other libraries accompanying \GTK:
The Pango libraries for font rendering; the Cairo
libraries for vector graphics; the GdkPixbuf libraries for image
manipulation; libglade for designing GUI layouts from an XML
description; ATK for the accessiblity toolkit;  and GDK, which
provides an abstract layer between the windowing system, such as X11,
and \GTK. These libraries are multi-platform and extensive and have been
used for many major projects, such as the linux versions of the
firefox browser and open office.

% Actually, the bindings to GTK are only part of the story. RGtk2 also
% offers complete bindings to Pango (font rendering), GDK (basic
% drawing, low-level device access), Cairo (vector graphics), GdkPixbuf
% (image manipulation), libglade (GUI's from XML descriptions),
% GtkMozEmbed (embeddable mozilla browser on linux), and ATK
% (accessibility devices). [Michael Lawrence's announcement]


\pkg{RGtk2}, for the most part, automatically creates \R\/ functions
that call into the \GTK\/ library. For example, the \R\/ function
\function{gtkContainerAdd} eventually calls the C function
\code{gtk\_container\_add}. The naming convention is the C name has its underscores
removed and each following letter capitalized (camelback).


The full API for \GTK\/ is quite large, and clearly can not be
documented here. However, the \GTK\/ documentation is converted into
\R\/ format in the building of \pkg{RGtk2}. This conveniently allows
the programmer to refer to the appropriate documentation within an
\R\/ session, without having to consult  a web page, such as
\url{http://library.gnome.org/devel/gtk/stable/}, which lists the API
of the stable versions \GTK.



%%% ------- OOP --------------

\section{How \GTK\/ is organized}
\label{sec:RGtk2:constructors}


\GTK\/ objects are created using constructors such as
\function{gtkWindowNew} and \function{gtkButtonNewWithLabel} (these
mapping to \code{gtk\_window\_new} and
\code{gtk\_button\_new\_with\_label} respectively). \pkg{RGtk2} also
provides constructors with names not ending in ``\code{New}'' that may,
depending on the arguments given, call different, but similar,
constructors. As such we prefer the shorter named constructors, such as
\function{gtkWindow} or \function{gtkButton}.

\subsection{Methods}


%% OO methods
The underlying \GTK\/ library is written in C, but still provides a a
singly inherited, object-oriented framework that leads naturally to
the use of S3 classes for the \R\/ package. In \GTK\/ the
\class{GtkWindow} class inherits methods, properties, and signals from
the \class{GtkBin}, \class{GtkContainer}, \class{GtkWidget},
\class{GtkObject}, \class{GInitiallyUnowned}, and \class{GObject}
classes. In \pkg{RGtk2}, we can see the class heiarchy by calling
\function{class} on a \command{gtkWindow} instance:~\footnote{We use
  the term ``instance'' of a constructor to refer to the object
  returned by the constructor, which is an instance of some class.}
\begin{Schunk}
\begin{Sinput}
 class(gtkWindow())
\end{Sinput}
\begin{Soutput}
[1] "GtkWindow"         "GtkBin"            "GtkContainer"     
[4] "GtkWidget"         "GtkObject"         "GInitiallyUnowned"
[7] "GObject"           "RGtkObject"       
\end{Soutput}
\end{Schunk}

The classes are identical except for the addition of the base \class{RGtkObject}
class. When a widget is destroyed, the \R\/ object is assigned
\class{<invalid>} class.

Methods of \pkg{RGtk2} do not use S3 dispatch, but rather an internal
one. The call \code{obj\$method(...)} resolves to a function call
\code{f(obj,...)}. The function is found by looking for any function
prefixed with with either an interface or a class from the object
followed by the method name. The interfaces are checked first.

For instance, if \code{win} is a \function{gtkWindow} instance, then to resolve the call
\code{win\$add(widget)} (or \code{win\$add(widget)}) \R\/ looks for methods with the name
\function{gtkBuildableAdd}, \function{atkImplementorIfaceAdd},
\function{gtkWindowAdd}, \function{gtkBinAdd} before finding
\function{gtkContainerAdd} and calling it as
\code{gtkContainerAdd(win,widget)}. The \method{\$}{GObject} method for \pkg{RGtk2} objects does the
work. Understanding this mechanism allows us to add to the \pkg{RGtk2}
API, as convenient. For instance, we can add to the button API with

\begin{Schunk}
\begin{Sinput}
 gtkButtonPrintHello <- function(obj) print("hello")
 b <- gtkButton()
 b$printHello()
\end{Sinput}
\begin{Soutput}
[1] "hello"
\end{Soutput}
\end{Schunk}

%% common methods
Some common methods are inherited by most widgets, as they are defined
in the base \class{gtkWidget} class. These include the methods 
\method{Show}{gtkWidget} to specify that the widget should be drawn;
\method{Hide}{gtkWidget} to hide the widget until specified;
\method{Destroy}{gtkWidget} to destroy a widget and clear up any
references to it; \method{getParent}{gtkWidget} to find the parent
container of the widget; \method{ModifyBg}{gtkWidget} to modify the
background color of a widget; and \method{ModifyFg}{gtkWidget} to
modify the foreground color.


\subsection{Properties}


%% --------- Properties ------------
Also inherited are widget properties. A list of properties that a
widget has is returned by its \method{GetPropInfo}{gObject}
method. \pkg{RGtk2} provides the \R\/ generic \method{names}{Gobject}
as a familiar alternative for this method. For the button just
defined, we can see the first eight properties listed with:
\begin{Schunk}
\begin{Sinput}
 head(names(b), n=8)                     # or b$getPropInfo()
\end{Sinput}
\begin{Soutput}
[1] "user-data"      "name"           "parent"         "width-request" 
[5] "height-request" "visible"        "sensitive"      "app-paintable" 
\end{Soutput}
\end{Schunk}

Some common properties are \code{parent} to store the parent widget
(if any); \code{user-data} which allow one to store arbitrary data
with the widget; \code{sensitive}, to control whether a widget can
receive user events;


There are a few different ways to access these properties. Consider
the \code{label} property of a \code{gtkButton} instance.  \GTK\/
provides the functions \function{gObjectGet} and \function{gObjectSet}
to get and set properties of a widget.  The set funtion using the
arguments names for the property key.

\begin{Schunk}
\begin{Sinput}
 b <- gtkButton("A button")
 gObjectGet(b,"label")
\end{Sinput}
\begin{Soutput}
[1] "A button"
\end{Soutput}
\begin{Sinput}
 gObjectSet(b,label="a new label for our button")
\end{Sinput}
\end{Schunk}
Additionally, most user-accessible properties have specific \code{Get} and
\code{Set} methods defined for them. In our example,  the methods
\method{getLabel}{gtkButton} and \method{setLabel}{gtkButton} can be used.
\begin{Schunk}
\begin{Sinput}
 b$getLabel()
\end{Sinput}
\begin{Soutput}
[1] "a new label for our button"
\end{Soutput}
\begin{Sinput}
 b$setLabel("Again, a new label for our button")
\end{Sinput}
\end{Schunk}

\pkg{RGtk2} provides the convenient and familiar \code{[} and
\code{[$<$-} methods to get and access the properties:
\begin{Schunk}
\begin{Sinput}
 b['label']
\end{Sinput}
\begin{Soutput}
[1] "Again, a new label for our button"
\end{Soutput}
\end{Schunk}

For ease of referencing the appropriate help pages, we tend to use the
full method name in the examples, although at times the move \R-like
vector notation will be used for commonly accessed properties.

%%% ------ constants --------

\subsection{Enumerated types and flags}


The \GTK\/ libraries have a number of constants that identify
different states. These enumerated types are defined in the C
code. For instance, for a toolbar, there are four possible styles: with
icons, just text, both text and icon, and both text and icon drawn
horizontally. The flags indicating the style are stored in C in an
enumeration \code{GtkToolbarStyle} with constants
\code{GTK\_TOOLBAR\_ICONS}, \code{GTK\_TOOLBAR\_TEXT}, etc. In \pkg{RGtk2}
these values are conveniently stored in the vector
\code{GtkToolbarStyle} with named integer values
\begin{Schunk}
\begin{Sinput}
 GtkToolbarStyle
\end{Sinput}
\begin{Soutput}
     icons       text       both both-horiz 
         0          1          2          3 
attr(,"class")
[1] "enums"
\end{Soutput}
\end{Schunk}

A  list of enumerated types for \GTK\/ is listed in the man
page \code{?gtk-Standard-Enumerations} and for \code{Pango} in
\code{?pango-Layout-Objects}. The \code{Gdk}  variables are
prefixed with \code{Gdk} and so can be found using \function{apropos},
say, using \code{ignore.case=TRUE}.

To use these enumerated types, one can specify them by name as
\begin{Schunk}
\begin{Sinput}
 tb <- gtkToolbar()
 tb$setStyle(GtkToolbarStyle['icons'] )
\end{Sinput}
\end{Schunk}

But \pkg{RGtk2} provides the convenience of specifying the style name
only, as in
\begin{Schunk}
\begin{Sinput}
 tb$setStyle("icons")
\end{Sinput}
\end{Schunk}

When more than one value is desired, they can be combined using
\function{c}.

%%% ------ Signals ----------

\subsection{Events and signals}


In \pkg{RGtk2} user actions, such as mouse clicks, keyboard usage,
drag and drops, etc. trigger \pkg{RGtk2} widgets to signal the action.
A GUI can be made interactive, by adding callbacks to respond when
these signals are emitted. In addition to signals, there are a number
of window manager events, such as a \code{button-press-event}. These
events have callbacks attached in a similar manner.

The signals and events that an object adds are returned by the method
\method{GetSignals}{gObject}. For example
\begin{Schunk}
\begin{Sinput}
 names(b$getSignals())
\end{Sinput}
\begin{Soutput}
[1] "pressed"  "released" "clicked"  "enter"    "leave"    "activate"
\end{Soutput}
\end{Schunk}
shows the ``clicked'' signal in addition to others.

To list all the inherited signals can be achieved using
\function{gtkTypeGetSignals}. For instance, the following code will print
out all the inherited signals and events.
\begin{Schunk}
\begin{Sinput}
 types <- class(b)
 lst <- sapply(head(types,n=-1), gtkTypeGetSignals)
 for(i in names(lst)) { cat(i,"\n"); print(lst[[i]])}
\end{Sinput}
\end{Schunk}



%% the gSignalConnect function
\paragraph{Binding a callback}
The \function{gSignalConnect} (or \function{gSignalConnect}) function is used
to add a callback to a widget's signal. Its signature is
\begin{Schunk}
\begin{Sinput}
 args(gSignalConnect)
\end{Sinput}
\begin{Soutput}
function (obj, signal, f, data = NULL, after = FALSE, user.data.first = FALSE)  
\end{Soutput}
\end{Schunk}

The \argument{obj}{gSignalConnect} is the widget the callback is attached to and
\argument{signal}{gSignalConnect} the signal name, for instance \code{"drag-drop"}. 
This may also be an event name.

The \argument{f}{gSignalConnect} argument is for the callback.
Although, it can be specified as an expression or a call, our examples
always use a function to handle the callback. More detail follows. The
\argument{after}{gSignalConnect} argument is a logical indicating if
the callback should be called after the default handlers (see
\command{?gSignalConnect}).

The \argument{data}{gSignalConnect} argument allows arbitrary data to be
passed to the callback.  The \argument{user.data.first}{gSignalConnect} argument
specifies if this \argument{data}{gSignalConnect} argument should be the first
argument to the callback or (the default) the last. As the signature of
the callback has varying length, setting this to \code{TRUE} can prove
useful.

%% the callback
The signature for the callback varies for each signal and 
window manager event. Unless the default for \code{user.data.first} is overridden, the
argument is the widget. For signals, other arguments are
possible depending on the type. For window events, the second argument is a
\class{GdkEvent} type, which can carry with it extra information about
the event that occurred. The \GTK\/ API lists each argument. 

As the callback is an \R\/ function, it is passed copies of the
object. Since \pkg{RGtk2} objects are pointers, there is no practical
difference. So changes within the body of a callback to \pkg{RGtk2}
objects are reflected outside the scope of the callback, unlike
changes to most other \R\/ objects.


\begin{Schunk}
\begin{Sinput}
 w <- gtkWindow(); w['title'] <- "test signals"
 x <- 1; 
 b <- gtkButton("click me"); w$add(b)
 ID <- gSignalConnect(b,signal="clicked",f = function(widget,...) {
   widget$setData("x",2)
   x <- 2
   return(TRUE)
 })
\end{Sinput}
\end{Schunk}
Then after clicking, we would have

\begin{Schunk}
\begin{Sinput}
 cat(x, b$getData("x"),"\n") # 1 and 2
\end{Sinput}
\begin{Soutput}
1 2 
\end{Soutput}
\end{Schunk}

Callbacks for signals emitted by window manager events are expected to
return a logical value. Failure to do so can cause errors to be
raised. For other callbacks the return value is ignored, so it is safe to
always return a logical value. When it is not ignored, a return value
of \code{TRUE} indicates that no further callbacks should be called,
whereas \code{FALSE} indicates that the next callback should be
called. So in the following example, only the first two callbacks are
executed when the user presses on the button.

\begin{Schunk}
\begin{Sinput}
 b <- gtkButton("click")
 w <- gtkWindow()
 w$add(b)
 id1 <- gSignalConnect(b, "button-press-event", 
                       function(b, event, data) {
                         print("hi"); return(FALSE)
                       })
 id2 <- gSignalConnect(b, "button-press-event", 
                       function(b, event, data) {
                         print("and"); return(TRUE)
                       })
 id3 <- gSignalConnect(b, "button-press-event", 
                       function(b, event, data) {
                         print("bye"); return(TRUE)
                       })
\end{Sinput}
\end{Schunk}

%% multiple callbacks; remove; block
Multiple callbacks can be assigned to each signal. They will be
processed in the order they were bound to the signal.  The
\function{gSignalConnect} function returns an ID that can be used to
disconnect a callback if desired using
\function{gSignalHandlerDisconnect} or temporarily blocked using
\function{gSignalHandlerBlock} and
\function{gSignalHandlerUnblock}. The man page for
\function{gSignalConnect} gives the details on this, and much more.


%% --------- Event Loop

\subsection{The eventloop}


The \pkg{RGtk2} eventloop integrates with the \R\/ event loop. In practice, such
integration is tricky. In a C program, \GTK\/ programs call the
function \code{gtk\_main} which puts control of the GUI into the main
event loop of \GTK. This sits idle until some event occurs. According
to the \pkg{RGtk2} website, ``The nature of the R event loop prevents
the continuous execution of the GTK main loop, thus preventing things
like timers and idle tasks from executing reliably. This manifests
itself when using functionality such as GtkExpander and
GtkEntryCompletion.''

During a long calculation, the GUI can seem unresponsive. To avoid
this the following construct can be used during the long calculation
to process pending events.

\begin{Schunk}
\begin{Sinput}
 while(gtkEventsPending()) 
   gtkMainIteration()
\end{Sinput}
\end{Schunk}



\section{RGtk2 and gWidgetsRGtk2}
\label{sec:RGtk2:gWidgetsRGtk2}


The widgets described above, are also available through
\pkg{gWidgetsRGtk2}. The two packages can be used together, for the
most part. The \code{add} method of \pkg{gWidgetsRGtk2} can be used to
add an \pkg{RGtk2} widget to a \code{gWidgetsRGtk2}
container. Whereas, the \code{getToolkitWidget} method will (usually)
return the \pkg{RGtk2} component to use within \pkg{RGtk2}.

%% Views example in next chapter?

\chapter{RGtk2: Basic Components}
\label{sec:top-level-windows}

\XXX{ --- GDkWindow role (events, coloring, event box): explain on need
to know basis}

\XXX{SetDecorated for gtkWindow}


This section covers some of the basic widgets and containers of
\GTK. We begin with a discussion of top level containers and box
containers. Then we continue with describing many of the simpler
controls -- essentially those without an underlying model, and then
finish by describing a few more containers. 

\section{Top-level windows}
\label{sec:RGtk2:gtkWindow}

%% constructor Show/Hide
Top-level windows are constructed by the \constructor{gtkWindow}
constructor. This function has arguments \code{type} to specify the
type of window to create. The default is a top-level window, which we
will always use, as the alternative is for ``popups'' which are used,
for example, with menus. The second argument is \code{show}, which by
default is \code{TRUE} indicating that the window should be shown. If
set to \code{FALSE}, the window, like other widgets, can later be
shown by calling its \method{Show}{gtkWidget} method. The \method{ShowAll}{gtkWidget} method
will also show any child components. These can be reversed with
\method{Hide}{gtkWidget} and \method{HideAll}{gtkWidget}.

%% title
As with all objects, windows have several properties. The window title
is stored in the \code{title} property. As usual, this property can be
accessed via the ``get'' and ``set'' methods
\method{GetTitle}{gtkWindow} and \method{SetTitle}{gtkWindow}, or
using the vector notion. To illustrate, the following sets up a new
window with some title.
\begin{Schunk}
\begin{Sinput}
 w <- gtkWindow(show=FALSE)              # use default type
 w$setTitle("Window title")              # set window title
 w['title']                              # or w$getTitle()
\end{Sinput}
\begin{Soutput}
[1] "Window title"
\end{Soutput}
\begin{Sinput}
 w$setDefaultSize(250,300)               # 250 wide, 300 high
 w$show()                                # show window
\end{Sinput}
\end{Schunk}

\paragraph{Window size}
The initial size of the window can be set with the
\method{setDefaultSize}{gtkWindow} method, as shown, which takes a
\argument{width}{gtkWindow} and \argument{height}{gtkWindow} argument
specified in pixels. This specification allows the window to be
resized, but must be made before the window is drawn. The
\method{SetSizeRequest}{gtkWidget} method will also set the size, but
does not allow for resizing smaller than the requested size. To really
fix the size of a window, the \code{resizable} property may be set to
\code{FALSE}.

\paragraph{Transient windows}
New windows may be standalone top-level windows, or may be associated
to some other window, such as a how a dialog is associated with some
parent window. In this case, the \method{SetTransientFor}{gtkWindow}
method can be used to specify which window. This allows the window
manager to keep the transient window on top. The position on top, can
be specified with \code{SetPostion} which takes a constant given by
\code{GtkWindowPosition}. Finally it can be specified that the dialog
be destroyed with its parent. For example
\begin{Schunk}
\begin{Sinput}
 ## create a window and a dialog window
 w <- gtkWindow(show=FALSE); w$setTitle("Top level window")
 d <- gtkWindow(show=FALSE); d$setTitle("dialog window")
 d$setTransientFor(w)
 d$setPosition(GtkWindowPosition["center-on-parent"])
 d$setDestroyWithParent(TRUE)
 w$show()
 d$Show()
\end{Sinput}
\end{Schunk}

The above code produces a non-modal dialog window. Due to its
transient nature, it can hide parts of the top-level window, but it
does not prevent that window from receiving events like a modal dialog
window. \GTK\/ provides a number of modal dialogs discussed later.


%% delete-event; destroy
\paragraph{Destroying windows}
The window can be closed through the window manager, by clicking on
its close icon, or programatically by calling its \method{Destroy}{gtkWidget}
method. When the window manager is clicked, the \code{delete-event}
event signal is raised, and can have a callback listen for it. If that callback returns \code{FALSE}, then the
window's \signal{destroy} signal is emitted. It is this signal that is
propogated to the windows child components. However, if a callback for
the \code{delete-event} signal returns \code{TRUE}, then the
\signal{destroy} signal will not be emitted. This can be useful if a
confirmation is desired before closing the window.


%% A container
\paragraph{Adding a child component to a window}
A window is a container. However, \command{gtkWindow} objects, inherit
from the \code{GtkBin} class which allows only one child
container. This child is added through the windows \method{Add}{gtkContainer}
method. This child is often another container that allows for more
than one component to be added.

We illustrate by adding a simple label to a window.
\begin{Schunk}
\begin{Sinput}
 w <- gtkWindow(); w$setTitle("Hello world")
 l <- gtkLabel("Hello world")
 w$add(l)
\end{Sinput}
\end{Schunk}
The method \method{GetChildren}{GtkContainer} will return the children of a
container as a list. In this case, as the \code{GtkWindow} window
class is a subclass of \code{GtkBin}, which holds only 1 child
component, the \method{GetChild}{GtkWidget} method may be used to access the label
directly. For instance, to retrieve the label's text one can do.
\begin{Schunk}
\begin{Sinput}
 w$getChild()['label']                   # return label property of child
\end{Sinput}
\begin{Soutput}
[1] "Hello world"
\end{Soutput}
\end{Schunk}


%% [[ for container
The \method{[[}{GObject} method 
%%]]
can be used to access
the child containers by number, as an shorcut for list extraction
from the return value of the \method{GetChildren} method.

From the label's perspective, the window is its immediate parent. The
\code{GetParent} method for \GTK\/ widgets, will return a widget's
parent container.

\section{Box containers}
\label{sec:RGtk2:BoxContainers}

Flexible containers for holding more than one child are the box
containers constructed by \function{gtkHBox} or \function{gtkVBox}.
These produce horizontal, or vertical ``boxes'' which allow packing of
child components in a manner analagous to packing a box. These
components can be subsequent box containers, allowing for very
flexible layouts.

Each child component is allocated a cell in the box.  The
\argument{homogeneous}{gtkHBox} argument can be set to \code{TRUE} to
ensure all the cells have the same size allocated to them. The
default, is so have non-homogeneous size allocations. 

\paragraph{Packing child components}
Adding a child component to the box is done with the methods
\method{PackStart}{gtkBox} or \method{PackEnd}{gtkBox}. The
\method{PackStart}{gtkBox} method adds children from left to right
when the box is horizontal, or top to bottom when vertical. the
\method{PackEnd}{gtkBox} method is opposite. These methods have initial argument
\argument{child}{gtkBoxPackStart} to specify the child component and \argument{padding}{gtkBoxPackStart} to specify a padding in pixels between child components. 


Once packed in the child components can be referenced through the \method{GetChildren}{GtkBox} method which is conveniently called through list extraction.


\paragraph{Removing and reordering children}
Once children are packed into a box container, they can be manipulated
in various ways.

The \method{Hide}{gtkWidget} method of a child component will cause it
not to be drawn. This can be reversed with the
\method{Show}{gtkWidget} method.

The \method{Remove}{gtkWidget} method for containers can cause a child
component to be removed. It can later be re-added using
\method{PackStart}{gtkBox}. For instance
\begin{Schunk}
\begin{Sinput}
 b <- g[[3]]
 g$remove(b)                             # removed
 g$packStart(b, expand=TRUE, fill=TRUE)
\end{Sinput}
\end{Schunk}

The \method{Reparent}{gtkWidget} method of widgets, will allow a
widget to specify a new parent container.

The \method{ReorderChild}{gtkBox} method can be used
to reorder the child components. The new position of the child is
specified using 0-based indexing. This code will move the last child
to the second position.
\begin{Schunk}
\begin{Sinput}
 b3 <- g[[3]]
 g$reorderChild(b3, 2 - 1)               # second is 2 - 1
\end{Sinput}
\end{Schunk}

\paragraph{Spacing}
There are several adjustments possible to add space around components
in a box container.  The \argument{spacing}{gtkHBox} argument for the
constructors specifies the amount of space, in pixels,  between the cells with a
default of 0.  The \code{Pack} methods also have a
\argument{padding}{gtkHBox} argument to specify the padding between
subsequent children, again with default 0. For horizontal packing, this space goes on both
the left and right of the child component, whereas the \code{spacing}
value is just between children. (The spacing between components is the sum of the \code{spacing} value and the two \code{padding} values when the children are added.) Child widgets also have
properties \code{xpad} and \code{ypad} for setting the padding around
themselves.
Example~\ref{eg:RGtk2:mac-buttons} provides an example and
Figure~\ref{fig:RGtk2-pack-start} an illustration.


\begin{figure}
  \centering
  \includegraphics[width=.85\textwidth]{ex-RGtk2-pack-start}
  \caption{Examples of packing widgets into a box container. The top
    row shows no padding, whereas the 2nd and 3rd illustrate the
    difference between \code{padding} (an amount around each child)
    and \code{spacing} (an amount between each child). The last two
    rows show the effect of \code{fill} when \code{expand=TRUE}. Illustration
    follows one in orignial \GTK\/ tutorial.}
  \label{fig:RGtk2-pack-start}
\end{figure}




%% Resizing
\paragraph{Component size}
Each component has properties \code{width} and \code{height} to determine the size of the component when mapped. When these are both $-1$, the natural size of the widget will be used. To set the requested size of a component, the method \method{SetSizeRequest}{gtkWidget} is used to specify minimum values for the\code{width} and \code{height} of the widget. The methods help page warns that it is impossible to adequately hardcode a size that will always be correct.

When a parent container is resized, it queries its children for their
preferred size (\method{GetSizeRequest}{gtkWidget}), If these children
have children, they then are asked, etc. This size
information is then passed back to the top-level component. It resizes
itself, then passes on the available space to its children to resize
themselves, etc. After resizing the \method{GetAllocation}{gtkWidget}
method returns the new width and height, as components in a list.
The space allocated to a cell, may be more than the space requested by
the widget. In this case, the \argument{expand}{gtkBoxPackStart} and
\argument{fill}{gtkBoxPackStart} arguments for the \code{Pack} methods
are important. If \code{expand=TRUE} is given, then the cell will
expand to fill the extra space. Furthermore, if also \code{fill=TRUE} then
the widget will expand to fill the space allocated to the cell.
Figure~\ref{fig:RGtk2:GtkBox-pack-start} illustrates.


%% alignment of widgets within cells
Widgets inherit the properties \code{xalign} and \code{yalign} from
the \class{GtkMisc} class. These properties are used to specify how
the widget is aligned within the cells when the widget size request is
less than that allocated to the cell. These properties take values between $0$ and
$1$, with $0$ being left and top. Their defaults are $0.5$, for centered alignment.


% \begin{figure}
%   \centering
%   \includegraphics[width=.6\textwidth]{RGtk2-GtkBox-expand-fill}
%   \caption{Illustration of \code{expand} and \code{fill} arguments for
%     a \code{GtkBox} container. The \code{expand} argument causes the
%     cell to expand to fill all possible space. The \code{fill}
%     argument then indicates if the widget should then stretch to fill
%     the space allocated to its cell.}
%   \label{fig:RGtk2:GtkBox-expand-fill}
% \end{figure}






\section{Buttons}
\label{sec:RGtk2:gtkButton}

A basic button is constructed using \constructor{gtkButton}. This is a
convenience wrapper for several constructors. With no argument, it
returns a simple button. When the first argument, \argument{label}{gtkButton}, is
used it returns a button with a label, calling
\function{gtkButtonNewWithLabel}. The \argument{stock.id}{gtkButton} argument
calls \function{gtkButtonNewFromStock}.  Buttons in \GTK\/ are
actually containers (of class \class{GtkBin}), By default, they have a
\code{label} and \code{image} property. The image is specified using a
stock id. The available stock icons are listed by
\function{gtkStockListIds}. Finally, if a mnemonic is desired, for the
button, the constructor \constructor{gtkButtonNewWithMnemonic} can be
used. Mnemonics are specified by prefixing the character with an
underscore, as illustrated in this example.

\begin{example}{Button constructors}{eg:RGtk2:button-constructors}
\begin{Schunk}
\begin{Sinput}
 w <- gtkWindow(show=FALSE)
 w$setTitle("Various buttons")
 w$setDefaultSize(400, 25)
 g <- gtkHBox(homogeneous=FALSE, spacing=5)
 w$add(g)
 b <- gtkButtonNew(); 
 b$setLabel("long way")
 g$packStart(b)
 g$packStart(gtkButton(label="label only") )
 g$packStart(gtkButton(stock.id="gtk-ok") )
 g$packStart(gtkButtonNewWithMnemonic("_Mnemonic") ) # Alt-m to "click"
 w$show()
\end{Sinput}
\end{Schunk}
\end{example}

\begin{figure}
  \centering
  \includegraphics[width=.8\textwidth]{RGtk2-various-button}
  \caption{Various buttons}
  \label{fig:RGtk2:various-buttons}
\end{figure}

Buttons are essentially containers with a decoration to give them a
button like appearance. The relief style of the button can be changed
so that the button is drawn like a label. The method
\method{SetRelief}{gtkButton} is used, with the available styles
found in the \code{GtkReliefStyle} enumeration.

A button, can be drawn with extra space all around it. The
\code{border-width} property, with default of 0, specifies this space.
One can use the method \method{SetBorderWidth}{gtkContainer} to make
a change.


%% signals
\paragraph{signals}
The \signal{clicked} signal is emitted when the button is
clicked on with the mouse or when the button has focus and the
\kbd{enter} key is pressed. A callback can listen for this event, to
initiate an action.  If one wishes to filter out the mouse
button that was pressed on the button, the \signal{button-press-event} signal
is also emitted. Since this is a window manager event, the second
argument to the callback is an event which contains the button
information. This can be retrieved using the event's
\method{getButton}{gtkEventButton} method. However, the
\signal{button-press-event} signal is not emitted when the keyboard initiates
the action.

%% Buttons initiate actions
If the action a button is to initiate is the default action for the
window it can be set so that it is activated
when the user presses  \kbd{enter} while the parent window has the
focus. To implement this, the property \code{can-default} must be
\code{TRUE} and the widget method \method{grabDefault}{gtkWidget} must
be called. (This is not specific to buttons, but any widget that can
be activatable.)


As buttons are intended to call an action immediately after being
clicked, it is customary to make them not sensitive to user input when the action is not possible. The \method{SetSensitive}{gtkWidget} method can adjust this for the button, as with other widgets.


If the action that a button initiates is to be represented elsewhere
in the GUI, say a menu bar, then a \code{GtkAction} object may be
appropriate. Action objects are covered in Section~\ref{sec:RGtk2:UIManager}.


\begin{example}{Callback example for \code{gtkButton}}{eg:RGtk2:gtkButton-callback}
\begin{Schunk}
\begin{Sinput}
 w <- gtkWindow(); b <- gtkButton("click me");
 w$add(b)
 ## only when clicked, but can retrieve button as shown
 ID <- gSignalConnect(b,"button-press-event",   # just mouse click
                      f = function(w,e,data) {
                        print(e$getButton())    # which button
                        return(FALSE)           # propogate
                      })
 ID <- gSignalConnect(b,"clicked",              # click or keyboard
                      f = function(w,...) {
                        print("clicked")
                      })
\end{Sinput}
\end{Schunk}
\end{example}


\begin{example}{Spacing between buttons}{eg:RGtk2:mac-buttons}
This example shows how to pack buttons into a box so that the spacing
between the similar buttons is 12 pixels, but between potentially
dangerous buttons is 24 pixels, as per the Mac human interface
guidelines.
\GTK\/ provides the constructor \constructor{gtkHButtonBox} for
holding buttons, which provides a means to apply consistent styles,
but the default styles do not allow such spacing as desired. (Had all
we wanted was to right align the buttons, then that style is certainly
supported.) As such, we will illustrate how this can be done through a
combination of \code{spacing} arguments.
We assume that our parent container, \code{g}, is a
horizontal box container.


\begin{figure}
  \centering
  \includegraphics[width=.7\textwidth]{ex-RGtk2-mac-buttons}
  \caption{Example using stock buttons with extra spacing added between the \code{delete} and \code{cancel} buttons.}
  \label{fig:ex-RGtk2-mac-buttons}
\end{figure}

We include standard buttons, so use the stock names and icons.
\begin{Schunk}
\begin{Sinput}
 cancel <- gtkButton(stock.id="gtk-cancel")
 ok <- gtkButton(stock.id="gtk-ok")
 delete <- gtkButton(stock.id="gtk-delete")
\end{Sinput}
\end{Schunk}

We will right align our buttons, so use the parent container's
\code{PackEnd} method. The \code{ok} button has no padding, the
12-pixel gap between it and the \code{cancel} button is ensured by  the
\code{padding} argument when the \code{cancel} button is
added. Treating the \code{delete} button as potentially irreversible,
we aim to have 24 pixels of seperation between it and the
\code{cancel} button. This is given by adding 12 pixels of padding
when this button is packed in, giving 24 in total. The blank label is
there to fill out space if the parent container expands.
\begin{Schunk}
\begin{Sinput}
 g$packEnd(ok, padding=0)
 g$packEnd(cancel, padding=12)
 g$packEnd(delete, padding=12)
 g$packEnd(gtkLabel(""), expand=TRUE, fill=TRUE)
\end{Sinput}
\end{Schunk}
We make \code{ok} the default button, so have it grab the focus and
add a simple callback when the button is either clicked or the
\kbd{enter} key is pressed when the button has the focus.
\begin{Schunk}
\begin{Sinput}
 ok$grabFocus()
 QT <- gSignalConnect(ok, "clicked", function(...) print("ok"))
\end{Sinput}
\end{Schunk}






\end{example}

\section{Labels}
\label{sec:RGtk2:gtkLabel}

Labels are created by the \constructor{gtkLabel} constructor. Its main
argument is \argument{str}{gtkLabel} to specify the button text,
through its \code{label} property. This text can be set with either
\method{SetLabel}{gtkLabel} or \method{SetText}{gtkLabel} and
retrieved with either \method{GetLabel}{gtkLabel} or \method{GetText}{gtkLabel}.
The difference being the former can respect formatting marks. 

The text can include line breaks, specified with ``\code{\backslashn}.''
Further formatting is available. Wrapping of long labels can be
specified using a logical value with the method
\method{SetLineWrap}{gtkLabel}. The line width can be specified in
terms of the number of characters thorugh
\method{SetWidthChars}{gtkLabel} or by setting the size request for
the label. This is not determined by the size of the parent
window. Long labels can also have ellipsis inserted into them to
shorten when there is not enough space. By default this is turned
off. The variable \code{PangoEllipsizeMode} contains the constants,
and the method \method{SetEllipsize}{gktkLabel} is used to set this.
The property \code{justify}, with values taken from
\code{GtkJustification}, controls the justification.


\GTK\/ allows markup of text elements using the Pango text attribute
markup language. The method \method{SetMarkup}{gtkLabel} is used to
specify the text in the format, which is similar to a basic subset of
HTML. Text is marked using tags to indicate the style. Some convenient
tags are \code{<b>} for bold, \code{<i>} for italics, \code{<ul>} for
underline, and \code{<tt>} for monospace text. More complicated markup
involves the \code{<span>} tag markup, such as \code{<span color='red'>some text</span>}. The text can may need to be escaped first, so that designated entities replace reserved characters.



By default, text in a label can not be copied and pasted into another
widget or application. To allow this, the \code{selectable} property
can be set to \code{TRUE} with \method{SetSelectable}{gtkLabel}.

Labels can hold mnemonics for other widgets. The constructor is \code{gtkLabelNewWithMnemonic}. The label needs to idenfy the widget it is holding a mnemonic for, this is done with the \method{SetMnemonicWidget}{gtkLabel} method.

\begin{example}{Label formatting}{eg:RGtk2:label-formatting}
\begin{Schunk}
\begin{Sinput}
 w <- gtkWindow(); w$setTitle("Label formatting")
 w$setSizeRequest(250,100)               # narrow
 g <- gtkVBox(spacing=2); g$setBorderWidth(5); w$add(g)
 string = "the quick brown fox jumped over the lazy dog"
 ## wrap by setting number of characters
 basicLabel <- gtkLabel(string)
 basicLabel$setLineWrap(TRUE); 
 basicLabel$setWidthChars(35)            # specify number of characters
 ## Set ellipsis to shorten long text
 ellipsized <- gtkLabel(string)
 ellipsized$setEllipsize(PangoEllipsizeMode["middle"])
 ## Right justify text
 ## use xalign property for label in cell
 rightJustified <- gtkLabel("right justify"); 
 rightJustified$setJustify(GtkJustification["right"])
 rightJustified['xalign'] <- 1
 ## PANGO markup
 pangoLabel <- gtkLabel(); 
 pangoLabel$setMarkup(paste("<span foreground='blue' size='x-large'>",
                            string,"</span>"));
 QT <- sapply(list(basicLabel, ellipsized, rightJustified, pangoLabel),
              function(i) g$packStart(i, expand=TRUE, fill=TRUE ))
 w$showAll()
\end{Sinput}
\end{Schunk}
\end{example}

%% signals
\paragraph{Signals}
Unlike buttons, labels do not emit any signals. Labels are intended to
hold static text. However, if one wishes to define callbacks to react
to events, then the label can be placed within an instance of
\constructor{gtkEventBox}. This creates a non-visible parent window
for the label that does signal
events. Example~\ref{eg:RGtk2:editable-label} will illustrate the use
of an event box.  Alternatively, one could use an instance of
\code{gtkButton} with its \code{relief} property assigned to
\code{GtkReliefStyle['none']}.


\subsection{Link Buttons}
\label{sec:link-buttons}

A link button is a special label which shows an underlined link, such
as is done by a web browser. (Newer versions of \GTK\/ allow the label
of a button to contain HTML links.) The \code{uri} is specified to the
\constructor{gtkLinkButton} constructor with an optional
\argument{label}{gtkLinkButton} argument. If none is specified, the
\code{uri} is used to provide the value. This \code{uri} is stored in
the \code{uri} property and the label in the \code{label} value. These
may be adjusted later.

As the link button inherits from the \class{gtkButton} class, the
\code{clicked} signal is emitted when a user clicks a mouse on the link.

\begin{example}{Basic link button usage}{eg:RGtk2:link-button}
\begin{Schunk}
\begin{Sinput}
 w <- gtkWindow()
 g <- gtkVBox(); w$add(g)
 lb <- gtkLinkButton(uri="http://www.r-project.org")
 lb1<- gtkLinkButton(uri="http://www.r-project.org", label="R Home")
 g$packStart(lb)
 g$packStart(lb1)
 f <- function(w,...) browseURL(w['uri'])
 ID <- gSignalConnect(lb,"clicked",f = f)
 ID <- gSignalConnect(lb1,"clicked",f = f)
\end{Sinput}
\end{Schunk}
\end{example}


\section{Images}
\label{sec:RGtk2:images}

Images in \pkg{RGtk2} are constructed with
\constructor{gtkImage}. This is a front end for several constructors:
\constructor{gtkImageNewFromIconSet},
\constructor{gtkImageNewFromPixmap},
\constructor{gtkImageNewFromImage}, \constructor{gtkImageNewFromFile},
\constructor{gtkImageNewFromPixbuf},
\constructor{gtkImageNewFromStock},
\constructor{gtkImageNewFromAnimation}. We only discuss loading an
image from a file, and so use the \constructor{gtkImageNewFromFile}
constructor. To add an image after construction of the main widget,
the \constructor{gtkImageNew} constructor can be used along with
methods such as \method{SetFromFile}{gtkImage}.

The image widget, like the label widget, does not have a parent
GdkWindow, which means it does not receive window events. As with the
label widget, the image widget can be placed inside a
\constructor{gtkEventBox} container if one wishes to connect to such
events.



\begin{example}{Using a pixbuf to present graphs}{ex:RGtk2:pixbuf}

  This example shows how to use a \constructor{gtkImage} object to
  embed a graphic within \pkg{RGtk2}, as an alternative to using the
  \pkg{cairoDevice} package. The basic idea is to use the
  \function{Cairo} device to create a file containing the graphic, and
  then use \constructor{gtkImageNewFromFile} to construct a widget to
  show the graphic.

  We begin by creating a window of a certain size.
\begin{Schunk}
\begin{Sinput}
 w <- gtkWindow(show=FALSE); w$setTitle("Graphic window");
 w$setSizeRequest(400,400)
 g <- gtkHBox(); w$add(g)
 w$showAll()
\end{Sinput}
\end{Schunk}


The size of the image is retrieved from the size allocated to the box
\code{g}. This allows the window to be resized prior to drawing the
graphic. Unllke an interactive device, after drawing, this graphic
does not resize itself when the window resizes.

\begin{Schunk}
\begin{Sinput}
 theSize <- g$getAllocation()
 width <- theSize$width; height <- theSize$height
\end{Sinput}
\end{Schunk}

Now we draw a basic  graphic as a png file (\code{Cairo}'s default) stored in a temporary file.
\begin{Schunk}
\begin{Sinput}
 require(Cairo)
 filename <- tempfile()
 Cairo(file = filename, width = width, height = height)
 hist(rnorm(100))
 QT <- dev.off()
\end{Sinput}
\end{Schunk}

The constructor may be called as \command{gtkImage(filename=filename)}
or as follows:
\begin{Schunk}
\begin{Sinput}
 image <- gtkImageNewFromFile(filename)
 g$packStart(image, expand=TRUE, fill = TRUE)
 unlink(filename)                        # tidy up
\end{Sinput}
\end{Schunk}

\end{example}

\begin{Schunk}
\begin{Sinput}
 ## Work this into an example ###
 makeIconRGtk2 <- function(w, giffile) {
   if(checkPtrType(w, "GtkWindow")) {
     img <- gdkPixbufNewFromFile(giffile)
     if(!is.null(img$retval))
       w$setIcon(img$retval)
   }
 }
\end{Sinput}
\end{Schunk}
\section{Stock icons}
\label{sec:RGtk2:stock-icons}

\GTK\/ comes with several ``stock'' icons. These are used by the
\constructor{gtkButton} constructor when its \code{stock.id} argument
is specified, and will be used for menubars, and toolbars. The size of
the icon used is one of the values returned by \code{GtkIconSize}.

As mentioned previously, the full list of stock icons are returned in
a list by \function{gtkStockListIds}. The first $4$ are:
\begin{Schunk}
\begin{Sinput}
 head(unlist(gtkStockListIds()), n=4)   
\end{Sinput}
\begin{Soutput}
[1] "gtk-zoom-out" "gtk-zoom-in"  "gtk-zoom-fit" "gtk-zoom-100"
\end{Soutput}
\end{Schunk}

To load a stock icon into an image widget, the
\constructor{gtkImageNewFromStock} can be used. The
\argument{stock.id}{gtkImageNewFromStock} contains the icon name and
\argument{size}{gtkImageNewFromStock} the size. 

The example below, we use the method \method{RenderIcon}{gtkWidget} to
return a pixbuf containing the icon that can be used with the
constructor \constructor{gtkImageNewFromPixbuf} to display the
icon. Here the stock id and size are specified to the
\method{RenderIcon}{gtkWidget} method.



\begin{example}{\constructor{gtkButtonNewFromStock} -- the hard way}{ex:RGtk2:stock-icon}
The following example, shows how to do the work of
\constructor{gtkButtonNewFromStock} by hand using an image and label together.
\begin{Schunk}
\begin{Sinput}
 b <- gtkButton()
 g <- gtkHBox()
 pbuf <- b$renderIcon("gtk-ok", size=GtkIconSize["button"]) 
 i <- gtkImageNewFromPixbuf(pbuf)
 i['xalign'] <- 1; i['xpad'] <- 5        # right align with padding
 g$packStart(i, expand=FALSE)
 l <- gtkLabel(gettext("ok")); 
 l['xalign'] <- 0 # left align
 g$packStart(l, expand=TRUE, fill=TRUE)
 b$add(g)
 ## show it
 w <- gtkWindow(); w$add(b)
\end{Sinput}
\end{Schunk}
\end{example}

%% Only for the package
%% Adding icons to stock
\begin{example}{Adding to the stock icons}{ex:RGtk2:add-stock-icons}
This example shows, without much explanation the steps to add images
to the list of stock icons. To generate some sample icons, we use
those provided by objects in the \pkg{ggplot2} package.




First we create the icons using the fact that the objects have a function \code{icon} to draw an image.
\begin{Schunk}
\begin{Sinput}
 require(ggplot2)
 require(Cairo)
 iconNames <- c("GeomBar","GeomHistogram")   # 2 of many ggplot functions
 icon.size <- 16
 iconDir <- tempdir()
 fileNames <- sapply(iconNames, function(name) {
   nm <- paste(iconDir, "/", name, ".png", sep="", collapse="")
   Cairo(file=nm, width=icon.size, height=icon.size, type="png")
   val <- try(get(name))
   grid.newpage()
   try(grid.draw(val$icon()), silent=TRUE)
   dev.off()
   nm
 })
\end{Sinput}
\end{Schunk}
The following function works through the steps to add a new icon. The
basic ideas are sketched out in the API for \code{GtkIconsSet}.
\begin{Schunk}
\begin{Sinput}
 addToStockIcons <- function(iconNames, fileNames, stock.prefix="new") {
   iconfactory <- gtkIconFactoryNew()
   
   for(i in seq_along(iconNames)) {
     
     iconsource = gtkIconSourceNew()
     iconsource$setFilename(fileNames[i])
     
     iconset = gtkIconSetNew()
     iconset$addSource(iconsource)
     
     stockName = paste(stock.prefix, "-", iconNames[i], sep="")
     iconfactory$add(stockName, iconset)
     
     items = list(test=list(stockName, iconNames[i],"","",""))
     gtkStockAdd(items)
   }
   iconfactory$AddDefault()
   invisible(TRUE)
 }
\end{Sinput}
\end{Schunk}
We call this function and then check that the values are added:
\begin{Schunk}
\begin{Sinput}
 addToStockIcons(iconNames, fileNames)
 nms <- gtkStockListIds()
 unlist(nms[grep("^new", nms)])
\end{Sinput}
\begin{Soutput}
[1] "new-GeomHistogram" "new-GeomBar"      
\end{Soutput}
\end{Schunk}

\end{example}



%% Alertpanel application
\begin{example}{An alert panel}{eg:RGtk2:alert-panel}


\begin{figure}
  \centering
  \includegraphics[width=.45\textwidth]{ex-RGtk2-alert-panel}
  \caption{The alert panel showing a message.}
  \label{fig:RGtk2-alert-panel}
\end{figure}

This example puts together images, buttons, labels and box containers
to create an alert panel, or information bar. This is an area that
seems to drop down from the menu bar to give users feedback about an
action that is less disruptive than a modal dialog. A similar widget
is used by Mozilla with its popup blocker. Although, as of version
2.18, a similar feature is available in \GTK\/ through the
\code{GtkInfoBar} widget, this example is given, as it shows how
several useful things in \GTK\/ can be combined to customize the user
experience.

This constructor for the widget specifies some properties and returns
an environment to store these properties, as our function calls will
need to update these properties and have be persistent.
\begin{Schunk}
\begin{Sinput}
 newAlertPanel <- function(wrap=35,
                           icon="gtk-dialog-warning",
                           message="",
                           panel.color="goldenrod",
                           evb=NULL,
                           image=NULL,
                           label=NULL # info
                     ) {
   x <- c("wrap","icon","message","panel.color","evb","image","label")
   e <- new.env()
   sapply(x, function(i) assign(i, envir=e, get(i)))
   return(e)
 }
\end{Sinput}
\end{Schunk}

An alert panel needs just a few methods: one to create the widget, one
to show the widget and one to hide the widget. We create a function
\code{getAlertPanelBlock} to return a component that can be added to a
container. An event box is used so that we can color the background,
as this isn't possible for a box container due to its lack of a gdk
window.  To this event box we add a box container that will hold an
icon indicating this is an alert, a label for the message, and another
icon to indicate to the user how to close the alert. Since we wish to
receive mouse clicks on close icon, we place this inside another event
box. To this, we bind a callback to the \signal{button-press-event} signal.
\begin{Schunk}
\begin{Sinput}
 getAlertPanelBlock <- function(obj) {
 
   obj$evb <- gtkEventBox(show=FALSE)
   obj$evb$ModifyBg(state="normal",color=obj$panel.color)
 
   g <- gtkHBox(homogeneous=FALSE, spacing=5)
   obj$evb$add(g)
 
   obj$image <- gtkImageNewFromStock(obj$icon, size="button",show=TRUE)
   obj$image['yalign'] <- .5
   g$packStart(obj$image, expand=FALSE)
 
   obj$label <- gtkLabel(obj$message)
   obj$label['xalign'] <- 0; obj$label['yalign'] <- .5
   obj$label$setLineWrap(TRUE)
   obj$label$setWidthChars(obj$wrap)
   g$packStart(obj$label, expand=TRUE, fill= TRUE)
 
   xbutton <- gtkEventBox()
   xbutton$modifyBg(state="normal",color=obj$panel.color) # same color
   xbutton$add(gtkImageNewFromStock("gtk-close",size="menu"))
   xbuttonCallback <- function(data,widget,...) {
     hideAlertPanel(data)
     return(FALSE)
   }
   ID <- gSignalConnect(xbutton,"button-press-event",
                  f = xbuttonCallback,
                  data=obj, user.data.first=TRUE)
   g$packEnd(xbutton, expand=FALSE, fill=FALSE)
 
   ## also close when event box is clicked
   obj$motionID <-
     gSignalConnect(obj$evb,signal="button-press-event",
                    f = xbuttonCallback,
                    data=obj, user.data.first=TRUE)
   return(obj$evb)
 }
\end{Sinput}
\end{Schunk}

The \code{showAlertPanel} function updates the message and then calls
the \method{Show}{gtkWidget} method of the event box.

\begin{Schunk}
\begin{Sinput}
 showAlertPanel <- function(obj) {
   obj$label$setText(obj$message)
   obj$evb$show()
 }
\end{Sinput}
\end{Schunk}

Our \code{hideAlertPanel} function simply calls the \method{hide}{gtkWidget}
method the event box.
\begin{Schunk}
\begin{Sinput}
 hideAlertPanel <- function(obj) obj$evb$hide()
\end{Sinput}
\end{Schunk}

To test it out, we create a simple GUI
\begin{Schunk}
\begin{Sinput}
 w <- gtkWindow()
 g <- gtkVBox(); w$add(g)
 ap <- newAlertPanel()
 g$packStart(getAlertPanelBlock(ap), expand=FALSE)
 g$packStart(gtkLabel("fill space"), expand=TRUE, fill=TRUE)
 ap$message <- "New Message"             # add message
 showAlertPanel(ap)
\end{Sinput}
\end{Schunk}

One could also add a time to close the panel after some delay. The \function{gTimeoutAdd} function is used to specify a function to call periodically until the function returns \code{FALSE}. 

\end{example}

\section{Text entry}
\label{sec:RGtk2:gtkEntry}

A one-line text entry widget is constructed by
\function{gtkEntry}. An  argument \argument{max}{gtkEntry} specifies
the maximum number of characters if positive, but this calls a
deprecated function, so this restriction should be set using the
method \method{SetMaxLength}{gtkEntry}.

The \code{text} property stores the text. This can be set with the
method \method{SetText}{gtkEntry} and retrieved with
\method{GetText}{gtkEntry}. To insert text, the method
\method{InsertText}{gtkEditable} is used.  Its argument
\argument{new.text}{gtkEditableInsertText} contains the text and
\argument{position}{gtkEditableInsertText} specifies the position of
the next text. The return value is a list with components
\code{position} indicating the position \textit{after} the new
text. The \method{DeleteText}{gtkEditable} method can be used to
delete text. This takes two integers indicating the start and finish
location of the text. 

\begin{example}{Insert and Delete text}{eg:RGtk2:insert-delete-text}
The example will show how to add then delete text.  
\begin{Schunk}
\begin{Sinput}
 e <- gtkEntry()
 e$setText("Where did that guy go?")
 add.pos <- regexpr("guy", e['text']) - 1 # before "guy"
 ret <- e$insertText("@$#%! ", position = add.pos)
 e$getText()                             # or e['text']
\end{Sinput}
\begin{Soutput}
[1] "Where did that @$#%! guy go?"
\end{Soutput}
\begin{Sinput}
 e$deleteText(start = add.pos, end= ret$position)
 e$getText()
\end{Sinput}
\begin{Soutput}
[1] "Where did that guy go?"
\end{Soutput}
\end{Schunk}
\end{example}

%% signals
The \class{GtkEntry} class adds three signals \signal{changed} when
text is changed, \signal{delete-text} for delete events, and
\signal{insert-text} for insert events. The \signal{changed} signal will
be emitted each time there is a keypress, while the widget has
focus. When the \kbd{enter} key is pressed the \signal{activate} signal
is also emitted. 

\begin{example}{Editable label}{eg:RGtk2:editable-label}
We use the \signal{activate} signal in the following example where we
make an editable label. When the label is clicked, a text entry
replaces the label allowing its content to be edited. The user
indicates the editing is over by pressing the \kbd{enter} key. This
example also illustrates the \function{gtkEventBox}, which emits a
\signal{button-press-event} signal, when the label is clicked on.  
\begin{Schunk}
\begin{Sinput}
 w <- gtkWindow(); w$setTitle("Editable label")
 evb <- gtkEventBox(); 
 w$add(evb); 
 e <- gtkEntry()
 l <- gtkLabel("Click me to edit")
 evb$setData("entry", e); evb$setData("label", l)
 evb$add(l)
 ID <- gSignalConnect(evb, "button-press-event", function(w, e, ...) {
   label <- w$getData("label"); entry <- w$getData("entry")
   entry$setText(label$getText())
   w$remove(label);  w$add(entry) # swap
 })
 ID <- gSignalConnect(e,"activate", function(userData, w, ...) {
   evb = userData$evb; label <- evb$getData("label")
   label$setText(w$getText())
   evb$remove(w); evb$add(label) #swap
 },
                      data=list(evb=evb),
                      user.data.first=TRUE)
\end{Sinput}
\end{Schunk}
\end{example}


\section{Check button}
\label{sec:RGtk2:gtkCheckbox}

A check button widget is constructed by \function{gtkCheckButton}. The
optional argument \argument{label}{gtkCheckButton} places a label next
to the button. The label can have a mnemonic, but then the constructor
is  \constructor{gtkCheckButtonewWithMnemonic}.

The \code{label} property stores the label. This can be set or
retrieved with the methods \method{SetLabel}{gtkButton} and
\method{GetLabel}{gtkButton}. 

A check button's state is stored as a logical variable in its
\code{active} property. It can be set or retrieved with the methods
\method{SetActive}{gtkToggleButton} and
\method{GetLabel}{gtkToggleButton}. 

When the state is changed the \signal{toggle} signal is emitted.


\subsection{Toggle buttons}
\label{sec:ToggleButtons}

A toggle button, is a useful way to set configuration values in an
obvious way to the user.  A toggle button has a depressed look when in
an active state. The \constructor{gtkToggleButton} constructor is
used to create toggle buttons. The \argument{label}{gtkToggleButton}
argument sets the \code{label} property. This can also be set or
retrieved with the methods \method{SetLabel}{gtkButton} and
\method{GetLabel}{gtkButton}.

%% active property
The \code{active} property is \code{TRUE} when the button is
depressed, and \code{FALSE} otherwise. This can be queried with the
\method{GetActive}{gtkToggleButton} method.

%% signal clicked
As with other buttons, the \code{clicked} signal is emitted when the user
clicks on the button.


\section{Radio groups}
\label{sec:RGtk2:gtkRadioButton}

The \function{gtkRadioButton} constructor is used to create linked
radio buttons.  The argument \argument{group}{gtkRadioButton} is
missing or \code{NULL} will create a new radio button group. If
specified as a list of radio buttons, will create a new button for the
group.  The constructor returns a single radio button widget.  The
labels for each individual button are determined by their \code{label}
property, This can be set at construction time through the
\argument{label}{gtkRadioButton}, or can be modified through the
\method{SetLabel}{gtkButton} method.

%% active
Each radio button in the group has its \code{active} property either
\code{TRUE} or \code{FALSE}, although only one can be \code{TRUE} at a
time. The methods \method{GetActive}{gtkToggleButton} and \method{SetActive}{gtkToggleButton} may be used to manipulate the state of an individual button. To determine which button is active, they can be queried
individually. The same property can be set to make a given button
active.  

%% --- toggled signal ---
When the state of a radio button is changed, it emits the
\signal{toggled} signal. To assign a callback to this event, each
button in the group must register a callback for this signal The
\code{active} property can be queried to decide if the toggle is from
being selected, or deselected.


\begin{example}{Radio group construction}{eg:RGtk2:radio-buttons}
Creating a new radio button group follows this pattern:
\begin{Schunk}
\begin{Sinput}
 vals <- c("two.sided", "less", "greater")
 l <- list()                                 # list for group
 l[[vals[1]]] <- gtkRadioButton(label=vals[1]) # group = NULL
 for(i in vals[-1]) 
   l[[i]] <- gtkRadioButton(l, label=i)  # group is a list
\end{Sinput}
\end{Schunk}
Each button needs to be managed. Here we illustrate a simple GUI doing so.
\begin{Schunk}
\begin{Sinput}
 w <- gtkWindow(); w$setTitle("Radio group example")
 g <- gtkVBox(FALSE, 5); w$add(g)
 QT <- sapply(l, function(i) g$packStart(i))
\end{Sinput}
\end{Schunk}
We can set and query which button is active, as follows:
\begin{Schunk}
\begin{Sinput}
 l[[3]]$setActive(TRUE)           
 sapply(l, function(i) i$getActive()) 
\end{Sinput}
\begin{Soutput}
two.sided      less   greater 
    FALSE     FALSE      TRUE 
\end{Soutput}
\end{Schunk}
Here is how we might register a callback for the \code{toggled} signal.
\begin{Schunk}
\begin{Sinput}
 QT <- sapply(l, function(i) 
        gSignalConnect(i, "toggled",     # attach each to "toggled"
                       f = function(w, data) {
                         if(w$getActive()) # set before callback
                           cat("clicked", w$getLabel(),"\n")
                       }))
\end{Sinput}
\end{Schunk}
\end{example}

The \pkg{RGtk2} package coverts a list in \R\/ to the appropriate list for \GTK. However,
you may wisth to refer to this list within a callback, but only the current radio button is passed through. Rather than passing the list through the \code{data} argument or using a global, The \method{GetGroup}{gtkRadioButton} method can be used to
reference the buttons stored within a radio group. This method returns
a list containing the radio button. However, it is in the
reverse order of how they were added (newest first). (As
 GLib list uses prepend to add elements, not append, as it
is more efficient.)

\begin{example}{Radio group using \code{GetGroup}}{eg:gtk:radio-group-get-group}
  In this example below, we illustrate two things: using the
  \method{NewWithLabelFromWidget}{gtkRadioButton} method to add new
  buttons to the group and the \method{GetGroup}{gtkRadioButton}
  method to reference the buttons. The \function{rev} function is used
  to pack the widgets, to get them to display first to last.
\begin{Schunk}
\begin{Sinput}
 radiogp <- gtkRadioButton(label=vals[1])
 for(i in vals[-1])
   radiogp$newWithLabelFromWidget(i)
 w <- gtkWindow(); w['title'] <- "Radio group example"
 g <- gtkVBox(); w$add(g)
 QT <- sapply(rev(radiogp$getGroup()),         # reverse list
              function(i) g$packStart(i))
\end{Sinput}
\end{Schunk}
\end{example}

\section{Combo boxes}
\label{sec:RGtk2:basic-combobox}
A basic combobox is constructed by
\constructor{gtkComboBoxNewText}. Later we will discuss more
complicated comboboxes. Unlike others, as of writing, this widget must
have its \method{Show}{gtkWidget} method called to be mapped.

For the basic combobox, items may be added to the combobox in a few manners: to add to the end or beginning we have
\method{AppendText}{gtkComboBox} and \method{PrependText}{gtkComboBox}; to insert within the list the \method{InsertText}{gtkComboBox} method is used with the argument
\argument{position}{gtkComboBoxInsertText} specified in addition to the argument 
\argument{text}{gtkComboBoxInsertText} to indicate the index where the values should added. (The prepend method would be index $0$, the append method would be with an index equal to the number of existing items.)

The currently selected value is specified by index with the method
\method{SetActive}{gtkComboBox} and returned by
\method{GetActive}{gtkComboBox}. The index, as usual, is $0$-based, and in
this case uses a value of $-1$ to specify that no value is selected.
The \method{GetActiveText}{gtkComboBox} method can be used to retrieve the text shown by the basic combo box

It can be difficult to use a combobox when there are a large number of
selections. The \method{SetWrapWidth}{gtkComboBox} method allows the
user to specify the preferred number of columns to be used to display
the data.


%% signal
The main signal to connect to is \signal{changed} which is emitted
when the active item is changed either by the user or the programmer
through the \code{GetActive} method.

\begin{example}{Combo box}{eg:RGtk2:simple-combo-box}
A simple combobox may be produced as follows:
\begin{Schunk}
\begin{Sinput}
 vals <- c("two.sided", "less", "greater")
 cb <- gtkComboBoxNewText()
 for(i in vals) cb$appendText(i)
 cb$setActive(0)                         # first one
 ID <- gSignalConnect(cb, "changed",
                f = function(w, ...) {
                  i <- w$getActive() + 1 # shift index
                  if(i == 0) 
                    cat("No value selected\n")
                  else
                    cat("Value is", w$getActiveText(), "\n")
                })
\end{Sinput}
\end{Schunk}
A simple GUI is shown, the call to \code{ShowAll} is use here, as this
widget does not get mapped otherwise.
\begin{Schunk}
\begin{Sinput}
 ## show it
 w <- gtkWindow(show=FALSE); w['title'] <- "Combobox example"
 w$add(cb)
 w$showAll()    # propogate down to cb
\end{Sinput}
\end{Schunk}
\end{example}




\subsection{Sliders}
\label{sec:RGtk2:sliders}

The slider widget and spinbutton widget allow selection from a
regulary spaced list of values. In \GTK\/ these values are stored in an
adjustment object, whose details are mostly hidden in normal use.

The slider widget in \GTK\/ may be oriented either horizontally or
vertically. The decision
is made through the choice of constructor: \constructor{gtkHScale} or
\constructor{gtkVScale}. For these widgets, the adjustment can be
specified -- if desired, or for convenience, will be created if the arguments
\argument{min}{gtkHScale}, \argument{max}{gtkHScale}, and
\argument{step}{gtkHScale} are given.  These arguments take  numeric
values. As the first argument (\argument{adjustment}{gtkHScale}) is
used to specify an adjustment,
these values are best specified by name. Alternatively, the
\constructor{gtkHScaleNewWithRange} constructor can be used with
positional arguments for \code{min}, \code{max} and \code{step}.


The methods \method{getValue}{gtkRange} and
\method{getValue}{gtkRange} can be used to return and set the value of
the widget. When assigning a value, values outside the bounds will be
set to the minimum or maximum value.

%% properties
A few properties can be used to adjust the appearance of the slider widget.
The \code{digits} property controls the number of digits after the
decimal point that are displayed.  The property \code{draw-value} can be
used to turn off the drawing of the selected value near the
slider. Finally, the property \code{value-pos} specifies where this
value will be drawn using values from \code{GtkPositionType}. The
default is \code{top}.

%% value-changed
Callbacks can be assigned to the \code{value-changed} signal, which is
emitted when the slider is moved.

\begin{example}{A slider controlling histogram bin selection}{ex:RGtk2:sliders}
  A simple mechanism to make a graph interactive, is to have the
  graph redraw wheneven a slider has its value changed. The following
  shows how this can be achieved.
\begin{Schunk}
\begin{Sinput}
 library(lattice)
 w <- gtkWindow(); w$setTitle("Histogram bin selection")
 slider <- gtkHScaleNewWithRange(1, 100, 1) # min, max, step
 slider['value-pos'] <- "bottom"
 w$add(slider)
 ## initial graphic
 x <- rnorm(100)
 print(histogram(x), nint = slider$getValue())
 ID <- gSignalConnect(slider,"value-changed",
                f = function(w, ...) {
                  val <- w$getValue()
                  print(histogram(x, nint=val))
                })
\end{Sinput}
\end{Schunk}
\end{example}

\subsection{Spinbuttons}
\label{sec:RGtk2:spinboxes}

The spinbutton widget is very similar to the slider widget in \GTK. Spinbuttons are constructed with
\constructor{gtkSpinButton}. As with sliders, this constructor allows a
specification of the adjustment with an actual adjustment, or through
the arguments \argument{min}{gtkSpinButton}, \argument{max}{gtkSpinButton}, and
\argument{step}{gtkSpinButton}. 

As with sliders, the methods
\method{GetValue}{gtkSpinButton} and \method{SetValue}{gtkSpinButton}
are used to get and set the widgets value. The property
\code{snap-to-ticks} can be set to \code{TRUE} to force the new value
to be one of sequence of values in the adjustment. The \code{wrap}
property indicates if the widget will ``wrap'' around when at the
bounds of the adjustment.

Again, as with sliders, the \code{value-changed} signal is emitted when the
spin button is changed. 

\begin{example}{A range widget}{ex:RGtk2-range-widget}
This example shows how to make a range widget that combines both the slider and spinbutton to choose a single number. Such a widget is popular, as the slider is easy to make large changes and the spinbutton better at finer changes. In \GTK\/ we use the same adjustment, so changes to one widget propogate without effort to the other.

\begin{Schunk}
\begin{Sinput}
 ## make a range widget combining both a slider and spinbutton to choose a number
 library(RGtk2)
\end{Sinput}
\end{Schunk}

Were this written as a function, an \R\/ user might expect the
arguments to match those of \code{seq}/
\begin{Schunk}
\begin{Sinput}
 from <- 0; to <- 100; by <- 1
\end{Sinput}
\end{Schunk}

The slider is drawn without a value, so that the user sees only that
in the spinbutton. This spinbutton is created by specifying the
adjustement created when the slider widget is.
\begin{Schunk}
\begin{Sinput}
 slider <- gtkHScale(min=from, max=to, step=by)
 slider['draw-value'] <- FALSE
 adjustment <- slider$getAdjustment()
 spinbutton <- gtkSpinButton(adjustment=adjustment)
\end{Sinput}
\end{Schunk}
Our layout places the two widgets in a horizontal box container with
the slider set to expand on a resize, but not the spinbutton.
\begin{Schunk}
\begin{Sinput}
 g <- gtkHBox()
 g$packStart(slider, expand=TRUE, fill=TRUE, padding=5)
 g$packStart(spinbutton, expand=FALSE, padding=5)
\end{Sinput}
\end{Schunk}


\end{example}

\subsection{The cairoDevice package}
\label{sec:cairodevice-package}

The package \pkg{cairoDevice} describes itself as a ``Cairo-based
cross-platform antialiased graphics device driver.''  It can be
embedded in a \pkg{RGtk2} GUI as with any other widget. Its basic
usage involves a few steps. First a new drawing area is made with
\constructor{gtkDrawingArea}. This drawing area can be
used by various drawing functions, that we do not describe. (In fact,
arbitrary widgets, such as pixbufs, can be used here.) The
\pkg{cairoDevice} package provides the function
\function{asCairoDevice} to coerce the drawing area to a graphics
device. This function has standard argument
\argument{pointsize}{asCairoDevice} and for some underlying widgets
\argument{width}{asCairoDevice} and \argument{height}{asCairoDevice} arguments.

\section{Containers}
\label{sec:containers}


In addition to boxes, there are a number of useful containers detailed
next. Each of the following, except for the table widget, inherit the
\method{Add}{gtkContainer} method and allow for only one child component.


\subsection{Framed containers}
\label{sec:RGtk2:gtkFrame}

The \function{gtkFrame} function constructs a container with a
decorative frame to set off the containers components. The optional
\argument{label}{gtkFrame} argument can be used to specify the
\code{label} property. This can be subsequently retrieved and set
using the \method{GetLabel}{gtkFrame} and \method{SetLabel}{gtkFrame}
methods. The label can be aligned using the
\method{SetLabelAlign}{gtkFrame} method. This has arguments
\code{xalign} and \code{yalign}, with values in $[0,1]$, to specify the position of the label
relative to the frame.

Frames have a decorative shadow whose type is stored in the
\code{shadow-type} property. This type is a value from \code{GtkShadowType}.

\subsection{Expandable containers}
\label{sec:RGtk2:gtkExpander}

Although they are a little unresponsive due to eventloop issues, an
expandable container proves quite useful to manage screen space. Expandable containers are
constructed by \function{gtkExpander}. Use
\function{gtkExpanderNewWithMnemonic} if a mnemonic is desired. The
containers are drawn with a trigger button and optional label, which can be
clicked on to hide or show the containers child.

The label can be given as an optional argument to the constructor, or
assigned later with the \method{SetLabel}{gtkExpander} method, The label can use
Pango markup. This is indicated by setting the \code{use-markup}
property  through \method{SetUseMarkup}{gtkExpander}. 

The state of the widget is stored in the \code{expanded} property,
which can be accessed with \method{GetExpanded}{gtkExpander} and
\method{SetExpander}{gtkExpander}. 
When the state changes, the \signal{activate} signal is emitted.

\section{Divided containers}
\label{sec:RGtk2:gtkPanedWindow}

The \constructor{gtkHPaned} and \constructor{gtkVPaned} create
containers with a ``gutter'' to allocate the space between its two
children. An example is presented in Example
\ref{eg:RGtk2:using-tree-content}. Like the \code{gtkBin} containers,
the two spaces allow only one child component. The two children may be
added two different ways. The methods \method{Add1}{gtkPaned} and
\method{Add2}{gtkPaned} simply add the child, whereas the methods
\method{Pack1}{gtkPaned} and \method{Pack2}{gtkPaned} have arguments
\argument{resize}{gtkPanedPack1} and \argument{shrink}{gtkPanedPack1}
which specify how the child will resize if the paned container is
resized. These two arguments take logical values.  After children are
added, they can be referenced from the container through its
\method{GetChild1}{gtkPaned} and \method{GetChild2}{gtkPaned} methods.

The position of the gutter can be set with the
\method{SetPosition}{gtkPaned} method. This is given in terms of
screen position. The
properties \code{min-position} and \code{max-position} can be used to
convert a percentage into a screen position.

The \signal{move-handle} signal is emitted when the gutter position is
changed. 



\section{Notebooks}
\label{sec:RGtk2:gtkNotebook}

The \constructor{gtkNotebook} constructor creates a notebook container. The
default position of the notebook tabs is on the top, starting on the
left. The property \code{tab-pos} property
(\method{SetTabPos}{gtkNotebook}) uses the \code{GtkPositionType}
values of \qcode{left},\qcode{right},\qcode{top}, or \qcode{bottom} to adjust this. The
property \code{scrollable} should be set to \code{TRUE} to have the
widget gracefully handle the case when there are more page tabs than
can be shown at once. If the same size tab for each page is desired,
the method \method{SetHomogeneousTabs}{gtkNotebook} can be called with
a value of \code{TRUE}. 


%% adding pages
\paragraph{Adding pages to a notebook}
New pages can be added to the notebook with the
\method{InsertPage}{gtkNotebook} method. Each page of a notebook holds
one child component. This is specified with the
\argument{child}{gtkNotebookInsertPage} argument. The tab label can be
specified with the \argument{tab.label}{gtkNotebookInsertPage}
argument, but can also be set later with
\method{SetTabLabel}{gtkNotebook} and retrieved with
\method{getTabLabel}{gtkNotebook}. The label is specified using a
widget, such as a \function{gtkLabel} instance, but this allows for
more complicated tabs, such as a box container with a close icon. The
\method{SetTabLabelText}{gtkNotebook} can be used if just a text label
is desired. To use this method, the child widget is needed. the
\method{[[}{GObject}
%%]]
 method can be used to retrieve this. So can the
\method{GetNthPage}{gtkNotebook} method. Both are an alternative to
getting all the children returned as a list through
\method{GetChildren}{gtkContainer}. By default, the new page will be at
the last position (the same as \method{AppendPage}{gtkNotebook}). This
can be changed by supplying the desired position to the argument
\argument{position}{gtkNotebookInsertPage} using $0$-based indexing.
The default value is $-1$, indictating the last page.


%% page motions: reordered, deleted
\paragraph{Rearranging pages}
Pages can be reordered using the \method{ReorderChild}{gtkNotebook}
method. The arguments are the \argument{child}{gtkNotebook} for the
child widget, and \argument{position}{gtkNotebook}, again 0-based with
$-1$ indicating appending at the end. Pages can be deleted using the
method \method{RemovePage}{gtkNotebook}. The
\argument{page.num}{gtkNotebookRemovePage} argument specifies the page
by its position. If the child is known, but not the number the method
\method{PageNum}{gtkNotebook} returns the page number. Its argument is \argument{child}{gtkNotebookPageNum}.

%% movement
The current page number is stored in the \code{page} property.
The number of pages can be found by inspecting the length of the
return value of \method{GetChildren}{gtkContainer}, but more directly
is done with the method \method{GetNPages}{gtkNotebook}. A given page
can be raised with the \method{SetCurrentPage}{gtkNotebook}
method. The argument \argument{page.num}{gtkNotebookSetCurrentPage}
specifies which page number to raise. If the child container should
not be hidden, or the page won't change. Incremental movements are
possible through the methods \method{NextPage}{gtkNotebook} and
\method{PrevPage}{gtkNotebook}. 

\paragraph{Signals}
The notebook widget emits various signals when its state is
changed. Among these: the signal \signal{focus-tab} is emitted when a tab receives the
focus, \signal{select-page} is similar and the \signal{switch-page} is
emitted when the current page is changed.

\begin{example}{Adding a page with a close button}{eg:RGtk2-notebook-close-icon}
A familiar element of notebook tabs from web browsing is a close button. The following
defines a new method \method{InsertPageWithCloseButton}{gtkNotebook}
that will use an ``x'' to indicate a close button. An icon would be
prettier, of course. The callback passes both the notebook and the
page through the \code{data} argument, so that the proper page can be
deleted. One caveat, the simpler command \command{nb\$getCurrentPage()}
will return the page of the focused tab prior to clicking the "x"
button, which may not be the correct page to close.

\begin{Schunk}
\begin{Sinput}
 gtkNotebookInsertPageWithCloseButton <- 
   function(object, child, label.text="", position=-1) {
     label <- gtkHBox()
     label$packStart(gtkLabel(label.text))
     label$packEnd(b <- gtkButton("x"))   # prettier with icon
     ID <- gSignalConnect(b,"clicked",
                    function(userData, b, ...) {
                      nb <- userData$nb 
                      page <- userData$page
                      nb$removePage(nb$pageNum(page))
                    },
                    data = list(nb=object, page=child),
                    user.data.first=TRUE)
     object$insertPage(child, label, position)
   }
\end{Sinput}
\end{Schunk}

We now show a simple usage of a notebook.
\begin{Schunk}
\begin{Sinput}
 w <- gtkWindow()
 nb <- gtkNotebook(); w$add(nb)
 nb$setScrollable(TRUE)
 QT <- nb$insertPageWithCloseButton(gtkButton("hello"), label.text="page 1")
 QT <- nb$insertPageWithCloseButton(gtkButton("world"), label.text="page 2")
\end{Sinput}
\end{Schunk}
  
\end{example}


\subsection{Scrollable windows}
\label{sec:RGtk2:scroll-windows}

Scrollbars allow components that are larger than the space allotted to
be interacted with, by allowing the user to ajdust the visible portion
of the componented. Scrollbars in \GTK\/ use adjustments (like the
spin button widget) to control the $x$ and $y$ position of the
displayed portion of the component.

The convenience function \constructor{gtkScrolledWindow} creates a
window that allows the user to scroll around its  child component. By
default, the horizontal and vertical adjustments are generated,
although, if desired, these
may be specified by the programmer.

Like a top-level window, a scrolled window is a \class{GtkBin} object
and has only one immediate child component. If this child is a tree
view, text view (discussed in the following), icon view, layout or
viewport the \method{add}{gtkContainer} method is used. Otherwise, the
method \method{addWithViewport}{gtkScolledWindow} can be used to create an
intermediate viewport around the child.


The properties \code{hscrollbar-policy} and \code{vscrollbar-policy}
determine if the scrollbars are drawn. By default, they are always
drawn. The \code{GtkPolicyType} enumeration, allows a specification of
\qcode{automatic} so that the scrollbars are drawn if needed, i.e, the
child component requests more space than can be allotted. The
\method{setPolicy}{gtkScrolledWindow} method allows both to be set at
once, as in the following example.

\begin{example}{Scrolled window example}{eg:RGtk2:scrolled-window}
This example shows how a scrolled window can be used to display a long list of values. The tree view
widget can also do this, but here we can very easily customize the display of each value. In the example, we simply locates where a label is placed.


\begin{Schunk}
\begin{Sinput}
 g <- gtkVBox(spacing=0)
 QT <- sapply(state.name, function(i) {
   l <- gtkLabel(i)
   l['xalign'] <- 0; l['xpad'] <- 10
   g$packStart(l, expand=TRUE, fill=TRUE)
 })
\end{Sinput}
\end{Schunk}

The scrolled window has just two basic steps in its construction. Here
we specify never using a scrolled window for the vertical display.
\begin{Schunk}
\begin{Sinput}
 sw <- gtkScrolledWindow()
 sw$setPolicy("never","automatic")
 sw$addWithViewport(g)                   # just "Add" for text, tree, ...
\end{Sinput}
\end{Schunk}

\begin{Schunk}
\begin{Sinput}
 w <- gtkWindow(show=FALSE)
 w$setTitle("Scrolled window example")
 w$setSizeRequest(-1, 300)
 w$add(sw)
 w$show()
\end{Sinput}
\end{Schunk}
\end{example}

\section{Tabular layout}
\label{sec:RGtk2:gtkTable}

The \constructor{gtkTable} constructor produces a container for laying
out objects in a tabular format. The container sets aside cells in a
grid, and a child component may occupy one or more cells. The
\argument{homogeneous}{gtkTable} argument can be used to make all
cells homogeneous in size. Otherwise, each column and row can have a
different size. At the time of construction, the number rows and
columns for the table many be specified with the
\argument{rows}{gtkTable} and \argument{columns}{gtkTable}
arguments. After construction, the \method{Resize}{gtkTable} method
can be used to resize these values.

%% adding children
Child components are added to this container through the
\method{AttachDefaults}{gtkTable} method. Its first argument,
\argument{child}{gtkTableAttachDefaults}, is the child component. This
component can span more than one cell. To specify which cells, the
arguments \argument{left.attach}{gtkTableAttachDefaults} and
\argument{right.attach}{gtkTableAttachDefaults} specify the columns through
the column number to attach the left (or right) side of the child to, and 
\argument{top.attach}{gtkTableAttachDefaults} and
\argument{bottom.attach}{gtkTableAttachDefaults} to specify the rows. 

The \method{Attach}{gtkTable} method is similar, but allows the
programmer more control over the placement of the child
component. This method has the arguments
\argument{xoptions}{gtkTableAttach} and
\argument{yoptions}{gtkTableAttach} to specify how the widget responds
to resize events. These arguments use the values of
\code{GtkAttachOptions} to specify either \qcode{expand},
\qcode{shrink} and/or \qcode{fill}. Just \qcode{fill} will cause the
widget to remain the same size if the window is enlarged, the
\qcode{expand} and \qcode{fill} combination will cause the component
to fill the available space, and the \code{shrink} option instructs
the widget to shrink if the table is made smaller through
resizing. Finally, the \argument{xpadding}{gtkTableAttach} and
\argument{xpadding}{gtkTableAttach} arguments allow the specification
of padding around the cell in pixels.


Anchoring of widgets within a cell can be done by setting the
\code{xalign} and \code{yalign} properties of the child widgets. 

\begin{example}{Layout of calculator buttons}{eg-RGtk2-calculator-buttons}
  The following example shows how to layout calculator buttons. In
  this example we want each button click to call the same callback,
  passing in user data to parameterize the call.  To do this
  efficiently, we define a new method
  \method{GetLabelsFromDataFrame}{gtkTable} which will map the entries
  in a data frame to the button labels. This method is defined as,
\begin{Schunk}
\begin{Sinput}
 gtkTableGetLabelsFromDataFrame <-
   function(object, dataFrame, callBack, userData=NULL) {
     checkPtrType(object, "GtkTable")
     dataFrame <- as.data.frame(dataFrame)
     d <- dim(dataFrame)
     object$resize(d[1], d[2])            # resize dynamically
     for(i in 1:d[1]) {
       for(j in 1:d[2]) {
         child <-  gtkButton(as.character(dataFrame[i,j]))
         id <- connectSignal(child,"clicked", f=callBack,
                             data=list(i=i, j=j, data=userData),
                             user.data.first=TRUE)
         ## 0-based for attaching
         object$attach(child, 
                       left.attach=j-1, j, top.attach=i-1, i)
       }}}
\end{Sinput}
\end{Schunk}
The callback function should have signature
\code{(userData,button)}. We use the following matrix to store our
calculator layout.
\begin{Schunk}
\begin{Sinput}
 m = rbind(
   c("^", "(", ")", " / ", " == "),
   c(7  ,   8,   9, " * ", " > "),
   c(4  ,   5,   6, " - ", " >= "),
   c(1  ,   2,   3, " + ", " < "),
   c(0  , ".", " ", " ! ", " <= ")
   )
\end{Sinput}
\end{Schunk}
Finally, we create our GUI using the method above to do the bulk of
the work.
\begin{Schunk}
\begin{Sinput}
 tab <- gtkTable(homogeneous=TRUE)       # same size buttons
 tab$getLabelsFromDataFrame(m, function(userData, b) 
                            cat(b$getLabel()))
 w <- gtkWindow(show=FALSE)
 w$setTitle("Show matrix")
 w$add(tab)
 w$show()
\end{Sinput}
\end{Schunk}
\end{example}


\begin{example}{Dialog layout}{ex-RGtk2-dialog-layout}
This example shows how to layout some controls for a dialog with some
attention paid to how the widgets are aligned and how they respond to
resizing of the window.

\begin{figure}
  \centering
  \includegraphics[width=.5\textwidth]{ex-RGtk2-dialog-layout}
  \caption{A basic dialog using a \code{gtkTable} container for layout.}
  \label{fig:RGtk2-dialog-layout}
\end{figure}


Our basic GUI is a table with 4 rows and 2 columns.
\begin{Schunk}
\begin{Sinput}
 w <- gtkWindow(show=FALSE)
 w$setTitle("example of gtkTable and attaching")
 tbl <- gtkTable(rows=4, columns=2, homogeneous=FALSE)
 w$add(tbl)
\end{Sinput}
\end{Schunk}

We define our widgets first then deal with their layout.
\begin{Schunk}
\begin{Sinput}
 l1 <- gtkLabel("Sample size")
 w1 <- gtkComboBoxNewText()
 QT <- sapply(c(5, 10, 15, 30), function(i) w1$appendText(i))
 l2 <- gtkLabel("Show diagnostic ")
 w2 <- gtkVBox()
 rb <- list()
 rb[["t"]] <- gtkRadioButton(label="t-statistic")
 for(i in c("mean","median")) rb[[i]] <- gtkRadioButton(rb, label=i)
 QT <- sapply(rb, function(i) w2$packStart(i))
 w3 <- gtkButton("Run simulation")
\end{Sinput}
\end{Schunk}

The basic \code{AttachDeafults} method will cause the widgets to
expand when resized, which we want to control here. As such we use
\code{Attach}. To get the control's label to center align yet still
have some breathing room we set its
\code{xalign} and  \code{xpad} properties.
For the combobox we avoid using \qcode{expand} as otherwise it resizes
to fill the space allocated to the cell in the \code{y} direction.
\begin{Schunk}
\begin{Sinput}
 tbl$attach(l1, left.attach=0,1, top.attach=0,1, yoptions="fill")
 l1["xalign"] <- 1; l1["xpad"] <- 5
 tbl$attach(w1, left.attach=1,2, top.attach=0,1, xoptions="fill", yoptions="fill")
\end{Sinput}
\end{Schunk}

We use \qcode{expand} here to attach the radio group, so that it
expands to fill the space. The label has its \code{yalign} proporty
set, so that it stays at the top of the cell, not the middle.
\begin{Schunk}
\begin{Sinput}
 tbl$attach(l2, left.attach=0,1, top.attach=1,2, yoptions="fill")
 l2["xalign"] <- 1; l2['yalign'] <- 0; l2["xpad"] <- 4
 tbl$attach(w2, left.attach=1,2, top.attach=1,2, xoptions=c("expand", "fill"))
\end{Sinput}
\end{Schunk}
A separator with a bit of padding provides a visual distinction
between the controls and the button to initiate an action.
\begin{Schunk}
\begin{Sinput}
 tbl$attach(gtkHSeparator(),left.attach=0,2, top.attach=2,3, ypadding=10, yoptions="fill")
 tbl$attach(w3, left.attach=1,2, top.attach=3,4, xoptions="fill", yoptions="fill")
\end{Sinput}
\end{Schunk}
Finally, we use the \code{ShowAll} method so that it propogates to the combobox.
\begin{Schunk}
\begin{Sinput}
 w$showAll()                             # propogate to combo
\end{Sinput}
\end{Schunk}
\end{example}




\section{Drag and drop}
\label{sec:RGtk2:dnd}

%% ------------ Drag and Drop

\GTK\/ has mechanisms to provide drag and drop features to
widgets. Only widgets which can receive signals will work for drag and
drop, so to drag or drop on a label an event box must be used. To
setup drag and drop actions requires setting a widget to be a source
for a drag request, and setting a widget to be a target for a drop
action. We illustrate how to set up the dragging of a text value from
one widget to another. Much more complicated examples are possible,
but we do not pursue it here.

When a drag and drop is initiated, different types of data may be
transferred. \GTK\/ allows the user to specify a target type. Below,
we define target types for text, pixmap and arbitrary objects.

\begin{Schunk}
\begin{Sinput}
 ## info arguments -- application assigned
 TARGET.TYPE.TEXT   = 80                 # 
 TARGET.TYPE.PIXMAP = 81                 # 
 TARGET.TYPE.OBJECT = 82
 widgetTargetTypes = list(
 ## target -- string representing the drag type
 ## flag delimiting drag scope. 0 -- no limit
 ## info -- application assigned value to identify
 text = gtkTargetEntry("text/plain", 0, TARGET.TYPE.TEXT),
 pixmap = gtkTargetEntry("image/x-pixmap", 0, TARGET.TYPE.PIXMAP),
 object = gtkTargetEntry("text/plain", 0, TARGET.TYPE.OBJECT)
 )
\end{Sinput}
\end{Schunk}

To set a widget to be a drag source requires calling
\function{gtkDragSourceSet}. This function has arguments
\argument{object}{gtkDragSourceSet} for the widget we are making a
source, \argument{start.button.mask}{gtkDragSourceSet}  to specify
which mouse buttons can initiate the drag,
\argument{targets}{gtkDragSourceSet} to specify the target type, and
\argument{actions}{gtkDragSourceSet} to indicate which of the
\code{GdkDragAction} types is in effect, for instance \code{copy} or
\code{move}. 

When a widget is a drag source, it sends the drag data in response to
the \signal{drag-data-get} signal using a callback. The signature of
this callback is important, although we only use the \code{selection}
argument, which is assigned the text that will be the data passed to
the target widget. (Text, as we are passing text information.)

\begin{Schunk}
\begin{Sinput}
 widget <-  gtkButton("Drag me")
 w <- gtkWindow(); w$add(widget)
 QT <- 
   gtkDragSourceSet(widget,
                    start.button.mask=c("button1-mask", "button3-mask"),
                    targets=widgetTargetTypes[["text"]],
                    actions="copy") ## can also be any of GdkDragAction
 ID <- 
   gSignalConnect(widget, "drag-data-get", 
                  f=function(userData, widget, context, 
                    selection, targetType, eventTime) {
                    selection$setText(str="some value") 
                    return(TRUE)
                  },
                  user.data.first=TRUE,
                  data="")
\end{Sinput}
\end{Schunk}

To make an object a drop target, we call \function{gtkDragDestSet} on
the object with new argument \argument{flags}{gtkDragDestSet} for
specifying the actions \GTK\/ will perform when the widget is dropped
on. We use the value \qcode{all} for \qcode{motion}, \qcode{highlight}, and
\qcode{drop}. Additionally, values for
\argument{targets}{gtkDragDestSet} and
\argument{actions}{gtkDragDestSet} are given.

\begin{Schunk}
\begin{Sinput}
 widget <- gtkButton("Drop here")
 w <- gtkWindow(); w$add(widget)
 QT <- gtkDragDestSet(widget,
                      flags="all", 
                      targets=widgetTargetTypes[["text"]],
                      actions="copy"
                      )
\end{Sinput}
\end{Schunk}

When data is dropped, the widget emits the
\code{drag-data-received}. The data is passed through the
\code{selection} argument. The \code{context} argument is a
\code{gdkDragContext}. The \code{x} and \code{y} arguments are integer
valued and pass in the position in the widget where the drop
occurred. In the example below, we see that text data is passed to
this function in \code{raw} format, so it is converted with
\function{rawToChar}.

\begin{Schunk}
\begin{Sinput}
 ID <- 
   gSignalConnect(widget, "drag-data-received", 
                  f=function(userData, widget, context, x, y, 
                    selection, targetType, eventTime) {
                    dropdata <- selection$getText()
                    if(class(dropdata)[1] == "raw")
                      val <- paste(rawToChar(dropdata), sep="")
                    else
                      val <- paste(dropdata, sep="")
                    print(val) ## some action
                  },
                  user.data.first=TRUE,
                  data = "")
\end{Sinput}
\end{Schunk}

If the action was \code{"move"} then the source widget emits the
\code{drag-data-delete} signal, so that a callback can be defined to
handle the deletion of the data.



% XXX-- 
% Explain signature -- 
% context: \code{?gdkDragContext} 
% x: integer with position of drop -- relative to widget dropped on
% y: integer with position of drop -- relative to widget dropped on
%   -- see cairoDevice for example
% target: the target (form of the data) to retrieve, can be use with switch()
% eventTime:  an integer recording time of drop
% --XXX



\chapter{RGtk2: Widgets Using Models}
\label{sec:RGtk2:widgets-with-models}

Many widgets in \GTK\/ use the model, view, controller paradigm. While
many times this is in the background, for the widgets in this chapter
one needs to be aware of the usage. This framework adds a layer of
complexity, in exchange for more flexibility.  



\section{Text views and text buffers} %% text buffer
\label{sec:RGtk2:textviews}

Multiline text areas are displayed through  \constructor{gtkTextView}
instances. These provide a view of an accompanying
\code{gtkTextBuffer}, which is the model that stores the text and
other objects to be rendered. The view is responsible for the display
of the text in the buffer, so has methods for adjusting tabs, margins,
indenting, etc. While the view stores the text so has methods for
adding and manipulating the text.

A text view is created with \constructor{gtkTextView}. The
\argument{buffer}{gtkTextView} argument is used to specify a text
buffer, otherwise one will be created. 
This buffer is returned by the
method \method{getBuffer}{gtkTextView} and may be set for a view with
the \method{setBuffer}{gtkTextView} method. Text views are typically
placed inside a scrolled window, and since a viewport is established,
this is done with the \method{add}{gtkBin} method for scrolled
windows.

%% simple use -- replace, Append, insert at cursor

Text may be added programmatically through various 
methods of the text buffer. The easiest to use are \method{setText}{gtkTextBuffer} which
simply replaces the current text with that specified by
\argument{text}{gtkTextBufferSetText}. The method
\method{insertAtCursor}{gtkTextBuffer} will add the text to the buffer
at the current position of the cursor. Other means are described after
the first example.

%% properties: editable, ...
\paragraph{Properties}
Key properties of the text view include \code{editable}, which if assigned
a value of \code{FALSE} will prevent users from editing the text. 
If the view is not editable, the cursor may be hidden by setting the
\code{cursor-visible} property to \code{FALSE}. 
The text in a buffer may be wrapped or not. The method \method{setWrapMode}{gtkTextView}
takes values from \code{GtkWrapMode} with default of \qcode{none}, but
options for \qcode{char}, \qcode{word}, or \qcode{word\_char}. The
justification for the entire buffer is controlled by the
\code{justification} property which takes values of \qcode{left},
\qcode{right}, \qcode{center}, or \qcode{fill} from \code{GtkJustification}.
The global value may be overwridden for parts of the text buffer
through the use of text tags. The left and right margins are adjusted
through the \code{left-margin} and \code{right-margin} properties. 

The text buffer has a few key properties, including \code{text} for
storing the text and \code{has-selection} to indicate if text is
currently selected in a view. The buffer also tracks if it has been
modified. This information is available through the buffer's
\method{getModified}{gtkTextBuffer} method, which returns \code{TRUE}
if the buffer has changes. The method
\method{setModifed}{gtkTextBuffer}, if given a value of \code{FALSE},
allows the programmer to change this state, say after saving a
buffer's contents.


%% buffer - wide font
\paragraph{Fonts}
The size and font can be globally set for a text view using the
\method{modifyFont}{gtkWidget} method. (Specifying fonts for parts of
the buffer requires the use of tags, described later.) The argument
\argument{font.desc}{gtkWidgetModifyFont} specifies the new font using
a Pango font description. These descriptions may be generated from a
string specifying the font through the function
\code{pangoFontDescriptionFromString}. These strings may contain up to
3 parts: the first is a comma-separated list of font families, the
second a white-space separated list of style options, and the third a
size in points or pixels if the units ``px'' are included. A typical
value might look like \code{"serif, monospace bold italic condensed
  16"}. The various style options are enumerated in \code{PangoStyle},
\code{PangoVariant}, \code{PangoWeight}, \code{PangoStretch}, and
\code{PangoGravity}. The help page for \code{PangoFontDescription}
contains more information.

\paragraph{Signals}
The text buffer emits many different types of signals. Most
importantly, the \signal{changed} signal is emitted when the content
of the buffer changes. The callback for a \signal{changed} signal has
signature that returns the text buffer and any user data. Other
signals are detailed in the help page for \function{gtkTextBuffer}.


\begin{example}{Simple textview usage}{eg:RGtk2:simple-textview-usage}
We illustrate the basics of using a text view, including setting some
of the view's properties.
\begin{Schunk}
\begin{Sinput}
 tv <- gtkTextView()
 sw <- gtkScrolledWindow()
 sw$setPolicy("automatic","automatic")
 sw$add(tv)
 w <- gtkWindow(); w$add(sw)
 tv['editable'] <- TRUE                  
 tv['cursor-visible'] <- TRUE            
 tv['wrap-mode'] <- "word"               # GtkWrapMode value
 tv['justification'] <- "left"           # GtkJustification value
 tv['left-margin'] <- 20                 # 0 is default
 tb <- tv$getBuffer()                    
 tb$setText("the quick brown fox jumped over the lazy dog")
 font <- pangoFontDescriptionFromString("Serif,monospace bold italic 8")
 tv$modifyFont(font)
\end{Sinput}
\end{Schunk}
\end{example}

\subsection{Tags, iterators, marks}
\label{sec:RGtk2:tags-iterators-marks}

In order to do more with a text buffer, such as retrieve the text, 
retrieve a selection, or modify attributes of just some of the text,
one needs to become familiar with how pieces of the buffer are
referred to within \pkg{Rgtk2}.

There are two methods: text iterators (iters) are a transient means to
mark begin and end boundaries within a buffer, whereas text marks
specify a location that remains when a buffer is modified. One can use
these with tags to modify attributes of pieces of the buffer.


\paragraph{Iterators}

%% text iters. GetStartIterm GetEndIter, GetSelectionBounds
%% transient
A text iterator is used to specify a position in a buffer in a transient
manner. Iterators become invalid as soon as a buffer changes.  In
\pkg{RGtk2}, iterators are stored as lists with components \code{iter}
to hold a pointer to the underlying iterator and component
\code{retval} to indicate whether the iterator when it was returned is
valid. Many methods of the text buffer will update the iterator. This
can happen inside a function call where the iterator is passed as an
argument, so is unexpected to the \R\/ user. The
\method{copy}{gtkTextIter} method will create a copy of an iterator,
in case one is to be modifed yet the original is desired.

%% methods to return an iterator
Several methods of the text buffer return iterators marking positions in the buffer. 
The beginning and end of the
buffer are returned by the methods
\method{getStartIter}{gtkTextBuffer} and
\method{getEndIter}{gtkTextBuffer}. Both of these iters are
returned at once by the method \method{getBounds}{gtkTextBuffer} again
as components of a list, in this case \code{start} and \code{end}.
The current selection is returned by the method method
\method{getSelectionBounds}{gtkTextBuffer}. Again, as a list of
iterators specifying the start and end positions of the current
selection. If there is no selection, then the component \code{retval}
will be \code{FALSE}, otherwise it is \code{TRUE}.  

The method \method{getIterAtLine}{gtkTextBuffer} will return an
iterator pointing to the start of the line, which is specifed by line
number starting with 0. The method \method{getIterAtLineOffset}{gtkTextBuffer} has an
additional argument to specify the offset for a given line. An offset
counts the number of individual characters and keeps track of the fact
that the text encoding, UTF-8, may use more than one byte per
character.  In addition to the text buffer, a text view also has the
method \method{getIterAtLocation}{gtkTreeView} to return the iterator
indicating the between-word space in the buffer closest to the point
specified in $x$-$y$ coordinates.

%% iter methods
There are several methods for iterators that allow one to refer to
positions in the buffer relative to the iterator, for example, these
with obvious names to move a character or characters:
\method{forwardChar}{gtkTextIter}, \method{forwardChars}{gtkTextIter},
\method{backwardChar}{gtkTextIter}, and
\method{backwordChars}{gtkTextIter}. As well, there are methods to
move to the end or beginning of the word the iterator is in or the end
or beginning of the sentence (\method{forwardWordEnd}{gtkTextIter},
\method{backwardWordStart}{gtkTextIter},
\method{backwardSentenceStart}{gtkTextIter}, and
\method{forwardSentenceEnd}{gtkTextIter}).  There are also various methods,
such as \method{insideWord}{gtkTextIter}, returning logical values
indicating if the condition is met.
To use these methods, the iterator in the
\code{iter} component is used, not the value returned as a
list. Example~\ref{eg-RGtk2-find-word} shows how some of the above are
used, in particular how these methods update the iterator rather than
return a new one.

%% using iters: Insert, Delte, GetText
\paragraph{Modifying the buffer}
Iterators are specified as arguments to several methods to set and
retrieve text. The \method{insert}{gtkTextBuffer} method will insert
text at a specified iterator. The argument
\argument{len}{gtkTextBufferInsert} specifies how many bytes of the
\code{text} argument are to be inserted. The default value of $-1$
will insert the entire text. This
method, by default, will also update the iterator to indicate the end
of where the text is inserted. The \method{delete}{gtkTextBuffer}
method will delete the text between the iterators specified to the
arguments \argument{start}{gtkTextBufferDelete} and
\argument{start}{gtkTextBufferDelete}. The
\method{getText}{gtkTextBuffer} will get the text between the
specified \argument{start}{gtkTextBufferDelete} and
\argument{end}{gtkTextBufferDelete} iters. A similar method
\method{getSlice}{gtkTextBuffer} will also do this, only it includes
offsets to indicate the presence of images and widgets in the text
buffer.

\begin{example}{Finding the word one clicks on}{eg:RGtk2-find-word}
This example shows how one can find the iterator corresponding to a
mouse-button-press event. The callback has an event argument which is
a \class{GdkEventButton} object with methods
\method{getX}{GdkEventButton} and \method{getY}{GdkEventButton} to
extract the \code{x} and \code{y} components of the event
object. These give the position relative to the widget. The methods
\method{getXRoot}{GdkEventButton} and
\method{getYRoot}{GdkEventButton} give the position relative to the
parent window the widget resides in.


\begin{Schunk}
\begin{Sinput}
 ID <- gSignalConnect(tv, "button-press-event", f=function(obj, e, data) {
   siter <- obj$getIterAtLocation(e$getX(), e$getY())$iter
   niter <- siter$copy()
   siter$backwardWordStart()
   niter$forwardWordEnd()
   val <- obj$getBuffer()$getText(siter, niter)
   print(val)                            # replace
   return(FALSE)                         # call next handler
   })
\end{Sinput}
\end{Schunk}
\end{example}


\paragraph{Marks}
In addition to iterators, \GTK\/ provides marks to indicate positions
in the buffer that persist through changes. For instance, the mark
\qcode{insert} always refers to the position of the cursor. Marks have a
gravity of \qcode{left} or \qcode{right}, with \qcode{right} being the
default. When the text surrounding a mark is deleted, if the gravity
is \qcode{right} the mark will remain to the right af any added
text. 


Marks can be defined in two steps by calling \function{gtkTextMark},
specifying and name and a value for the gravity, and then 
positioned within a buffer, specified by an iterator, through the
buffers \method{addMark}{gtkTextBuffer} method. The
\method{createMark}{gtkTextBuffer} method combines the two steps.

There are many text buffer methods to work with marks. The
\method{getMark}{gtkTextBuffer} method will return the mark object for
a given name. (There are functions which refer to the name of a mark,
and others requiring the mark object.) The method \method{getIterAtMark}{gtkTextBuffer} will return
an iterator for the given mark to be used when an iterator is needed.



%% tag Table
\paragraph{Tags}
Marks and iterators can be used to specify different properties for
different parts of the text buffer. \GTK\/ uses tags to specify how
pieces of text will differ from those of the textview overall. To
create a tag, the \method{createTag}{gtkTextBuffer} method is
used. This has optional argument
\code{tag.name}{gtkTextBufferCreateTag} which can be used to refer to
the tag later, and otherwise uses named arguments so specify a properties
names and the corresponding values. These tags may be applied to the text between
two iters using the methods \method{applyTag}{gtkTextBuffer} or
\method{applyTagByName}{gtkTextBuffer}.

\begin{example}{Using text tags}{eg:RGtk2:using-text-tags}
We define two text tags to make text bold or italic and illustrate how
to apply them.
\begin{Schunk}
\begin{Sinput}
 tv <- gtkTextView()
 tb <- tv$getBuffer()
 tb$setText("the quick brown fox jumped over the lazy dog")
 ##
 tag.b <- tb$createTag(tag.name="bold", weight=PangoWeight["bold"])
 tag.em <- tb$createTag(tag.name="em", style=PangoStyle["italic"])
 tag.large <- tb$createTag(tag.name="large", font="Serif normal 18")
 ##
 iter <- tb$getBounds()                  # or get iters another way
 tb$applyTag(tag.b, iter$start, iter$end) # iters updated
 tb$applyTagByName("em", iter$start, iter$end)
\end{Sinput}
\end{Schunk}
\end{example}


\paragraph{Interacting with the clipboard}
\GTK\/ can create clipboards and provides convenient access to the
default clipboard so that the standard cut, copy and paste actions can
be implemented. The function \function{gtkClipboardGet} returns the
default clipboard if given no arguments. The clipboard is the lone argument for
the method \method{copyCliboard}{gtkTextBuffer} to copy the current
selection to the clipboard. The method
\method{cutClipboard}{gtkTextBuffer} has an extra argmument,
\code{default.editable}, which is typically \code{TRUE}. To paste the
clipboard contents into the buffer, the second argument should be an
iterator specifying where the text should be inserted and the third
argument \code{TRUE} if the text is to be editable.

\begin{example}{A simple command line interface}{eg:RGtk2-command-line}
This example shows how the text view widget can be used to make a
simple command line. While programming a command line isn't likely to
be the most common task in designing a GUI for a statistics
application (presumably you are already using a good one), the example
is familiar and shows several different, but useful, aspects
of the widget.

\begin{figure}
  \centering
  \includegraphics[width=.6\textwidth]{ex-RGtk2-terminal}
  \caption{A basic \R\/ terminal implemented using a \code{gtkTextView} widget.}
  \label{fig:RGtk2-terminal}
\end{figure}


We begin by defining our text view widget and retrieving its
buffer. We also specify a fixed-width font for the buffer.
\begin{Schunk}
\begin{Sinput}
 tv <- gtkTextView()
 tb <- tv$getBuffer()
 font <- pangoFontDescriptionFromString("Monospace")
 tv$modifyFont(font)                     # widget wide
\end{Sinput}
\end{Schunk}

Our main object will be the text buffer which will have only one
view. As there is no built-in method to return a corresponding view
from the buffer, we use the \method{setData}{gtkWidget} method to
associate the view with the buffer.
\begin{Schunk}
\begin{Sinput}
 tb$setData("textview", tv)
\end{Sinput}
\end{Schunk}

We will use a few formatting tags, defined next. We don't need the tag
objects, as we refer to them later by name.
\begin{Schunk}
\begin{Sinput}
 aTag <- tb$createTag(tag.name="cmdInput")
 aTag <- tb$createTag(tag.name="cmdOutput", 
                      weight=PangoWeight["bold"])
 aTag <- tb$createTag(tag.name="cmdError", 
                      weight=PangoStyle["italic"], foreground="red")
 aTag <- tb$createTag(tag.name="uneditable", editable=FALSE)
\end{Sinput}
\end{Schunk}

We define one new mark to mark the prompt for a new line. We
need to be able to identify a new command, and this marks the
beginning of this command.
\begin{Schunk}
\begin{Sinput}
 startCmd <- gtkTextMark("startCmd", left.gravity=TRUE)
 tb$addMark(startCmd, tb$getStartIter()$iter)
\end{Sinput}
\end{Schunk}

We define several functions, which we think of as methods of the text
buffer (not the text view). This first shows how to move the viewport
so that the command line is visibile.
\begin{Schunk}
\begin{Sinput}
 moveViewport <- function(obj) {
   tv <- obj$getData("textview")
   endIter <- obj$getEndIter()
   QT <- tv$scrollToIter(endIter$iter, 0)
 }
\end{Sinput}
\end{Schunk}

There are two types of prompts needed. This function adds a new one or a
continuation one. An argument allows one to specify that the
\code{startCmd} mark is set.
\begin{Schunk}
\begin{Sinput}
 addPrompt <- function(obj, prompt=c("prompt","continue"), 
                       setMark=TRUE) {
   prompt <- match.arg(prompt)
   prompt <- getOption(prompt)
   
   endIter <- obj$getEndIter()
   obj$insert(endIter$iter, prompt)
   tv <- obj$getData("textview")
   if(setMark)
     obj$moveMarkByName("startCmd", endIter$iter)
 }
 addPrompt(tb) ## place an initial prompt
\end{Sinput}
\end{Schunk}

This helper method is used to write the output of a command to the
text buffer. We arrange to truncate large outputs. By passing in the
tag name, we could reuse this function. If we were to streamline the
code for this example, we might use this function to also write out the error
messages, but leave that to the similarly defined function
\code{addErrorMessage} (not shown).


\begin{Schunk}
\begin{Sinput}
 addOutput <- function(obj, output, tagName="cmdOutput") {
   if(length(output) > 100)              # shorten if needed
     out <- c(output[1:100], "...")
 
   endIter <- obj$getEndIter()
   if(length(output) > 0) 
     sapply(output, function(i)  {
       obj$insertWithTagsByName(endIter$iter, i, tagName)
       obj$insert(endIter$iter, "\n", len=-1)
     })
   
   addPrompt(obj, "prompt", setMark=TRUE)
   obj$applyTagByName("uneditable", obj$getStartIter()$iter, 
                      obj$getEndIter()$iter)
   moveViewport(obj)
 }
\end{Sinput}
\end{Schunk}


This next function uses the \code{startCmd} mark and the end of the buffer
to extract the current command. Multi-line commands are handled through
a regular expression which should not be hard-coded to the standard
continue prompt, but for sake of simplicity is.
\begin{Schunk}
\begin{Sinput}
 findCMD <- function(obj) {
   endIter <- obj$getEndIter()
   startIter <- obj$getIterAtMark(startCmd)
   cmd <- obj$getText(startIter$iter, endIter$iter, TRUE)
 
   cmd <- unlist(strsplit(cmd, "\n[+] ")) # hardcoded "+"
   cmd
 }
\end{Sinput}
\end{Schunk}

The following function takes the current command and does the
appropriate thing. It uses a hack (involving \code{grep}) to
distinguish between an incomplete command and a true syntax error. The
\code{addHistory} call refers to a function that is not shown, but is
left to illustrate where one would add to a history stack if desired.

\begin{Schunk}
\begin{Sinput}
 evalCMD <- function(obj, cmd) {
   cmd <- paste(cmd, sep="\n")
   out <- try(parse(text=cmd), silent=TRUE)
   if(inherits(out, "try-error")) {
     if(length(grep("end", out))) {      # unexpected end of input
       ## continue
       addPrompt(obj, "continue", setMark=FALSE)
       moveViewport(obj)
     } else {
       ## error
       addErrorMessage(obj, out)
     }
     return()
   }
   addHistory(obj, cmd)  ## if keeping track of history
   
   out <- capture.output(eval(parse(text = cmd), envir=.GlobalEnv))
   addOutput(obj, out)
 }
\end{Sinput}
\end{Schunk}

The \code{evalCMD} command is called when the \kbd{return} key is
pressed. The \signal{key-release-event} signal passes the event
information through to the second argument. We inspect the key value
and compare to that of the return key. 

\begin{Schunk}
\begin{Sinput}
 ID <- gSignalConnect(tv, "key-release-event", f=function(w, e, data) {
   obj <- w$getBuffer()                  # w is textview
   keyval <- e$getKeyval()
   if(keyval == GDK_Return) {
     cmd <- findCMD(obj)
     if(nchar(cmd) > 0)
       evalCMD(obj, cmd)
   }
   return(FALSE)                         # events need return value
 })
\end{Sinput}
\end{Schunk}
Figure~\ref{fig:RGtk2-terminal} shows the widget placed into a very
simple GUI.



\end{example}


% \begin{example}{Adding parentheses highlighting}{eg:RGtk2:paren-highlight}
%   \SweaveInput{ex-RGtk2-match-parentheses}
% \end{example}

%% insert images or widgets
\paragraph{Inserting non-text items}
If desired, one can insert images and/or widgets into a text buffer,
although this isn't a common use within statistical GUIs. The method
\method{insertPixbuf}{gtkTextBuffer} will insert into a position
specified by an iter a \class{GdkPixbuf} object. In the buffer, this
will take up one character, but will not be returned by
\method{getText}{gtkTextBuffer}. 

Arbitrary child components can also be inserted. To do so an anchor
must first be created in the text buffer. The method
\method{createChildAnchor}{gtkTextBuffer} will return such an anchor, and
then the text view method \method{addChildAtAnchor}{gtkTextView} can
be used to add the child.



\section{Views of tabular and heirarchical data}
\label{sec:RGtk2:tabular-heirarchical-data}

Widgets to create comboboxes, display tabular data values, and to
display tree-like data are treated similarly in \GTK. Each uses the
MVC paradigm and for these, the models are defined similarly.
We begin by discussing the models, then present the various
views. Each view is described by its column, which in turn have their
cells specifed by cell renderers.


\subsection{Tabular  stores and tree stores}
\label{sec:tabular-stores-tree}

\GTK\/ provides list stores and tree stores as models to hold tabular and
heirarchical data to be viewed through various widgets, such as the
combo box or tree view. Like a data frame, each row in these stores
contains data of varying types. The main difference between the two is
that tree stores also have information about about whether a row has
any offspring. The list store is just a tree store where there
are no children of the top-level offspring.

For speed, much greater convenience and familiarity purposes, \pkg{RGtk2} provides a third store
through \constructor{rGtkDataFrame} for storing data frames.

\paragraph{rGtkDataFrame}

\R\/ uses data frames to hold tabular data, where each column is of a
certain class, and each row is related to some observational
unit. This is also the way tree views are organized when no
heirarchical structure is needed. As such it is natural to have a
means to map a data frame into a store for a tree view. The
\constructor{rGtkDataFrame} constructor does this, producing an object
that can be used as the model for a view. This \R-specific
addition to \GTK\/ not only is more convenient, it has the added bonus
of being especially fast. The constructor takes a data frame as an
argument. The column classes are important, so even if this data frame
is empty, it should specify the desired column classes.

The constructor produces an object of class \class{RGtkDataFrame}
which has a few methods defined for it. The familiar S3 methods
\method{[}{RGtkDataFrame}, \method{[\ASSIGN}{RGtkDataFrame},
\method{dim}{RGtkDataFrame}, and \method{as.data.frame}{RGtkDataFrame}
are defined. The \code{dimnames} attributes are kept, but have no
well-defined meaning
for this model. The \code{[$<$-} method does not have quite the same
functionality, as it does for a data frame. Columns can not be removed
by assigning values to \code{NULL}, column types
should not be changed which can be an issue with coercion to character from numeric
say, rows can not be dropped. To add a new column or row, the methods
\method{appendColumns}{rGtkDataFrame} and
\method{appendRows}{rGtkDataFrame} may be used, where the new column
or row may be given as the argument.

To remove rows from this model, the \method{setFrame}{rGktDataFrame}
method can be used to specify the new data. This method can also be used
to replace the existing data in the model with a new data
frame. There are few issues though. If the new data frame has more
rows or columns, then the appropriate \code{append} method should be
used first. As well, one should not change the column classes with the
new frame, as views of the model may be expecting a certain class of data.

\begin{example}{Defining and manipulating a data store}{eg-RGtk2-manipulate-rGtkDataframe}
  The basic data frame methods are similar.
\begin{Schunk}
\begin{Sinput}
 data(Cars93, package="MASS")            # mix of classes
 model <- rGtkDataFrame(Cars93)
 model[1, 4] <- 12
 model[1, 4]                              # get value
\end{Sinput}
\begin{Soutput}
[1] 12
\end{Soutput}
\end{Schunk}
Factors are treated differently from character values, as is done with
data frames, so assignment to a factor must be from one of the
possible levels.

To change the backend data, we can use the \code{SetFrame} method:
\begin{Schunk}
\begin{Sinput}
 QT <- model$setFrame(Cars93[1:5, 1:5])
\end{Sinput}
\end{Schunk}
\end{example}


\paragraph{List stores and tree stores}
Although the \code{rGtkDataFrame} model is very useful, there are
times when it can't be employed. List stores can be used when the
underlying data contains values that can not be stored in a data frame
(such a images) and tree stores are used for heirarchical data. 

%% construction
A tree store or list store is constructed using
\constructor{gtkTreeStore} or  \constructor{gtkListStore}. Both are
interfaces for the abstract \class{GtkTreeModel} class. The
column types are specified through a character vector at the time of
construction. The specification uses ``GTypes'' such as
\code{gchararray} for character data, \code{gboolean} for logical data,
\code{gint} for integer data, \code{gdouble} for numeric data, and
\code{GObject} for \GTK\/ objects, such as pixbufs.

%% iters vs. paths
\paragraph{Iterators and tree paths}
Similar to a text buffer, a list store uses transient iterators to
refer to position -- in this case the row -- within a store. One can
also refer to position through a path, which for a list store is
essentially the row number, $0$-based, as a character; and for a tree
is a colon-separated set of values referring to the offspring
(\qcode{a:b:c} indicates the \code{c}th child of the \code{b}th child
of \code{a}).  A third way, through a row reference, is not discussed
here.

%% paths
A \class{GtkTreePath} object is created by the constructor
\code{gtkTreePathNewFromString} which takes a string specifying the
position. To retrieve this string from a path object, the
\method{toString}{gtkTreePath} method can be used.

%% iters
Paths are convenient, as they are human readable, but iterators are
empolyed by the various methods and more easily allow the programmer
to traverse the store. One can flip between the two
representations. The iterator referring to the path can be returned by
the method \method{getIterFromString}{gtkTreeModel}. The method
\method{getStringFromIter}{gtkTreeModel} will return the string. The
tree path object itself is returned by the method
\method{getPath}{gtkTreeModel}.  In \pkg{RGtk2} iterators are lists
with component \code{retval} indicating if this is a valid iterator
and a component \code{iter} holding the object of the
\class{GtkTreeIter} class.

%% adding to a storex
\paragraph{Adding values to a store}
Values are added to and returned from a store by specifying the row
and column for the value. The row is specified by an iterator and the
columns by its index, $0$-based.  The method
\method{setValue}{gtkListStore} is used to specify value by value,
whereas an entire row can be assigned through the
\method{set}{gtkTreeStore} method. The former has arguments
\code{iter}, \code{column}, \code{value}, in that order; the latter
has no \code{column} or \code{value} argument. Instead, \code{Set}
uses positional arguments to specify the column and the value. The
column index appears as an even argument (say $2k$) and the
corresponding value in the odd argument (say $2k+1$).  When calling
\code{setValue} or \code{set} the iterator updates to the next row
Values are returned by the \method{getValue}{gtkListStore} method,
a list with component \code{value} storing the value.

\paragraph{Finding iterators}
For a list or tree model, an iterator for the first child is returned
by \method{getIterFirst}{gtkTreeStore}. This iterator corresponds to
the path \qcode{0}.
The \method{append}{gtkStore}
method for the store returns an iterator indicating the next value at
the end of the store (this is slightly different from the \GTK\/ function which
modifies an iterator passed as an argument).  The
\method{prepend}{gtkTreeStore} method is similar, only returning an
iterator pointing to the initial row.  Other methods allow for
specifying postion relative to some row. The
\method{insert}{gtkListStore} method is used to return an iterator
that allows one to insert a row at a position specified to its
\code{position} argument. (The \code{Prepend} method is similar to
using \code{postion=0}). To avoid the two-step approach of getting the
iterator, then assigning the value, the method
\method{insertWithValues}{gtkListStore} can be used, where values are
specified as with the \code{Set} method.  The
\method{insertBefore}{gtkListStore}, and
\method{insertAfter}{gtkListStore}, methods take an iterator,
\code{sibling} and will return an iterator indicating the postion just
before or after the sibling.




\begin{example}{Appending to a list store}{eg:RGtk2:list-store}
  To illustrate, to create a simple list store to hold a column
of text we have:  
\begin{Schunk}
\begin{Sinput}
 lstore <- gtkListStore("gchararray")
 for(i in 1:nrow(Cars93)) {
   iter <- lstore$append()
   if(is.null(iter$retval)) 
     lstore$setValue(iter$iter, 0, Cars93[i,1])
 }
\end{Sinput}
\end{Schunk}
To retrieve a value, we have this example to get the first one in the store:
\begin{Schunk}
\begin{Sinput}
 iter <- lstore$getIterFirst()           # first row
 lstore$getValue(iter$iter, column = 0)
\end{Sinput}
\begin{Soutput}
$retval
NULL

$value
[1] "Acura"
\end{Soutput}
\end{Schunk}
\end{example}

\paragraph{Adding heirarchical information}

For a tree store, the methods \method{append}{gtkTreeStore},
\method{prepend}{gtkTreeStore} etc. are similar to that for a list
store with the difference being that a
\argument{parent}{gtkTreeStoreAppend} argument is used for tree stores
to specify in iterator for the parent of the new row, thereby creating
the heirarchical structure of a tree.


\begin{example}{Defining a tree}{eg:RGtk2:tree-store}
  As an application, we can create a tree with parents the car
  manufacturers in the \code{Cars93} data set, and children the makes
  of their cars, as follows:
\begin{Schunk}
\begin{Sinput}
 tstore <- gtkTreeStore("gchararray")
 Manufacturers <- Cars93$Manufacturer
 Makes <- split(Cars93[,"Model"], Manufacturers)
 for(i in unique(Manufacturers)) {
   piter <- tstore$append()              # parent
   tstore$setValue(piter$iter, column=0, value=i)
   for(j in Makes[[i]]) { 
     sibiter <- tstore$append(parent=piter$iter) # child
     if(is.null(sibiter$retval)) 
       tstore$setValue(sibiter$iter,column=0, value=j)
   }
 }
\end{Sinput}
\end{Schunk}
To retrieve a value from the tree store using its path we have:
\begin{Schunk}
\begin{Sinput}
 iter <- tstore$getIterFromString("0:0") #  the 1st child of root
 tstore$getValue(iter$iter,column=0)$value
\end{Sinput}
\begin{Soutput}
[1] "Integra"
\end{Soutput}
\end{Schunk}
\end{example}


%% Manipulating rows
\paragraph{Manipulating rows}
Rows within a store can be rearranged using the methods
\method{swap}{gtkTreeStore} to swap rows referenced by their
iterators; \method{moveAfter}{gtkTreeStore} to move one row after
another, both referenced by iterators, although if the last is blank,
the end of the store is assumed; and
\method{moveBefore}{gtkTreeStore}, where if the second iterator is
blank the first position is assumed. To totally reorder the store, the
\method{reorder}{gtkTreeStore} method is available. Its
\argument{new.order}{gtkTreeStore} argument specifies the new
order as row indices. For tree stores, these rows are the children of the
\argument{parent}{gtkTreeStoreReorder} argument.

%% clearing contents
Once added, rows may be removed using the
\method{remove}{gtkTreeStore} method. The iterator for the row to
delete is given as an argument. The store's entire contents can be
removed by its \method{clear}{gtkTreeStore} method.


%% Traversing
\paragraph{Traversing the store}
An iterator points to a row in a tree or list store. For both lists
and trees, an interator pointing to the next row (at the same level
for trees) is produced by the method
\method{iterNext}{gtkTreeStore}. This method returns \code{FALSE} if
no next row exists. Otherwise, it updates the iterator in place. (That
is, calling \command{store\$iterNext(iter\$iter)} updates \code{iter},
despite it not being assigned to.) The path method \method{prev}{gtkTreePath}
will point to the previous child at the same depth in a tree, but no
such method is defined for iterators. One could be, for example the
following will do so for both list and tree stores.

\begin{Schunk}
\begin{Sinput}
 gtkTreeModelIterPrev <- function(object, iter) {
   path <- object$getPath(iter)
   ret <- path$prev()
   if(ret)
     return(list(retval=NULL, iter=object$getIter(path)$iter))
   else
     return(list(retval=FALSE,iter=NA))
 }
\end{Sinput}
\end{Schunk}

For trees, the method \method{iterParent}{gtkTreeModel} returns an
iterator to point to the parent row, if no parent is found. The
\code{retval} component is \code{FALSE}.
There are several methods when a row has children.  The
method \method{iterHasChild}{gtkTreeModel} returns a logical
indicating if a row has children. The method
\method{iterChildren}{gtkTreeModel} returns an iterator to point to
the first child of parent. If no child exists, the \code{retval}
component is \code{FALSE}. If an iterator for the $n$th child is desired, the
method \method{iterNthChild}{gtkTreeModel} can be used. Again, it returns
an iterator referring to the $n$th child, or has a \code{retval} of
\code{FALSE} if
none exists. To find the number of children, the method
\method{iterNChildren}{gtkTreeModel} is provided. This method returns
$0$ if there are no children.

\begin{figure}
%%  \centering
  \includegraphics[width=.7\textwidth]{traverse-tree}
  \caption{[REPLACEME!] Graphical illustration of the functions used by iters to traverse a tree store. }
  \label{fig:traverse-iter}
\end{figure}

\subsection{Cell renderers}
\label{sec:RGtk2:cellrenderers}
The various views ultimately display the information in the model
column by column (a combobox having one column). Within each column,
the display is controlled by cell renderers, which are used to specify
how each cell is layed out. Cell renderers are used whenever the 
\code{gtkCellLayout} interface is implemented, such as with comboboxes
and tree views, but also other widgets not discussed here.

A cell renderer is customized by adjusting its attributes. These
attributes are documented in the help pages for the corresponding
constructor. These attributes can be set to one value for all rows, or
can be set to depend on a corresponding row in the model. The latter
allows them to change from cell to cell. For example, the \code{text}
attribute of the text cell renderer would usually get its values from
the model, as that would vary from cell to cell, but a background
color (\code{background}) might be common to the column. The
\method{addAttribute}{gtkCellLayout} method is used to associate a
column in the store with a cell renderer's attribute.

There are many different cell renderers, we mention first the text and
pixbuf renderer, as they are commonly used in comboboxes. With the
discussion of tree views, we mention others.

%% text/ numbers
\paragraph{Text cell renderers}
The \constructor{gtkCellRendererText} constructor is used to display
text and numeric values. Numeric values are shown as strings, but are
not converted in the model.  For the text renderer, important
properties are \code{text} to indicate the column in the data store
that the text for the cell is to come from, \code{font} to specify the
font from a string, \code{size} for the font size, \code{background}
for the background color and \code{foreground} for the text color (as
strings).

To display right-aligned text in a Helvetica font, the following could be used:
\begin{Schunk}
\begin{Sinput}
 cr <- gtkCellRendererText()
 cr['xalign'] <- 1                       # default 0.5 = centered
 cr['family'] <- "Helvetica"  
\end{Sinput}
\end{Schunk}
The \code{wrap} attribute can be specified as \code{TRUE}, if the
entries are expected to be long. There are several other attributes that can
changed. 

%% pixbuf
\paragraph{Pixbuf cell renderers}
Graphics can be added to the cell with the renderer
\constructor{gtkCellRendererPixbuf}. The graphic can be spepcified by
its \code{stock-id} attribute as a character string, or
\code{icon-name} for a themed icon. It can also be specifed as an
image object, through the \code{pixbuf} attribute. Pixbuf objects can
be placed in a list store using the \code{GObject} type. A simple use,
might be the following:
\begin{Schunk}
\begin{Sinput}
 cr <- gtkCellRendererPixbuf()
 cr['stock.id'] <- "gtk-ok" ## or from a column in a model
\end{Sinput}
\end{Schunk}


% <<CellRendererExample>>=
% cr <- gtkCellRendererText()
% cr['family'] <- "Helvetica"             # font family for whole column
% cr['background'] <- "goldenrod"         # background color for column
% crp <- gtkCellRendererPixbuf();         # a pixbuf
% crp['xalign'] <- 0
% @

% Cellrenderers define how data is to be displayed. The binding of the
% data to the cell renderer is handled by the view, not the
% cellrenderer.

\subsection{Combo boxes}
\label{sec:RGtk2:combobox}

The basic combo box usage was discussed in
Section~\ref{sec:RGtk2:basic-combobox}, here we discuss more
complicated comboboxes that use a model for the backend This data is
tabular and may be kept in a \function{rGtkDataFrame} instance or a
list store.  

The basic \constructor{gtkComboBoxEntryNewWithModel} constructor
allows one to specify the model, and a column where the values are
found. For this, the cell renderers (below) are not needed.

If some layout of the values in the combobox is desired, such as adding an image,
then the constructor
\constructor{gtkComboBox} is used. The model may be specified at the time of
construction through the optional \argument{model}{gtkComboBox}
argument.  This model may be changed or set through the
\method{setModel}{gtkComboBox} method and is returned by
\method{getModel}{gtkComboBox}.

The constructor \constructor{gtkComboBoxEntry} returns a combobox
widget that allows the user to add their own values. This constructor
does not allow the model to be specified, so the \code{SetModel}
method must be used. The editable combobox uses a \code{gtkEntry}
object, which can be accessed directly through the
\method{getChild}{gtkBin} method of the combobox.

\paragraph{Cellrenderers}
Comboboxes display rows of data, each row referred to as a cell. In \GTK\/ each cell is like a box
container and can show different bits of information, like an image or
text. Each bit of information is presented by a cellrenderer.
Cellrenderers are added to the combo box by its
\method{packStart}{gtkCellLayout} method. As with box containers, more
than one cell renderer can be added per row.

To specify the data from the model to be
displayed, the \method{addAttribute}{gtkCellLayout} method maps
columns of the model to attributes of the cellrenderer. 



%% get value from widget
\paragraph{Retrieving the selected value}
For a non-editable combobox, the selected value may be retrieved by
index or by iterator. The \method{getActive}{gtkComboBox} method
returns the index of the current selection, $0$-based. The value is $-1$
if no selection has been made. The
\method{getActiveIter}{gtkComboBox} method returns an iterator
pointing to the row in the data store. If no row has been selected,
the \code{retval} component of the iterator is \code{FALSE}. These may
be used with the data store to retrieve the value. The data store
itself is returned by the \method{getModel}{gtkComboBox} method.

To set the combobox to a certain index is done through the
\method{setActive}{gtkComboBox} method, using a $0$-based index to
specify the row.

For editable comboboxes, one can first get the entry widget then call
its \code{GetText} method. The \code{SetText} method of the entry
widget would be used to specify the text.


%% signals
\paragraph{Signals}
When a user selects a value with the mouse, the \code{changed} signal
is emitted. For editable combo boxes, the user may also make changes
by typing in the new value. The underlying widget is a \code{gtkEntry}
widget, so the signal \code{changed} is emitted each time the text is
changed and the signal \code{activate} is emitted by the
\code{gtkEntry} widget when the \kbd{enter} key is pressed. One binds to
the signal of the entry widget, not the combobox widget, to have a
callback for that event.


\begin{example}{Modifying the values in a combobox}{eg-RGtk2-combobox-dynamic}
This example shows how to use two comboboxes to achieve a useful
task. That being, allowing the user a means to select from the
available variables in a data frame. We use a \code{rGtkDataFrame}
model for each, but for one use the basic constructor and the other
the more involved, as an illustration.
\begin{Schunk}
\begin{Sinput}
 data("Cars93", package="MASS")
 dfNames <- c("mtcars", "Cars93")
 dfModel <- rGtkDataFrame(dfNames)
 dfCb <- gtkComboBoxEntryNewWithModel(dfModel, text.column=0)
\end{Sinput}
\end{Schunk}

The variable names are initially just an empty string. We use an
\code{rGtkDataFrame}  as the
model and also specify a cell renderer to view the data.

\begin{Schunk}
\begin{Sinput}
 variableNames <- character(0)
 varModel <- rGtkDataFrame(variableNames)
 varCb <- gtkComboBoxNewWithModel(varModel)
 cr <- gtkCellRendererText()
 varCb$packStart(cr)
 varCb$addAttribute(cr, "text", 0)
\end{Sinput}
\end{Schunk}

Our basic GUI uses a table for layout. Comboboxes fill and expand to fill the cell.
\begin{Schunk}
\begin{Sinput}
 tbl <- gtkTableNew(rows=2, columns=2, homogeneous=FALSE)
 tbl$attach(gtkLabel("Data frame"), left.attach=0,1, top.attach=0,1)
 tbl$attach(dfCb, left.attach=1,2, top.attach=0,1)
 tbl$attach(gtkLabel("Variables"), left.attach=0,1, top.attach=1,2)
 tbl$attach(varCb, left.attach=1,2, top.attach=1,2)
\end{Sinput}
\end{Schunk}
This callback will be used for both the entry widget and the combobox,
so we first check which it is and if it is the combobox, we get the
entry widget from it. To update the display we replace the model. The
option of replacing the frame within the current model requires us to
be careful when adding additional rows.

\begin{Schunk}
\begin{Sinput}
 newDfSelected <- function(varCb, w, ...) {
   if(inherits(w, "GtkComboBox"))        # get entry widget
     w <- w$getChild()
   val <- w$getText()
   df <- try(get(val, envir=.GlobalEnv), silent=TRUE)
   if(!inherits(df, "try-error") && is.data.frame(df)) {
     nms <- names(df)
     ## update model
     newModel <- rGtkDataFrame(nms)
     varCb$setModel(newModel)
     varCb$setActive(-1)
   }
 }
\end{Sinput}
\end{Schunk}
Our callbacks for the data frame combobox simply call the above
function. As for the variable combobox, we show how to get the
selected value, but for no real purpose.
\begin{Schunk}
\begin{Sinput}
 QT <- gSignalConnect(dfCb, "changed", f=newDfSelected,
                      user.data.first=TRUE,
                      data=varCb)
 QT <- gSignalConnect(dfCb$getChild(), "activate", f=newDfSelected,
                      user.data.first=TRUE,
                      data=varCb)
 QT <- gSignalConnect(varCb, "changed", f=function(w, ...) {
   model <- w$getModel()
   iter <- w$getActiveIter()
   val <- model$getValue(iter$iter, column=0)
   print(val$value)                      # add real purpose
 })
\end{Sinput}
\end{Schunk}
\end{example}

\begin{example}{A color selection widget}{eg:RGtk2:combobox}
%%
This examples shows how a combobox can be used as an alternative to
\function{gtkColorButton} to select a color. We use two cellrenderers
for each row, one to hold an image and the other a text label.


This function uses the \pkg{grid} package to produce a graphic that
will read into the pixbuf.

\begin{Schunk}
\begin{Sinput}
 require(Cairo)
 makePixbufFromColor <- function(color) {
   filename <- tempfile()
   Cairo(file=filename, width=25,height=10)
   grid.newpage()
   grid.draw(rectGrob(gp = gpar(fill = color)))
   dev.off()
   image <- gdkPixbufNewFromFile(filename)
   unlink(filename)
   return(image$retval)
 }
\end{Sinput}
\end{Schunk}

Our data store has one column for the pixbuf and one for the color
text. The pixbuf is stored using the \code{GObject} class.

\begin{Schunk}
\begin{Sinput}
 store <- gtkListStore(c("GObject","gchararray"))
\end{Sinput}
\end{Schunk}

This loop adds the colors and their name to the data store.
\begin{Schunk}
\begin{Sinput}
 theColors <- palette()                  # some colors
 for(i in theColors) {
   iter <- store$append()
   store$setValue(iter$iter, 0, makePixbufFromColor(i))
   store$setValue(iter$iter, 1, i)
 }
\end{Sinput}
\end{Schunk}

Next we define the combobox using the store as the model. There are
two cell renderers to add.
\begin{Schunk}
\begin{Sinput}
 combobox <- gtkComboBox(model=store)
 ## pixbuf
 crp <- gtkCellRendererPixbuf(); crp['xalign'] <- 0
 combobox$packStart(crp, expand=FALSE)                
 combobox$addAttribute(crp, "pixbuf", 0)
 ## text
 crt <- gtkCellRendererText(); 
 crt['xpad'] <- 5                        # give some space
 combobox$packStart(crt)
 combobox$addAttribute(crt, "text", 1)
\end{Sinput}
\end{Schunk}


% This shows how the method \method{GetActiveIter}{gtkComboBox}
% indicates the selected value, which can be used by the
% \method{GetValue}{gtkTreeStore} method of the  data store to retrieve
% the value.
% <<changedsignal>>=
% ID <- gSignalConnect(combobox, "changed",
%               f = function(cb, data) {
%                 store <- cb$getModel()
%                 iter <- cb$getActiveIter()
%                 if(iter$retval) {
%                   val <- store$getValue(iter$iter,1)$value 
%                   print(val)
%                 }
%                 return(TRUE)
                
%               })
% @ 
% \end{example}



% \begin{example}{Editable combo box}{eg:RGtk2:editable combo box}

%   This example is similar to the previous one, except it adds the
%   ability to edit the value in the combobox.

%   The \argument{text.column}{gtkComboBoxEntryNewWithModel} argument in
%   the constructor specifies which column in the data store contains
%   the text. We do not need a cell renderer to display the text.

% <<>>=
% comboentry <- gtkComboBoxEntryNewWithModel(store, text.column = 1)
% @ 

% To draw the image next to the text is similar to before.

% <<>>=
% crp <- gtkCellRendererPixbuf(); crp['xalign'] <- 0
% comboentry$packStart(crp, expand=FALSE)                 # icon first
% comboentry$addAttribute(crp, "pixbuf", 0)
% @ 

% We need to call the \method{Show}{gtkWidget} method for this widget to
% be visible.
% <<>>=
% comboentry$show()
% @ 

% Again, we simply add the widget to a top-level window.
% <<>>=
% win <- gtkWindow();win$setTitle("Combo box with entry")
% win$add(comboentry)
% @ 


% These two callbacks show how to attach a callback to changes due to
% either selecting a value from the available values or pressing
% \kbd{enter} after entering a value.

% <<>>=

% ID <- gSignalConnect(comboentry, "changed", # changed via popup
%                f = function(cb, data) {
%                  if(cb$getActive() != -1)
%                    print(cb$getChild()$getText())
%                  return(TRUE)
%                })
% ## just enter will call handler. Widget is gtkEntry instance
% ID <- gSignalConnect(comboentry$getChild(), "activate", # on entry
%               f = function(w, data) {
%                   print(w$getText())
%                 return(TRUE)
%               })

% @ 
\end{example}


\subsection{Text entry widgets with completion}
\label{sec:RGtk2:entry-completion}

A common alternative to a combobox, implemented on many websites, is to add the completion features to a
\constructor{gtkEntry} instance. When a user types a partial match,
all available matches are offered to select from. To implement completion, one creates
a completion object with the constructor
\constructor{gtkEntryCompletion}. The values to complete from are
stored in a model, the example uses an \code{rGtkDataFrame} instance,
which is assigned to the completion object through its
\method{setModel}{gtkEntryCompletion} method. To set the completion
for the entry widget, the entry widget's
\method{setCompletion}{gtkEntry} method is used. The
\code{text-column} property is used to specify which column in
the model is used to find the matches, we use the method
\code{SetTextColumn} in the example to set this.


There are several properties that can be adjusted to tailor the
completion feature, we metion some of them. Setting the property
\code{inline-selection} to \code{TRUE} will place the completion
suggestion to the entry inline as the completions are scrolled
through; \code{inline-completion} will add the common prefix
automatically to the entry widget; \code{popup-single-match} is a
logical indicating if a popup is displayed on a single match;
\code{minimum-key-length} takes an integer specifying the number of
characters needed in the entry before completion is checked, the
default is $1$.

By default, the rows in the data model that match the
current value of the entry widget in a case insensitive manner are displayed. This
matching function can be overridden by setting a new function through
the \method{setMatchFunc}{gtkEntryCompletion} method. The signature of
this function is the completion object, the string from the entry
widget (lower case), an interator pointing to a row in the model and optionally
user data that is passed through the \code{func.data} argument of the
\code{SetMatchFunc} method. This method should return \code{TRUE} or
\code{FALSE} depending on whether that row should be displayed in the
set of completions.


\begin{example}{Text entry with completion}{eg:RGtk2:text-entry-comletion}
This example illustrates the basic steps to add completion to a text entry.


The two basic widgets are defined as follows:
\begin{Schunk}
\begin{Sinput}
 entry <- gtkEntry()
 completion <- gtkEntryCompletionNew()
 entry$setCompletion(completion)
\end{Sinput}
\end{Schunk}

We will use a \code{rGtkDataFrame} instance for our completion model,
taking a convenient list of names for our example.
We set the completion objects's model and text column using the
similarly named methods, and then set some properties to customize how
the completion is handled.
\begin{Schunk}
\begin{Sinput}
 store <- rGtkDataFrame(data.frame(name=I(state.name)))
 completion$setModel(store)
 completion$setTextColumn(0)             # which column in model
 completion['inline-completion'] <- TRUE # inline with text edit
 completion['popup-single-match'] <- FALSE
\end{Sinput}
\end{Schunk}

If we wanted to set a different matching function, one would do
something along the lines of the following where \function{grepl} is
used to indicate any match, not just the initial part of the
string. We get the string from the entry widget, not the value passed
in, as that has been standardized to lower case.

\begin{Schunk}
\begin{Sinput}
 f <- function(comp, str, iter, user.data) {
   model <- comp$getModel()
   rowVal <- model$getValue(iter, 0)$value   # column 0 in model
   
   str <- comp$getEntry()$getText()      # case sensitive
   grepl(str, rowVal)
 }
 QT <- completion$setMatchFunc(func=f)
\end{Sinput}
\end{Schunk}

\begin{Schunk}
\begin{Sinput}
 ## Our basic GUI is basic:
 w <- gtkWindow(show=FALSE)
 w$setTitle("Test of entry with completion")
 w$add(entry)
 w$showAll()
\end{Sinput}
\end{Schunk}



\end{example}



\subsection{Tree Views}
\label{sec:RGtk2-tree-view}

%% intro
Both tabular data and tree-like data are displayed through tree
views. The visual difference is that a trigger icon appears in rows which
represent parents with children. When these are expanded, the
children are indicated by indentation. The children are all displayed
in a consistent tabular format.

%% constructor
A tree view is constructed by \constructor{gtkTreeView}. The
\argument{model}{gtkTreeView} can be used to specify the underlying
model. If not specified at the time of construction, the
\method{setModel}{gtkTreeView} can be used. The accompanying
\method{getModel}{gtkTreeView} model returns the model from the view.

%% tree view properties
Tree views have several properties. The \code{headers-clickable}
property, when set to \code{TRUE}, allows the column headers to
receive mouse clicks. This is used for sorting, when the underlying
data store allows for that. The tree view widget can popup a search
box when the user types \kbd{control-f} if the property
\code{enable-search} is \code{TRUE} (the default). To turn on
searching, a column needs to be specified through the
\code{search-column} property. Rows may be rearranged through
drag-and-drop if the \code{reorderable} property is set to
\code{TRUE}. The \code{rules-hint}, if \code{TRUE}, will instruct the
theme that the rows hold associated data. Themes will typically use this
information to stripe alternating rows.


%% treeviewcolumns
\paragraph{Tree view columns}
For speed purposes, the rendering of a tree view centers around the
display of its columns. Each column is displayed through a tree view
column, given by the \constructor{gtkTreeViewColumn}.

%% basic properties
One can set basic properties of the column. Each column has an
optional header that can contain a title or even an arbitrary
widget. The \method{setTitle}{gtkTreeViewColumn} method is used to set
the title. This area can be \qcode{clickable}, in which case this area
receives mouse clicks. This is most commonly used to allow sorting of
the column by clicking on the headers, but can also be used to add
popup menus (with a bit of wizardry).

The property \qcode{resizable} determines wheter the user can resize
the column, by dragging with the mouse. The size properties
\qcode{width}, \qcode{min-width}, and \qcode{fixed-width} control the
size.

The visibility of the column can be adjusted through the
\method{setVisible}{gtkTreeViewColumn} method. 


Tree view columns are added to the tree view with the
method \method{insertColumn}{gtkTreeView}. The
\argument{column}{gtkTreeViewInsertColumn} argument specifies the tree
view column, and the \argument{position}{gtkTreeViewInsertColumn}
argument the column to insert into ($0$-based). A column can be moved
with the \method{moveColumnAfter}{gtkTreeView} method, and removed
with the \method{removeColumn}{gtkTreeView} method. The tree view's
\method{getChildren}{gtkTreeView} method returns a list containing all
of the tree view columns.

%% cell renderers
\paragraph{More on cell renderers}

In addition to the text and pixbuf cell renderers discussed in
Section~\ref{sec:RGtk2:combobox}, there are cell renderers that allow
one to display other types of data available for the tree view widget.
Some only make sense if the underlying data is to be edited, for
example the combobox renderer, for these, the \code{editable}
attribute for the cell renderer
should be set to \code{TRUE}.


As with comboboxes, the mapping of values in the data store to the
attributes of a cell render is done by the
\method{addAttribute}{gtkCellLayout} method of the tree view column.


\paragraph{Toggle cell renderers}
Binary data can be represented by a toggle. The
\constructor{gtkCellRendererToggle} will create a check box in the
cell, that  will look checked or not depending on the value of its
\code{active} attribute. If this value is found in a boolean column of
the model, then changes to the model will be reflected in the state of
the GUI. However, the programmer must propogate changes to the GUI (the
view) back to the model. The \signal{toggled} signal is emitted when
the state is changed. The \code{activatable} attribute for the cell
must be \code{TRUE} in order for it to receive user input.

\begin{Schunk}
\begin{Sinput}
 cr <- gtkCellRendererToggle()
 cr['activatable'] <- TRUE               # cell can be edited
 cr['active'] <- TRUE
 QT <- gSignalConnect(cr, "toggled", function(w, path) {
   print(as.numeric(path) + 1) ## the row
 })
\end{Sinput}
\end{Schunk}

%% combo
\paragraph{Combobox cell renderers}
A cell can show a combobox for selection. The
\constructor{gtkCellRendererCombo} produces the object. Its
\code{model} attribute is set to give the values to choose from. The
attribute \code{has-entry} can be set to \code{TRUE} to allow a user
to enter values, if \code{FALSE} they can only select from the
available ones. 

\begin{Schunk}
\begin{Sinput}
 cr <- gtkCellRendererCombo()
 store <- rGtkDataFrame(state.name)
 cr['model'] <- store
 cr['text-column'] <- 0
 cr['editable'] <- TRUE                  # needed
\end{Sinput}
\end{Schunk}

%% progress bars
\paragraph{Progress bar cell renderers}
A progress bar can be used to display the percentage of some task. The
\constructor{gtkCellRendererProgress} function returns the cell
renderer. Its \code{value} attribute takes a value between $0$ and
$100$ indicating the amount finished, with a default value of
$0$. Values out of this range will be signaled by an error message.  The
\code{orientation} property, with values from
\code{GtkProgressBarOrientation}, can adjust the direction that the
bar grows.  For example,

\begin{Schunk}
\begin{Sinput}
 cr <- gtkCellRendererProgress()
 cr["value"] <- 50                       # fixed 50%
 cr['orientation'] <- "right-to-left"
\end{Sinput}
\end{Schunk}

%% numbers
\paragraph{Cell data functions}
Formatting numbers is a bit trickier, as the cell renderer properties
are geared around text values. For example, to align floating point
numbers one can do so in the model and then display as text. However,
to do so through the cell rendererer requires one to get the value
from the model and modify it before the cell rendererer gets it. For
this, a cell data function is used, and not with comboboxes, but only
tree views. A cell data function has arguments pointing to a tree view
column, the cell renderer, the model, an
iterator pointing to the row in the model and a data argument for user
data. The function should set the appropriate attributes of the cell
renderer. For example, this function could be used to format floating
point numbers:
\begin{Schunk}
\begin{Sinput}
 func <- function(viewCol, cellRend, model, iter, data) {
   curVal <- model$GetValue(iter, 0)$value
   fVal <- sprintf("%.3f", curVal)
   cellRend['text'] <- fVal
   cellRend['xalign'] <- 1
 }
\end{Sinput}
\end{Schunk}

%% This function will be set for the tree view column and illustrated in Example~\ref{ex:RGtk2:rGtk2DataFrame}.



%% editable cells \paragraph{Editable cells} When the \code{editable}
property of a text cell (or \code{activatable} property of a toggle
cell) is set to \code{TRUE}, then the cell contents can be
changed. This allows the user to make changes to the underlying model
through the GUI. Although the view automatically reflects changes made
to the model, the reverse is not true. A callback must be assigned to
the \code{editable} (\code{toggled}) signal for the cell renderer to
imlement the change. The callback for the \qcode{editable} signal has
arguments \code{renderer}, \code{path} for the path of the selected
row (as a string), and \code{new.text} containing the value of the edited text as a
string. The tree view and which column was edited are not passed in by
default. These can be passed through the user data arugment, or set as
data for the widget if needed within the callback.
\begin{Schunk}
\begin{Sinput}
 cr['editable'] <- TRUE
 ID <- gSignalConnect(cr, "edited", 
                      f=function(cr, path, newtext, data) {
                        curRow <- as.numeric(path) + 1
                        model <- data$model
                        curCol <- data$column
                        model[curRow, curCol] <- newtext
                      }, data=list(model=store, column=1))
\end{Sinput}
\end{Schunk}

Users may expect that once a cell is edited, the next cell is then set
up to be edited. In order to do this, one must set the cursor to the
appropriate place and set the state to editing. This is done through
the tree view's \method{setCursor}{gtkTreeView} method. The
\code{path} argument takes a tree path instance, the \code{column}
argument is for a tree view column object, and the flag
\code{start.editing} should be set to \code{TRUE} to initiate
editing. The tree view method \method{getColumn}{gtkTreeView} can be
used to get the tree view column by index ($0$-based) and the path
object can be found from a string through \function{gtkTreePathNewFromString}.


% \begin{example}{A basic usage of displaying a data frame using a tree view}{ex:RGtk2-minimal-rGtkDataFrame}
%   \SweaveInput{ex-RGtk2-minimal-rGtkDataFrame}
% \end{example}

\begin{example}{Displaying text columns in a tree view}{ex:RGtk2-add-toggle-to-df}
This example shows how to select one or more rows from a data frame
that contains some information. We write it so any data frame could be
used, although in the specific case we show a list of the installed
packages that can be upgraded from CRAN.


To get the installed packages that can be upgraded, we use some of the
functions provided by the  \pkg{utils} package.
\begin{Schunk}
\begin{Sinput}
 avail <- available.packages()
 installed <- installed.packages()
 tmp <- merge(avail, installed, by="Package")
 need.upgrade <- with(tmp, as.character(Version.x) != as.character(Version.y))
 d <- tmp[need.upgrade, c(1, 2, 16, 6)]
 names(d) <- c("Package", "Available", "Installed", "Depends")
\end{Sinput}
\end{Schunk}


This function will be called on the selected rows. The \code{print}
call would be replaced with something more reasonable, such as a call
to \function{install.packages}.
\begin{Schunk}
\begin{Sinput}
 doThis <- function(d) print(d)
\end{Sinput}
\end{Schunk}

The rest of this code is independent of the details of \code{d}. We first
append a column to the data frame to store the selection information.
\begin{Schunk}
\begin{Sinput}
 n <- ncol(d)
 nms <- names(d)
 d$.toggle <- rep(FALSE, nrow(d))
 store <- rGtkDataFrame(d)
\end{Sinput}
\end{Schunk}

Our tree view shows each column using a simple text cell renderer,
except for an initial one where the user can select the packages they
want to call \code{doThis} on.
\begin{Schunk}
\begin{Sinput}
 view <- gtkTreeView()
 # add toggle
 togglevc <- gtkTreeViewColumn()
 view$insertColumn(togglevc, 0)
\end{Sinput}
\begin{Soutput}
[1] 1
\end{Soutput}
\begin{Sinput}
 cr <- gtkCellRendererToggle()
 togglevc$packStart(cr)
 cr['activatable'] <- TRUE
 togglevc$addAttribute(cr, "active", n)
 gSignalConnect(cr, "toggled", function(cr, path, user.data) {
   view <- user.data
   row <- as.numeric(path) + 1
   model <- view$getModel()
   n <- dim(model)[2]
   model[row, n] <- !model[row, n]
 },
                data=view)
\end{Sinput}
\begin{Soutput}
toggled 
    726 
attr(,"class")
[1] "CallbackID"
\end{Soutput}
\end{Schunk}

The text columns are added all in a similar manner.
\begin{Schunk}
\begin{Sinput}
 sapply(1:n, function(i) {
   vc <- gtkTreeViewColumn()
   vc$setTitle(nms[i])
   view$insertColumn(vc, i)
 
   cr <- gtkCellRendererText()
   vc$packStart(cr)
   vc$addAttribute(cr, "text", i-1)
 })
\end{Sinput}
\begin{Soutput}
[[1]]
NULL

[[2]]
NULL

[[3]]
NULL

[[4]]
NULL
\end{Soutput}
\end{Schunk}

Our basic GUI places the view into a box container that also holds a
button to initiate the action.
\begin{Schunk}
\begin{Sinput}
 w <- gtkWindow(show=FALSE)
 w$setTitle("Installed packages that need upgrading")
 w$setSizeRequest(300, 300)
 g <- gtkVBox(); w$add(g)
 sw <- gtkScrolledWindow()
 g$packStart(sw, expand=TRUE, fill=TRUE)
 sw$add(view)
 sw$setPolicy("automatic", "automatic")
\end{Sinput}
\end{Schunk}

We add the button and its callback. We pass in the view, rather than
the model, in case the model would be modified by the \code{doThis}
call. In a real application, once a package is upgraded it would be
removed from the display.
\begin{Schunk}
\begin{Sinput}
 b <- gtkButton("click me")
 gSignalConnect(b, "clicked", function(w, data) {
   view <- data
   model <- view$getModel()
   n <- dim(model)[2]
   vals <- model[model[, n], -n, drop=FALSE]
   doThis(vals)
 }, data=view)
\end{Sinput}
\begin{Soutput}
clicked 
    768 
attr(,"class")
[1] "CallbackID"
\end{Soutput}
\begin{Sinput}
 g$packStart(b, expand=FALSE)
\end{Sinput}
\end{Schunk}

Finally, we connect the store to the model and show the top-level window
\begin{Schunk}
\begin{Sinput}
 view$setModel(store)
 w$show()
\end{Sinput}
\end{Schunk}
\end{example}
%% filtering
\paragraph{Using filtered models to restrict the displayed rows}
\GTK\/ provides a means to show a filtered selection of rows. The
basic idea is that an extra column in the store stores logical values
to indicate if a row should be visible. To implement this, a filtered
store must be made from the original store. The
\method{filterNew}{gtkTreeModel} method of a data store returns a
filtered data store. The original model is found from the filtered one
through its \code{getModel} method. The method
\method{setVisibleColumn}{gtkTreeModelFilter} specifies which column
in the model holds the logical values.  Finally, to use the filtered store, it is simply set as
the model for a tree view.

\begin{Schunk}
\begin{Sinput}
 df <- data.frame(col=letters[1:3], vis=c(TRUE, TRUE, FALSE))
 store <- rGtkDataFrame(df)
 filtered <- store$filterNew()
 filtered$setVisibleColumn(1)            # 0-based
 view <- gtkTreeView(filtered)
\end{Sinput}
\end{Schunk}

\begin{Schunk}
\begin{Soutput}
[1] 1
\end{Soutput}
\end{Schunk}

%% searching
\paragraph{Sorting the display}
One can implement sorting of the display by clicking on the column
headers. This is done by creating a model that can be sorted from the
original store. The function \function{gtkTreeModelSortNewWithModel}
will produce a new store that is assigned as the model for the tree
view. Then to allow a column to be sorted, one specifies first the
\code{clickable} property of the view column, and then specifies a
column to sort by when the column header is clicked (it can be
different if desired). The following shows the basic steps:

\begin{Schunk}
\begin{Sinput}
 store <- rGtkDataFrame(mtcars)
 sorted <- gtkTreeModelSortNewWithModel(store)
 #
 view <- gtkTreeView(sorted)
 vc <- gtkTreeViewColumn()
 view$insertColumn(vc, 0)                  # first column
\end{Sinput}
\begin{Soutput}
[1] 1
\end{Soutput}
\begin{Sinput}
 vc$setTitle("Click to sort")
 vc$setClickable(TRUE)
 vc$setSortColumnId(0)                   
 #
 cr <- gtkCellRendererText()
 vc$packStart(cr)
 vc$addAttribute(cr, "text", 0)
\end{Sinput}
\end{Schunk}
The default sorting function can be changed. The sortable stores
method \method{setSortFunc}{gtkTreeSortable} is used for this.
The following function -- which can easily be modified to taste -- shows
how the default sorting might be implemented.
\begin{Schunk}
\begin{Sinput}
 f <- function(model, iter1, iter2, data) {
   column <- data
   val1 <- model$GetValue(iter1, column)$value
   val2 <- model$GetValue(iter2, column)$value
   val1 > val2
 }
 QT <- sorted$setSortFunc(sort.column.id=0, sort.func=f, 
                          user.data=0)   # column
\end{Sinput}
\end{Schunk}


\paragraph{Selection}
%% selection: none, single browse multiple, getSelected

\GTK\/ provides a class to handle the selection of rows that the user
makes. The selection object is returned from the tree view, through
its \method{getSelection}{gtkTreeView} method. 
To modify the selection possibilities, the selection's method
\method{setMode}{gtkTreeSelection} method is used, with values from
\code{GtkSelectionMode}, including \qcode{multiple} for allowing more
than one row to be selected and \qcode{single} for no more than one row.
Additionally, the selection object has various
methods to interact with the selection.  

When only a single selection is possible, the method
\method{getSelected}{gtkTreeSelection} returns a list with components
\code{retval} to indicate success, \code{model} containing the model
and \code{iter} containing an iterator to the selected row in the
model.

\begin{Schunk}
\begin{Sinput}
 store <- rGtkDataFrame(mtcars)
 view <- gtkTreeView(store)
 selection <- view$getSelection()
 QT <- selection$setMode("single")
\end{Sinput}
\end{Schunk}

\begin{Schunk}
\begin{Soutput}
[1] 1
\end{Soutput}
\end{Schunk}
If this tree view is shown and a selection made, this code will return the value in the first column:
\begin{Schunk}
\begin{Sinput}
 selection$selectPath(gtkTreePathNewFromString("3")) # set for example
 # 
 curSel <- selection$getSelected()       # retrieve selection
 with(curSel, model$getValue(iter, 0)$value) # both model and iter in curSel
\end{Sinput}
\begin{Soutput}
[1] 21.4
\end{Soutput}
\end{Schunk}


When multiple selection is permitted, then the method
\method{getSelectedRows}{gtkTreeSelection} returns a list with
componets \code{model} pointing to the model, and \code{retval} a list
of tree paths. No column information is passed back by this method.

For example, if we change the selection mode then select a few rows
\begin{Schunk}
\begin{Sinput}
 selection$setMode("multiple")
\end{Sinput}
\end{Schunk}
This code will print the selected values in the first column:
\begin{Schunk}
\begin{Sinput}
 curSel <- selection$getSelectedRows()
 if(length(curSel$retval)) {
   rows <- sapply(curSel$retval, function(path) {
     as.numeric(path$toString()) + 1
   })
   curSel$model[rows, 1]
 }
\end{Sinput}
\begin{Soutput}
[1] 21.4
\end{Soutput}
\end{Schunk}
                 

%% signals, callback
%% row-activated -- double click
\paragraph{Signals}
Tree views can be used different ways: if the cells are not editable,
then they are basically list boxes which allow the user to select one of
several rows. A single click selects the value, and a double click is
often used to initiate an action. If the cells are editable, then this
action is to edit the content.

%% selection
When a selection is made or changed, the signal \signal{changed} is emitted by
the selection. 

%% double click
When a row is not editable, then the double-click event or a keyboard
command triggers the \signal{row-activated} signal for the tree
view. The callback has arguments \code{tree.view} pointing to the
widget that emits the signal, \code{path} storing a tree path of the
selected row, and \code{column} containing the tree view column. The
column number is not returned. If that is of interest, it can be
passed in via the user data argument, or matched against the children
of the tree view through a command like

\begin{Schunk}
\begin{Sinput}
 sapply(tree.view$getColumns(), function(i) i == column)
\end{Sinput}
\end{Schunk}


%% tree stores
For tree stores, the user can click to expand or collapse a part of
the tree. The signals \code{row-expanded} and \code{row-collapsed} are
emitted respectively by the tree view. The signature of the callback is similar to
above with the view, a tree path and a view column.



%% filter
\begin{example}{Using filtering}{ex:RGtk2-filtered}
This example shows how to use \GTK's filtering feature to restrict the rows of the model shown by matching against the values entered into a text entry box. The end result is similar to an entry widget with completion.

\begin{figure}
  \centering
  \includegraphics[width=.45\textwidth]{ex-RGtk2-filtered}
  \caption{Example of a data store filtered by values typed into a text-entry widget.}
  \label{fig:RGtk2-filtered}
\end{figure}

We use a convenient set of names and create a data frame. The
\code{VISIBLE} column will be added to the \code{rGtkDataFrame}
instance to adjust the visible rows.
\begin{Schunk}
\begin{Sinput}
 df <- data.frame(state.name)
 df$VISIBLE <- rep(TRUE, nrow(df))
 store <- rGtkDataFrame(df)
\end{Sinput}
\end{Schunk}

The filtered store needs to have the column specified that contains
the logical values, in this example it is the last column.
\begin{Schunk}
\begin{Sinput}
 filteredStore <- store$filterNew()
 filteredStore$setVisibleColumn(ncol(df)-1)      # offset
 view <- gtkTreeView(filteredStore)
\end{Sinput}
\end{Schunk}

This example uses just one column, we create a basic view of it below.
\begin{Schunk}
\begin{Sinput}
 vc <- gtkTreeViewColumn()
 cr <- gtkCellRendererText()
 vc$packStart(cr, TRUE)
 vc$setTitle("Col")
 vc$addAttribute(cr, "text", 0)
 QT <- view$insertColumn(vc, 0)
\end{Sinput}
\end{Schunk}

An entry widget will be used to control the filtering. In the
callback, we adjust the \code{VISIBLE} column of the
\code{rGtkDataFrame} instance, to reflect the rows to be shown. When
\code{val} is an empty string, the result \function{grep} is just
\code{TRUE}, so all rows will be shown. The
\code{getModel} method of the filtered store is used, although we
could have passed in that store itself.
\begin{Schunk}
\begin{Sinput}
 e <- gtkEntry()
 ID <- gSignalConnect(e, "changed", function(w, data) {
   val <- w$getText()
   df <- data$getModel()
   values <- df[,1]
   df[, dim(df)[2]] <- sapply(values, function(i) 
                              as.logical(length(grep(val,i))))
 },
                      data=filteredStore)
\end{Sinput}
\end{Schunk}


Figure~\ref{fig:RGtk2-filtered} shows the two widgets placed within a
simple GUI.
\end{example}


%% ping pong
\begin{example}{A widget for variable selection}{ex:RGtk2-pingpong}
This example shows a combination widget that is familiar from other
statistics GUIs. It provides two tree views listing variable names and
has arrows to move variable names from one side to the other. Often
such widgets are used for specifying statistical models. 

\begin{figure}
  \centering
  \includegraphics[width=.4\textwidth]{ex-RGtk2-pingpong}
  \caption{An example showing to tree views with buttons to move entries from one to the other. This is a common method for variable selection.}
  \label{fig:RGtk2-pingpong}
\end{figure}


We will use Example~\ref{ex:RGtk2:add-stock-icons}, in particular its
function \code{addToStockIcons}, to add some custom
stock icons to identify the variable type.
\begin{Schunk}
\begin{Sinput}
 nms <- c("factor","numeric")
 fileNms <- c(system.file("images","factor.gif", package="gWidgets"),
              system.file("images","numeric.gif", package="gWidgets"))
 QT <- addToStockIcons(nms, fileNms)
\end{Sinput}
\end{Schunk}

To keep track of the variables in the two tree views we use a single
model. It has a column for all the variable names, a column for the
icon, and two columns to keep track of which variable names are to be
displayed in the respective tree views.
\begin{Schunk}
\begin{Sinput}
 d <- data.frame(varNames=c("response", "trt1", "trt2"),
                 stock.id=c("new-numeric", "new-factor", "new-factor"),
                 leftView  = rep(TRUE, 3),
                 rightView = rep(FALSE, 3),
                 stringsAsFactors=FALSE)
 model <- rGtkDataFrame(d)
\end{Sinput}
\end{Schunk}

We will use a filtered data store to show each tree view.
As the two tree views are identical, except for the rows that are
displayed, we use a function to generate them. The \code{vis.col}
indicates which column in the \code{rGtkDataFrame} object contains the
visibility information. Our tree view packs in both a pixbuf cell
rendererer and a text one.
\begin{Schunk}
\begin{Sinput}
 ## make a view
 makeView <- function(model, vis.col) {
   filteredModel <- model$filterNew()
   filteredModel$setVisibleColumn(vis.col - 1)
   tv <- gtkTreeView(filteredModel)
   tv$getSelection()$setMode("multiple")
   ##
   vc <- gtkTreeViewColumn()
   vc$setTitle("Variable")
   tv$insertColumn(vc, 0)
   ##
   cr <- gtkCellRendererPixbuf()
   vc$PackStart(cr, expand=FALSE)
   cr['xalign'] <- 1
   vc$addAttribute(cr, "stock-id", 1)
   ##
   cr <- gtkCellRendererText()
   vc$PackStart(cr, expand=TRUE)
   cr['xalign'] <- 0
   cr['xpad'] <- 5
   vc$addAttribute(cr, "text", 0)
 
   return(tv)
 }
\end{Sinput}
\end{Schunk}
We know create the tree views and store the selections associated to each.
\begin{Schunk}
\begin{Sinput}
 views <- list()
 views[["left"]] <- makeView(model,3)
 views[["right"]] <- makeView(model,4)
 selections <- lapply(views, gtkTreeViewGetSelection)
\end{Sinput}
\end{Schunk}
We need buttons to move the values left and right, these are stored
in a list for convenience later on.
\begin{Schunk}
\begin{Sinput}
 buttons <- list()
 buttons[["fromLeft"]] <- gtkButton(">")
 buttons[["fromRight"]] <- gtkButton("<")
\end{Sinput}
\end{Schunk}

Our basic GUI is shown in Figure~\ref{fig:RGtk2-pingpong} where the
two tree views are placed side-by-side.
%The basic GUI just lays out the two tree views with the buttons in
%between. We left out command to add scrollwindows etc.

The key handler moves the selected value from one side to the
other. The issue here is that when the view is using filtering the
selection returns values relative to the child model (the filtered
one). In general the methods of the filtered model
\code{convertChildPathToPath} and \code{convertChildIterToIter} will
translate between the two models, but in this case we pass in the
\code{rGtkDataFrame} instance, not the filtered model. So we use the
columns indicating visibility to identify which index is being
referred to. This handler assumes the model and a value indicating the
view (\code{from}) is passed in through the user data.
\begin{Schunk}
\begin{Sinput}
 moveSelected <- function(b, data) {
   model <- data$model
 
   selection <- selections[[data$from]]
   selected <- selection$getSelectedRows()
   if(length(selected$retval)) {
     childRows <- sapply(selected$retval, function(childPath) {
       childRow <- as.numeric(childPath$toString()) + 1
     })
     shownIndices <- which(model[, 2 + data$from])
     rows <- shownIndices[childRows]
 
     model[rows, 2 + data$from] <- FALSE
     model[rows, 2 + (3-data$from)] <- !model[rows, 2 + data$from]
   }
 }
\end{Sinput}
\end{Schunk}
We connect the handler to the \qcode{clicked} signal for the buttons.
\begin{Schunk}
\begin{Sinput}
 IDs <- sapply(1:2, function(i) 
               gSignalConnect(buttons[[i]], signal="clicked", 
                              f=moveSelected,
                              data=list(from=i, model=model)))
\end{Sinput}
\end{Schunk}

We add one flourish, namely ensuring that the arrows are not sensitive
when the corresponding selection is not set. This handler for the
selections is used.
\begin{Schunk}
\begin{Sinput}
 disableButton <- function(sel, data) {
   selected <- sel$getSelectedRows()
   buttons[[data]]$setSensitive(length(selected$retval) != 0)
 }
 IDs <- sapply(1:2, function(i) 
               gSignalConnect(selections[[i]], signal="changed",
                              f=disableButton,
                              data=i))
\end{Sinput}
\end{Schunk}
As the initial state has no selection,  we set the buttons sensitivity accordingly.
\begin{Schunk}
\begin{Sinput}
 QT <- sapply(buttons, function(i) i$setSensitive(FALSE))
\end{Sinput}
\end{Schunk}
 \end{example}

%% editable
% \begin{example}{An editable data frame}{ex:RGtk2-editable-dataframe}
%   \SweaveInput{ex-RGtk2-editable-dataframe}
% \end{example}


%% Tree stores
The \constructor{gtkTreeView} widget displays either list stores or
tree  stores. The difference for the programmer is in the creation of
the data store, not the tree view.

\begin{example}{A simple tree display}{eg:RGtk2-simple-tree}
This example illustrates that the same tree view can display both
rectangular and heirarchical data. The data we use will come from the
\code{Cars93} data set used in Example~\ref{eg:RGtk2:tree-store}. In
that example we defined a simple tree store from a data frame, with a
level for manufacturer and make for different cars. We refer to that
model by \code{tstore} below. 



Now, we make a simple rectangular store for the make information with
the following:

\begin{Schunk}
\begin{Sinput}
 store <- rGtkDataFrame(Cars93[,"Model", drop=FALSE])
\end{Sinput}
\end{Schunk}

The basic view is similar to that for rectangular data already presented.
\begin{Schunk}
\begin{Sinput}
 view <- gtkTreeView()
 vc <- gtkTreeViewColumn()
 vc$setTitle("Make")
 QT <- view$insertColumn(vc, 0)
 cr <- gtkCellRendererText()
 vc$packStart(cr)
 vc$addAttribute(cr, "text", 0)
\end{Sinput}
\end{Schunk}


Finally, we illustrate that the same view can be used with either model:
\begin{Schunk}
\begin{Sinput}
 view$setModel(store)                    # the rectangular store
 view$setModel(tstore)                   # or the tree store
\end{Sinput}
\end{Schunk}
\end{example}

%% ## dynagmic
\begin{example}{Dynamically growing a tree}{eg:RGtk2:tree-dynamic}
This example uses a tree to explore an \R\/ list object, such as what
is returned by one of \R's modelling functions.  As the depth of these
lists is not specified in advance, we use a dynamical approach to
creating the tree store, modifying the tree store when the tree view
is expanded or collapsed.
  


We begin by defining a function that gets the ``children'' of a list
object. For a level of the list, this function returns the named
components, their class and a logical indicating if the component is
recursive.
\begin{Schunk}
\begin{Sinput}
 getChildren <- function(path=character(0)) {
   
   pathToObject <- function(path) {      
     x <- try(eval(parse(text=paste(path,collapse="$")),
                   envir=.GlobalEnv),silent=TRUE)
     if(inherits(x,"try-error")) {
       cat(sprintf("Error with %s",path))
       return(NA)
     }
     return(x)
   }
 
   theChildren <- function(path) {
     if(length(path) == 0)
       ls(envir=.GlobalEnv)
     else
       names(pathToObject(path))
   }
   hasChildren <- function(obj) is.recursive(obj) && !is.null(names(obj))
   
   getType <- function(obj) head(class(obj), n=1)
 
   children <- theChildren(path)
   objs <- sapply(children,function(i) pathToObject(c(path,i)))
   d <- data.frame(children=children,
                   class=sapply(objs, getType),
                   offspring=sapply(objs, hasChildren))
   ## filter out Gtk ones
   ind = grep("^Gtk", d$class)
   if(length(ind) == 0) return(d) else return(d[-ind,])
 }
\end{Sinput}
\end{Schunk}

This function is used to add the children to a tree store.
\begin{Schunk}
\begin{Sinput}
 addChildren <- function(store, children, parentIter=NULL) {
   if(nrow(children) == 0) 
     return(NULL)
   for(i in 1:nrow(children)) {
     iter <- store$append(parent=parentIter)$iter
     ## use last column to indicate logical
     sapply(1:(ncol(children) - 1), function(j)              
            store$setValue(iter, column=j-1, children[i,j]))
     ## Add a branch if there are children
     ## no better way, as this adds an extra blank line
     ## we remove ir later.
     if(children[i, "offspring"])
       store$append(parent=iter)
   }
 }
\end{Sinput}
\end{Schunk}

The various callbacks for the tree view pass back the view and a tree
path. We define some functions to relate these values with iterators.
\begin{Schunk}
\begin{Sinput}
 tpathToPIter <- function(view, tpath) {
   ## view$getModel -- sstore, again store
   sstore <- view$getModel()
   store <- sstore$getModel()
   uspath <- sstore$convertPathToChildPath(tpath)
   p.iter <- store$getIter(uspath)$iter
   return(p.iter)
 }
\end{Sinput}
\end{Schunk}

A ``path'' is made up of the names of each component that makes up an
element in the list. This function returns the path for a component
specified by its iterator.
\begin{Schunk}
\begin{Sinput}
 iterToPath <- function(view, iter) {
   sstore <- view$getModel()
   store <- sstore$getModel()
   string <- store$getPath(iter)$toString()
   indices <- unlist(strsplit(string,":"))
   thePath <- c()
   for(i in seq_along(indices)) {
     path <- paste(indices[1:i],collapse=":")
     iter <- store$getIterFromString(path)$iter
     thePath[i] <- store$getValue(iter,0)$value
   }
   return(thePath[-1])
 }
\end{Sinput}
\end{Schunk}

Now we can begin defining our tree store. This example allows sorting,
so calls the \constructor{gtkTreeModelSortNewWithModel} function.
\begin{Schunk}
\begin{Sinput}
 store = gtkTreeStore(rep("gchararray",2))
 sstore = gtkTreeModelSortNewWithModel(store)
\end{Sinput}
\end{Schunk}

We set an initial root.
\begin{Schunk}
\begin{Sinput}
 iter <- store$append(parent=NULL)$iter
 store$setValue(iter,column=0,"GlobalEnv")
 store$setValue(iter,column=1,"")
 iter <- store$append(parent=iter)
\end{Sinput}
\end{Schunk}
The call of \method{append}{gtkTreeStore} is used to allow the object
to have an expandable icon. 


Now to define the tree view. We allow multiple selection, as an illustration.
\begin{Schunk}
\begin{Sinput}
 view = gtkTreeViewNewWithModel(sstore)
 sel = view$getSelection()
 sel$setMode(GtkSelectionMode["multiple"])
\end{Sinput}
\end{Schunk}

The view will have two similar columns.
\begin{Schunk}
\begin{Sinput}
 ## add two cell renderers -- 1 for name, 1 for type
 nms <- c("Variable name","type")
 for(i in 1:2) {
   cr <- gtkCellRendererText()
   vc <- gtkTreeViewColumn()
   vc$setSortColumnId(i-1) # allow sorting
   vc$setResizable(TRUE)
   vc$setTitle(nms[i])
   vc$packStart(cr,TRUE)
   vc$addAttribute(cr,"text",i-1)
   view$insertColumn(vc, i-1)
 }
\end{Sinput}
\end{Schunk}

We put the tree view widget into a basic GUI.
\begin{Schunk}
\begin{Sinput}
 sw <- gtkScrolledWindow()
 sw$setPolicy("automatic","automatic")
 sw$add(view)
 w <- gtkWindow()
 w$setTitle("Tree view")
 w$add(sw)
\end{Sinput}
\end{Schunk}


At this point, we can see the first level, but nothing will happen if
we click on the trigger icon. We need to specify a handler for the
\code{row-expanded} signal. The odd thing here, is that we appended an
fake child, so that an expand icon would appear. In this, we remove it.

\begin{Schunk}
\begin{Sinput}
 ID <- gSignalConnect(view,signal="row-expanded",
                      f = function(view, iter, tpath, user.data) {
                        store <- user.data
                        p.iter <- tpathToPIter(view, tpath)
                        path <- iterToPath(view, p.iter)
                        children = getChildren(path)
                        addChildren(store, children, parentIter=p.iter)
                        ## remove errant 1st offspring. See addChildren
                        ci <- store$iterChildren(p.iter)
                        if(ci$retval) store$remove(ci$iter)
                      },
                      data=store)
\end{Sinput}
\end{Schunk}

Since the new data is generated when the row is expanded, we need
to remove the old data when the row is closed.
\begin{Schunk}
\begin{Sinput}
 ID <- gSignalConnect(view,signal="row-collapsed",
                   f = function(view, iter, tpath, user.data) {
                     store <- user.data
                     p.iter <- tpathToPIter(view,tpath)
 
                     n = store$iterNChildren(p.iter)
                     if(n > 1) { ## n=1 gets removed when expanded
                       for(i in 1:(n-1)) {
                         child.iter = store$iterChildren(p.iter)
                         if(child.iter$retval)
                           store$remove(child.iter$iter)
                       }
                     }
                   }, data=store)
\end{Sinput}
\end{Schunk}

Finally, this handler simply shows how to get the value (or values if
multiple selection is okay). 
\begin{Schunk}
\begin{Sinput}
 ID <- gSignalConnect(view,signal="row-activated",
                      f = function(view, tpath, tcol) {
                        p.iter <- tpathToPIter(view, tpath)
                        path <- iterToPath(view, p.iter)
                        sel <- view$getSelection()
                        out <- sel$getSelectedRows()
                        if(length(out) == 0) return(c()) # nothing
                        vals <- c()
                        for(i in out$retval) {  # multiple selections
                          iter <- out$model$getIter(i)$iter
                          vals <- c(vals, out$model$getValue(iter,0)$value)
                        }
                        print(vals)      # Insert Real Function Here
                      })
\end{Sinput}
\end{Schunk}



\end{example}
%% % ## mapes a list, shows how to update text view
% \SweaveInput{ex-RGtk2-tree-show}






\chapter{RGtk2: Menus and Dialogs}
\label{sec:RGtk2-menus}
% Menus in RGtk2

\section{Actions}
\label{sec:RGtk2:actions}

Actions are a means to create reusable representations for some action
to be initiated. The \constructor{gtkAction} constructor creates
actions, taking arguments \argument{name}{gtkAction},
\argument{label}{gtkAction} (what gets shown),
\argument{tooltip}{gtkAction}, and \argument{stock.id}{gtkAction}.
The act associated with an action is specified by adding a callback to
the \signal{activate} signal.

Actions are connected to widgets, through the method
\method{connectProxy}{gtkAction}. For buttons, the stock id
information must be added to the button through the button's
\code{setImage} method. The action's \method{createIcon}{gtkAction}
method, with argument coming from a value of \code{GtkIconSize}, will
return the needed image.

Actions can have their \code{sensitivity} property adjusted through
their \method{SetSensitive}{gtkAction} method. This will propogate to
all the widgets the action has a proxy connection with.


\begin{example}{An action object}{ex:RGtk2-action-object}
A basic action can be defined as follows:
\begin{Schunk}
\begin{Sinput}
 a <- gtkAction(name="ok", label="_Ok", tooltip="An OK button", stock.id="gtk-ok")
 ID <- gSignalConnect(a, "activate", f = function(w, data) {
   print(a$GetName())                    # of some useful thing
 })
\end{Sinput}
\end{Schunk}
To connect the action to a button, is straightforward.
\begin{Schunk}
\begin{Sinput}
 b <- gtkButton()
 a$connectProxy(b)
\end{Sinput}
\end{Schunk}

The image must be manually placed, which is facilitated by methods for
the button and the action object.
\begin{Schunk}
\begin{Sinput}
 b$setImage(a$createIcon('button')) # GtkIconSize value
\end{Sinput}
\end{Schunk}

\end{example}
\section{Menus}
\label{sec:RGtk2:menus}

A menu allows access to the GUI's actions in an organized way. This
organization relies on a choice of top-level menu items, their
possible submenus, and grouping within the same level of a
menu. Menubars are typically nested. Toolbars allow access more
quickly to common actions, but do not allow for nesting.

Menubars and popup menus  may be constructed by appending each menuitem
and submenu separately, as illustrated below. An alternative, using a
UI manager, is described in a subsequent section,

To specify a menubar step-by-step consists of defining a top-level
menu bar (\command{gtkMenuBar}). To a menu bar we append menu
items. Menu items may have sub menus (\constructor{gtkMenu}) appended, which gives
the heirarchical nature of a menu. Popup menus are similar, although begin with a \constructor{gtkMenu} instance.  

%% attach
\paragraph{Building the the menu}
Submenus and added to a menu item through the
\method{SetSubMenu}{gtkMenuItem} method. Menu items are added to a
menu through the methods
\method{Append}{gtkMenuShell}; \method{Prepend}{gtkMenuShell}; and
\method{Insert}{gtkMenuShell}, the latter requiring an index where the
insertion is to take place, with 0 being the same as \code{Prepend},
in addition to the child. After a child is added, the method
\method{ReorderChild}{gtkMenuShell} can be used to move it to a new
position ($0$-based). Menuitems are not typically removed, rather they
are disabled through their \code{SetSensitive} method, but if desired
their \method{Show}{gtkWidget} and \method{Hide}{gtkWidget} methods
can be used to stop them from being drawn.

%% Menu itmes
%% From API docs for gtkAccelGroup
% Note that accelerators are different from mnemonics. Accelerators are shortcuts for activating a menu item; they appear alongside the menu item they're a shortcut for. For example "Ctrl+Q" might appear alongside the "Quit" menu item. Mnemonics are shortcuts for GUI elements such as text entries or buttons; they appear as underlined characters. See gtk_label_new_with_mnemonic(). Menu items can have both accelerators and mnemonics, of course. 

\paragraph{Menu items}
Menu items represent actions to be taken and are created by several
different constructors.  A basic menu item is created by
\constructor{gtkMenuItem}. The argument \argument{label}{gtkMenuItem}
allows one to specify the label at construction time. The related
constructor, \constructor{gtkMenuItemNewWithMnemonic} also allows the
specification of a label, only underscores within the string specify
the mnemonic for the menu item.  To group menu items, one use
separators (\constructor{gtkSeparatorMenuItem}).

A menu item for a \code{gtkAction} object can be created by its
\method{createMenuItem}{gtkAction} method. For window managers that
display them, any icon specified the action's \code{stock.id} argument
will be displayed.


To add a different image in the menu bar, the
\constructor{gtkImageMenuItem} can be used. Although, the
\argument{stock.id}{gtkImageMenuItem} argument can be used to specify
the icon, we don't use this, as then the argument
\argument{accel.group}{gtkImageMenuItem} must be specified. An
accelerator group defines a set of keyboard shortcuts to initiate
actions, such as a \kbd{Ctrl+Q} to quit. (A mnemonic is a keyboard
shortcut to indicate a GUI element). We don't discuss creating those
here (they are given in the UI manager example). If instead of the
\code{stock.id} argument, just the \code{label} is specified for the
image menu item, the image can be added later through the
\method{SetImage}{gtkImageMenuItem} method. This takes an image object
for an argument. 

A check button menu item can be created by
\constructor{gtkCheckMenuItem}. This menu item shows a check box when
the \code{active} state for the menu is set. The default is not
active. Use the \method{GetActive}{gtkCheckMenuItem} to test if the
state is active or not.  It may be best to set this state to active,
so the user can identify that the item is a toggle (use the method
\code{SetActive(TRUE)}. When the user clicks on the menu item, the
\signal{toggled} signal is emitted.

\begin{example}{A basic menu bar}{eg:Rgtk2-Basic}
We illustrate how to make a basic menu bar with a plain item, an
item with an icon, and check item. Our GUI, just adds a menubar to a
top-level window.

We create top-level menubar and a menu item for our top level File entry with a mnemonic.
\begin{Schunk}
\begin{Sinput}
 mb <- gtkMenuBar()
 fileMi <- gtkMenuItemNewWithMnemonic(label="_File")
 mb$append(fileMi)
\end{Sinput}
\end{Schunk}

For the menu item we attach a submenu.
\begin{Schunk}
\begin{Sinput}
 fileMb <- gtkMenu()
 fileMi$setSubmenu(fileMb)
\end{Sinput}
\end{Schunk}
We now define some menu items. First a basic one:
\begin{Schunk}
\begin{Sinput}
 open <- gtkMenuItem(label="open")
\end{Sinput}
\end{Schunk}

Next we show how an \code{gtkAction} item can define a menuitem.
\begin{Schunk}
\begin{Sinput}
 saveAction <- gtkAction("save", "save", "Save object", "gtk-save")
 save <- saveAction$CreateMenuItem()
\end{Sinput}
\end{Schunk}

This illustrates how to add an image to the menu bar using a stock icon. The size specification is important to get the correct look.
\begin{Schunk}
\begin{Sinput}
 quit <- gtkImageMenuItem(label="quit")
 quit$setImage(gtkImageNewFromStock("gtk-quit", size=GtkIconSize["menu"]))
\end{Sinput}
\end{Schunk}

A simple check menu item can be created, as follows:
\begin{Schunk}
\begin{Sinput}
 happy <- gtkCheckMenuItem(label="happy")
 happy$setActive(TRUE)
\end{Sinput}
\end{Schunk}

These items are appended in the desired order, by
\begin{Schunk}
\begin{Sinput}
 Qt <- sapply(list(open, save, happy, gtkSeparatorMenuItem(), quit), function(i) {
        fileMb$append(i)
      })
\end{Sinput}
\end{Schunk}
We specify a handler for the toggle button
\begin{Schunk}
\begin{Sinput}
 ID <- gSignalConnect(happy, "toggled", function(b,data) {
   if(b$getActive())
     print("User is now happy")
 })
\end{Sinput}
\end{Schunk}
For the other  items, we specify a generic action for the \signal{activate} signal.
\begin{Schunk}
\begin{Sinput}
 QT <- sapply(list(open, quit, saveAction), function(i) 
        gSignalConnect(i, "activate", f=function(mi, data) {
          cat("item selected\n")
        })
        )
\end{Sinput}
\end{Schunk}

We make as simple GUI for the menubar.
\begin{Schunk}
\begin{Sinput}
 w <- gtkWindow(show=FALSE)
 w['title'] <- "Menubar example"
 w$add(mb)
 w$ShowAll()
\end{Sinput}
\end{Schunk}
\end{example}

\begin{example}{Popup menus}{ex:RGtk2-popup-menus}
To illustrate popup menus, we show how define a one and connect it to
a third-mouse click. We start with a \code{gtkMenu} instance, to which we add some items.
\begin{Schunk}
\begin{Sinput}
 popup <- gtkMenu()                       # top level
 for(i in c("cut","copy","----","paste")) {
   if(i == "----")
     popup$append(gtkSeparatorMenuItem())
   else
     popup$append(gtkMenuItem(i))
 }
\end{Sinput}
\end{Schunk}

This menu will be shown by \code{gtkMenuPopup}. This function is
called with the menu, some optional arguments for placement, and
values for the button that was clicked and the time of
activation. These values can be retrieved from the second argument of
the callback (a \code{GdkEvent}), as shown.
\begin{Schunk}
\begin{Sinput}
 b <- gtkButton("Click me with right mouse button")
 w <- gtkWindow(); w$setTitle("Popup menu example")
 w$add(b)
 ID <- gSignalConnect(b,"button-press-event",
                     f = function(w, e, userData) {
                       if(e$getButton() == 3 ||
                          (e$getButton() == 1 && # a mac
                           e$getState() == GdkModifierType['control-mask'])
                          ) {
                         gtkMenuPopup(userData$mb,
                                      button = e$getButton(),
                                      activate.time = e$getTime())
                         }
                       return(FALSE)
                     },
                     data=list(mb=popup)
                     )
\end{Sinput}
\end{Schunk}

The above will popup a menu, but until we bind to the \code{activate}
signal, nothing will happen when a menu item is selected. Below we
supply a stub for sake of illustration. The children of a popup menu
are the menu items, including the separator which we avoid.
\begin{Schunk}
\begin{Sinput}
 IDs <- sapply(popup$getChildren(), function(i) {
   if(!inherits(i, "GtkSeparatorMenuItem")) # skip these
     gSignalConnect(i, "activate",
                    f = function(w, data) print("replace me"))
 })
\end{Sinput}
\end{Schunk}
\end{example}


The above can easily be automated. The menubar widget in
\pkg{gWidgetsRGtk2} simply maps a list with named components to the
above, by setting menu items for each top-level component, submenus for
each component that contains children, and menu items for components
that do not have children. However, the next approach is preferred for
larger menubars, as it separates out the presentation from the actions.

\section{Toolbars}
\label{sec:RGtk2:toolbars}

Toolbars are like menubars only they only contain actions, there are
no submenus. Toolbar objects are constructed by
\constructor{gtkToolbar}. The placement of the widget at the top of a
top-level window is done by the programmer. Toolbar items are added to
the toolbar using the \method{Add}{gtkContainer} method. Once added,
items can be referred to by index using the \code{[[} method.

Toolbar items have some common properties. The buttons are comprised
of an icon and text, and the style of their layout is specified by the
toolbar method \method{SetStyle}{gtkToolbar}, with values coming from
the \code{GtkToolbarStyle} enumeration. Toolbar items can have a
tooltip set for them through the methods
\method{SetTooltipText}{gtkToolItem} or
\method{SetTooltipMarkup}{gtkToolItem}, the latter if PANGO markup is
desired. Toolbar items can be disabled, through the method \method{SetSensitive}{gtkWidget}.

The items can be one of a few different types. A stock toolbar item is
constructed by \constructor{gtkToolbarButtonNewFromStock}, with the
stock id as the argument. The constructor
\constructor{gtkToolbarButton} creates a button that can have its
label and icon value set through methods
\method{SetLabel}{gtkToolbarButton} and
\method{SetIconWidget}{gtkToolbarButton}. Additionally, there are
methods for setting a tooltip or specifying a stock id after
construction. A toggle button, which toggles between looking depressed
or not when clicked is created by \constructor{gtkToggleToolButton} or \constructor{gtkToggleToolButtonNewFromStock}.
Additionally there are constructors to place menus
(\constructor{gtkMenuToolButton}) and radio groups (\constructor{gtkRadioToolButton}).
  
% signale
The \code{clicked} signal is emitted when a toolbar button is
pressed. For the toggle button, the \code{toggle} signal is emitteed. Other



\begin{example}{Basic toolbar usage}{ex:RGtk2-basic-toolbar}
We illustrate with a toolbar, whose buttons are produced in various ways.
\begin{Schunk}
\begin{Sinput}
 tb <- gtkToolbar()
\end{Sinput}
\end{Schunk}
A button with a stock icon is produced by a call to the appropriate constructor.
\begin{Schunk}
\begin{Sinput}
 b1 <-  gtkToolButtonNewFromStock("gtk-open") 
 tb$add(b1)
\end{Sinput}
\end{Schunk}
To use a custom icons, requires a few steps.
\begin{Schunk}
\begin{Sinput}
 f <- system.file("images/dataframe.gif", package="gWidgets")
 image <- gtkImageNewFromFile(f)
 b2 <- gtkToolButton()
 b2$setIconWidget(image)
 b2$setLabel("Edit")
 tb$add(b2)
\end{Sinput}
\end{Schunk}
Adding a toggle button also is just a matter of calling the
appropriate constructor. In this, example we illustrate how to
initiate the callback only when the button is depressed.
\begin{Schunk}
\begin{Sinput}
 b3 <- gtkToggleToolButtonNewFromStock("gtk-fullscreen")
 tb$add(b3)
 QT <- gSignalConnect(b3, "toggled", f=function(button, data) {
   if(button$getActive())
     cat("toggle button is depressed\n")
   })
\end{Sinput}
\end{Schunk}
We give the other buttons a simple callback when clicked:
\begin{Schunk}
\begin{Sinput}
 QT <- sapply(1:2, function(i) gSignalConnect(tb[[i]], "clicked", function(button, data) {
   cat("You clicked", button$getLabel(), "\n")
 }))
\end{Sinput}
\end{Schunk}

\begin{Schunk}
\begin{Sinput}
 w <- gtkWindow(show=FALSE)
 w['title'] <- "Toolbar example"
 g <- gtkVBox()
 w$add(g)
 g$packStart(tb, expand=FALSE)
 g$packStart(gtkLabel("filler"), expand=TRUE, fill=TRUE)
 w$showAll()
\end{Sinput}
\end{Schunk}
\end{example}

\section{Statusbars}
\label{sec:RGtk2:statusbars}

In \GTK, a statusbar is constructed through by the
\constructor{gtkStatusbar} function. Statusbars must be placed at the
bottom of a top-level window by the programmer. In \GTK, a statusbar
keeps various stacks of messages for display. One adds a message to
display for given stack through the \method{Push}{gtkStatusbar} method
by specifying first an integer value for \code{context.id} and a
message. To pop the top message on a stack and display the next, the
method \method{Pop}{gtkStatusbar} method is available.


\section{UI Managers}
\label{sec:RGtk2:UIManager}


A GUI is designed around actions that are accessible through the
menubar and the toolbar. The notion of a \dfn{user interface manager}
(\acronym{UI} manager) separates out the definitions of the actions
from the user interface. The steps required to use \GTK's UI manager
are 
\begin{enumerate}
\item define a UI manager,
\item  set up an accelarator group for
keyboard shortcuts,
\item define our actions,
\item create action groups to
specify the name, label (with possible mnuemonic), keyboard
accelerator, tooltip, icon and callback for the graphical elements
that call the action,
\item specify where the menu items and toolbar
items will be placed,
\item connect the action group to the UI manager,
and finally
\item display the widgets.
\end{enumerate}

We show by an example how this is done. 

\begin{example}{UI Manager example}{ex:RGtk2:UImanager}
We define the UI manager as follows

\begin{figure}
  \centering
  \includegraphics[width=.6\textwidth]{ex-RGtk2-UI}
  \caption{A GUI made using a UI manager to layout the menubar and toolbar.}
  \label{fig:RGtk2-UI}
\end{figure}


\begin{Schunk}
\begin{Sinput}
 uimanager = gtkUIManager()
\end{Sinput}
\end{Schunk}

Our actions either open a dialog to gather more information or issue a
command. A \class{GtkAction} element is passed to the action. We
define a stub here, that simply updates a \code{gtkStatusbar}
instance, defined below.
\begin{Schunk}
\begin{Sinput}
 someAction <- function(action,...) 
   statusbar$push(statusbar$getContextId("message"), action$getName())
 Quit <- function(...) win$destroy()
\end{Sinput}
\end{Schunk}

To show how we can sequentially add interfaces, we break up our action
group definitions into one for ``File'' and ``Edit'' and another one
for ``Help.'' The key is the list defining the entries. Each component
specifies (in this order) the name; the icon; the label, with
\code{\_} specifying the mnemonic; the keyboard accelerator, with
\code{<control>}, \code{<alt>}, \code{<shift>} as possible prefixes, a
tooltip, which may not work with the \R\/ event loop, and finally the
callback. Empty values can be defined as \code{NULL} or, except for
the callback, an empty string.


\begin{Schunk}
\begin{Sinput}
 firstActionGroup = gtkActionGroup("firstActionGroup")
 firstActionEntries = list(
   ## name,ID,label,accelerator,tooltip,callback
   file = list("File",NULL,"_File",NULL,NULL,NULL),
   new = list("New","gtk-new","_New","<control>N","New document",someAction),
   sub = list("Submenu",NULL,"S_ub",NULL,NULL,NULL),
   open = list("Open","gtk-open","_Open","<ctrl>0","Open document",someAction),
   save = list("Save","gtk-save","_Save","<alt>S","Save document",someAction),
   quit = list("Quit","gtk-quit","_Quit","<ctrl>Q","Quit",Quit),
   edit = list("Edit",NULL,"_Edit",NULL,NULL,NULL),
   undo = list("Undo","gtk-undo","_Undo","<ctrl>Z","Undo change",someAction),
   redo = list("Redo","gtk-redo","_Redo","<ctrl>U","Redo change",someAction)
 )
\end{Sinput}
\end{Schunk}
We now add the actions to the action group, then add this action group
to the first spot in the UI manager.
\begin{Schunk}
\begin{Sinput}
 QT <- firstActionGroup$addActions(firstActionEntries)
 uimanager$insertActionGroup(firstActionGroup,0) # 0 -- first spot
\end{Sinput}
\end{Schunk}

The ``Help'' actions we do a bit differently. We define a ``Use
tooltips'' mode to be a toggle, as an illustration of that feature. One can also
incorporate radio groups, although this is not shown.

\begin{Schunk}
\begin{Sinput}
 helpActionGroup = gtkActionGroup("helpActionGroup")
 helpActionEntries = list(
   help = list("Help","","_Help","","",NULL),
   about = list("About","gtk-about","_About","","",someAction)
   )
 QT <- helpActionGroup$AddActions(helpActionEntries)
\end{Sinput}
\end{Schunk}

A toggle is defined with \command{gtkToggleAction} which has signature
in a different order than the action entry. Notice, we don't have an
icon, as the toggled icons is used.  To add a callback, we connect to
the \code{toggled} signal of the action element. This callback allows
for user data, as illustrated.

\begin{Schunk}
\begin{Sinput}
 toggleAction <- gtkToggleAction("UseTooltips",label="_Use tooltips",
                                 tooltip="Use tooltips ")
 toggleAction$setActive(TRUE)            # initially set
 ID <- gSignalConnect(toggleAction,signal = "toggled",
                     f=function(ta,userData) cat(userData,ta$getName(),"\n"),
                     data="toggled")
 helpActionGroup$addAction(toggleAction)
\end{Sinput}
\end{Schunk}
We insert the help action group in position 2.
\begin{Schunk}
\begin{Sinput}
 uimanager$insertActionGroup(helpActionGroup,1)
\end{Sinput}
\end{Schunk}
The \code{SetActive} method can set the state, use \code{GetActive} to
retrieve the state.


Our UI Manager's layout is specified in a file. The file uses XML to
specify where objects go. The structure of the file can be grasped
quickly from the example. Each entry is wrapped in \code{ui} tags. The
type of UI is either a \code{menubar}, \code{toolbar}, or
\code{popup}.  The \code{name} properties are used to reference the
widgets later on.  Menuitems are added with a \code{menuitem} entry
and toolbar items the \code{toolitem} entry. These have an
\code{action} value and an optional name (defaulting to the
\code{action} value). The \code{separator} tags allow for some
formatting.  The nesting of the menuitems is achieved using 
the \code{menu} tags. A \code{placeholder} tag can be used to add
entries at a later time.

\begin{verbatim}
<ui>
  <menubar name="menubar">
    <menu name="FileMenu" action="File">
      <menuitem name="FileNew" action="New"/>
      <menu action="Submenu">
	<menuitem name="FileOpen" action="Open" />
      </menu>
      <menuitem name="FileSave" action="Save"/>
      <separator />
      <menuitem name="FileQuit" action="Quit"/>
    </menu>
    <menu action="Edit">
      <menuitem name="EditUndo" action="Undo" />
      <menuitem name="EditRedo" action="Redo" />
    </menu>
    <menu action="Help">
      <menuitem action="UseTooltips"/>
      <menuitem action="About"/>
    </menu>
  </menubar>
  <toolbar name="toolbar">
    <toolitem action="New"/>
    <toolitem action="Open"/>
    <toolitem action="Save"/>
    <separator />
    <toolitem action="Quit"/>
  </toolbar>
</ui>
\end{verbatim}
%\VerbatimInput{ex-menus.xml}

This file is loaded into the UI manager as follows
\begin{Schunk}
\begin{Sinput}
 id <- uimanager$addUiFromFile("ex-menus.xml")
\end{Sinput}
\end{Schunk}

The \code{id} value can be used to merge and delete UI components, but
this is not illustrated here. The menus can also be loaded from strings.

Now we can setup a basic window template with menubar, toolbar, and
status bar. We first get the three main widgets. We use the names from
the UI layout to get the widgets through the \command{GetWidget}
method of the UI manager. The menubar and toolbar are returned as
follows, for our choice of names in the XML file.
\begin{Schunk}
\begin{Sinput}
 menubar <- uimanager$getWidget("/menubar")
 toolbar <- uimanager$getWidget("/toolbar")
\end{Sinput}
\end{Schunk}
The statusbar is constructed with
\begin{Schunk}
\begin{Sinput}
 statusbar <- gtkStatusbar()
\end{Sinput}
\end{Schunk}
Statusbars have a simple API. The \method{push} method, as used in the
definition of the callback \code{f}, is used to add new text to the
statusbar. The \code{pop} method reverts to the previous message.


Now we define a top-level window and attach a keyboard accelerator
group to the window so that when the window has the focus, the
specified keyboard shortcuts can be used.

\begin{Schunk}
\begin{Sinput}
 win <- gtkWindow(show=TRUE)
 win$setTitle("Window example")
 accelgroup = uimanager$getAccelGroup()  # add accel group
 win$addAccelGroup(accelgroup)
\end{Sinput}
\end{Schunk}


Now it is a simple matter of packing the widgets into a box.
\begin{Schunk}
\begin{Sinput}
 box <- gtkVBox()
 win$add(box)
 box$packStart(menubar, expand=FALSE, fill=FALSE,0)
 box$packStart(toolbar, expand=FALSE, fill= FALSE,0)
 contentArea = gtkVBox()
 box$packStart(contentArea, expand=TRUE, fill=TRUE,0)
 contentArea$packStart(gtkLabel("Content Area"))
 box$packStart(statusbar, expand=FALSE, fill=FALSE, 0)
\end{Sinput}
\end{Schunk}

The redo feature should only be sensitive to mouse events after a user
has undone an action and has not done another. To set the sensitivity
of a menu item is done through the \method{SetSensitive} method called
on the widget. We again retrieve the menuitem or toolbar item widgets
through their names.

\begin{Schunk}
\begin{Sinput}
 uimanager$getWidget("/menubar/Edit/EditRedo")$setSensitive(FALSE)
\end{Sinput}
\end{Schunk}
To reenable, use \code{TRUE} for the argument to \command{SetSensitive}

We can also use the \method{SetText} method on the menuitems. For
instance, instead of a generic ``Undo'' label, one might want to
change the text to list the most previous action.  The method is not
for the menu item though, but rather a \code{gtkLabel} which is the
first child. We use the list notation to access that.
\begin{Schunk}
\begin{Sinput}
 a <- uimanager$getWidget("/menubar/Edit/EditUndo")
 a[[1]]$setText("Undo add text")
\end{Sinput}
\end{Schunk}
\end{example}


\label{sec:RGtk2:dialogs}
\section{Dialogs}
\label{sec:dialogs}
\GTK\/ comes with a variety of dialogs to create simple, usually
single purpose, popup windows
for the user to interact with.

\subsection{The \code{gtkDialog} constructor}

The constructor \constructor{gtkDialog} creates a basic dialog box,
which is a display containing a top section with optionally an icon, a
message, and a secondary message. The bottom section, the action area,
shows buttons, such as \kbd{yes}, \kbd{no} and/or
\kbd{cancel}. The convenience functions
\constructor{gtkDialogNewWithButtons} and
\constructor{gtkMessageDialog} simplify the construction.

In \GTK\/ dialogs can be modal or not. Thre are a few ways to make a
dialog modal. The method window \method{setModal}{gtkWindow} will do
so, as will passing in a \code{modal} flag to some of the
constructors. These make other GUI elements inactive, but not the \R\/
session. Whereas, calling the \method{run}{gtkDialog} method,
will stop the flow until the dialog is dismissed, The return value can
then be inspected for the action, such as what button was
pressed. These values are from \code{GtkResponseType}, which lists
what can happen.


\paragraph{Basic message dialogs} The \constructor{gtkMessageDialog} has
an argument \argument{parent}{gtkMessageDialog}, to specify a parent
window the dialog should appear relative to. The
\argument{flags}{gtkMessageDialog} argument allows one to specify
values (from \code{GtkDialogFlags}) of \code{destroy-with-parent} or
\code{modal}. The \argument{type}{gtkMessageDialog} is used to specify
the message type, using a value in \code{GtkMessageType}. The
\argument{buttons}{gtkMessageDialog} is used to specify which buttons
will be drawn. The message is the following argument. The dialog has a
\code{secondary-text} property that can be set to give a secondary message.

\begin{Schunk}
\begin{Sinput}
 w <- gtkWindow()
 w['title'] <- "Parent window"
 dlg <- gtkMessageDialog(parent=w, flags="destroy-with-parent",
                         type="question", buttons="ok",
                         "My message")
 dlg['secondary-text'] <- "A secondary message"
 response <- dlg$run()
 if(response == GtkResponseType["cancel"] || # for other buttons
    response == GtkResponseType["close"] ||
    response == GtkResponseType["delete-event"]) {
   ## pass
 } else if(response == GtkResponseType["ok"]) {
   print("Ok")
 }
 dlg$Destroy()
\end{Sinput}
\end{Schunk}

\paragraph{Making your own dialogs} The \constructor{gtkDialog}
constructor returns a dialog object which can be customized for more
involved dialogs. In the example below, we illustrate how to make a
dialog to accept user input. We use the
\constructor{gtkDialogNewWithButtons}, which allows us to specify a
stock buttosn and a response value. We use standard responses, but could
have used custom ones by specifying a positive integer. The dialog is
a window object containing a box container, which is returned by the
\method{getVbox}{gtkDialog} method. This box has a separator and
button box packed in at the end, we pack in another box at the
beginning below to hold a label and our entry widget. 

When one of the buttons is clicked, the \signal{response} signal is
emitted by the dialog. We connect to this close the dialog.

\begin{Schunk}
\begin{Sinput}
 dlg <- gtkDialogNewWithButtons(title="Enter a value", 
                                parent=NULL, flags=0,
                                "gtk-ok", GtkResponseType["ok"],
                                "gtk-cancel", GtkResponseType["cancel"],
                                show=FALSE)
 g <- dlg$getVbox()                           # content area
 vg <- gtkVBox()
 vg['spacing'] <- 10
 g$packStart(vg)
 vg$packStart(gtkLabel("Enter a value"))
 entry <- gtkEntry()
 vg$packStart(entry)
 ID <- gSignalConnect(dlg, "response", f=function(dlg, resp, user.data) {
   if(resp == GtkResponseType["ok"])
     print(entry$getText())
   dlg$Destroy()
 })
 dlg$showAll()
 dlg$setModal(TRUE)
\end{Sinput}
\end{Schunk}

\subsection{File chooser}
\label{sec:RGtk2:file-chooser}

\GTK\/ has a \class{GtkFileChooser} backend to implement selecting a file
from the file system. The same widget allows one to open or save a
file and select or create
a folder (directory). The action is specified through one of the
\code{GtkFileChooserAction} flags.
This backend presented in various ways through
\constructor{gtkFileChooserDialog}, which pops up a modal dialog;
\constructor{gtkFileChooserButton}, which pops up the dialog when the
button is clicked; and \constructor{gtkFileChooserWidget}, which
creates a widget that can be placed in a GUI to select a file.

The dialog constructor allows one to spcify a title, a parent and an
action. In addition, the dialog buttons must be specified, as with the last
example using \code{gtkDialogNewWithButtons}. 

\begin{example}{An open file dialog}{ex:RGtk2:open-file}
An open file dialog can be created with:

\begin{Schunk}
\begin{Sinput}
 dlg <- gtkFileChooserDialog(title="Open a file", parent=NULL, action="open",
                             "gtk-ok", GtkResponseType["ok"],
                             "gtk-cancel", GtkResponseType["cancel"])
\end{Sinput}
\end{Schunk}

One can use the \code{run} method to make this modal or connect to the
\signal{response} signal. The file selected is found from the file
chooser method \method{getFilename}{gtkFileChooser}. One can enable
multiple selections, by passing
\method{setSelectMultiple}{gtkFileChooser} a \code{TRUE} value. In
this case, the \method{getFilenames}{gtkFileChooser} returns a list of filenames,

\begin{Schunk}
\begin{Sinput}
 ID <- gSignalConnect(dlg, "response", f=function(dlg, resp, data) {
   if(resp == GtkResponseType["ok"]) {
     filename <- dlg$getFilename()
     print(filename)
   }
   dlg$destroy()
 })
\end{Sinput}
\end{Schunk}

For the open dialog, one may wish to specify one or more filters, to narrow the
available files for selection. A filter object is returned by the
\constructor{gtkFileFilter} function.
This object is added to the file chooser, through its
\method{addFilter}{gtkFileChooser} method. The filter has a name
property set through the \method{setName}{gtkFileFilter} method. The
user can select a filter through a combobox, and this provides the
label. To use
the filter, one can add a pattern (\method{addPattern}{gtkFileFilter}),
a MIME type (\method{addMimeType}{gtkFileFilter}), or a custom filter. 

\begin{Schunk}
\begin{Sinput}
 fileFilter <- gtkFileFilter()
 fileFilter$setName("R files")
 dlg$addFilter(fileFilter)
 QT <- sapply(c("*.R", "*.Rdata"), 
              function(i) fileFilter$addPattern(i))
 QT <- sapply(c("text/plain"), 
              function(i) fileFilter$addMimeType(i))
\end{Sinput}
\end{Schunk}
\end{example}

The save file dialog is similar. The
\method{setFilename}{gtkFileChooser} can be used to specify a
default file and \method{setFolder}{gtkFileChooser} can specify an
initial directory. To be careful as to not overwrite an existing file, the
method \method{setDoOverwriteConfirmation}{gtkFileChooser} can be
passed a \code{TRUE} value.

\subsection{Date picker}
\label{sec:RGkt2:date-picker}

A calendar widget is produced by \constructor{gtkCalendar}. This
widget allows selection of a day, month or year. To specify these
values, the properties \code{day}, \code{month} ($0$-$11$), and
\code{year} store these values as integers. One can assign to these
directly, or use the methods
\method{selectDay}{gtkCalendar} and
\method{selectMonth}{gtkCalendar} (no select year method). The method
\method{getData}{gtkCalendar} returns a list with components for the
year, month and day. If there is no selection, the \code{day}
component is $0$.

The widget emits various signals when a selection is changed. The
\signal{day-selected} and \signal{day-selected-double-click} ones are
likely the most useful of these.







% \chapter{\code{cairoDevice}}
% \label{sec:RGtk2-cairoDevice}
% \SweaveInput{cairoDevice}

% \chapter{Using \code{glade} to design GUIs}
% \label{sec:RGtk2-glade}
% \SweaveInput{Glade}

%\section{End of chapter notes}
%\label{sec:RGtk2:end-of-chapter}

% * overview in inst
% * gtk tutorial in R
% * API at http://developer.gimp.org/api/2.0/gtk/
% * ...

% * widget gallery
% http://library.gnome.org/devel/gtk/stable/ch02.html

%  good source: http://developer.gnome.org/doc/GGAD/ggad.html

% * http://gtk.org/documentation.html

% * gtk API
% http://library.gnome.org/devel/gtk/stable/

% * pango manual 
% http://library.gnome.org/devel/pango/stable/PangoMarkupFormat.html

% * Missing discussion on standard dialogs: message, confirmations, etc.
% filebrowser, colors selector, font selector, 

% * Mention history: GIMP; GNOME; DTL RGtk; Michael Lawrence RGtk2
% * Installation: windows; linux; mac OS X

% * Info
% php cookbook good source of info

% DTL examples from omegahat

% pygtk tutorial easier to read than C one (no types specifed)

%%%%%%%%%%%%%%%%%%%%%%%%%%%%%%%%%%%%%%%%%%%%%%%%%%




% \chapter{RGtk2 Example: Extending the data frame class}
% \label{chap:RGtk2}
% \input{ch-RGtk2Example/ch-RGtk2Example}

% %% RwxWidgets
% \chapter{RwxWidgets}
% \label{chap:RwxWidgets}
% \input{ch-RwxWidgets/ch-RwxWidgets}

%% tcltk
%%\part{The \pkg{tcltk} package}
\label{chap:tcltk}
%% gWidgets introduction
 
%%\newcommand{\ONLYIN}[1]{[only in #1]}
\newcommand{\Event}[1]{$<$#1$>$}
\newcommand{\VirtualEvent}[1]{$<<$#1$>>$}

\XXX{add in tcl(``source''); external TCL packages}
\XXX{comment about scope -- no user data}


% makeIconTcltk(w, gifFile) {
%   if(as.character(tkwinfo("class", w)) == "Toplevel" &&
%      file.exists(giffile)) {
%     tkimage.create("photo","::icon::name", file=giffile)
%     tcl("wm","iconphoto", w, "::icon::name")
%   }
% }
\chapter{Tcl/Tk: Overview}
\label{sec:tcltk:overview}

%%\section{Overview}
\XXX{Where to put loading in external TCL source, packages, ...}
\XXX{tcl("update","idletasks")}
\XXX{use svMisc -- atleast comment on Parse, Complete, CompletePlus}



\TCL\/ (``tool command language'') is a scripting language and
interpreter of that language.  Originally developed in the late 80s by
John Ousterhout as a ``glue'' to combine two or more complicated
applications together, it evolved overtime to find use not just as
middleware, but also as a standalone development tool.

\TK~\footnote{
  %% Documentation sources
  \Tk{} has a well documented API~\citep{TclTk:Api}
  (\url{www.tcl.tk/man/tcl8.5}).  There are also several books
  to supplement. We consulted the one by Welch, Jones and
  Hobbs~\citep{beedub} often in the developement of this material. In
  addition, the Tk Tutorial of Mark Roseman~\citep{TclTk:Tutorial}
  (\url{www.tkdocs.com/tutorial}) provides much detail. \R{} specific
  documentation include two excellent R News
  articles and a proceedings paper~\citep{Rnews:Dalgaard:2001a},~\citep{Rnewse:Dalgaard:2002}
  and \citep{Dalgaard-DSC} by Peter Dalgaard, the
  package author. A set of examples due to James
  Wettenhall~\citep{Wettenhall} are also quite instructive. A main use
  of \pkg{tcltk} is within the \pkg{Rcmdr} framework. Writing
  extensions for that is well documented in an R News
  article~\citep{Rnews:Fox:2007} by John Fox, the package author.
}
is an extension of \TCL\/ that provides GUI components through
\TCL.  This was first developed in 1990, again by John
Ousterhout. \TK\/ quickly found widespread usage, as it made
programming GUIs for X11 easier and faster. Over the years, other
graphical toolkits have evolved and surpassed this one, but \TK\/
still has numerous users.

\TK\/ has a large number of bindings available for it,
e.g. \proglang{Perl}, \proglang{Python}, \proglang{Ruby}, and through
the \pkg{tcltk} package, \R. The \pkg{tcltk} package was developed by
Peter Dalgaard and included in \R\/ from version 1.1.0. Since then,
the package has been used in a number of GUI projects for \R, most
notably, the \pkg{Rcmdr} GUI. In addition, the \pkg{tcltk2} package
provides additional bindings and bundles in some useful external TCL
code. Our focus here is limited to the base \pkg{tcltk} package.

\TK\/ had a major change between versions 8.4 and 8.5, with the latter
introducing themed widgets. Many widgets were rewritten and their API
dramatically simplified. In \pkg{tcltk} there can be two different
functions to construct a similar widget. For example,
\function{tklabel} or \function{ttklabel}. The latter, with the
\code{ttk} prefix, corresponds to the newer themed variant of the
widget. We assume the \TK\/ version is 8.5 or higher, as this was a
major step forward.

Despite its limitations as a graphical toolkit, as compared to \GTK\/
or \Qt, the \TK\/ libraries are widely used for \R\/ GUIs, as for most
users there are no installation issues. As of version 2.7.0, \R\/ for
Windows has been bundled with the necessary \TK\/ version, so there are no
installation issues for that platform. For linux users, it is
typically trivial to install the newer libraries and for Mac OS X
users, the provided binary installations include the newer \TK\/ libraries.

\section{Interacting with \TCL}
\label{sec:tcltk:interacting-with-tcl}


%% Tclk
Although both are scripting languages, the basic syntax of \TCL\/ is a bit unlike \R. For
example a simple string assignment would be made at tclsh, the \TCL\/
shell with (using \code{\%} as a prompt):
\begin{verbatim}
% set x {hello world}
hello world
\end{verbatim}
Unlike \R\/ where braces are used to form blocks, this example shows
how \TCL\/ uses braces instead of quotes to group the words as a
single string. The use of braces, instead of quotes, in this example
is optional, but in general isn't, as expressions within braces are
not evaluated.  

The example above assigns to the variable \code{x} the
value of \code{hello world}. Once assignment has been made, one can
call commands on the value stored in \code{x} using the \code{\$}
prefix:
\begin{verbatim}
% puts $x
hello world
\end{verbatim}
The \code{puts} command, in this usage, simply writes back its argument to the terminal. Had
we used braces the argument would not have been substituted:
\begin{verbatim}
% puts {$x}
$x
\end{verbatim}

More typical within the \pkg{tcltk} package is the idea of a subcommand. For
example, the \code{string} command provides the subcommand
\code{length} to return the number of characters in the string.
\begin{verbatim}
% string length $x
11
\end{verbatim}

%% .Tcl
The \pkg{tcltk} package provides the low-level function \function{.Tcl} for direct
access to the \TCL\/ interpreter:
\begin{Schunk}
\begin{Sinput}
 library(tcltk)
 .Tcl("set x {some text}")               # assignment
\end{Sinput}
\begin{Soutput}
<Tcl> some text 
\end{Soutput}
\begin{Sinput}
 .Tcl("puts $x")                         # prints to stdout
\end{Sinput}
\end{Schunk}

%% must hard code this, as output isn't printed
\begin{Soutput}
some text
\end{Soutput}

\begin{Schunk}
\begin{Sinput}
 .Tcl("string length $x")                # call a command
\end{Sinput}
\begin{Soutput}
<Tcl> 9 
\end{Soutput}
\end{Schunk}

the \dfn{\function{.Tcl}} function simply sends a command as a text
string to the \TCL\/ interpreter and returns the result as an object
of class \dfn{\class{tclObj}} (cf. \code{?.Tcl}).  The \function{.Tcl}
function can be used to read in \TCL\/ scripts as with
\code{.Tcl("source filename")}. This allowing arbitrary \TCL\/ scripts
to run within an \R\/ session. \TCL\/ packages may be read in with
\function{tclRequire}.



The \class{tclObj} objects print with the leading \code{<Tcl>}. The string representation
of objects of class \class{tclObj} is returned by \function{tclvalue}
or by coercion through the \function{as.character} function. They
differ in how they treat spaces and new lines.  Conversion to numeric
values is also possible through \code{as.numeric}, but conversion to
logical may require two steps as only modes \code{character},
\code{double} or \code{integer} are stored. In general though,
conversion of complicated \TCL\/ expressions is not supported.

To simplify coercion to logical, we define a new method:
\begin{Schunk}
\begin{Sinput}
 as.logical.tclObj <- function(x, ...) as.logical(as.numeric(x))
\end{Sinput}
\end{Schunk}


The \TK\/ extensions to \TCL\/ have a complicated command structure,
and thankfully, \pkg{tcltk} provides some more conveniently named
functions. To illustrate, the \TCL\/ command to set the text value for
a label object (\code{.label}) would look like
\begin{verbatim}
% .label configure -text "new text"
\end{verbatim}
The \pkg{tcltk} package provides a corresponding function
\code{tkconfigure}. The above would be done in an \R-like way as (assuming \code{l} is a
label object):


\begin{Schunk}
\begin{Sinput}
 tkconfigure(l, text="new text")
\end{Sinput}
\end{Schunk}


% Although the \TCL\/ statement appears to have the object-oriented form
% of ``object method arguments,'' behind the scenes \TCL\/ creates a
% command with the same name as the widget with \code{configure} as a
% subcommand. This is followed by options passed in using the form
% \texttt{-key value}.  


The \TK\/ API for \code{ttklabel}'s \code{configure} subcommand is

\begin{quotation}
  \textit{pathName} \textbf{configure} \textit{?option? ?value option value ...?}
\end{quotation}

The \textit{pathName} is the ID of the label widget. This can be found
from the object \code{l} above, in \code{l\$ID}, or in some cases is a
return value of some other command call.  In the \TK\/ documentation
paired question marks indicate optional values. In this case, one can
specify nothing, returning a list of all options; just an option, to
query the configured value; the option with a value, to modify the
option; and possibly do more than one at at time.  For commands such
as \code{configure}, there will usually correspond a function in
\R\/ of the same name with a \code{tk} prefix, as in this case
\function{tkconfigure}.~\footnote{The package \pkg{tcltk} was written before
namespaces were implemented in \R, so the ``tk'' prefix serves that role.}

To make consulting the \TK\/ manual pages easier in the text we would
describe the configure subcommand as
\subcommanda{configure}{ttklabel}{[options]}. (The \R\/ manual pages
simply redirect you to the original \TK\/ documentation, so
understanding this is important for reading the API.) However, if such
a function shortcut is present, we will typically use the shortcut when we
illustrate code. 

Some subcommands have further subcommands. An example
is to set the selection. In the \R\/ function, the second command is
appended with a dot, as in \code{tkselection.set}. (There are a few
necessary exceptions to this.)

\paragraph{The tcl function} Within \pkg{tcltk}, the \function{tkconfigure} function is defined by

\begin{Sinput}
function(widget, ...) tcl(widget, "configure", ...)
\end{Sinput}

The \dfn{\function{tcl}} function is the workhorse used to piece
together \TCL\/ commands, call the interpreter, and then return an
object of class \code{tclObj}.  Behind the scenes it turns an \R\/
object, \code{widget}, into the \textit{pathName} above (using its ID
component). It converts \R\/ \code{key=value} pairs into \code{-key
  value} options for \TCL. As named arguments are only for the
\code{-key value} expansion, we follow the \TCL\/ language and call
the arguments ``options'' in the following. Finally, it adjusts any
callback functions. The \function{tcl} function uses position to
create its command. The order of the subcommands needs to match that
of the \TK\/ API, so although it is true that often the \R\/ object is
first, this is not always the case.

\begin{figure}
  \centering
\begin{verbatim}
          tcl(widget, subcommand, key=value, callback)
             /            |           |           \
          widget$ID  subcommand   -key value   makeCallback
\end{verbatim}
  \caption{How the \code{tcl} function maps its arguments}
  \label{fig:tcl-function-map}
\end{figure}






%% constructors

\section{Constructors}
\label{sec:tcltk:constructors}

In this Chapter, we will stick to a few basic widgets: labels and
buttons; and top-level containers to illustrate the basic usage of
\pkg{tcltk}, leaving for later more detail on containers and widgets.

Unlike \GTK, say, the construction of widgets in \pkg{tcltk} is linked
to the widget hierarchy. \TK\/ widgets are constructed as children of
a parent container with the parent specified to the constructor. When
the \TK\/ shell, wish, is used or the \TK\/ package is loaded through
the \TCL\/ command \code{package require Tk}, a top level window named
``\code{.}'' is created. In the variable name \code{.label}, from
above, the dot refers to the top level window.  In \pkg{tcltk} a
top-level window is created separately through the
\constructor{tktoplevel} constructor, as with
\begin{Schunk}
\begin{Sinput}
 w <- tktoplevel()
\end{Sinput}
\end{Schunk}

Top-level windows will be explained in more detail in
Chapter~\ref{sec:tcltk:basic-containers}. For now, we just use one to
be a parent container for a label widget. Like all constructors but
the one for toplevel windows, the label constructor
(\constructor{ttklabel}) requires a specification of the parent
container followed by any other options that are desired. A typical
invocation would look like:
\begin{Schunk}
\begin{Sinput}
 l <- ttklabel(w, text="label text")
\end{Sinput}
\end{Schunk}

To see the label, we can pack it into the toplevel window, which was
used as its parent at construction.
\begin{Schunk}
\begin{Sinput}
 tkpack(l)
\end{Sinput}
\end{Schunk}

\paragraph{Options}
The first argument of a widget constructor is the parent container,
subsequent arguments are used to specify the options for the
constructor given as \code{key=value} pairs. The \TK\/ API lists these
options along with their description.

For a simple label, the following options are possible: \code{anchor},
\code{background}, \code{font}, \code{foreground}, \code{justify},
\code{padding}, \code{relief}, \code{text}, and \code{wraplength}.
This is in addition to the standard options \code{class},
\code{compound}, \code{cursor}, \code{image}, \code{state},
\code{style}, \code{takefocus}, \code{text}, \code{textvariable},
\code{underline}, and \code{width}. (Although clearly lengthy, this
list is significantly reduced from the options for \code{tklabel}
where options for the many style properties are also included.)

Many of the options are clear from their name.  The main option,
\code{text}, takes a character string. The label will be multiline if
this contains new line characters.  The \argument{padding}{ttklabel}
argument allows the specification of space in pixels between the text
of the label and the widget boundary. This may be set as four values
\code{c(left, top, right, bottom)}, or fewer, with \code{bottom}
defaulting to \code{top}, \code{right} to \code{left} and \code{top}
to \code{left}. The \argument{relief}{ttklabel} argument specifies how
a 3-d effect around the label should look, if specified. Possible
values are \qcode{flat}, \qcode{groove}, \qcode{raised},
\qcode{ridge}, \qcode{solid}, or \qcode{sunken}.

\paragraph{The functions tkconfigure, tkcget}

Option values may be set through the constructor, or adjusted
afterwards by \function{tkconfigure}. A listing (in \TCL\/ code) of possible options
that can be adjusted may be seen by calling \function{tkconfigure}
with just the widget as an argument.

\begin{Schunk}
\begin{Sinput}
 head(as.character(tkconfigure(l)))      # first 6 only
\end{Sinput}
\begin{Soutput}
[1] "-background frameColor FrameColor {} {}"   
[2] "-foreground textColor TextColor {} {}"     
[3] "-font font Font {} {}"                     
[4] "-borderwidth borderWidth BorderWidth {} {}"
[5] "-relief relief Relief {} {}"               
[6] "-anchor anchor Anchor {} {}"               
\end{Soutput}
\end{Schunk}

The \function{tkcget} function returns the value of an
option (again as a \code{tclObj} object). The option can be specified
two different ways. Either using the \TK\/ style of a leading dash or
using the \R{} convention that \code{NULL} values mean to return the value,
and not set it.


\begin{Schunk}
\begin{Sinput}
 tkcget(l, "-text")                      # retrieve text property
\end{Sinput}
\begin{Soutput}
<Tcl> label text 
\end{Soutput}
\begin{Sinput}
 tkcget(l, text=NULL)                    # alternate syntax
\end{Sinput}
\begin{Soutput}
<Tcl> label text 
\end{Soutput}
\end{Schunk}

\paragraph{Coercion to character}
As mentioned, the \code{tclObj} objects can be coerced to character class two ways.
The conversion through \code{as.character} breaks the return value along whitespace:
\begin{Schunk}
\begin{Sinput}
 as.character(tkcget(l, text=NULL))
\end{Sinput}
\begin{Soutput}
[1] "label" "text" 
\end{Soutput}
\end{Schunk}
%
Whereas, conversion by the \function{tclvalue} function does not:
\begin{Schunk}
\begin{Sinput}
 tclvalue(tkcget(l, text=NULL))
\end{Sinput}
\begin{Soutput}
[1] "label text"
\end{Soutput}
\end{Schunk}
%


\paragraph{Buttons}
Buttons are constructed using the \constructor{ttkbutton} constructor.
\begin{Schunk}
\begin{Sinput}
 b <- ttkbutton(w, text="click me", 
                command=function() print("thanks"))
\end{Sinput}
\end{Schunk}

Buttons and labels share many of the same options. However, in
addition, buttons have a \argument{command}{ttkbutton} option to
specify a callback for when the button is invoked. Buttons may be
invoked by clicking and releasing the mouse on the button, by pressing
the space bar when the button has the focus or by calling the
\subcommand{invoke}{ttkbutton} subcommand. In the above example, clicking on the
button will call the function causing a simple message to be printed. More on callbacks in \pkg{tcltk}
will be explained in Section~\ref{sec:tcltk:callbacks}.

\paragraph{The tkwidget function}
Constructors call the \function{tkwidget} function which returns an
object of class \code{tkwin}. (In \TK\/ the term ``window'' is used to
refer to the drawn widget and not just a top-level window)

\begin{Schunk}
\begin{Sinput}
 str(b)
\end{Sinput}
\begin{Soutput}
List of 2
 $ ID : chr ".2.2"
 $ env:<environment: 0x100da8b50> 
 - attr(*, "class")= chr "tkwin"
\end{Soutput}
\end{Schunk}

The returned widget objects are lists with two components: an ID and an
environment. The \code{ID} component keeps a unique ID of the
constructed widget. This is a character string, such as ``.1.2.1''
coming from the the widget hierarchy of the object. This value is
generated behind the scenes by the \pkg{tcltk} package using numeric
values to keep track of the hierarchy. The \code{env} component
contains an environment that keeps a count of the subwindows, the parent
window and any callback functions. This helps ensure that any copies
of the widget refer to the same object~\citep{Dalgaard-DSC}. As the
construction of a new widget requires the \code{ID} and environment of
its parent, the first argument to \function{tkwidget}, \code{parent},
must be an \R\/ \TK\/ object, not simply its character ID, as is
possible for the \function{tcl} function.

%%The latter is useful in
%%a callback, as only the ID may be known to the callback function.


\paragraph{State of themed widgets}

The themed widgets (those with a \code{ttk} constructor) have a state
to determine which style is to be applied when painting the
widget. These states can be adjusted through the \code{state} command
and queried with the \code{instate} command. For example, to see if
button widget \code{b} has the focus we have:
\begin{Schunk}
\begin{Sinput}
 as.logical(tcl(b, "instate", "focus"))
\end{Sinput}
\begin{Soutput}
[1] FALSE
\end{Soutput}
\end{Schunk}
To set a widget to be not sensitive to user input we have:
\begin{Schunk}
\begin{Sinput}
 tcl(b, "state", "disabled")             # not sensitive
\end{Sinput}
\begin{Soutput}
<Tcl> !disabled 
\end{Soutput}
\end{Schunk}
The states are bits and can be negated by prefacing the value with \code{!}:
\begin{Schunk}
\begin{Sinput}
 tcl(b, "state", "!disabled")            # sensitive again
\end{Sinput}
\begin{Soutput}
<Tcl> disabled 
\end{Soutput}
\end{Schunk}

The full list of states is in the manual page for \code{ttk::widget}.

\subsection{Geometry managers}
\label{sec:tcltk:overview:geometry-managers}

As with \Qt, when a new widget is constructed it is not automatically
mapped. \TK\/ uses geometry managers to specify how the widget will be
drawn within the parent container. We will discuss two such geometry
managers in Chapter~\ref{sec:tcltk:basic-containers}, but for now, we note
that the simplest way to view a widget in its parent window is through
\function{tkpack}, as in:
\begin{Schunk}
\begin{Sinput}
 tkpack(b)
\end{Sinput}
\end{Schunk}

This command packs the widgets into the top-level window (the parent
in this case) in a box-like manner. Unlike \GTK\/ more than one child
can be packed into a top-level window, although we don't demonstrate
this further, as later we will use an intermediate \function{ttkframe}
box container so that themes are properly displayed.

\subsection{\TCL\/ variables}
\label{sec:tcltk:overview:textvariables}


%% textvariables
For the button and label widgets, there is an option
\code{textvariable} to specify a \TCL\/ variable to store the text
property. These variables are dynamically bound to the widget, so that
changes to the variable are propagated to the GUI. (The \TCL\/
variable is a model and the widget a view of the model.)  Other
widgets allow for tcl variables to be used for this and other purposes
The basic functions involved are \function{tclVar} to create a \TCL\/
variable, \function{tclvalue} to get the assigned value and
\function{tclvalue\ASSIGN} to modify the value.

\begin{Schunk}
\begin{Sinput}
 textvar <- tclVar("another label")
 l2 <- ttklabel(w, textvariable=textvar)
 tkpack(l2)
 tclvalue(textvar)
\end{Sinput}
\begin{Soutput}
[1] "another label"
\end{Soutput}
\begin{Sinput}
 tclvalue(textvar) <- "new text"         
\end{Sinput}
\end{Schunk}

\TCL\/ variables have a unique identifier, returned by \function{as.character}:
\begin{Schunk}
\begin{Sinput}
 as.character(textvar)
\end{Sinput}
\begin{Soutput}
[1] "::RTcl1"
\end{Soutput}
\end{Schunk}

The advantages of \TCL\/ variables are like those of the MVC paradigm
-- a single data source can have its changes propagated to several
widgets automatically. If the same text is to appear in different
places, their usage is recommended.  One disadvantage, is that in a
callback, the variable is not passed to the callback and must be found
through \R's scoping rules. (In Section~\ref{sec:tcltk:checkboxes} we
show a workaround.)

The package also provides the function \dfn{\function{tclArray}} to
store an array of \TCL\/ variables. The usual list methods \code{[[}
and \code{\$} and their forms for assignment are available for arrays,
but values are only referred to by name, not index:



\begin{Schunk}
\begin{Sinput}
 x <- tclArray()                    # no init
 x$one <- 1; x[[2]] <- 2            # $<- and [[<-
 x[[1]]                             # no match by index
\end{Sinput}
\begin{Soutput}
NULL
\end{Soutput}
\begin{Sinput}
 names(x)                           # the stored names
\end{Sinput}
\begin{Soutput}
[1] "2"   "one"
\end{Soutput}
\begin{Sinput}
 x[['2']]                           # match by name, not index
\end{Sinput}
\begin{Soutput}
<Tcl> 2 
\end{Soutput}
\end{Schunk}


\subsection{Window properties and state: \code{tkwinfo}}
\label{sec:tcltk:overview:widget-properties}

For a widget, the function \function{tkcget} is used to get the values
of its options. If it is a themed widget, the \code{instate} command
returns its state values. Finally, there is also \function{tkwinfo} to
return the properties of the containing window of the widget.  When
widgets are mapped, the ``window'' they are rendered to has properties,
such as a class or size. If the API is of the form

\begin{quotation}
\texttt{winfo} \textbf{subcommand\_name} \textit{window}  
\end{quotation}
the specification to \function{tkwinfo} is in the same order (the
widget is not the first argument). For instance, the
class~\footnote{The class of a widget is more like a attribute and should
  not be confused with class in the object oriented sense. The class
  is used internally for bindings and styles.} of a label
is returned by the \texttt{class} subcommand:

\begin{Schunk}
\begin{Sinput}
 tkwinfo("class", l)
\end{Sinput}
\begin{Soutput}
<Tcl> TLabel 
\end{Soutput}
\end{Schunk}
%

The window, in this example \texttt{l}, can be specified as an \R\/
object, or by its character ID. This is useful, as the return value of
some functions is the ID. For instance, the \texttt{children}
subcommand returns IDs. Below the \code{as.character} function will
coerce these into a vector of IDs.

\begin{Schunk}
\begin{Sinput}
 (children <- tkwinfo("children", w))
\end{Sinput}
\begin{Soutput}
<Tcl> .2.1 .2.2 .2.3 
\end{Soutput}
\begin{Sinput}
 sapply(as.character(children), function(i) tkwinfo("class", i))
\end{Sinput}
\begin{Soutput}
$`.2.1`
<Tcl> TLabel 

$`.2.2`
<Tcl> TButton 

$`.2.3`
<Tcl> TLabel 
\end{Soutput}
\end{Schunk}

There are several possible subcommands, here we list a few. The
\subcommand{geometry}{tkwinfo} sub command returns the location and
size of the widgets' window in the form \code{width x height + x + y};
the sub commands \subcommand{height}{tkwinfo},
\subcommand{width}{tkwinfo}, \subcommand{x}{tkwinfo}, or
\subcommand{y}{tkwinfo} can be used to return just those parts. The
\subcommand{exists}{tkwinfo} command returns 1 (\code{TRUE}) if the
window exists and 0 otherwise; the \subcommand{ismapped}{tkwinfo} sub
command returns 1 or 0 if the window is currently mapped (drawn); the
\subcommand{viewable}{tkwinfo} sub command is similar, only it checks
that all parent windows are also mapped.  For traversing the widget
hierarchy, one has available the \subcommand{parent}{tkwinfo} sub
command which returns the immediate parent of the component,
\subcommand{toplevel}{tkwinfo} which returns the ID of the top-level
window, and \subcommand{children}{tkwinfo} which returns the IDs of
all the immediate child components, if the object is a container, such
as a top-level window.


\subsection{Themes}
\label{sec:tcltk:overview:themes}


%% themes -- ttkframe
As mentioned, the newer themed widgets have a style that determines
how they are drawn based on the state of the widget. The separation of
style properties from the widget, as opposed to having these set for
each construction of a widget, makes it much easier to change the look
of a GUI and easier to maintain the code. A collection of styles makes
up a theme. The available themes depend on the system. The default
theme allows the GUI to have the native look and feel of the operating
system. (This was definitely not the case for the older \TK\/
widgets.) There is no built in command to return the theme, so we use
\code{.Tcl} to call the appropriate \TCL\/ command. The \code{names}
sub command will return the available themes:

\begin{Schunk}
\begin{Sinput}
 .Tcl("ttk::style theme names")
\end{Sinput}
\begin{Soutput}
<Tcl> clam alt default classic 
\end{Soutput}
\end{Schunk}
%
The \code{use} sub command is used to set the theme:
\begin{Schunk}
\begin{Sinput}
 .Tcl("ttk::style theme use clam")
\end{Sinput}
\end{Schunk}

The writing of themes will not be covered, but in
Example~\ref{ex-tcltk-toolbar} we show how to create a new style for a
button. This topic is covered in some detail in the \Tk\/ tutorial by Roseman.
\\

The example we have shown so far, would not look quite right for some
operating systems, as the toplevel window is not a themed widget
(Figure~\ref{fig:tcltk-frame-noframe-example}). To work around that, a
\function{ttkframe} widget is usually used to hold the child
components of the top-level window. The following shows how to place a
frame inside the window, with some arguments to be explained later
that allow it to act reasonably if the window is resized.

\begin{figure}
  \centering
  \includegraphics[width=.4\textwidth]{fig-tcltk-no-frame}
  \includegraphics[width=.4\textwidth]{fig-tcltk-with-frame}
  \caption{Similar GUIs, one using a frame within the toplevel window
    (right one) and one without. The one without has widgets whose background does not match the toplevel window.}
  \label{fig:tcltk-frame-noframe-example}
\end{figure}




\begin{Schunk}
\begin{Sinput}
 w <- tktoplevel()
 f <- ttkframe(w, padding=c(3,3,12,12))  # Some breathing room
 tkpack(f, expand=TRUE, fill="both")     # For window expansion
 l <- ttklabel(f,  text="label")         # some widget
 tkpack(l)
\end{Sinput}
\end{Schunk}


\subsection{Colors and fonts}
\label{sec:tcltk:overview:colors-fonts}
Colors and fonts are typically specified through a theme, but at times
it is desirable to customize the preset ones.

The label color can be set through its \code{foreground}
property. Colors can be specified by name -- for common colors -- or
by hex RGB values which are common in web programming.
\begin{Schunk}
\begin{Sinput}
 tkconfigure(l, foreground="red")
 tkconfigure(l, foreground="#00aa00")
\end{Sinput}
\end{Schunk}

To find the hex RGB value, one can use the \code{rgb} function to
create RGB values from intensities in $[0,1]$.  The \R\/ function
\function{col2rgb} can translate a named color into RGB values. The
\code{as.hexmode} function will display an integer in hexadecimal
notation.

In Example~\ref{ex:tcltk-entry-initial-message} we show how to modify
a style, as opposed to the \code{foreground} option, to change the
text color in an entry widget.

%% fonts
\paragraph{Fonts}
Fonts are a bit more involved than colors. \TK\/ version 8.5 made it
more difficult to change font properties of individual widgets, this
following the practice of centralizing style options for consistency,
ease of maintaining code and ease of theming.  To set a font for a
label, rather than specifying the font properties, one configures the
\code{font} options using a pre-defined font name, such as
\begin{Schunk}
\begin{Sinput}
 tkconfigure(l, font="TkFixedFont")
\end{Sinput}
\end{Schunk}

The \qcode{TkFixedFont} value is one of the standard font names, in
this case to use a fixed-width font. A complete list of the standard
names is provided in Table~\ref{tab:tcltk-std-fonts}. Each theme sets
these defaults accordingly.
\begin{table}
\centering
\label{tab:tcltk-std-fonts}
\caption{Standard font names defined by a theme.}
\begin{tabular}{@{}ll@{}}
\toprule

Standard font name&Description\\
\midrule
\code{TkDefaultFont}&Default font for all GUI items not otherwise specified\\\code{TkTextFont}&Font for text widgets\\\code{TkFixedFont}&Fixed-width font\\\code{TkMenuFont}&Menu bar fonts\\\code{TkHeadingFont}&Font for column headings\\\code{TkCaptionFont}&Caption font (dialogs)\\\code{TkSmallCaptionFont}&Smaller caption font\\\code{TkIconFont}&Icon and text font
\\ \bottomrule
\end{tabular}
\end{table}%
\paragraph{tkfont.create}
The \function{tkfont.create} function can be used to create a new font, as with the following commands:
\begin{Schunk}
\begin{Sinput}
 tkfont.create("ourFont", family="Helvetica", size=12, 
               weight="bold")
\end{Sinput}
\begin{Soutput}
<Tcl> ourFont 
\end{Soutput}
\begin{Sinput}
 tkconfigure(l, font="ourFont")
\end{Sinput}
\end{Schunk}

As font families are system dependent, only \qcode{Courier},
\qcode{Times} and \qcode{Helvetica} are guaranteed to be there. A list
of an installation's available font families is returned by the
function \function{tkfont.families}.
Figure~\ref{fig:fig-tcltk-all-fonts} shows a display of some available
font families on a Mac OS X machine.  See
Example~\ref{ex:tcltk-scrollable-frame} for details.

The arguments for \function{tkfont.create} are optional. The
\argument{size}{tkfont.create} argument specifies the pixel size. The
\argument{weight}{tkfont.create} argument can be used to specify
\qcode{bold} or \qcode{normal}.  Additionally, a
\argument{slant}{tkfont.create} argument can be used to specify either
\qcode{roman} (normal) or \qcode{italic}. Finally,
\argument{underline}{tkfont.create} and
\argument{overstrike}{tkfont.create} can be set with a \code{TRUE} or
\code{FALSE} value.


\begin{figure}
  \centering
  \includegraphics[width=.6\textwidth]{fig-tcltk-all-fonts.png}
  \caption{A scrollable frame widget (cf. Example~\ref{ex:tcltk-scrollable-frame}) showing the available fonts on a system.}
  \label{fig:fig-tcltk-all-fonts}
\end{figure}


\paragraph{Font metrics}
The average character size is important in setting the width and
height of some components. (For example the text widget specifies its
height in lines, not pixels.) These sizes can be found using the
\function{tkfont.measure} and \function{tkfont.metrics} functions as
follows:
\begin{Schunk}
\begin{Sinput}
 chars <- c(0:9,LETTERS, letters)
 tmp <- tkfont.measure("TkTextFont",paste(chars,collapse=""))
 fontWidth <- ceiling(as.numeric(tclvalue(tmp))/length(chars))
 tmp <- tkfont.metrics("TkTextFont","linespace"=NULL)
 fontHeight <- as.numeric(tclvalue(tmp))
 c(width=fontWidth, height=fontHeight)
\end{Sinput}
\begin{Soutput}
 width height 
     8     15 
\end{Soutput}
\end{Schunk}


\subsection{Images}
\label{sec:tcltk:overview:images}


Many \pkg{tcltk} widgets, including both labels and buttons, can show
images. In these cases, either with or without an accompanying text
label. Constructing images to display is similar to constructing new
fonts, in that a new image object is created and can be reused by
various widgets. Images are created by the \function{tkimage.create}
function. 

The following command shows how an image object can be made from the
file \code{tclp.gif} in the current directory:

\begin{Schunk}
\begin{Sinput}
 tkimage.create("photo", "::img::tclLogo", file = "tclp.gif")
\end{Sinput}
\begin{Soutput}
<Tcl> ::img::tclLogo 
\end{Soutput}
\end{Schunk}


The first argument, \qcode{photo} specifies that a full color image is
being used. (This option could also be \qcode{bitmap} but that is more
a legacy option.) The second argument specifies the name of the
object. We follow the advice of the \TK\/ manual and preface the name
with \code{::img::} so that we don't inadvertently overwrite any
existing \TCL\/ commands. (The command \code{tcl("image", "names")}
will return all defined image names.) The third argument
\argument{file}{tkimage.create} specifies the graphic file. The basic
\TK\/ \code{image} command can only show GIF and PPM/PNM
images. Unfortunately, not many \R\/ devices output in these
formats. (The \code{GDD} device driver can.) One may need system
utilities to convert to the allowable formats or install add-on \TCL\/
packages that can display other formats.

To use the image, one specifies the image name to the
\option{image}{ttklabel} option:
\begin{Schunk}
\begin{Sinput}
 l <- ttklabel(w, image="::img::tclLogo", text="logo text", 
               compound = "top")
\end{Sinput}
\end{Schunk}

By default the text will not show. The \argument{compound}{ttklabel}
argument takes a value of either \qcode{text}, \qcode{image}
(default), \qcode{center}, \qcode{top}, \qcode{left}, \qcode{bottom},
or \qcode{right} specifying where the label is in relation to the
text.

\paragraph{Image manipulation}
Once an image is created, there are several options to manipulate the
image. These are found in the \TK\/ man page for \code{photo}, not
\code{image}. For instance, to change the palette so that instead of
\code{fullcolor} only 16 shades of gray are used to display the icon,
one could issue the command
\begin{Schunk}
\begin{Sinput}
 tkconfigure("::img::tclLogo", palette=16)
\end{Sinput}
\end{Schunk}

%% changed to active, !active
% Another useful manipulation to draw attention to an image is to change
% the \code{gamma} value when something happens, such as a mouse-over
% event (cf. Example~\ref{ex-tcltk-toolbar}).



\section{Events and Callbacks}
\label{sec:tcltk:overview:events-callbacks}

The button widget has the \code{command} option for assigning a
callback which is invoked (among other ways) when the user clicks the
mouse on the button. In addition to such commands, one may use
\function{tkbind} to invoke callbacks in response to many other events
that the user may initiate.

The basic call is \code{tkbind(tag, event, script)}. 

%% (Unlike \pkg{RGtk2}, that the event triggers a widget to emit a signal that a callback listens for is notneeded)

\subsection{The tag}

The \code{tag} object is more general than just a widget, or its
id. It can be:

\begin{description}
\item[the name of a widget,] in which case the command will be bound to that widget;
\item[a top-level window,] in which case the command will be be bound
  to the event for the window and all its internal widgets;
\item[a class of widget,] such as \qcode{TButton}, in which case all
  such widgets will get the binding; or
\item[the value \qcode{all},] in which case all widgets in the
  application will get the binding.
\end{description}

This flexibility makes it easy to create keyboard accelerators. For
example, the following mimics the linux shortcut \code{Control-q} to
close a window.
\begin{Schunk}
\begin{Sinput}
 w <- tktoplevel()
 l <- ttkbutton(w, text="Some widget with the focus"); tkpack(l)
 tkbind(w, "<Control-q>", function() tkdestroy(w))
\end{Sinput}
\end{Schunk}

By binding to the top-level window, we ensure that no matter which
widget has the focus the command will be invoked by the keyboard shortcut.


\subsection{Events}
\label{sec:tcltk:events}

%% possible events
%% keys
The possible events (or sequences of events) vary from widget to
widget. The events can be specified in a few ways. 

The example below uses two types of events. A single key press event, such as
``C'' or ``O'' can invoke a command and is specified by just its
character. Whereas, the event of pressing the return key is specified
using \code{\event{Return}}. In the following we bind the key presses to the
top-level window and the return event to any button with the default
class \code{TButton}.

\begin{Schunk}
\begin{Sinput}
 w <- tktoplevel()
 l <- ttklabel(w, text="Click Ok for a message")
 b1 <- ttkbutton(w, text="Cancel", command=function() tkdestroy(w))
 b2 <- ttkbutton(w, text="Ok", command=function() {
   print("initiate an action")
 })
 sapply(list(l,b1,b2), tkpack)
 tkbind(w, "C", function() tcl(b1, "invoke"))
 tkconfigure(b1, underline=0)
 #
 tkbind(w, "O", function() tcl(b1, "invoke"))
 tkconfigure(b2, underline=0)
 tkfocus(b2)
 #
 tkbind("TButton", "<Return>", function(W) {
   tcl(W, "invoke")
 })
\end{Sinput}
\end{Schunk}
%
We modified our buttons using the \code{underline} option to give the
user an indication that the ``C'' and ``O'' keys will initiate some
action. Our callbacks simply cause the appropriate button to
\code{invoke} their command. The latter one uses a percent
substitution (below), which is how \TK\/ passes along information
about the event to the callback. 

%% tkbind(widget,''<modifier-modifier-type-detail'>'', command)
\paragraph{Events with modifiers}
More complicated events can be described with the pattern

\begin{quotation}
\code{\Event{modifier-modifier-type-detail}}.   
\end{quotation}

Examples of a ``type'' are \code{\Event{KeyPress}} or
\code{\Event{ButtonPress}}. The event \code{\Event{Control-q}}, used
above, has the type \code{q} and modifier \code{Control}. Whereas
\code{\Event{Double-Button-1}} also has the detail \code{1}. The full
list of modifiers and types are described in the man page for
\code{bind}. Some familiar modifiers are \code{Control}, \code{Alt},
\code{Button1} (also \code{B1}), \code{Double} and \code{Triple}. The
event types are the standard X event types along with some
abbreviations. These are also listed in the \code{bind} man page. Some
commonly used ones are \code{Return} (as above), \code{ButtonPress},
\code{ButtonRelease}, \code{KeyPress}, \code{KeyRelease},
\code{FocusIn}, and \code{FocusOut}.

\paragraph{Window manager events}
Some events are based on window manager events. The \code{\Event{Configure}}
event happens when a component is resized. The \code{\Event{Map}} and
\code{\Event{Unmap}} events happen when a component is drawn or undrawn.

\paragraph{Virtual events}
Finally, the event may be a ``virtual event.'' These are represented
with \code{\VirtualEvent{EventName}}. There are predefined virtual
events listed in the \code{event} man page. These include
\code{\VirtualEvent{MenuSelect}} when working with menus,
\code{\VirtualEvent{Modified}} for text widgets,
\code{\VirtualEvent{Selection}} for text widgets, and
\code{\VirtualEvent{Cut}}, \code{\VirtualEvent{Copy}} and
\code{\VirtualEvent{Paste}} for working with the clipboard. New
virtual events can be produced with the \code{tkevent.add}
function. This takes at least two arguments, an event name and a
sequence that will initiate that event. The \code{event} man page has
these examples coming from the Emacs world:
\begin{Schunk}
\begin{Sinput}
  tkevent.add("<<Paste>>", "<Control-y>")
  tkevent.add("<<Save>>", "<Control-x><Control-s>")
\end{Sinput}
\end{Schunk}
%
In addition to virtual events occurring when the sequence is performed,
the \function{tkevent.generate} can be used to force an event for a
widget. This function requires a widget (or its ID) and the event
name. Other options can be used to specify substitution values,
described below. To illustrate, this command will generate the
\code{\VirtualEvent{Save}} event for the button \code{b}:
\begin{Schunk}
\begin{Sinput}
 tkevent.generate(b, "<<Save>>")
\end{Sinput}
\end{Schunk}
%



\subsection{Callbacks}
\label{sec:tcltk:callbacks}
%% \XXX{ use of tcl(``eval'',''break'') to avoid calling subsequent callbacks} -- doesn't work
%% See PD's comments here on callbacks http://article.gmane.org/gmane.comp.lang.r.general/136705

The \pkg{tcltk} package implements callbacks in a manner different
from \TK, as the callback functions are \R\/ functions, not \TK\/
procedures. This is much more convenient, but introduces some slight
differences.  In \pkg{tcltk} these callbacks can be expressions
(unevaluated calls) or functions. We use only the latter, for more
clarity. The basic callback function need not have any arguments and
those that do only have percent substitutions passed in.


%% scope of callback?
The callback's return value is generally not important, although we
shall see that within the validation framework of entry widgets
(Section~\ref{sec:tcltk:entry-widgets}) it can matter.~\footnote{The
  difference in processing of return values can make porting some
  \Tk\/ code to \pkg{tcltk} difficult. For example, the \code{break}
  command to stop a chain of call backs does not work.}



In \pkg{tcltk} only one callback can be associated with a widget and
event through the call
\code{tkbind(widget,event,callback)}. (Although, callbacks for the
widget associated with classes or toplevel windows can differ.)
Calling \code{tkbind} another time will replace the callback. To
remove a callback, simply assign a new callback which does
nothing.~\footnote{This event handling can prevent being able to port
  some \Tk\/ code into \pkg{tcltk}. In those cases, one may consider
  sourcing in \Tcl\/ code directly.}



\subsection{\% Substitutions}
\label{sec:tcltk-percent-substitutions}

One can not pass arbitrary user data to a callback, rather such values
must be found through \R's usual scoping rules. However, \TK\/
provides a mechanism called \defn{percent substitution} to pass
information about the event to callbacks bound to the event. The basic
idea is that in the \TCL\/ callback expressions of the type
\code{\%X}, for different characters \code{X}, will be replaced by values
coming from the event. In \pkg{tcltk}, if the callback function has an
argument \code{X}, then that variable will correspond to the value
specified by \code{\%X}. The complete list of substitutions is in the
\code{bind} man page. Some useful ones are \code{x} and \code{X} to
specify the relative or absolute $x$-postion of a mouse click (the
difference can be found through the \code{rootx} property of a
widget), \code{y} and \code{Y} for the $y$-position, \code{k} and
\code{K} for the keycode (ASCII) and key symbol of a
\code{\Event{KeyPress}} event, and \code{W} to refer to the ID of the
widget that signaled the event the callback is bound
to. 

The following trivial example illustrates, whereas
Example~\ref{ex-tcltk-dnd} will put these to use.

\begin{Schunk}
\begin{Sinput}
 w <- tktoplevel()
 b <- ttkbutton(w, text="Click me to record the x,y position")
 tkpack(b)
 tkbind(b, "<ButtonPress-1>", function(W, x,y, X, Y) {
   print(W)                              # an ID
   print(c(x, X))                        # character class
   print(c(y, Y))
   })
\end{Sinput}
\end{Schunk}




%% After
\paragraph{The after command}
The \TCL\/ command \code{after} will execute a command after a certain
delay (specified in milliseconds as an integer) while not interrupting
the control flow while it waits for its delay. The function is called
in a manner like:
\begin{Schunk}
  \begin{Sinput}
    ID <- tcl("after", 1000, function() print("1 second passed"))    
  \end{Sinput}
\end{Schunk}
The ID returned by \code{after} may be used to \code{cancel} the
command before it executes. To execute a command repeatedly, can be
done along the lines of:
\begin{Schunk}
\begin{Sinput}
 afterID <- ""
 someFlag <- TRUE
 repeatCall <- function(ms=100, f) {
   afterID <<- tcl("after", ms, function() {
     if(someFlag) {                      
       f()
       afterID <<- repeatCall(ms, f)
     }  else {
       tcl("after", "cancel", afterID)
     }
   })
 }
 repeatCall(2000, function() {
   print("Running. Set someFlag <- FALSE to stop.")
 })
\end{Sinput}
\end{Schunk}
%


\begin{example}{Drag and Drop}{ex-tcltk-dnd}
This relatively involved example~\footnote{The idea for the example
  code originated with \url{http://wiki.tcl.tk/416}} shows several
different uses of the event framework to implement drag and drop
behavior between two widgets. In \pkg{tcltk} much more work is
involved with drag and drop, than with \pkg{RGtk2} and \pkg{qtbase}. Steps are needed to
make one widget a drop source, and other steps are needed to make the
other widget a drop target.

The basic idea is that when a value is being dragged, virtual events
are generated for the widget the cursor is over. If that widget has
callbacks bound to these events, then the drag and drop can be
processed.


To begin, we create a simple GUI to hold three widgets. We use buttons
for drag and drop, but only because we haven't yet discussed more
natural widgets such as the text widgets. 

\begin{Schunk}
\begin{Sinput}
 w <- tktoplevel()
 bDrag <- ttkbutton(w, text="Drag me")
 bDrop <- ttkbutton(w, text="Drop here")
 tkpack(bDrag)
 tkpack(ttklabel(w, text="Drag over me"))
 tkpack(bDrop)
\end{Sinput}
\end{Schunk}


Before beginning, we define three global variables that can be shared
among drop sources to keep track of the drag and drop state. 
\begin{Schunk}
\begin{Sinput}
 .dragging <- FALSE                 # currently dragging?
 .dragValue <- ""                   # value to transfer
 .lastWidgetID <- ""                # last widget dragged over
\end{Sinput}
\end{Schunk}
%
%
To set up a drag source, we bind to three events: a mouse button
press, mouse motion, and a button release. For the button press, we
set the values of the three global variables.
\begin{Schunk}
\begin{Sinput}
 tkbind(bDrag,"<ButtonPress-1>",function(W) {
   .dragging <<-  TRUE
   .dragValue <<- as.character(tkcget(W,text=NULL))
   .lastWidgetID <<- as.character(W)
 })
\end{Sinput}
\end{Schunk}
%
This initiates the dragging immediately. A more common strategy is to
record the position of the mouse click and then initiate the dragging
after a certain minimal movement is detected.

%
For mouse motion, we do several things. First we set the cursor to
indicate a drag operation. The choice of cursor is a bit outdated. The
comment refers to a web page showing how one can put in a custom
cursor from an xbm file, but this doesn't work for all platforms
(e.g., OS X and aqua). After setting the cursor, we find the ID of the
widget the cursor is over. We use \function{tkwinfo} to find the
widget containing the $x,y$-coordinates of the cursor position.  We
then generate the \code{\VirtualEvent{DragOver}} virtual event for
this widget, and if this widget is different from the previous last
widget, we generate the \code{\VirtualEvent{DragLeave}} virtual event.

%%  ## This failed with OS X: "
%%  ##  .Tcl(paste(as.character(bDrag$ID),' configure -cursor "@', getwd(),'/cursor.xbm black"', sep=""))

\begin{Schunk}
\begin{Sinput}
 tkbind(w, "<B1-Motion>", function(W, X, Y) {
   if(!.dragging) return()
   ## see cursor help page in API for more options
   ## For custom cursors cf. http://wiki.tcl.tk/8674. 
   tkconfigure(W, cursor="coffee_mug")   # set cursor
 
   w = tkwinfo("containing", X, Y)       # widget mouse is over
   if(as.logical(tkwinfo("exists", w)))  # does widget exist?
     tkevent.generate(w, "<<DragOver>>")
 
   ## generate drag leave if we left last widget
   if(as.logical(tkwinfo("exists", w)) &&
      nchar(as.character(w)) > 0 && 
      length(.lastWidgetID) > 0          # if not tcltk character(0) 
      ) {
     if(as.character(w) != .lastWidgetID) 
       tkevent.generate(.lastWidgetID, "<<DragLeave>>")
   }
   .lastWidgetID <<- as.character(w)
 })
\end{Sinput}
\end{Schunk}


Finally, if the button is released, we generate the
\code{\VirtualEvent{DragLeave}} and, most importantly,
\code{\VirtualEvent{DragDrop}} virtual events for the widget we are
over.
\begin{Schunk}
\begin{Sinput}
  tkbind(bDrag, "<ButtonRelease-1>", function(W, X, Y) {
   if(!.dragging) return()
   w <- tkwinfo("containing", X, Y)
     
   if(as.logical(tkwinfo("exists", w))) {
     tkevent.generate(w, "<<DragLeave>>")
     tkevent.generate(w, "<<DragDrop>>")
     tkconfigure(w, cursor="")
   }
   .dragging <<- FALSE
   .lastWidgetID <<- "" 
   tkconfigure(W, cursor="")
 })
\end{Sinput}
\end{Schunk}
%
%
To set up a drop target, we bind callbacks for the virtual events
generated above to the widget. For the \code{\VirtualEvent{DragOver}} event
we make the widget \code{active} so that it appears ready to receive a
drag value.
\begin{Schunk}
\begin{Sinput}
 tkbind(bDrop,"<<DragOver>>",function(W) {
   if(.dragging) 
     tcl(W, "state", "active")
 })
\end{Sinput}
\end{Schunk}
%
If the drag event leaves the widget without dropping, we change the
state back to not active.
\begin{Schunk}
\begin{Sinput}
 tkbind(bDrop,"<<DragLeave>>", function(W) {
   if(.dragging)  {
     tkconfigure(W, cursor="")
     tcl(W, "state", "!active")  
    }
 })
\end{Sinput}
\end{Schunk}
%
Finally, if the \code{\VirtualEvent{DragDrop}} virtual event occurs, we set
the widget value to that stored in the global variable
\code{.dragValue}.
\begin{Schunk}
\begin{Sinput}
 tkbind(bDrop,"<<DragDrop>>", function(W) {
   tkconfigure(W, text=.dragValue)
   .dragValue <- ""
 })
\end{Sinput}
\end{Schunk}
\end{example}





\chapter{Tcl/Tk: Layout and Containers}
\label{sec:tcltk:basic-containers}
%% parent windows, frames etc.
%% Example

\section{Top-level windows}
\label{sec:tcltk:top-level-windows}
%%\XXX{Window Styles}

%% constructor
Top level windows are created through the \function{tktoplevel}
constructor. Basic options include the ability to specify the
preferred width and height and to specify a menubar through
the \argument{menu}{tktoplevel} argument. (Menus will be covered in
Section~\ref{sec:tcltk:menus}.)


Other properties can be queried and set through the \TK\/ command
\function{wm}. This command has several subcommands, leading to
\pkg{tcltk} functions with names such as \function{tkwm.title}, the
function used to set the window title. As with all such functions,
either the top-level window object, or its ID must be the first
argument. In this case, the new title is the second.

\paragraph{Suppressing the initial drawing}
When a top-level window is constructed there is no option for it not
to be shown.  However, one can use the \function{tclServiceMode}
function to suspend/resume drawing of any widget through \TK. This
function takes a logical value indicating the updating of widgets
should be suspended. One can set the value to \code{FALSE}, initiate
the widgets, then set to \code{TRUE} to display the widgets. To
iconify an already drawn window can be done through the
\function{tkwm.withdraw} function and reversed with the
\function{tkwm.deiconify} function. Either of these can be useful in
the construction of complicated GUIs, as the drawing of the widgets
can seem slow. (The same can be done through the \function{tkwm.state}
function with an option of \qcode{withdraw} or \qcode{normal}.)
 


\paragraph{Window sizing}

The preferred size of a top-level window can be configured through the
\code{width} and \code{height} arguments of the constructor.  Negative
values means the window will not request any size. The absolute size
and position of a top-level window in pixels can be queried or
specified through the \function{tkwm.geometry} function. The geometry
is specified as a string, as was described for \function{tkwinfo} in
Section~\ref{sec:tcltk:overview:widget-properties}. If this string is
empty, then the window will resize to accomodate its child components.

The \function{tkwm.resizable} function can be used to prohibit the
resizing of a top-level window. The syntax allows either the width or
height to be constrained. The following command would  prevent
resizing of both the width and height of the toplevel window \code{w}. 

\begin{Schunk}
  \begin{Sinput}
tkwm.resizable(w, FALSE, FALSE)    # width first
  \end{Sinput}
\end{Schunk}
%
When a window is resized, you can constrain the minimum and maximum
sizes with \function{tkwm.minsize} and \function{tkwm.maxsize}. The
aspect ratio (width/height) can be set through \function{tkwm.aspect}.


%% sizegrip
For resizable windows, the \constructor{ttksizegrip} widget can be
used to add a visual area (usually the lower right corner) for the
user to grab on to with their mouse for resizing the window. On some
OSes (e.g., Mac OS X) these are added by the window manager
automatically.


%% transient
\paragraph{Dialog windows}
For dialogs, a top-level window can be related to
a different top-level window. The function \function{tkwm.transient}
allows one to specify the master window as its second
argument (cf. Example~\ref{ex-tcltk-window}). The
new window will mirror the state of the master window, including if
the master is withdrawn.

%% overridedirect
For some dialogs it may be desirable to not have the
window manager decorate the window with a title bar etc. The command
\subcommanda{wm overrideredirect}{tktoplevel}{logical} takes a logical
value indicating if the window should be decorated. Though, not all
window managers respect this.



%% binginds
\paragraph{Bindings}
Bindings for top-level windows are propagated down to all of their
child widgets. If a common binding is desired for all the children,
then it need only be specified once for the top-level window
(cf. Section~\ref{sec:tcltk:events} where keyboard accelerators are
defined this way).


%% wm protocol
The \function{tkwm.protocol} function (not \function{tkbind}) is used
to assign commands to window manager events, most commonly, the delete
event when the user clicks the close button on the windows
decorations. A top-level window can be removed through the
\function{tkdestroy} function, or through the user clicking on the
correct window decorations. When the window decoration is clicked, the
window manager issues a \qcode{WM\_DELETE\_WINDOW} event. To bind to
this, a command of this form
\code{tkwm.protocol(win, "WM\_DELETE\_WINDOW", callback)} is used.

To illustrate, if \code{w} is a top-level window, and \code{e} a text
entry widget (cf. \function{tktext} in
Section~\ref{sec:tcltk:multi-line-text}) then the following snippet of
code would check to see if the text widget has been modified before
closing the window. This uses a modal message box described in
Section~\ref{sec:tcltk:dialogs}.



\begin{Schunk}
\begin{Sinput}
 tkwm.protocol(w,"WM_DELETE_WINDOW", function() {
   modified <- tcl(e, "edit", "modified")
   if(as.logical(modified)) {
     response <- 
       tkmessageBox(icon="question",
                    message="Really close?",
                    detail="Changes need to be saved",
                    type="yesno",
                    parent=w)
     if(as.character(response) == "no")
       return()
   }
   tkdestroy(w)                          # otherwise close
 })
\end{Sinput}
\end{Schunk}

%% Isn't working?
% %% stack of windows
% A window can be made to always be the topmost window through the
% \code{attributes} subcommand of the \code{wm} command. However, there
% is no direct \pkg{tcltk} function, so if \code{w} was to be on top, one would use the \function{tcl}
% function as follows: 
% \begin{Schunk}
% \begin{Sinput}
% tcl("wm", "attributes", w, topmost=TRUE)  
% \end{Sinput}
% \end{Schunk}

% When more than top-level window is in use, there is a stacking order
% describing how they are displayed. This stacking order is returned
% through the IDs of the windows through the \code{stackorder}
% subcommand of the \code{wm} command. There is no \pkg{tcltk} function
% for this, but the command \code{tcl("wm","stackorder", win)}, where
% \code{win} is the top-level window object will return the list.

% Stackign order of others; topmost
\begin{example}{A window constructor}{ex-tcltk-window}
  This example shows a possible constructor for top-level windows
  allowing some useful options to be passed in. We use the upcoming
  \function{ttkframe} constructor and \function{tkpack} command.
\begin{Schunk}
\begin{Sinput}
 newWindow <- function(title, command, parent,
                       width, height) {
   w <- tktoplevel()
 
   if(!missing(title)) tkwm.title(w, title)
 
   if(!missing(command)) 
     tkwm.protocol(w, "WM_DELETE_WINDOW", function() {
       if(command())            # command returns logical
         tkdestroy(w)
     })
 
   if(!missing(parent)) {
     parentWin <- tkwinfo("toplevel", parent)
     if(as.logical(tkwinfo("viewable", parentWin))) {
       tkwm.transient(w, parent)
     }
   }
   
   if(!missing(width)) tkconfigure(w, width=width)
   if(!missing(height)) tkconfigure(w, height=height)
 
   windowSystem <- tclvalue(tcl("tk", "windowingsystem"))
   if(windowSystem == "aqua") {
     f <- ttkframe(w, padding=c(3,3,12,12))
   } else {
     f1 <- ttkframe(w, padding=0)
     tkpack(f1, expand=TRUE, fill="both")
     f <- ttkframe(f1, padding=c(3,3,12,0))
     sg <- ttksizegrip(f1)
     tkpack(sg, side="bottom", anchor="se")
   }
   tkpack(f, expand=TRUE, fill="both", side="top")
 
   return(f)
 }
\end{Sinput}
\end{Schunk}
\end{example}

\section{Frames}
\label{sec:tcltk:frames}

The \function{ttkframe} constructor produces a themeable container
that can be used to organize visible components within a GUI. It is
often the first thing packed within a top-level window. 

The options include \option{width}{ttkframe} and
\option{height}{ttkframe} to set the requested size,
The \option{padding}{ttkframe}
option can be used to to put space within the border between the
border and subsequent children. Frames can be decorated. Use the
option \option{borderwidth}{ttkframe} to specify a border around the frame of
a given width, and \option{relief}{ttkframe} to set the border
style. The value of \code{relief} is chosen from (the default)
\qcode{flat}, \qcode{groove}, \qcode{raised}, \qcode{ridge},
\qcode{solid}, or \qcode{sunken}.  

\subsection{Label Frames}
\label{sec:tcltk:label-frames}

The \constructor{ttklabelframe} constructor produces a frame with an
optional label that can be used to set off and organize components of
a GUI. The label is set through the option
\option{text}{ttklabelframe}. Its position is determined by the option
\option{labelanchor}{ttklabelframe} taking values labeled by compass
headings (combinations of \code{n}, \code{e}, \code{w}, \code{s}. The
default is theme dependent, although typically \qcode{nw} (upper
left).

\paragraph{Separators}
As an alternative to a border, the \constructor{ttkseparator} widget can be used
to place a single line to separate off areas in a GUI. The lone
widget-specific option is \option{orient}{ttkseparator} which takes
values of \qcode{horizontal} (the default) or \qcode{vertical}. This
widget must be told to stretch when added to a container, as described
in the next section.

\section{Geometry Managers}
\label{sec:tcltk:geometry-managers}

\TCL\/ uses \secdfn{geometry managers}{tcltk} to place child
components within their parent windows. There are three such managers,
but we describe only two, leaving the lower-level \code{place} command
for the official documentation. The use of geometry managers, allows
\TK\/ to quickly reallocate space to a GUI's components when a window is
resized.  The \function{tkpack} function will place children into
their parent in a box-like manner. We have seen several examples in
the text that use nested boxes to construct quite flexible layouts.
Example~\ref{ex-tcltk-non-modal-dialog} will illustrate that once
again. When simultaneous horizontal and vertical alignment of child
components is desired, the \function{tkgrid} function can be used to
manage the components.
\\

%% warn against mixing XXX how to format warning?
A GUI may use a mix of \code{pack} and \code{grid} to mangage the child components,
but all immediate siblings in the widget hierarchy must be managed the same
way. Mixing the two will typically result in a lockup of the \R\/
session.


\subsection{Pack}
\label{sec:tcltk:pack}

%%\XXX{Is there a method to redraw the GUI?}
%%\XXX{Comment that pack can pack into other parent?}

%% side
We have illustrated how \constructor{tkpack} can be used to manage how
child components are viewed within their parent. The basic usage
\code{tkpack(child)} will pack in the child components from top to
bottom. The \option{side}{tkpack} option can take a value of
\qcode{left}, \qcode{right}, \qcode{top} (default), or \qcode{bottom}
to adjust where the children are placed. These can be mixed and
matched, but sticking to just one direction is typical, with nested
frames to give additional flexibility.

\paragraph{after, before}
The \option{after}{tkpack} and \option{before}{tkpack} options can be
used to place the child before or after another component. These are
used as with \code{tkpack(child1, after=child2)}. The object
\code{child2} can be an \R\/ object or its ID. 


\paragraph{forget}
Child components can be forgotten by the window manager, unmapping
them but not destroying them, with the \subcommand{forget}{tkpack}
subcommand, or the convenience function
\function{tkpack.forget}. After a child component is removed this way,
it can be re-placed in the GUI using a geometry manager. 

\paragraph{Introspection}
The subcommand \subcommand{slaves}{tkpack} will return a list of the
child components packed into a frame. Coercing these return values to
character via \code{as.character} will produce the IDs of the child
components. The subcommand \subcommand{info}{tkpack} will provide the
packing info for a child.
\\

These commands are illustrated below, where we show how one might
implement a ticker tape effect, where words scroll to the left.
\begin{Schunk}
\begin{Sinput}
 w <- tktoplevel()
 f <- ttkframe(w, padding=c(3,3,12,12))
 tkpack(f, expand=TRUE, fill="both")
 #
 x <- strsplit("Lorem ipsum dolor sit amet ...", "\\s")[[1]]
 labels <- lapply(x, function(i) ttklabel(f, text=i))
 sapply(labels, function(i) tkpack(i, side="left"))
 #
 rotateLabel <- function() {
   children <- as.character(tkpack("slaves", f))
   tkpack.forget(children[1])
   tkpack(children[1], after=children[length(children)], 
          side="left")
 }
\end{Sinput}
\end{Schunk}

One could use the \code{after} command to do this in the background,
but here we just rotate the values in a blocking loop:
\begin{Schunk}
\begin{Sinput}
 for(i in 1:20) {rotateLabel(); Sys.sleep(1)}
\end{Sinput}
\end{Schunk}


\paragraph{Specifying space around the children}

\begin{figure}
  \centering
  \includegraphics[width=.6\textwidth]{fig-tcltk-padding-pady-ipady}
 \caption{Various ways to put padding between widgets using \function{tkpack}. The \code{padding} option for the box container puts padding around the cavity for all the widgets. The \code{pady} option for \function{tkpack} puts padding around the top and bottom on the border of each widget. The \code{ipady} option for \function{tkpack} puts padding within the top and bottom of the border for each child (breaking the theme under Mac OS X).}
  \label{fig:fig-pack-example}
\end{figure}


In addition to the \code{padding} option for a frame container, the
\option{ipadx}{tkpack}, \option{ipady}{tkpack}, \option{padx}{tkpack},
and \option{pady}{tkpack} options can be used to add space around the
child components. Figure~\ref{fig:fig-pack-example} has an
example. In the above options, the \code{x} and \code{y} indicate left-right space or
top-bottom space. The \code{i} stands for internal padding that is
left on the sides or top and bottom of the child within the border,
for \code{padx} the external padding added around the border of the
child component. The value can be a single number or pair of numbers
for asymmetric padding.


This sample code shows how one can easily add padding around all the
children of the frame \code{f} using the
\subcommand{"configure"}{tkpack} subcommand.

\begin{Schunk}
\begin{Sinput}
 allChildren <- as.character(tkwinfo("children", f))
 sapply(allChildren, function(i) {
   tkpack("configure", i,  padx=10, pady=5)
 })
\end{Sinput}
\end{Schunk}


\paragraph{Cavity model}
The packing algorithm, as described in the \Tk\/ documentation, is based
on arranging where to place a slave into the rectangular unallocated
space called a ``cavity.'' We use the nicer terms ``child component'' and ``box''
to describe these. When a child is placed inside the box, say on the
top, the space allocated to the child is the rectangular space with
width given by the width of the box, and height the sum of the
requested height of the child plus twice the \code{ipady} amount (or
the sum if specified with two numbers). The packer then chooses the
dimension of the child component, again from the requested size plus
the \code{ipad} values for \code{x} and \code{y}. These two spaces
may, of course, have different dimensions.



By default, the child  will be placed centered along the edge of
the box within the allocated space and blank space, if any, on both
sides.  If there is not enough space for the child in the allocated
space, the component can be drawn oddly. Enlarging the top-level
window can adjust this. 



\paragraph{anchor, expand, fill} 
When there is more space in the box than requested by the child
component, there are other options. The \option{anchor}{tkpack} option
can be used to anchor the child to a place in the box by specifying
one of the valid compass points (eg. \code{"n"} or \code{"se"})
leaving blank space around the child. 


An alternative is to have one or more of the widgets expand to fill
the available space. Each child packed in with the 
option \option{expand}{tkpack} set to \code{TRUE} will have
the extra space allocated  to it in an even manner. The
\option{fill}{tkpack} option is used to base the size of the child on
the available cavity in the box -- not on the requested size of the
child. The \code{fill} option can be \qcode{x}, \qcode{y} or
\qcode{both}. The first two expanding the child's size in just one
direction, the latter in both.

\begin{example}{Packing dialog buttons}{ex-tcltk-pack}


This example shows how one can pack in action buttons, such as when a
dialog is created.

\begin{figure}
  \centering
  \includegraphics[width=.5\textwidth]{fig-tcltk-pack-buttons.png}
  \caption{Demonstration of using \code{tkpack} options showing
    effects of using the \code{side}
    and \code{padx} options to create
    dialog buttons.}
  \label{fig:tcltk-pack-buttons}
\end{figure}


The first example just uses \code{tkpack} without any arguments except
the side to indicate the buttons are packed in left to right, not top
to bottom.
\begin{Schunk}
\begin{Sinput}
 f1 <- ttklabelframe(f, text="plain vanilla")
 tkpack(f1, expand=TRUE, fill="x")
 l <- function(f) 
   list(ttkbutton(f, text="cancel"), ttkbutton(f, text="ok"))
 sapply(l(f1), function(i) tkpack(i, side="left"))
\end{Sinput}
\end{Schunk}

Typically the buttons are right justified. One way to do this is to
pack in using \code{side} with a value of \qcode{right}. This shows
how to use a blank expanding label to take up the space on the left.
\begin{Schunk}
\begin{Sinput}
 f2 <- ttklabelframe(f, text="push to right")
 tkpack(f2, expand=TRUE, fill="x")
 tkpack(ttklabel(f2, text=" "), expand=TRUE, fill="x", 
        side="left")
 sapply(l(f2), function(i) tkpack(i, side="left"))
\end{Sinput}
\end{Schunk}

Finally, we add in some padding to conform to Apple's design specification that such
buttons should have a 12 pixel separation.
\begin{Schunk}
\begin{Sinput}
 f3 <- ttklabelframe(f, text="push to right with space")
 tkpack(f3, expand=TRUE, fill="x")
 tkpack(ttklabel(f3, text=" "), expand=TRUE, fill="x", 
        side="left")
 sapply(l(f3), function(i) tkpack(i, side="left", padx=6))
\end{Sinput}
\end{Schunk}
\end{example}

\begin{example}{A non-modal dialog}{ex-tcltk-non-modal-dialog}
This example shows how to use  a window, frames,  labels, buttons,
icons, packing and bindings to create a non-modal dialog. 

\begin{figure}
  \centering
  \includegraphics[width=.4\textwidth]{fig-tcltk-simple-dialog.png}
  \caption{Example of a simple dialog}
  \label{fig:fig-tcltk-simple-dialog}
\end{figure}

Although not written as a function, we set aside the values that would
be passed in were it.
\begin{Schunk}
\begin{Sinput}
 title <- "message dialog"
 message <- "Do you like tcltk so far?"
 parent <- NULL
 tkimage.create("photo", "::img::tclLogo", 
                file = system.file("images","tclp.gif",
                  package="ProgGUIinR"))
\end{Sinput}
\end{Schunk}

The main top-level window is given a title, then withdrawn while
the GUI is created. 
\begin{Schunk}
\begin{Sinput}
 w <- tktoplevel(); tkwm.title(w, title)
 tkwm.state(w, "withdrawn")
 f <- ttkframe(w,  padding=c(3, 3, 12, 12))
 tkpack(f, expand=TRUE, fill="both")
\end{Sinput}
\end{Schunk}
As usual, we added a frame so that any themes are respected.

If the parent is non-null and is viewable, then the dialog is made
transient to a parent, The parent need not be a top-level window, so
\function{tkwinfo} if used to find the parent's top-level window. For
Mac OS X, we use the \code{notify} attribute to bounce the dock icon
until the mouse enters the window area.

\begin{Schunk}
\begin{Sinput}
 if(!is.null(parent)) {
   parentWin <- tkwinfo("toplevel", parent)
   if(as.logical(tkwinfo("viewable", parentWin))) {
     tkwm.transient(w, parent)
     if(as.character(tcl("tk", "windowingsystem")) == "aqua") {
       tcl("wm","attributes",parentWin, notify=TRUE) # bounce
       tkbind(parentWin,"<Enter>", function()        # stop
              tcl("wm","attributes",parentWin, notify=FALSE)) 
     }
   }
 }
\end{Sinput}
\end{Schunk}

We will use a standard layout for our dialog with an icon on the left,
a message and buttons on the right. We pack the icon into the left side of the frame,
\begin{Schunk}
\begin{Sinput}
 l <- ttklabel(f, image="::img::tclLogo", padding=5) # recycle
 tkpack(l,side="left")
\end{Sinput}
\end{Schunk}

A nested frame will be used to layout the message area and button area. Since the \function{tkpack} default is to pack in top to bottom, no \code{side} specification is made.
\begin{Schunk}
\begin{Sinput}
 f1 <- ttkframe(f)
 tkpack(f1, expand=TRUE, fill="both")
 #
 m <- ttklabel(f1, text=message)
 tkpack(m, expand=TRUE, fill="both")
\end{Sinput}
\end{Schunk}

The buttons have their own frame, as they are layed out horizontally. 
\begin{Schunk}
\begin{Sinput}
 f2 <- ttkframe(f1)
 tkpack(f2)
\end{Sinput}
\end{Schunk}
%
The callback function for the OK button prints a message then destroys the window.
\begin{Schunk}
\begin{Sinput}
 okCB <- function() {
   print("That's great")
   tkdestroy(w)
 }
 okButton <- ttkbutton(f2, text="OK", command=okCB)
 cancelButton <- ttkbutton(f2, text="Cancel", 
                           command=function() tkdestroy(w))
 #
 tkpack(okButton, side="left", padx=12)  # give some space
 tkpack(cancelButton)
\end{Sinput}
\end{Schunk}
%

As our interactive behavior is consistent for both buttons, we make a
binding to the \code{TButton} class, not individually. The first will
invoke the button command when the \kbd{return} key is pressed, the
latter two will highlight a button when the focus is on it.
\begin{Schunk}
\begin{Sinput}
 tkbind("TButton", "<Return>", function(W) tcl(W, "invoke"))
 tkbind("TButton", "<FocusIn>", function(W) 
        tcl(W, "state", "active"))
 tkbind("TButton", "<FocusOut>", function(W) 
        tcl(W, "state", "!active"))
\end{Sinput}
\end{Schunk}
%
Now we bring the dialog back from its withdrawn state, fix the size
and set the initial focus on the OK button.
\begin{Schunk}
\begin{Sinput}
 tkwm.state(w, "normal")
 tkwm.resizable(w, FALSE, FALSE)
 tkfocus(okButton)
\end{Sinput}
\end{Schunk}
\end{example}

\subsection{Grid}
\label{sec:tcltk:grid}
The \function{tkgrid} geometry manager is used to align child widgets
in rows and columns.  In its simplest usage, a command like
\begin{Schunk}
  \begin{Sinput}
tkgrid(child1, child2,..., childn)    
  \end{Sinput}
\end{Schunk}
will place the $n$ children in a new row, in columns $1$ through
$n$. If desired, the specific row and column can be specified through the
\option{row}{tkgrid} and \option{column}{tkgrid} options, counting of
rows and columns starts with $0$.  Spanning of multiple rows and columns
can be specified with integers $2$ or greater by the
\option{rowspan}{tkgrid} and \option{colspan}{tkgrid} options. These
options, and others, can be adjusted through the
\function{tkgrid.configure} function.


\paragraph{The tkgrid.rowconfigure, tkgrid.columnconfigure commands}
When the managed container is resized, the grid manager consults
weights that are assigned to each row and column to see how to
allocate the extra space. Allocation is based on proportions, not
specified sizes. The weights are configured with the
\function{tkgrid.rowconfigure} and \function{tkgrid.columnconfigure}
functions through the option \option{weight}{tkgrid.columnconfigure}.
The weight is a value between $0$ and $1$. If there are just two rows, and
the first row has weight $1/2$ and the second weight $1$, then the extra
space is allocated twice as much for the second row. The specific row
or column must also be specified. Again. rows and columns are referenced
starting with $0$ not the usual \R-like $1$. To specify a weight of $1$
to the first row would be done with a command like:

%
\begin{Schunk}
\begin{Sinput}
 tkgrid.rowconfigure(parent, 0, weight=1)
\end{Sinput}
\end{Schunk}
%
\paragraph{The sticky option}
The \function{tkpack} command had options \code{anchor} and
\code{expand} and \code{fill} to control what happens when more space
is available then requested by a child component. The
\option{sticky}{tkgrid} option for \function{tkgrid} combines
these. The value is a combination of the compass points
\qcode{n},\qcode{e},\qcode{w}, and \qcode{s}. A specification
\qcode{ns} will make the child component ``stick'' to the top and
bottom of the cavity that is provided, similar to the \code{fill="y"}
option for \function{tkpack}. A value of \qcode{news} will make the
child component expand in all the direction like \code{expand=TRUE, fill="both"}.

\paragraph{Padding}
As with \function{tkpack}, \function{tkgrid} has options
\option{ipadx}{tkgrid}, \option{ipady}{tkgrid}, \option{padx}{tkgrid},
and \option{padx}{tkgrid} to give internal and external space around a
child.

\paragraph{Size}
The function \function{tkgrid.size} will return the number of columns
and rows of the specified parent container that is managed by a
grid. This can be useful when trying to position child components
through the options \code{row} and \code{column}.

\paragraph{Forget}
To remove a child from the parent, the \function{tkgrid.forget}
function can be used with the child object as its argument.


% \begin{example}{Using \function{tkgrid} and \function{tkpack} to draw some world flags}{ex-tcltk-flags}
%   \SweaveInput{ex-tcltk-flags.Rnw}
% \end{example}


\begin{example}{Using \function{tkgrid} to create a toolbar}{ex-tcltk-toolbar}

\begin{figure}
  \centering
  \includegraphics[width=.4\textwidth]{fig-tcltk-toolbar.png}
  \caption{Illustration of using \code{tkpack} and \code{tkgrid} to make a toolbar. }
  \label{fig:fig-tcltk-toolbar}
\end{figure}




\TK\/ does not have a toolbar widget. Here we use \function{tkgrid} to
show how we can add one to a top-level window in a manner that is not
affected by resizing. We begin by packing a frame into a
top-level window.
\begin{Schunk}
\begin{Sinput}
 w <- tktoplevel(); tkwm.title(w, "Toolbar example")
 f <- ttkframe(w, padding=c(3,3,12,12))
 tkpack(f, expand=TRUE, fill="both")
\end{Sinput}
\end{Schunk}
Our example has two main containers: one to hold the toolbar buttons
and one to hold the main content.
\begin{Schunk}
\begin{Sinput}
 tbFrame <- ttkframe(f, padding=0)
 contentFrame <- ttkframe(f, padding=4)
\end{Sinput}
\end{Schunk}
The \function{tkgrid} geometry manager is used to place the toolbar at
the top, and the content frame below. The choice of sticky and the weights ensure that
the toolbar does not resize if the window does.
\begin{Schunk}
\begin{Sinput}
 tkgrid(tbFrame, row=0, column=0, sticky="we")
 tkgrid(contentFrame, row=1, column=0, sticky = "news")
 tkgrid.rowconfigure(f, 0, weight=0)
 tkgrid.rowconfigure(f, 1, weight=1)
 tkgrid.columnconfigure(f, 0, weight=1)
 #
 txt <- "Lorem ipsum dolor sit amet..." # sample text
 tkpack(ttklabel(contentFrame, text=txt))
\end{Sinput}
\end{Schunk}

Now to add some buttons to the toolbar. We first show how to create a
new style for a button (\code{Toolbar.TButton}), slightly modifying the themed button to set
the font and padding, and eliminate the border if the operating system allows. 
\begin{Schunk}
\begin{Sinput}
 tcl("ttk::style", "configure", "Toolbar.TButton", 
     font="helvetica 12", padding=0, borderwidth=0)
\end{Sinput}
\end{Schunk}
%
This \code{makeIcon} function finds stock icons from the
\pkg{gWidgets} package and adds them to a button.
\begin{Schunk}
\begin{Sinput}
 makeIcon <- function(parent, stockName, command=NULL) {
   iconFile <- system.file("images", 
                           paste(stockName,"gif",sep="."), 
                           package="gWidgets")
   if(nchar(iconFile) == 0) {
     b <- ttkbutton(parent, text=stockName, width=6)
   } else {
     iconName <- paste("::img::",stockName, sep="")
     tkimage.create("photo", iconName, file = iconFile)
     b <- ttkbutton(parent, image=iconName, 
                    style="Toolbar.TButton", text=stockName, 
                    compound="top", width=6)
     if(!is.null(command))
       tkconfigure(b, command=command)
   }
   return(b)
 }
\end{Sinput}
\end{Schunk}
%
To illustrate, we pack in some icons. Here we use \function{tkpack}.  
One does not use \function{tkpack} and \function{tkgrid} to manage
children of the same parent, but these are children of \code{tbFrame},
not \code{f}.
\begin{Schunk}
\begin{Sinput}
 tkpack(makeIcon(tbFrame, "ok"), side="left")
 tkpack(makeIcon(tbFrame, "quit"), side="left")
 tkpack(makeIcon(tbFrame, "cancel"), side="left")
\end{Sinput}
\end{Schunk}

These two bindings change the state of the buttons as the mouse hovers
over it:

\begin{Schunk}
\begin{Sinput}
 setState <- function(W, state) tcl(W, "state", state)
 tkbind("TButton", "<Enter>", function(W) setState(W, "active"))
 tkbind("TButton", "<Leave>", function(W) setState(W, "!active"))
\end{Sinput}
\end{Schunk}

If one wished to restrict the above to just the toolbar buttons, one
could check for the style of the button, as with:

\begin{Schunk}
\begin{Sinput}
 function(W) {
   if(as.character(tkcget(W, "-style")) == "Toolbar.TButton")
     cat("... do something for toolbar buttons ...")
 }
\end{Sinput}
\begin{Soutput}
function (W) 
{
    if (as.character(tkcget(W, "-style")) == "Toolbar.TButton") 
        cat("... do something for toolbar buttons ...")
}
\end{Soutput}
\end{Schunk}
\end{example}

\begin{example}{Using \function{tkgrid} to layout a calendar}{ex-tkgrid-calendar}
This example shows how to create a simple calendar using a grid
layout. (No such widget is standard with \pkg{tcltk}.) The following
relies on some date functions in the \pkg{ProgGUIInR} package.




\begin{figure}
  \centering
  \includegraphics[width=.4\textwidth]{fig-tcltk-grid-calendar}
  \caption{A monthly calendar illustrating various layouts.}
  \label{fig:qt-gridlayout-calendar}
\end{figure}


\begin{Schunk}
\begin{Sinput}
 makeMonth <- function(w, year, month) {
   ## add headers
   days <- c("S","M","T","W","Th","F","S")
   sapply(1:7, function(i) {
     l <- ttklabel(w, text=days[i])       
     tkgrid(l, row=0, column=i-1, sticky="")
   })
   ## add days
   sapply(seq_len(days.in.month(year, month)),  function(day) {
     l <- ttklabel(w, text=day)
     tkgrid(l, row=1 + week.of.month(year, month, day),
            column=day.of.week(year, month, day),
            sticky="e")
   })
 }
\end{Sinput}
\end{Schunk}

Next, we would like to incorporate the calendar widget into an interface
that allows the user to scroll through month-by-month beginning with:
\begin{Schunk}
\begin{Sinput}
 year <- 2000; month <- 1
\end{Sinput}
\end{Schunk}

Our basic layout will use a box layout with a nested layout
for the step-through controls and another holding the calendar widget.
\begin{Schunk}
\begin{Sinput}
 w <- tktoplevel()
 f <- ttkframe(w, padding=c(3,3,12,12))
 tkpack(f, expand=TRUE, fill="both", side="top")
 cframe <- ttkframe(f)
 calframe <- ttkframe(f)
 tkpack(cframe, fill="x", side="top")
 tkpack(calframe, expand=TRUE, anchor="n")
\end{Sinput}
\end{Schunk}

Our step through controls are packed in through a horizontal
layout. We use anchoring and \code{expand=TRUE} to keep the arrows on the edge and the
label with the current month centered.
\begin{Schunk}
\begin{Sinput}
 prevb <- ttklabel(cframe, text="<")
 nextb <- ttklabel(cframe, text=">")
 curmo <- ttklabel(cframe)
 #
 tkpack(prevb, side="left", anchor="w")
 tkpack(curmo, side="left", anchor="center", expand=TRUE)
 tkpack(nextb, side="left", anchor="e")
\end{Sinput}
\end{Schunk}

The \code{setMonth} function first removes the previous calendar
container and then
redefines one to hold the monthly calendar. Then it adds in a new
monthly calendar to match the year and month. The call to
\code{makeMonth} creates the calendar. Packing in the frame after
adding its child components makes the GUI seem much more responsive.
\begin{Schunk}
\begin{Sinput}
 setMonth <- function() {
   tkpack("forget", calframe)
   calframe <<- ttkframe(f)
   makeMonth(calframe, year, month)
   tkconfigure(curmo,                    # month label
               text=sprintf("%s %s", month.abb[month], year))
   tkpack(calframe)
 }
 setMonth()                              # initial calendar
\end{Sinput}
\end{Schunk}

The arrow labels are used to scroll, so we connect to the
\event{Button-1} event the corresponding commands. This shows the
binding to decrement the month and year using the global variables
\code{month} and \code{year}.
\begin{Schunk}
\begin{Sinput}
 tkbind(prevb, "<Button-1>", function() {
   if(month > 1) {
     month <<- month - 1
   } else {
     month <<- 12; year <<- year - 1
   }
   setMonth()
 })
\end{Sinput}
\end{Schunk}


Our calendar is static, but if we wanted to add interactivity to a
mouse click, we could make a binding as follows:
  
\begin{Schunk}
\begin{Sinput}
 tkbind("TLabel", "<Button-1>", function(W) {
   day <- as.numeric(tkcget(W, "-text"))
   if(!is.na(day))
     print(sprintf("You selected: %s/%s/%s", month, day, year))
 })
\end{Sinput}
\end{Schunk}


\end{example}

\section{Other containers}
\label{sec:tcltk:other-containers}
\TK\/ provides just a few other basic containers, here we describe paned windows and notebooks.

\subsection{Paned Windows}
\label{sec:tcltk:paned-windows}

A paned window, with sashes to control the individual pane sizes, is constructed by the function
\constructor{ttkpanedwindow}. The primary option, outside of setting
the requested width or height with \option{width}{ttkpanedwindow} and
\option{height}{ttkpanedwindow}, is \option{orient}{ttkpanedwindow},
which takes a value of \qcode{vertical} (the default) or
\qcode{horizontal}. This specifies how the children are stacked, and
is opposite of how the sash is drawn.

%% adding
The returned object can be used as a parent container, although one
does not use the geometry managers to manage them. Instead, the
\method{add}{ttkpanedwindow} command is used to add a child component. For example:
\begin{Schunk}
\begin{Sinput}
 w <- tktoplevel(); tkwm.title(w, "Paned window example")
 pw <- ttkpanedwindow(w, orient="horizontal")
 tkpack(pw, expand=TRUE, fill="both")
 left <- ttklabel(pw, text="left")
 right <- ttklabel(pw, text="right")
 #
 tkadd(pw, left, weight=1)
 tkadd(pw, right, weight=2)
\end{Sinput}
\end{Schunk}
%
When resizing which child gets the space is determined by the
associated \code{weight}, specified as an integer. The default uses
even weights.  Unlike \GTK\/ more than two children are allowed.

\paragraph{Forget}
The subcommand \subcommand{forget}{ttkpanedwindow} can be used to
unmanage a child component. For the paned window, we have no
convenience function, so we call as follows:
\begin{Schunk}
\begin{Sinput}
 tcl(pw, "forget", right)
 tkadd(pw, right, weight=2) ## how to add back
\end{Sinput}
\end{Schunk}
%
\paragraph{Sash position}
The sash between two children can be adjusted through the subcommand
\subcommand{sashpos}{ttkpanedwindow}. The index of the sash needs
specifying, as there can be more than one. Counting starts at 0. The
value for \code{sashpos} is in terms of pixel width (or height) of the
paned window. The width can be returned and used as follows:
\begin{Schunk}
\begin{Sinput}
 width <- as.integer(tkwinfo("width", pw))  # or "height"
 tcl(pw, "sashpos", 0, floor(0.75*width))
\end{Sinput}
\begin{Soutput}
<Tcl> 45 
\end{Soutput}
\end{Schunk}
%

\subsection{Notebooks}
\label{sec:tcltk:notebooks}

%% constructor

%
\begin{figure}
  \centering
  \includegraphics[width=.6\textwidth]{fig-tcltk-notebook}
 \caption{A basic notebook under \OSX{}}
  \label{fig:fig-notebook-example}
\end{figure}

Tabbed notebook containers are produced by the
\constructor{ttknotebook} constructor.  Notebook pages can be added
through the \subcommand{add}{ttknotebook} subcommand or inserted after
a page through the \subcommand{insert}{ttknotebook} subcommand. The
latter requires a tab ID to be specified, as described below.
Typically, the child components would be containers to hold more
complicated layouts. The tab label is configured similarly to
\function{ttklabel} through the options \option{text}{ttknotebook} and
(the optional) \option{image}{ttknotebook}, which if given has its
placement determined by \option{compound}{ttknotebook}.  The placement
of the child component within the notebook page is manipulated
similarly as \function{tkgrid} through the
\option{sticky}{ttknotebook} option with values specified through
compass points. Extra padding around the child can be added with the
\option{padding}{ttknotebook} option.

\paragraph{Tab identifiers} %%integer (0-based), object (ID), "current", "end"
Many of the commands for a notebook require a specification of a
desired tab. This can be given by index, starting at 0; by the values
\code{"current"} or \code{"end"}; by the child object added to the
tab, either as an \R\/ object or an ID; or in terms of $x$-$y$
coordinates in the form \code{"@x,y"} (likely found through a
binding).

%% illustrate add, inser
To illustrate, if \code{nb} is a \code{ttknotebook} object, then these
commands would add pages (cf. Figure~\ref{fig:fig-notebook-example}):
\begin{Schunk}
\begin{Sinput}
 iconFile <- system.file("images",paste("help","gif",sep="."),
                         package="gWidgets")
 iconName <- "::tcl::helpIcon"
 tkimage.create("photo", iconName, file = iconFile)
 #
 l2 <- ttklabel(nb, text="Page 2")
 tkadd(nb, l2, sticky="nswe", text="label 2", 
     image=iconName, compound="right")
 ## put l1 first (a tabID of 0); use tkinsert
 l1 <- ttklabel(nb, text="Page 1")
 tkinsert(nb, 0, l1, sticky="nswe", text="label 1")
\end{Sinput}
\end{Schunk}
%
There are several useful subcommands to extract information from the
notebook object.  For instance, \code{index} to return the page index
(0-based), \code{tabs} to return the page IDs, \code{select} to select
the displayed page, and \code{forget} to remove a page from the
notebook. (There is no means to place close icons on the tabs.)
Except for \code{tabs}, these require a specification of a tab ID.
\begin{Schunk}
\begin{Sinput}
 tcl(nb, "index", "current")           # current page for tabID
\end{Sinput}
\begin{Soutput}
<Tcl> 1 
\end{Soutput}
\begin{Sinput}
 length(as.character(tcl(nb,"tabs")))  # number of pages
\end{Sinput}
\begin{Soutput}
[1] 2
\end{Soutput}
\begin{Sinput}
 tcl(nb, "select", 0)        # select viewable page by index
 tcl(nb, "forget", l1)       # "forget" removes a page
 tcl(nb, "add", l1)          # can be managed again.
\end{Sinput}
\end{Schunk}
%

%% keyboard
The notebook state can be manipulated through the keyboard, provided traversal is enabled. This can be done through
\begin{Schunk}
\begin{Sinput}
 tcl("ttk::notebook::enableTraversal", nb)
\end{Sinput}
\end{Schunk}

If enabled, the shortcuts such as \kbd{control-tab} to move to the
next tab are implemented. If new pages are added or inserted with the
option \option{underline}{ttknotebook}, which takes an integer value
(0-based) specifying which character in the label is underlined, then
a keyboard accelerator is added for that letter.

\paragraph{Bindings}
%% virtualevent
Beyond the usual events, the notebook widget also generates a
\code{\VirtualEvent{NotebookTabChanged}} virtual event after a new tab is
selected.

%% limitations: no easy way to get close buttons (to me anyways); no
%% graceful way to handle too man tabs.
The notebook container in \TK\/ has a few limitations. For instance,
there is no graceful management of too many tabs, as there is with
\GTK, as well there is no easy way to implement close buttons as an
icon, as in \Qt.


\chapter{Tcl/Tk: Widgets}
\label{sec:tcltk:widgets}

This chapter covers both the standard dialogs provided by \TK\/ and
the various controls used to create custom dialogs. We begin with a
discussion of these standard dialogs, then cover the basic controls
before finishing up with the more involved \constructor{tktext},
\constructor{ttktreeview}, and \constructor{tkcanvas} widgets.

%% Rewrite
% \Tk\/ has widgets for the common GUI controls. As mentioned in
% Chapter~\ref{sec:tcltk:overview} -- where we illustrated both buttons
% and labels -- the constructors for these widgets call the function
% \function{tkwidget} which calls the appropriate \TK\/ command and adds
% in extra information including an ID and an environment. As with
% labels and buttons, one primarily uses \function{tkconfigure} and
% \function{tkcget} to set and get properties of the widget when a
% \TCL\/ variable is not used to store the data for the widget.



\section{Dialogs}
\label{sec:tcltk:dialogs}
\subsection{Modal dialogs}
\label{sec:modal-dialogs}

\begin{figure}
  \centering
  \includegraphics[width=.6\textwidth]{fig-tcltk-confirm-dialog.png}
  \caption{A basic modal dialog constructed by \code{tkmessageBox}.}
  \label{fig:fig-tcltk-confirm-dialog}
\end{figure}

%% messageBox
The \constructor{tkmessageBox} constructor can be used to create
simple modal dialogs allowing a user to confirm an action, using the
native toolkit if possible. This constructor replaces the older
\code{tkdialog} dialogs. The arguments \argument{title}{tkmessageBox},
\argument{message}{tkmessageBox} and \argument{detail}{tkmessageBox}
are used to set the text for the dialog. The \code{title} may not
appear for all operating systems. A messageBox dialog has an
\argument{icon}{tkmessageBox} argument. The default icon is
\qcode{info} but could also be one of \qcode{error}, \qcode{question},
or \qcode{warning}. The buttons used are specified through the
\argument{type}{tkmessageBox} argument with values of \qcode{ok},
\qcode{okcancel}, \qcode{retrycancel}, \qcode{yesno}, or
\qcode{yesnocancel}. When a button is clicked the dialog is destroyed
and the button label returned as a value. The argument
\argument{parent}{tkmessageBox} can be given to specify which window
the dialog belongs to. Depending on the operating system this may be
used when drawing the dialog.

A sample usage is:
\begin{Schunk}
\begin{Sinput}
 tkmessageBox(title="Confirm", message="Really exit?", 
              detail="Changes need saving.", 
              icon="question", type="okcancel")
\end{Sinput}
\end{Schunk}
%% 

\paragraph{The tkwait function}
If the default modal dialog is not enough -- for instance there is no
means to gather user input -- then a toplevel window can be made
modal. The \function{tkwait} function will cause a top-level window to
be modal and \function{tkgrab.release} will return the interactivity
for the window. We illustrate a simple use by example, beginning by
adding a label to a window:

\begin{Schunk}
\begin{Sinput}
 message <- "We care ..."
 dlg <- tktoplevel(); tkwm.withdraw(dlg)
 tkwm.overrideredirect(dlg, TRUE)   # no decoration
 f <- ttkframe(dlg, padding=5)
 tkpack(f, expand=TRUE, fill="both")
 tkpack(ttklabel(f, text=message), pady=5)
\end{Sinput}
\end{Schunk}

We will use \function{tkwait.variable} which waits for a change to
variable, in this case \code{flag} defined next. In the button's command
we release the window then change this value, ending the wait.
\begin{Schunk}
\begin{Sinput}
 flag <- tclVar("")
 tkpack(ttkbutton(f, text="dismiss", command=function() {
   tkgrab.release(dlg)
   tclvalue(flag) <- "Destroy"
 }))
\end{Sinput}
\end{Schunk}
Now we show the window and wait on the \code{flag} variable to change.
\begin{Schunk}
\begin{Sinput}
 tkwm.deiconify(dlg)
 tkwait.variable(flag)
\end{Sinput}
\end{Schunk}

When the flag is changed, the flow returns to the program. Here we
print a message then destroy the dialog.
\begin{Schunk}
\begin{Sinput}
 print("Thanks")
\end{Sinput}
\begin{Soutput}
[1] "Thanks"
\end{Soutput}
\begin{Sinput}
 tkdestroy(dlg)
\end{Sinput}
\end{Schunk}


\subsection{File and directory selection}
\label{sec:file-direct-select}

\Tk\/ provides constructors for selecting a file, for selecting a
directory or for specifying a filename when saving. These are
implemented by \constructor{tkgetOpenFile},
\constructor{tkchooseDirectory}, and \constructor{tkgetSaveFile}
respectively. Each of these can be called with no argument, and
each returns a \code{tclObj} object. The value is empty when there is no selection made.

The dialog will appear in a relationship with a toplevel window if the argument
\argument{parent}{tkgetOpenFile} is specified The
\argument{initialdir}{tkgetOpenFile} and
\argument{initialfile}{tkgetOpenFile} can be used to specify the
initial values in the dialog.  The
\argument{defaultextension}{tkgetSaveFile} argument can be used to
specify a default extension for the file.

When browsing for files, it can be convenient to filter the available
file types that can be selected. The \argument{filetypes}{tkgetOpenFile} argument is used for this task. However,
the file types are specified using \TCL\/ brace-notation, not \R\/ code. For example,
to filter out various image types, one could have 
\begin{Schunk}
\begin{Sinput}
 tkgetOpenFile(filetypes = paste(
                 "{{jpeg files} {.jpg .jpeg} }",
                 "{{png files} {.png}}",
                 "{{All files} {*}}", sep=" ")) # needs space
\end{Sinput}
\end{Schunk}
Extending this is hopefully clear from the pattern above.

\begin{example}{A ``File'' menu}{ex-tcltk-file-menu}
  To illustrate, a simple example for a file menu could be:
\begin{Schunk}
\begin{Sinput}
 w <- tktoplevel(); tkwm.title(w, "File menu example")
 mb <- tkmenu(w); tkconfigure(w, menu=mb)
 fileMenu <- tkmenu(mb)
 tkadd(mb, "cascade", label="File", menu=fileMenu)
 tkadd(fileMenu,"command",label="Source file...",
       command= function() {
         fName <- tkgetOpenFile(filetypes=
                         "{{R files} {.R}} {{All files} *}")
         if(file.exists(fName <- as.character(fName)))
            source(tclvalue(fName))
       })
 tkadd(fileMenu, "command", label="Save workspace as...",
       command=function() {
         fName <- tkgetSaveFile(defaultextension="Rsave")
         if(nchar(fname <- as.character(fName)))
           save.image(file=fName)
       })
 tkadd(fileMenu, "command", label="Set working directory...",
       command=function() {
         dName <- tkchooseDirectory()
         if(nchar(dName <- as.character(dName)))
           setwd(dName)
       })
\end{Sinput}
\end{Schunk}
\end{example}

\subsection{Choosing a color}
\Tk\/ provides the command \code{tk\_chooseColor} to construct a dialog for selection of a color by RGB value. There are three optional arguments \argument{initialcolor}{tk\_chooseColor} to specify an inital color such as \qcode{\#efefef}, \argument{parent}{tk\_chooseColor} to make the dialog a child of a specified window and \argument{title}{tk\_chooseColor} to specify a title for the dialog. The return value is in hex-coded RGB quantitles. 
There is no constructor in \pkg{tcltk}, but one can use the dialog as follows:
\begin{Schunk}
\begin{Sinput}
 w <- tktoplevel(); tkwm.title(w, "Select a color")
 f <- ttkframe(w, padding=c(3,3,3,12))
 tkpack(f, expand=TRUE, fill="both")
 colorWell <- tkcanvas(f, width=40, height=16, 
                       background="#ee11aa",
                       highlightbackground="#ababab") 
 tkpack(colorWell)
 tkpack(ttklabel(f, text="Click color to change"))
 #
 tkbind(colorWell,"<Button-1>", function(W) {
   color <- tcl("tk_chooseColor", parent=W, 
                title="Set box color")
   color <- tclvalue(color)
   print(color)
   if(nchar(color))
     tkconfigure(W, background = color)
 })
\end{Sinput}
\end{Schunk}



\section{Selection Widgets}
\label{sec:tcltk:selection-widgets}

This section covers the many different ways to present data for the
user to select a value. The widgets can use \TCL\/ variables to refer
to the value that is displayed or that the user selects.  Recall,
these were constructed through \function{tclVar} and manipulated
through \code{tclvalue}.  For example, a logical value can be stored as
\begin{Schunk}
\begin{Sinput}
 value <- tclVar(TRUE)
 tclvalue(value) <- FALSE
 tclvalue(value)
\end{Sinput}
\begin{Soutput}
[1] "0"
\end{Soutput}
\end{Schunk}
As \code{tclvalue} coerces the logical into the  character string  \qcode{0} or \qcode{1}, some coercion may be desired.

\subsection{Checkbutton}
\label{sec:tcltk:checkboxes}

The \constructor{ttkcheckbutton} constructor returns a check button
object. The checkbuttons value (\code{TRUE} or \code{FALSE}) is linked
to a \TCL\/ variable which can be specified using a logical value.
The checkbutton label can also be specified through a \TCL\/ variable
using the \option{textvariable}{ttkcheckbutton} option.  Alternately,
as with the \code{ttklabel} constructor, the label can be specified
through the \option{text}{ttkcheckbutton} option. As well, one can
specify an image and arrange its display -- as is done with
\function{ttklabel} -- using the \option{compound}{ttkcheckbutton}
option.

The \option{command}{ttkcheckbutton} argument is used at construction
time to specify a callback when the button is clicked. The callback is
called when the state toggles, so often a callback considers the state
of the widget before proceeding.  To add a callback with
\function{tkbind} use \code{\Event{ButtonRelease-1}}, as the callback
for the event \code{\Event{Button-1}} is called before the variable is
updated.

For example, if \code{f} is a frame, we can create a new check button
with the following:

\begin{Schunk}
\begin{Sinput}
 value <- tclVar(TRUE)
 callback <- function() print(tclvalue(value))   # uses global
 labelVar <- tclVar("check button label")
 cb <- ttkcheckbutton(f, variable=value, 
                      textvariable=labelVar, command=callback)
 tkpack(cb)
\end{Sinput}
\end{Schunk}

To avoid using a global variable is not trivial here. There is no easy
way to pass user data through to the callback, and there is no easy
way to get the \R\/ object from the values passed through the \%
substitution values. The variable holding the value can be found
through
\begin{Schunk}
\begin{Sinput}
 tkcget(cb, "variable"=NULL)
\end{Sinput}
\begin{Soutput}
<Tcl> ::RTcl7 
\end{Soutput}
\end{Schunk}

But then, one needs a means to lookup the variable from this id. Here is a
wrapper for the \function{tclVar} function and a lookup function that
use an environment created by the \pkg{tcltk} package in place of a
global variable.

\begin{Schunk}
\begin{Sinput}
 ourTclVar <- function(...) {
   var <- tclVar(...)
   .TkRoot$env[[as.character(var)]] <- var
   var
 }
 ## lookup function
 getTclVarById <- function(id) {
   .TkRoot$env[[as.character(id)]]
 }
\end{Sinput}
\end{Schunk}

Assuming we used \function{ourTclVar} above, then the callback above
could be defined to avoid a global variable by:


\begin{Schunk}
\begin{Sinput}
 callback <- function(W) {
   id <- tkcget(W, "variable"=NULL)
   print(getTclVarById(id))
 }
\end{Sinput}
\end{Schunk}


In Section~\ref{sec:tcltk:scale-widgets} we demonstrate how to encapsulate the widget and its
variable in a reference class so that one need not worry about scoping
rules to reference the variable. 


\paragraph{A toggle button}

By default the widget draws with a check box. Optionally the widget
can be drawn as a button, which when depressed indicates a \code{TRUE}
state. This is done by using the style \code{Toolbutton}, as in:
\begin{Schunk}
\begin{Sinput}
 tkconfigure(cb, style="Toolbutton")
\end{Sinput}
\end{Schunk}

The ``Toolbutton'' style is for placing widgets into toolbars.

\subsection{Radio Buttons}
\label{sec:tcltk:radio-buttons}

Radiobuttons are basically differently styled checkbuttons linked through a shared \TCL\/
variable. Each radio button is constructed through the
\constructor{ttkradiobutton} constructor. Each button has both a value and
a text label, which need not be the same. The
\option{variable}{ttkradiobutton} option refers to the
value. As with labels, the radio button labels may be specified
through a text variable or the \option{text}{ttkradiobutton} option,
in which case, as with a \code{ttklabel}, an image may also be
incorporated through the \option{image}{ttkradiobutton} and
\option{compound}{ttkradiobutton} options. In \TK\/ the placement of
the buttons is managed by the programmer.


This small example shows how radio buttons could be used for selection
of an alternative hypothesis, assuming \code{f} is a parent container.

\begin{Schunk}
\begin{Sinput}
 values <- c("less", "greater", "two.sided")
 var <- tclVar(values[3])                # initial value
 callback <- function() print(tclvalue(var))
 sapply(values, function(i) {
   rb <- ttkradiobutton(f, variable=var, 
                        text=i, value=i, 
                        command=callback)
   tkpack(rb, side="top", anchor="w")
 })
\end{Sinput}
\end{Schunk}


\subsection{Combo boxes}
\label{sec:tcltk:comboboxes}

The \constructor{ttkcombobox} constructor returns a combo box object
allowing for
selection from a list of values, or, with the appropriate option, allowing
the user to specify a value. Like radiobuttons and checkbuttons, the
value of the combo box can be specified using a \TCL\/ variable to the
option \option{textvariable}{ttkcombobox}, making the getting and
setting of the displayed value straightforward. The possible values to
select from are specified as a character vector through the
\option{values}{ttkcombobox} option. (This may require one to coerce
the results to the desired class.)

Unlike \GTK{} and \Qt{} there is no option to include images in the
displayed text. One can adjust the alignment through the
\option{justify}{ttkcombobox} options.  By default, a user can add in
additional values through the entry widget part of the combo box. The
\option{state}{ttkcombobox} option controls this, with the default
\qcode{normal} and the value \qcode{readonly} as an alternative.

To illustrate, again suppose \code{f} is a parent container. Then we
begin by defining some values to choose from and a \TCL\/ variable.


\begin{Schunk}
\begin{Sinput}
 values <- state.name
 var <- tclVar(values[1])              # initial value
\end{Sinput}
\end{Schunk}

The constructor call is as follows:
\begin{Schunk}
\begin{Sinput}
 cb <- ttkcombobox(f,
                   values=values,
                   textvariable=var,
                   state="normal",     # or "readonly"
                   justify="left")
 tkpack(cb)
\end{Sinput}
\end{Schunk}


The possible values the user can select from can be configured after
construction through the \option{values}{ttkcombobox} option:
\begin{Schunk}
\begin{Sinput}
 tkconfigure(cb, values=tolower(values))
\end{Sinput}
\end{Schunk}

There is one case where the above won't work: when there is a single
value and this value contains spaces. In this case, one can coerce the
value to be of class \class{tclObj}:
\begin{Schunk}
\begin{Sinput}
 tkconfigure(cb, values=as.tclObj("New York"))
\end{Sinput}
\end{Schunk}

\paragraph{Setting the value}
Setting values can be done through the \TCL\/ variable. As well, the
value can be set by value using the \subcommand{set}{ttkcombobox} sub
command through \function{tkset} or by index (0-based) using the
\subcommand{current}{ttkcombobox} sub command.

\begin{Schunk}
\begin{Sinput}
 tclvalue(var) <- values[2]            # using tcl variable
 tkset(cb, values[4])                  # by value
 tcl(cb, "current", 4)                 # by index
\end{Sinput}
\end{Schunk}


\paragraph{Getting the value}
One can retrieve the selected object in various ways: from the \TCL\/
variable. Additionally, the \subcommand{get}{ttkcombobox} subcommand
can be used through \function{tkget}.

\begin{Schunk}
\begin{Sinput}
 tclvalue(var)                           # TCL variable
\end{Sinput}
\begin{Soutput}
[1] "california"
\end{Soutput}
\begin{Sinput}
 tkget(cb)                               # get subcommand
\end{Sinput}
\begin{Soutput}
<Tcl> california 
\end{Soutput}
\begin{Sinput}
 tcl(cb, "current")                      # 0-based index
\end{Sinput}
\begin{Soutput}
<Tcl> 4 
\end{Soutput}
\end{Schunk}


\paragraph{Events}
The virtual event \code{\VirtualEvent{ComboboxSelected}} occurs with
selection. When the combo box may be edited, a user may expect some
action when the \kbd{return} key is pressed. This triggers a
\code{\Event{Return}} event. To bind to this event, one can do something
like the following:

\begin{Schunk}
\begin{Sinput}
 tkbind(cb, "<Return>", function(W) {
   val <- tkget(W)
   cat(as.character(val), "\n")
 })
\end{Sinput}
\end{Schunk}

For editable combo boxes, the widget also supports some of the
\function{ttkentry} commands discussed in
Section~\ref{sec:tcltk:entry-widgets}.


% \subsection{Listboxes}
% \label{sec:tcltk:listboxes}

% The \constructor{tklistbox} is a non-themed widget that can be used
% to select from a table of values. One can do a similar thing using
% the more general tree widget provided by \function{ttktreeview}
% widget, but the listbox is more convenient to use.



\subsection{Scale widgets}
\label{sec:tcltk:scale-widgets}

The \constructor{ttkscale} constructor to produce a themeable scale
(slider) control is missing~\footnote{As of the version of \pkg{tcltk}
  accompanying \R{} 2.12.0}. You can define your own simply enough:
\begin{Schunk}
\begin{Sinput}
 ttkscale <- function(parent, ...) 
   tkwidget(parent, "ttk::scale", ...)
\end{Sinput}
\end{Schunk}

The orientation is set through the option \option{orient}{ttkscale}
taking values of \qcode{horizontal} (the default) or
\qcode{vertical}. For sizing the slider, the \option{length}{ttkscale}
option is available.  

To set the range, the basic options are \option{from}{ttkscale} and
\option{to}{ttkscale}. There is no \code{by} option as of \TK\/
8.5. The constructor \constructor{tkscale}, for a non-themeable slider,
has the option \option{resolution}{tkscale} to set that. Additionally,
the themeable slider does not have any label or tooltip indicating its
current value.


As a workaround, we show how to display a vector of values by
sliding through the indices and place labels at the ends of the slider
to indicate the range. We write this using an \R{} reference class. 

\begin{Schunk}
\begin{Sinput}
 Slider <-
   setRefClass("TtkSlider",
      fields=c("frame", "widget", "v", "x"),
      methods=list(
        initialize=function(parent, x) {
          x <<- x
          v <<- tclVar(1)
          frame <<- ttkframe(parent)
          widget <<- ttkscale(frame, from=1, to=length(x),
                              variable=v, orient="horizontal")
          #
          tkgrid(widget, row=0, column=0, columnspan=3, 
                 sticky="we")
          tkgrid(ttklabel(frame, text=x[1]), 
                 row=1, column=0)
          tkgrid(ttklabel(frame, text=x[length(x)]), 
                 row=1, column=2)
          tkgrid.columnconfigure(frame, 1, weight=1)
          #
          .self
        },
        get_value=function() x[as.numeric(tclvalue(v))],
        set_value=function(value) {
          "Set value. Value must be in x"
          ind <- match(value, x)
          if(!is.na(ind)) {
            v_local <- v
            tclvalue(v_local) <- ind
          }
        }
        ))
\end{Sinput}
\end{Schunk}

To use this, we have:
\begin{Schunk}
\begin{Sinput}
 w <- tktoplevel()
 x <- seq(0,1,by=0.05)
 s <- Slider$new(parent=w, x=x)
 tkpack(s$frame, expand=TRUE, fill="x")
 #
 s$set_value(0.5)
 print(s$get_value())
\end{Sinput}
\begin{Soutput}
[1] 0.5
\end{Soutput}
\end{Schunk}

As seen, the \option{variable}{ttkscale} option can be used for
specifying a \TCL\/ variable to record the value of the slider. This
is convenient when the variable and widget are encapsulated into a
class, as above. Otherwise the \option{value}{ttkscale} option is
available.  The \function{tkget} and \function{tkset} function (using
the \subcommand{get}{ttkscale} and \subcommand{set}{ttkscale} sub
commands) can be used to get and set the value shown. They are used in
the same manner as the same-named subcommands for a combo box.

Again, the \option{command}{ttkscale} option can be used to specify a
callback for when the slider is manipulated by the user. E.g.:
\begin{Schunk}
\begin{Sinput}
 tkconfigure(s$widget, command=function(...) {
   print(s$get_value())
 })
\end{Sinput}
\end{Schunk}
For this widget, the callback is passed a value which we ignore above.


\subsection{Spin boxes}
\label{sec:tcltk:spinboxes}

In \TK\/ version 8.5 there is no themeable spinbox widget. In \TK\/ the
\code{spinbox} command produces a non-themeable spinbox. Again, there
is no direct \constructor{tkspinbox} constructor, but one can be
defined with:
\begin{Schunk}
\begin{Sinput}
 tkspinbox <- function(parent, ...) 
   tkwidget(parent, "tk::spinbox", ...)
\end{Sinput}
\end{Schunk}

The non-themeable widgets have many more options than the themeable
ones, as style properties can be set on a per-widget basis. We won't
discuss those here. The spinbox can be used to select from a sequence
of numeric values or a vector of character values.


For example, the following allows a user to scroll either direction through the 50
states of the U.S.

\begin{Schunk}
\begin{Sinput}
 w <- tktoplevel()
\end{Sinput}
\end{Schunk}
\begin{Schunk}
\begin{Sinput}
 sp <- tkspinbox(w, values=state.name, wrap=TRUE)
\end{Sinput}
\end{Schunk}

Whereas, this invocation will allow scrolling through a numeric sequence:
\begin{Schunk}
\begin{Sinput}
 sp1 <- tkspinbox(w, from=1, to=10, increment=1)
\end{Sinput}
\end{Schunk}

\begin{Schunk}
\begin{Sinput}
 tkpack(sp)
 tkpack(sp1)
\end{Sinput}
\end{Schunk}


The basic options to set the range for a numeric spinbox are
\option{from}{tkspinbox}, \option{to}{tkspinbox}, and
\option{increment}{tkspinbox}.  The \option{textvariable}{tkspinbox}
option can be used to link the spinbox to a \TCL\/ variable. As usual,
this allows the user to easily get and set the value
displayed. Otherwise, the \function{tkget} and \function{tkset}
functions may be used for these tasks. 

As seen, in \TK, spin boxes can also be used to select from a list of
text values. These are specified through the
\option{values}{tkspinbox} option. In the \code{state.name} example
above, we set the \option{wrap}{tkspinbox} option to \code{TRUE} so
that the values wrap around when the end is reached.
 
The option \option{state}{tkspinbox} can be used to specify whether
the user can enter values, the default of \qcode{normal}; not edit the
value, but simply select one of the given values (\qcode{readonly}),
or not select a value (\qcode{disabled}).  As with a combo box, when
the \TK\/ spinbox displays character data and is in the \qcode{normal}
state, the widget can be controlled like the entry widget of
Section~\ref{sec:tcltk:entry-widgets}.



% \begin{example}{A GUI for selecting a numeric range}{ex-tcltk-doublescale}
%   \SweaveInput{ex-tcltk-doublescale}
% \end{example}

%% Too much? It ain't pretty
\begin{example}{A GUI for \code{t.test}}{ex-tcltk-t-test}
This example illustrates how the basic widgets can be combined to make
a dialog for gathering information to run a $t$-test. A realization is shown in Figure~\ref{fig:fig-tcltk-t-test}.

\begin{figure}
  \centering
  \includegraphics[width=.75\textwidth]{fig-tcltk-t-test.png}
  \caption{A dialog to collect values for a $t$ test.}
  \label{fig:fig-tcltk-t-test}
\end{figure}





We will use a data store to hold the values to be passed to
\code{t.test}. For the data store, we  use an environment to hold \Tcl\/ variables.

\begin{Schunk}
\begin{Sinput}
 e <- new.env()
 e$x <- tclVar(""); e$f <- tclVar(""); e$data <- tclVar("")
 e$mu <- tclVar(0); e$alternative <- tclVar("two.sided")
 e$conf.level <- tclVar(95); e$var.equal <- tclVar(FALSE)
\end{Sinput}
\end{Schunk}

This allows us to write a function to evaluate a $t$-test easily
enough, although we don't illustrate that.






Our layout is basic. Here we pack a label frame into the window to give the dialog a nicer look.
We will use the \code{tkgrid} geometry manager below.
\begin{Schunk}
\begin{Sinput}
 lf <- ttklabelframe(f, text="t.test()", padding=10)
 tkpack(lf, expand=TRUE, fill="both", padx=5, pady=5)
\end{Sinput}
\end{Schunk}


This helper function simplifies the task of adding a label.
\begin{Schunk}
\begin{Sinput}
 putLabel <- function(parent, text, row, column) {
   label <- ttklabel(parent, text=text)
   tkgrid(label, row=row, column=column, sticky="e")
 }
\end{Sinput}
\end{Schunk}
%

Our first widget will be one to select a data frame. For this, a
combo box is used, although if a large number of data frames are a
possibility, a different widget may be better suited. The
\code{getDfs} function is not shown, but simply returns the names of
all data frames in the global environment. Also not shown are two
similar calls to create combo boxes \code{xCombo} and \code{fCombo}
which allow the user to specify parts of a formula.

\begin{Schunk}
\begin{Sinput}
 putLabel(lf, "data:",0,0)
 dataCombo <- ttkcombobox(lf, state="readonly", values=getDfs(), 
                          textvariable=e$data)
 tkgrid(dataCombo, row=0, column=1, sticky="ew", padx=2)
 tkfocus(dataCombo)                      # give focus
\end{Sinput}
\end{Schunk}



We jump ahead and use a \constructor{ttkentry} widget for the user to specify
a mean. For this purpose, the use is straightforward.
\begin{Schunk}
\begin{Sinput}
 putLabel(lf, "mu:", 2, 0)
 muCombo <-  ttkentry(lf,  textvariable=e$mu)
 tkgrid(muCombo, row=2, column=1, sticky="ew", padx=2)
\end{Sinput}
\end{Schunk}

The selection of an alternative hypothesis is a natural choice for a
combo box or a radio button group, we use the latter.
\begin{Schunk}
\begin{Sinput}
 putLabel(lf, "alternative:", 3, 0)
 rbFrame <- ttkframe(lf)
 sapply(c("two.sided","less","greater"), function(i) {
   rb <- ttkradiobutton(rbFrame, variable=e$alternative, 
                        text=i, value=i)
   tkpack(rb, side="left")
 })
 tkgrid(rbFrame, row=3, column=1, sticky="ew", padx=2)
\end{Sinput}
\end{Schunk}

Here we use two widgets to specify the confidence level. The slider
is quicker to use, but less precise than the spinbox. By sharing a
text variable, the widgets are automatically synchronized.
\begin{Schunk}
\begin{Sinput}
 putLabel(lf, "conf.level:", 3, 2)
 confFrame <- ttkframe(lf)
 tkgrid(confFrame, row=3, column=3, columnspan=2, 
        sticky="ew", padx=2)
 confScale <- ttkscale(confFrame, from=75, to=100, 
                      variable=e$conf.level)
 tkpack(confScale, expand=TRUE, fill="y", side="left")
 confSpin <- tkspinbox(confFrame, from=75, to=100, increment=1, 
                      textvariable=e$conf.level, width=5)
 tkpack(confSpin, side="left")
\end{Sinput}
\end{Schunk}

A checkbox is used to set the binary variable for \code{var.equal}
\begin{Schunk}
\begin{Sinput}
 putLabel(lf, "var.equal:", 4, 0)
 veCheck <- ttkcheckbutton(lf, variable=e$var.equal)
 tkgrid(veCheck, row=4, column=1, stick="w", padx=2)
\end{Sinput}
\end{Schunk}

When assigning grid weights, we don't want the labels (columns 0 and 2) to expand the same way we want the other columns to do, so we assign different weights.
\begin{Schunk}
\begin{Sinput}
 tkgrid.columnconfigure(lf, 0, weight=1)
 tkgrid.columnconfigure(lf, 1, weight=10)
 tkgrid.columnconfigure(lf, 2, weight=1)
 tkgrid.columnconfigure(lf, 1, weight=10)
\end{Sinput}
\end{Schunk}

The dialog has standard cancel and ok buttons.
\begin{Schunk}
\begin{Sinput}
 bf <- ttkframe(f)
 cancel <- ttkbutton(bf, text="cancel")
 ok <- ttkbutton(bf, text="ok")
 #
 tkpack(bf, fill="x", padx=5, pady=5)
 tkpack(ttklabel(bf, text=" "), expand=TRUE, fill="y", 
        side="left")                     # add a spring
 tkpack(cancel, padx=6, side="left")
 tkpack(ok, padx=6, side="left")
\end{Sinput}
\end{Schunk}

For the \code{ok} button we want to gather the values and run the
function. The \code{runTTest} function does this.  We configure both
buttons, then add to the default button bindings to invoke either of the button's commands
when they have the focus and \kbd{return} is pressed.
\begin{Schunk}
\begin{Sinput}
 tkconfigure(ok, command=runTTest)
 tkconfigure(cancel, command=function() tkdestroy(w))
 tkbind("TButton", "<Return>", function(W) tcl(W, "invoke"))
\end{Sinput}
\end{Schunk}

At this point, our GUI is complete, but we would like to have it
reflect any changes to the underlying \R\/ environment that effect its
display. A such, we define
a function, \code{updateUI}, which does two basic things: it searches for
new data frames and it adjusts the controls depending on the current
state.
\begin{Schunk}
\begin{Sinput}
 updateUI <- function() {
   dfName <- tclvalue(e$data)
   curDfs <- getDfs()
   tkconfigure(dataCombo, values=curDfs)
   if(!dfName %in% curDfs) {
     dfName <- ""
     tclvalue(e$data) <- ""
   }
 
   if(dfName == "") {
     ## 3 ways to disable needed !
     x <- list(xCombo, fCombo, muCombo,  confScale, veCheck, ok)
     sapply(x, function(W) tcl(W, "state", "disabled"))
     sapply(as.character(tkwinfo("children", rbFrame)), 
            function(W) tcl(W, "state", "disabled"))
     tkconfigure(confSpin, state="disabled")
   } else {
     ## enable univariate, ok
     sapply(list(xCombo,  muCombo, confScale, ok),
            function(W) tcl(W, "state", "!disabled"))
     sapply(as.character(tkwinfo("children", rbFrame)), 
            function(W) tcl(W, "state", "!disabled"))
     tkconfigure(confSpin, state="normal")
     
     df <- get(dfName, envir=.GlobalEnv)
     numVars <- getNumericVars(df)
     tkconfigure(xCombo, values=numVars)
     if(! tclvalue(e$x) %in% numVars)
       tclvalue(e$x) <- ""
 
     ## bivariate
     availFactors <- getTwoLevelFactor(df)
     sapply(list(fCombo, veCheck),
            function(W) {
              val <- if(length(availFactors)) "!" else ""
              tcl(W, "state", sprintf("%sdisabled", val))
            })
     tkconfigure(fCombo, values=availFactors)
     if(!tclvalue(e$f) %in% availFactors)
       tclvalue(e$f) <- ""
       
          }
 }
 updateUI()
 tkbind(dataCombo, "<<ComboboxSelected>>", updateUI)
\end{Sinput}
\end{Schunk}

This function could be bound to a ``refresh'' button or we could
arrange to have it called in the background. Using the \code{after}
command we could periodically check for new data frames, using a task
callback we can call this every time a new command is issued.  As the
call could potentially be costly, we only call if the available data
frames have been changed. Here is one way to arrange that:
\begin{Schunk}
\begin{Sinput}
 require(digest)
 create_function <- function() {
   .dfs <- digest(getDfs())
   f <- function(...) {
     if((val <- digest(getDfs())) != .dfs) {
       .dfs <<- val
       updateUI()
     }
     return(TRUE)
   }
 }
\end{Sinput}
\end{Schunk}
Then to create a task callback we have
\begin{Schunk}
\begin{Sinput}
 id <- addTaskCallback(create_function())
\end{Sinput}
\end{Schunk}

% <<>>=
% updateUI()                              # run once
% tkbind("TCombobox","<<ComboboxSelected>>", updateUI) ## misses update on new data
% @ 
\end{example}


\section{Text widgets}
\label{sec:tcltk:text-widgets}
\Tk\/ provides both single- and multi-line text entry widgets. The
section describes both and introduces scrollbars which are often
desired for multi-line text entry.

\subsection{Entry Widgets}
\label{sec:tcltk:entry-widgets}

The \constructor{ttkentry} constructor provides a single line text
entry widget. The widget can be associated with a \TCL\/ variable at
construction to facilitate getting and setting the displayed values
through its argument \argument{textvariable}{ttkentry}. The width of
the widget can be adjusted at construction time through the
\argument{width}{ttkentry} argument. This takes a value for the number
of characters to be displayed, assuming average-width characters.  The
text alignment can be set through the \argument{justify}{ttkentry}
argument taking values of \qcode{left} (the default), \qcode{right}
and \qcode{center}. For gathering passwords, the argument
\argument{show}{ttkentry} can be used, such as with
\code{show=}\qcode{*}, to show asterisks in place of all the
characters.

The following constructs a basic example
\begin{Schunk}
\begin{Sinput}
 eVar <- tclVar("initial value")
 e <- ttkentry(w, textvariable=eVar)
 tkpack(e)
\end{Sinput}
\end{Schunk}

We can get and set values using the \TCL\/ variable.
\begin{Schunk}
\begin{Sinput}
 tclvalue(eVar)
\end{Sinput}
\begin{Soutput}
[1] "initial value"
\end{Soutput}
\begin{Sinput}
 tclvalue(eVar) <- "set value"
\end{Sinput}
\end{Schunk}

The \code{get} command can also be used.
\begin{Schunk}
\begin{Sinput}
 tkget(e)
\end{Sinput}
\begin{Soutput}
<Tcl> set value 
\end{Soutput}
\end{Schunk}

\paragraph{Indices}
The entry widget uses an index to record the different positions
within the entry box. This index can be a number (0-based), an
$x$-coordinate of the value (\code{@x}), or one of the values
\qcode{end} and \qcode{insert} to refer to the end of the current text
and the insert point as set through the keyboard or mouse. The mouse
can also be used to make a selection. In this case the indices
\qcode{sel.first} and \qcode{sel.last} describe the selection.

With indices, we can insert  text with the \subcommand{insert}{ttkentry} command
\begin{Schunk}
\begin{Sinput}
 tkinsert(e, "end", "new text")
\end{Sinput}
\end{Schunk}

Or, we can delete a range of text, in this case the first 4
characters, using \subcommand{delete}{ttkentry}. The first value is
the left most index to delete (0-based), the second value the index to
the right of the last value deleted.
\begin{Schunk}
\begin{Sinput}
 tkdelete(e, 0, 4)
\end{Sinput}
\end{Schunk}

The \subcommand{icursor}{ttkentry} command can be used to set the
cursor position to the specified index.
\begin{Schunk}
\begin{Sinput}
 tkicursor(e, 0)                         # move to beginning
\end{Sinput}
\end{Schunk}

Finally, we note that the selection can be adjusted using the
\subcommand{selection range}{ttkentry} subcommand. This takes two
indices. Like \code{delete}, the first index specifies the first character of
the selection, the second indicates the character to the right of the
selection boundary. The following example would select all the text.
\begin{Schunk}
\begin{Sinput}
 tkselection.range(e, 0, "end")
\end{Sinput}
\end{Schunk}
The \subcommand{selection clear}{ttkentry} subcommand clears the selection and \subcommand{selection present}{ttkentry} signals if a selection is currently made.

\paragraph{Events}
Several useful events include \code{\Event{KeyPress}} and
\code{\Event{KeyRelease}} for key presses and \code{\Event{FocusIn}}
and \code{\Event{FocusOut}} for focus events.

% The examples show a bit more about the entry widget. The first  shows how styles can be used to adjust the look of an entry widget, and the second how to validate the users data entry in an entry widget.

%% XXX This is too technical. -- put into package?
% \begin{example}{Using styles to adjust the look of an entry widget}{ex:tcltk-searchentry}
%   \SweaveInput{ex-tcltk-searchentry.Rnw}
% \end{example}

\begin{example}{Putting in default text}{ex:tcltk-entry-initial-message}

In this example we show how to augment the \function{ttkentry} widget
to allow the inclusion of an initial message to direct the user. As
soon as the user focuses the entry area, say by clicking their mouse
on it, the initial message clears and the user can type in their value.

We use an \R{} reference class for our programming, as it nicely allows us to
encapsulate the entry widget, its tclvariable and the initial
message. For formatting purposes, we define the methods first, then
the class.

We begin with two methods to get and set the text. The field
\code{v} will hold our tclvariable.

\begin{Schunk}
\begin{Sinput}
 get_text <- function() {
   "Get the text value"
   if(!is_init_msg())
     as.character(tclvalue(v))
   else
     ""
 }
 #
 set_text <- function(text, hide=TRUE) {
   "Set text into widget"
   if(hide)
     hide_init_msg()
   v_local <- v
   tclvalue(v_local) <- text
 }
\end{Sinput}
\end{Schunk}

For the initial message we define three methods. The first checks if
the current state is the initial message. Rather than use a flag, we
just check if the text matches the initial message, which is stored in
the field \code{init\_msg}.

\begin{Schunk}
\begin{Sinput}
 is_init_msg <- function() {
   "Is the init text showing?"
   as.character(tclvalue(v)) == init_msg
 }
\end{Sinput}
\end{Schunk}
To indicate to the user that the initial message is not the current
text, we define a style for when the initial message is being
shown. It simply sets the foreground (text) color to gray.

\begin{Schunk}
\begin{Sinput}
 .Tcl("ttk::style configure Gray.TEntry -foreground gray") 
\end{Sinput}
\end{Schunk}

Now our two methods to hide and show the initial message are defined. Outside of
changing the text, we adjust the style option accordingly.
\begin{Schunk}
\begin{Sinput}
 hide_init_msg <- function() {
   "Hide the initial text"
   if(is_init_msg()) {
     tkconfigure(e, style="TEntry")
     set_text("", hide=FALSE)
   }
 }
 #
 show_init_msg <- function() {
   "Show the initial text"
   tkconfigure(e, style="Gray.TEntry")
   set_text(init_msg, hide=FALSE)
 }
\end{Sinput}
\end{Schunk}

To switch between the initial message and the entry area we make
bindings to the focus in and focus out events for the entry
widget. The focus out will show the initial message if there is no
text specified.

\begin{Schunk}
\begin{Sinput}
 add_bindings <- function() {
   "Add focus bindings to make this work"
   tkbind(e, "<FocusIn>", hide_init_msg)
   tkbind(e, "<FocusOut>", function() {
     if(nchar(get_text()) == 0)
       show_init_msg()
   })
 }
\end{Sinput}
\end{Schunk}

Now to create the reference class to hold our widget. First we set as old
classes the class for the entry widget and the tclvariable. This
allows us to use these as fields in the class.
\begin{Schunk}
\begin{Sinput}
 setOldClass("tkwin"); setOldClass("tclVar")
\end{Sinput}
\end{Schunk}
%
Our entry class has a straightforward initialization method, otherwise
we simply piece together the components we have just defined.
\begin{Schunk}
\begin{Sinput}
 ttkEntry <- 
   setRefClass("TtkEntry",
         fields=list(
           e="tkwin", v="tclVar",
           init_msg="character"
           ),
         methods=list(
           initialize=function(parent, text, init_msg="") {
             v <<- tclVar()
             e <<- ttkentry(parent, textvariable=v)
             init_msg <<- init_msg
             if(missing(text))
               show_init_msg()
             else
               set_text(text)
             add_bindings()
             .self
           },
           get_text=get_text,
           set_text=set_text,
           is_init_msg=is_init_msg,
           hide_init_msg=hide_init_msg,
           show_init_msg=show_init_msg,
           add_bindings=add_bindings
           )
         )
\end{Sinput}
\end{Schunk}

Finally, to use this widget we call its \meth{new} method to create an
instance. The actual entry widget is kept in the \code{e} field, so we
pack in that below.
\begin{Schunk}
\begin{Sinput}
 w <- tktoplevel()
 e <- ttkEntry$new(parent=w, init_msg="type value here")
 tkpack(e$e)
 #
 b <- ttkbutton(w, text="focus out onto this", 
                command=function() {
                  print(e$get_text())
                })
 tkpack(b)
\end{Sinput}
\end{Schunk}

\end{example}


\begin{example}{Using validation for dates}{ex:tcltk-validation-dates}



As previously mentioned, there is no native calendar widget in \pkg{tcltk}. This example shows how one can use the validation framework for entry widgets to check that user-entered dates conform to an expected format. 

%% The entry widget man page has many details.
Validation happens in a few steps.  A validation command is assigned
to some event. This call can come in two forms. Prevalidation is when
a change is validated prior to being committed, for example when each key
is pressed.  Revalidation is when the value is checked
after it is sent to be committed, say when the entry widget loses
focus or the enter key is pressed.

When a validation command is called it should check
whether the current state of the entry widget is valid or not. If
valid, it returns a value of \code{TRUE} and \code{FALSE}
otherwise. These need to be \TCL\/ Boolean values, so in the following,
the command \code{tcl("eval","TRUE")} (or \code{tcl("eval", "FALSE")}) is used. If
the validation command returns \code{FALSE}, then a subsequent call to
the specified invalidation command is made.

%% Put these into a table???
For each callback, a number of substitution values are possible, in
addition to the standard ones such as \code{W} to refer to the
widget. These are: \code{d} for the type of validation being done: 1
for insert prevalidation, 0 for delete prevalidation, or -1 for
revalidation; \code{i} for the index of the string to be inserted or
deleted or -1; \code{P} for the new value if the edit is accepted (in
prevalidation) or the current value in revalidation; \code{s} for the
value prior to editing; \code{S} for the string being inserted or
deleted, \code{v} for the current value of \code{validate} and
\code{V} for the condition that triggered the callback.

In the following callback definition we use \code{W} so that we can change the entry text color to black and format the data in a standard manner and \code{P} to get the entry widget's value just prior to validation.


To begin,  we define some patterns for acceptable date formats.
\begin{Schunk}
\begin{Sinput}
 datePatterns <- c()
 for(i in list(c("%m","%d","%Y"),        # U.S. style
               c("%m","%d","%y"))) {
   for(j in c("/","-"," ") )
     datePatterns[length(datePatterns)+1] <- 
       paste(i,sep="", collapse=j)
 }
\end{Sinput}
\end{Schunk}

Our callbacks set the color to black or red, depending on whether we
have a valid date. First our validation command.
\begin{Schunk}
\begin{Sinput}
 isValidDate <- function(W, P) { # P is the current value
   for(i in datePatterns) {
     date <- try( as.Date(P, format=i), silent=TRUE)
     if(!inherits(date, "try-error") && !is.na(date)) {
       tkconfigure(W, foreground="black")  # or use style
       tkdelete(W, 0,"end")
       tkinsert(W, 0, format(date, format="%m/%d/%y"))
       return(tcl("expr","TRUE"))        
     } 
   }
   return(tcl("expr","FALSE"))
 }
\end{Sinput}
\end{Schunk}

This is our invalid command.
\begin{Schunk}
\begin{Sinput}
 indicateInvalidDate <- function(W) {
   tkconfigure(W,foreground="red")
   tcl("expr","TRUE")
 }
\end{Sinput}
\end{Schunk}


The \argument{validate}{ttkentry} argument is used to specify when the
validation command should be called. This can be a value of
\qcode{none} for validation when called through the \code{validation}
command; \qcode{key} for each key press; \qcode{focusin} for when the
widget receives the focus; \qcode{focusout} for when it loses focus;
\qcode{focus} for both of the previous; and \qcode{all} for any of the
previous. We use \qcode{focusout} below, so also give a button widget
so that the focus can be set elsewhere. 
\begin{Schunk}
\begin{Sinput}
 e <- ttkentry(f, validate="focusout",   # f some parent
               validatecommand=isValidDate,
               invalidcommand=indicateInvalidDate)
 tkpack(e, side="left")
 b <- ttkbutton(f,text="click")       # something to focus on
 tkpack(b, side="bottom")
\end{Sinput}
\end{Schunk}
              
\end{example}


\subsection{Scrollbars}
\label{sec:tcltk:scrollbars}

\TK\/ has several scrollable widgets -- those that use scrollbars.
Widgets which accept a scrollbar (without too many extra steps) have
the options \code{xscrollcommand} and \code{yscrollcommand}.  To use
scrollbars in \pkg{tcltk} requires two steps: the scrollbars must be
constructed and bound to some widget, and that widget must be told it
has a scrollbar. This way changes to the widget can update the
scrollbar and vice versa. Suppose, \code{parent} is a container and
\code{widget} has these options, then the following will set up both
horizontal and vertical scrollbars. The scrollbars are defined first, as follows, using the
\option{orient}{ttkscrollbar} option and a command of the following
form.
%
\begin{Schunk}
\begin{Sinput}
 xscr <- ttkscrollbar(parent, orient="horizontal",
                  command=function(...) tkxview(widget, ...))
 yscr <- ttkscrollbar(parent, orient="vertical",
                  command=function(...) tkxview(widget, ...))
\end{Sinput}
\end{Schunk}
%
The \code{view} commands set what part of the widget is being shown.

To link the widget back to the scrollbar, the \code{set} command is
used in a callback to the scroll command.  For this example we
configure the options after the widget is constructed, but this can be
done at the time of construction as well. Again, the command takes a
standard form:
\begin{Schunk}
\begin{Sinput}
 tkconfigure(widget,
             xscrollcommand=function(...) tkset(xscr,...),
             yscrollcommand=function(...) tkset(yscr,...))
\end{Sinput}
\end{Schunk}

Although scrollbars can appear anywhere, the conventional place is on
the right and lower side of the parent. The following adds scrollbars
using the grid manager. The combination of weights and stickiness
below will have the scrollbars expand as expected if the window is
resized.
\begin{Schunk}
\begin{Sinput}
 tkgrid(widget, row=0, column=0, sticky="news")
 tkgrid(yscr,row=0,column=1, sticky="ns")
 tkgrid(xscr, row=1, column=0, sticky="ew")
 tkgrid.columnconfigure(parent, 0, weight=1)
 tkgrid.rowconfigure(parent, 0, weight=1)
\end{Sinput}
\end{Schunk}
%
Although a bit tedious, this gives the programmer some flexibility in
arranging scrollbars. To avoid doing all this in the sequel, we turn
the above into the function \function{addScrollbars} for subsequent
usage (not shown). In base \Tk, there is no means to hide scrollbars
when not needed. The \pkg{tcltk2} has some code that may be employed
for that.


\subsection{Multi-line Text Widgets}
\label{sec:tcltk:multi-line-text}

The \constructor{tktext} widget creates a multi-line text editing
widget. If constructed with no options but a parent container, the
widget can have text entered into it by the user.
%

%% arguments width, height
The text widget is not a themed widget, hence has numerous arguments
to adjust its appearance. We mention a few here and leave the rest to
be discovered in the manual page (along with much else). The argument
\argument{width}{tktext} and \argument{height}{tktext} are there to
set the initial size, with values specifying number of characters and
number of lines (not pixels, to convert see
Section~\ref{sec:tcltk:overview:colors-fonts}). The actual size is
font dependent, with the default for 80 by 24 characters. The
\argument{wrap}{tktext} argument, with a value from \qcode{none},
\qcode{char}, or \qcode{word}, indicates if wrapping is to occur and
if so, does it happen at any character or only a word boundary. The
argument \argument{undo}{tktext} takes a logical value indicating if
the undo mechanism should be used. If so, the subcommand
\subcommand{edit}{tktext} can be used to undo a change (or the
\kbd{control-z} keyboard shortcut).



\paragraph{Indices}
As with the entry widget, several commands take indices to specify
position within the text buffer. Only for the multi-line widget both a
line and character are needed in some instances. These indices may be
specified in many ways. One can use row and character numbers
separated by a period in the pattern \code{line.char}. The line is
$1$-based, the column $0$-based (e.g., \code{1.0} says start on the
1st row and first character). In general, one can specify any line
number and character on that line, with the keyword \code{end} used to
refer to the last character on the line. Text buffers may carry
transient marks, in which case the use of this mark indicates the next
character after the mark. Predefined marks include \code{end}, to  specify the end of the
buffer, \code{insert}, to track the insertion point in the text
buffer were the user to begin typing, and \code{current}, which 
follows the character closest to the mouse position. As well, pieces
of text may be tagged. The format \code{tag.first} and \code{tag.last}
index the range of the tag \code{tag}. Marks and tags are described
below. If the $x$-$y$ postion of the spot is known (through percent
substitutions say) the index can be specified by postion, as \code{\@x,y}.

Indices can also be adjusted relative to the above
specifications. This adjustment can be by a number of characters
(\code{chars}), index positions (\code{indices}) or \code{lines}. For
example, \code{insert + 1 lines} refers to 1 line under the insert
point. The values \code{linestart}, \code{lineend}, \code{wordstart}
and \code{wordend} are also available. For instance, \code{insert
  linestart} is the beginning of the line from the insert point, while
\code{end -1 wordstart} and \code{end - 1 chars wordend} refer to the
beginning and ending of the last word in the buffer. (The \code{end}
index refers to the character just after the new line so we go back 2
steps.)

\paragraph{Getting text}
%% multiline? tclvalue versus as.character
The \subcommand{get}{tktext} subcommand is used to retrieve the text
in the buffer. Coercion to character should be done with
\function{tclvalue} and not \function{as.character} to preserve the
distinction between spaces and line breaks.

\begin{Schunk}
\begin{Sinput}
 value <- tkget(t, "1.0", "end")
 as.character(value)                     # wrong way
\end{Sinput}
\begin{Soutput}
character(0)
\end{Soutput}
\begin{Sinput}
 tclvalue(value)
\end{Sinput}
\begin{Soutput}
[1] "\n"
\end{Soutput}
\end{Schunk}

\paragraph{Inserting text}
Inserting text can be done through the \subcommand{insert}{ttktext}
subcommand by specifying first the index then the text to add. One can
use \code{$\backslash$n} to add new lines.
\begin{Schunk}
\begin{Sinput}
 tkinsert(t, "end", "more text\n new line")  
\end{Sinput}
\end{Schunk}
Images and other windows can be added to a text buffer, but we do not discuss that here.

The buffer can have its contents cleared using \code{tkdelete}, as
with \code{tkdelete(t, "0.0", "end")}.

\paragraph{Panning the buffer: tksee}
After text is inserted, the visible part of buffer may not be what is
desired. The \subcommand{see}{ttktext} sub command is used to position
the buffer on the specified index, its lone argument.


\paragraph{tags}
Tags are a means to assign a name to characters within the text
buffer. Tags may be used to adjust the foreground, background and font
properties of the tagged characters from those specified globally at
the time of construction of the widget, or configured thereafter. Tags
can be set when the text is inserted by appending to the argument
list, as with
\begin{Schunk}
\begin{Sinput}
 tkinsert(t, "end", "last words", "lastWords") # lastWords tag
\end{Sinput}
\end{Schunk}

Tags can be set after the text is added through the
\subcommand{tag add}{tktext} subcommand using indices to specify
location. The following marks the first word with the \code{firstWord}
tag:
\begin{Schunk}
\begin{Sinput}
 tktag.add(t,"firstWord","1.0 wordstart", "1.0 wordend")
\end{Sinput}
\end{Schunk}
The \subcommand{tag configure}{tktext} can be used to configure properties of the tagged characters, for example:
\begin{Schunk}
\begin{Sinput}
 tktag.configure(t, "firstWord", foreground="red", 
                 font="helvetica 12 bold")
\end{Sinput}
\end{Schunk}
%
There are several other configuration options for a tag. From within
an \R\/ session, a cryptic list can be produced by calling the
subcommand \subcommand{tag configure}{tktext} without a value for
configuration.


\paragraph{selection}
The current selection, if any, is indicated by the \code{sel} tag,
with \code{sel.first} and \code{sel.last} providing indices to refer
to the selection (assuming the option \code{exportSelection} was not
modified). These tags can be used with \code{tkget} to retrieve the
currently selected text. An error will be thrown if there is no
current selection. To check if there is a current selection, the following may be used:
\begin{Schunk}
\begin{Sinput}
 hasSelection <- function(W) {
   ranges <- tclvalue(tcl(W, "tag", "ranges", "sel"))
   length(ranges) > 1 || ranges != ""
 }
\end{Sinput}
\end{Schunk}

The cut, copy and paste commands are implemented through the \Tk\/ functions
\code{tk\_textCut}, \code{tk\_textCopy} and
\code{tk\_textPaste}. Their lone argument is the text widget. These
work with the current selection and insert point. For example to cut
the current selection, one has
\begin{Schunk}
\begin{Sinput}
 tcl("tk_textCut", t)
\end{Sinput}
\end{Schunk}

\paragraph{marks}
Tags mark characters within a buffer, marks denote positions within a
buffer that can be modified. For example, the marks \code{insert} and
\code{current} refer to the position of the cursor and the current
position of the mouse. Such information can be used to provide
context-sensitive popup menus, as in this code example:
\begin{Schunk}
\begin{Sinput}
 popupContext <- function(W, x, y) {
   ## or use sprintf("@%s,$s", x, y) for "current"
   cur <- tkget(W, "current wordstart", "current wordend") 
   cur <- tclvalue(cur)
   popupContextMenuFor(cur, x, y)        # some function
 }
\end{Sinput}
\end{Schunk}

To assign a new mark, one uses the \subcommand{mark set}{tktext}
subcommand specifying a name and a position through an index. Marks
refer to spaces within characters. The \code{gravity} of the mark can
be \code{left} or \code{right}. When \code{right} (the default), new
text inserted is to the left of the mark. For instance, to keep track
of an initial insert point and the current one, the initial point
(marked \code{leftlimit} below) can be marked with
\begin{Schunk}
\begin{Sinput}
 tkmark.set(t,"leftlimit","insert")
 tkmark.gravity(t,"leftlimit","left")    # keep onleft
 tkinsert(t,"insert","new text")
 tkget(t, "leftlimit", "insert")
\end{Sinput}
\begin{Soutput}
<Tcl> new text 
\end{Soutput}
\end{Schunk}
%
The use of the subcommand \subcommand{mark gravity}{tktext} is done so
that the mark attaches to the left-most character at the insert
point. The rightmost one changes as more text is inserted, so would
make a poor choice here.


\paragraph{The edit command}
The subcommand \subcommand{edit}{tktext} can be used to undo text. As well, it can be used to test if the buffer has been modified, as follows:
\begin{Schunk}
\begin{Sinput}
 tcl(t, "edit", "undo")                  # no output
 tcl(t, "edit", "modified")              # 1 = TRUE
\end{Sinput}
\begin{Soutput}
<Tcl> 1 
\end{Soutput}
\end{Schunk}



\paragraph{Events}
The text widget has a few important events.  The widget defines
virtual events \code{\VirtualEvent{Modified}} and
\code{\VirtualEvent{Selection}} indicating when the buffer is modified
or the selection is changed. Like the single-line text widget, the
events \code{\Event{KeyPress}} and \code{\Event{KeyRelease}} indicate
key activity. The \%-substitution \code{k} gives the keycode and
\code{K} the key symbol as a string (\code{N} is the decimal number).



\begin{example}{Displaying commands in a text buffer}{ex-tcltk-text}
This example shows how a text buffer can be used to display the output
of \R\/ commands, using an approach modified from \pkg{Sweave}.

We begin by defining tags for formatting purposes.
\begin{Schunk}
\begin{Sinput}
 tktag.configure(t, "commandTag", foreground="blue", 
                 font="courier 12 italic")
 tktag.configure(t, "outputTag", font="courier 12")
 tktag.configure(t, "errorTag", foreground="red", 
                 font="courier 12 bold")
\end{Sinput}
\end{Schunk}

The following function does the work of evaluating a command chunk
then inserting the values into the text buffer, using the different
markup tags specified above to indicate commands from output.

\begin{Schunk}
\begin{Sinput}
 evalCmdChunk <- function(t, cmds) {
   
   cmdChunks <- try(parse(text=cmds), silent=TRUE)
   if(inherits(cmdChunks,"try-error")) {
     tkinsert(t, "end", "Error", "errorTag") # add markup tag
   }
 
   for(cmd in cmdChunks) {
     cutoff <- 0.75 * getOption("width")
     dcmd <- deparse(cmd, width.cutoff = cutoff)
     command <- 
       paste(getOption("prompt"),
             paste(dcmd, collapse=paste("\n", 
                           getOption("continue"), sep="")),
             sep="", collapse="")
     tkinsert(t, "end", command, "commandTag")
     tkinsert(t, "end","\n")
     ## output, should check for errors in eval!
     output <- capture.output(eval(cmd, envir=.GlobalEnv))
     output <- paste(output, collapse="\n")
     tkinsert(t, "end", output, "outputTag")
     tkinsert(t, "end","\n")
   }
 }
\end{Sinput}
\end{Schunk}


We envision this as a piece of a larger GUI which generates the
commands to evaluate. For this example though, we make a simple GUI.

\begin{Schunk}
\begin{Sinput}
 w <- tktoplevel(); tkwm.title(w, "Text buffer example")
 f <- ttkframe(w, padding=c(3,3,3,12))
 tkpack(f, expand=TRUE, fill="both")
 t <- tktext(f, width=80, height = 24)   # default size
 addScrollbars(f, t)
\end{Sinput}
\end{Schunk}
 
This is how it can be used.
\begin{Schunk}
\begin{Sinput}
 evalCmdChunk(t, "2 + 2; lm(mpg ~ wt, data=mtcars)")
\end{Sinput}
\end{Schunk}
\end{example}


\section{Treeview widget}
\label{sec:tcltk:treeview-widget}

The themed treeview widget can be used to display rectangular data,
like a data frame, or hierarchical data. The usage is similar for each
beyond the need to indicate the hierarchical structure of a tree.

\subsection{Rectangular data}

\XXX{Images -- add comment}

%% constructor
The \constructor{ttktreeview} constructor creates the tree
widget. There is no separate model for this widget, but there is a
means to adjust what is displayed.  The argument
\argument{columns}{ttktreeview} is used to specify internal names for
the columns and indicate the number of columns. A value of \code{1:n}
will work here unless explicit names are desired. The argument
\argument{displaycolunms}{ttktreeview} is used to control which of the
columns are actually displayed. The default is \qcode{all}, but a
vector of indices or names can be given.  The size of the widget is
specified two different ways.  The \argument{height}{ttktreeview}
argument is used to adjust the number of visible rows. The width of
the widget is determined by the combined widths of each column, whose
adjustments are mentioned later.



If \code{f} is a frame, then the following call will create a widget
with just one column showing 25 rows, like the older, non-themed,
listbox widget of \Tk.

\begin{Schunk}
\begin{Sinput}
 tr <- ttktreeview(f, 
                   columns=1,        # column identifier is "1"
                   show="headings",  # not "#0"
                   height=25)        
 addScrollbars(f, tr)                # scrollbar function
\end{Sinput}
\end{Schunk}



The treeview widget has an initial column for showing the tree-like
aspect with the data. This column is referenced by \code{\#0}. The
\argument{show}{ttktreeview} argument controls whether this column is
shown. A value of \qcode{tree} leaves just this column shown,
\qcode{headings} will show the other columns, but not the first, and
the combined value of \qcode{tree headings} will display both (the
default).  Additionally, the treeview is a scrollable widget, so has
the arguments \argument{xscrollcommand}{ttktreeview} and
\argument{yscrollcommand}{ttktreeview} for specifying scrollbars.

\paragraph{Adding values}

Rectangular data has a row and column structure. In \R, data frames
are internally stored by column vectors, so each column may have its
own type. The treeview widget is different, it stores all data as
character data and one interacts with the data row by row.

Values can be added to the widget through the
\subcommanda{insert}{ttktreeview}{parent item [text] [values]}
subcommand. This requires the specification of a parent (always
\qcode{} for rectangular data) and an index for specifying the
location of the new child amongst the previous children. The special
value \qcode{end} indicates placement after all other children, as
would a number larger than the number of children. A value of 0 or a
negative value would put it at the beginning.


In the example this is how we can add a list of possible CRAN mirrors
to the treeview display.
\begin{Schunk}
\begin{Sinput}
 x <- getCRANmirrors()
 Host <- x$Host
 shade <- c("none", "gray")                     # tag names
 for(i in 1:length(Host))
   ID <- tkinsert(tr, "", "end", values=as.tclObj(Host[i]),
                  tag=shade[i %% 2])            # none or gray
 tktag.configure(tr, "gray", background="gray95") # shade rows
\end{Sinput}
\end{Schunk}

For filling in each row's content the \code{values} option is
used. If there is a single column, like the current example, care
needs to be taken when adding a value. The call to
\function{as.tclObj} prevents the widget from dropping values after
the first space. Otherwise, we can pass a character vector of the
proper length.


There are a number of other options for each row. If column \code{\#0}
is present, the \code{text} option is used to specify the text for the
tree row and the option \code{image} can be given to specify an image
to place to the left of the text value. Finally, we mention the
\code{tag} option for \code{insert} that can be used to specify a tag
for the inserted row. This allows the use of the subcommand
\subcommand{tag configure}{ttktreeview} to configure the foreground
color, background color, font or image of an item.



\paragraph{Column properties}
%% column properties: heading, width, minwidth, stretch
The columns can be configured on a per-column basis. Columns can be
referred to by the name specified through the \code{columns} argument
or by number starting at 1 with \qcode{\#0} referring to the tree
column. The column headings can be set through the
\subcommand{heading}{ttktreeview} subcommand. The heading, similar to
the button widget, can be text, an image or both. The text placement
of the heading may be positioned through the \code{anchor} option. For
example, this command will center the text heading of the first
column:
\begin{Schunk}
\begin{Sinput}
 tcl(tr, "heading", 1, text="Host", anchor="center")
\end{Sinput}
\end{Schunk}

The \subcommand{column}{ttktreeview} subcommand can be used to adjust
a column's properties including the size of the column. The option
\code{width} is used to specify the pixel width of the column (the
default is large); As the widget may be resized, one can specify the
minimum column width through the option \code{minwidth}. When more
space is allocated to the tree widget, than is requested by the
columns, column with a \code{TRUE} value specified to the option
\code{stretch} are resized to fill the available space. Within each
column, the placement of each entry within a cell is controlled by the
\code{anchor} option, using the compass points.

For example, this command will adjust properties of the lone column of \code{tr}:
\begin{Schunk}
\begin{Sinput}
 tcl(tr, "column", 1, width=400,  stretch=TRUE, anchor="w")
\end{Sinput}
\end{Schunk}

\paragraph{Item IDs}
%% referring to rows ID
Each row has a unique item ID generated by the widget when a row is
added. The base ID is \qcode{} (why this is specified for the value of
\code{parent} for rectangular data). For rectangular displays, the
list of all IDs may be found through the \subcommand{children}{ttktreeview}
sub command, which we will describe in the next section.  Here we see
it used to find the children of the root. As well, we show how the
\subcommand{index}{ttktreeview} command returns the row index.
\begin{Schunk}
\begin{Sinput}
 children <- tcl(tr, "children", "")
 (children <- head(as.character(children)))     # as.character
\end{Sinput}
\begin{Soutput}
[1] "I001" "I002" "I003" "I004" "I005" "I006"
\end{Soutput}
\begin{Sinput}
 sapply(children, function(i) tclvalue(tkindex(tr, i)))
\end{Sinput}
\begin{Soutput}
I001 I002 I003 I004 I005 I006 
 "0"  "1"  "2"  "3"  "4"  "5" 
\end{Soutput}
\end{Schunk}

%% retrieving values
\paragraph{Retrieving values}
The \subcommand{item}{ttktreeview} subcommand can be used to get the
values and other properties stored for each row. One specifies the item and the
corresponding option:
\begin{Schunk}
\begin{Sinput}
 x <- tcl(tr, "item", children[1], "-values") # no tkitem
 as.character(x)
\end{Sinput}
\begin{Soutput}
[1] "University of Melbourne"
\end{Soutput}
\end{Schunk}
%
The value returned from the \code{item} command can be difficult to
parse, as \TCL\/ places braces around values with blank spaces. The coercion through
\code{as.character} works much better at extracting the individual
columns. A possible alternative to using the \code{item} command, is
to instead keep the original data frame and use the index of the item
to extract the value from the original.

%% deleting values
\paragraph{Moving and deleting items}
The \subcommand{move}{ttktreeview} subcommand can be used to replace a
child. As with the \code{insert} command, a parent and an index for
where the new child is to go among the existing children is needed. The
item to be moved is referred to by its ID. The
\subcommand{delete}{ttktreeview} and \subcommand{detach}{ttktreeview}
can be used to remove an item from the display, as specified by its
ID. The latter command allows for the item to be reinserted at a later
time.


\paragraph{Selection}
The user may select one or more rows with the mouse, as controlled by
the option \argument{selectmode}{ttktreeview}. Multiple rows may be
selected with the default value of \qcode{extended}, a restriction to
a single row is specified with \qcode{browse}, and no selection is
possible if this is given as \code{none}.

%% getting the selection
The \subcommand{select}{ttktreeview} command will return the current
selection. The current selection marks 0, 1 or more than 1 items if
\qcode{extended} is given for the \code{selectmode} argument.  If
converted to a string using \code{as.character} this will be a
character vector of the selected item IDs. Further subcommands
\code{set}, \code{add}, \code{remove}, and \code{toggle} can be used
to adjust the selection programatically.

For example, to select the first 6 children, we have:
\begin{Schunk}
\begin{Sinput}
 tkselect(tr, "set", children)
\end{Sinput}
\end{Schunk}
%
To toggle the selection, we have:
\begin{Schunk}
\begin{Sinput}
 tkselect(tr, "toggle", tcl(tr, "children", ""))
\end{Sinput}
\end{Schunk}
%
Finally, the selected IDs are returned with:
\begin{Schunk}
\begin{Sinput}
 IDs <- as.character(tkselect(tr))
\end{Sinput}
\end{Schunk}

%% Events; handlers.
\paragraph{Events and callbacks}
In addition to the keyboard events \code{\Event{KeyPress}} and
\code{\Event{KeyRelease}} and the mouse events \code{\Event{ButtonPress}},
\code{\Event{ButtonRelease}} and \code{\Event{Motion}}, the virtual event
\code{\VirtualEvent{TreeviewSelect}} is generated when the selection changes.

Within a key or mouse event callback, the clicked on column and row can
be identified by position, as illustrated in this example callback.
\begin{Schunk}
\begin{Sinput}
 callbackExample <- function(W, x, y) {
   col <- as.character(tkidentify(W, "column", x, y))
   row <- as.character(tkidentify(W, "row", x, y))
   ## now do something ...
 }
\end{Sinput}
\end{Schunk}


%% example: filter through data -- table
\begin{example}{Filtering a table}{ex-tcltk-table}

We illustrate the above with a slightly improved GUI for selecting a CRAN mirror. This adds in a text box to filter the possibly large display of items to avoid scrolling through a long list. 
\begin{Schunk}
\begin{Sinput}
 df <- getCRANmirrors()[, c(1,2,5,4)]
\end{Sinput}
\end{Schunk}


We use a text entry widget to allow the user to filter the values in the display as the user types.
\begin{Schunk}
\begin{Sinput}
 f0 <- ttkframe(f); tkpack(f0, fill="x")
 l <- ttklabel(f0, text="filter:"); tkpack(l, side="left")
 filterVar <- tclVar("")
 filterEntry <- ttkentry(f0, textvariable=filterVar)
 tkpack(filterEntry, side="left")
\end{Sinput}
\end{Schunk}

\begin{figure}
  \centering
  \includegraphics[width=.8\textwidth]{fig-tcltk-filter-table.png}
  \caption{Using \code{ttktreeview} to show various CRAN sites. This
    illustration adds a search-like box to filter what repositories
    are displayed for selection.}
  \label{fig:fig-tcltk-filter-table}
\end{figure}


The treeview  will only show the first three columns of the data frame, although we store the fourth which contains the URL.
\begin{Schunk}
\begin{Sinput}
 f1 <- ttkframe(f); tkpack(f1, expand=TRUE, fill="both")
 tr <- ttktreeview(f1, columns=1:ncol(df), 
                   displaycolumns = 1:(ncol(df) - 1), 
                   show = "headings",     # not "tree" 
                   selectmode = "browse") # single selection
 addScrollbars(f1, tr)
\end{Sinput}
\end{Schunk}

We configure the column widths and titles as follows:
\begin{Schunk}
\begin{Sinput}
 widths <- c(100, 75, 400)            # hard coded
 nms <- names(df)
 for(i in 1:3) {
   tcl(tr, "heading", i, text=nms[i])
   tcl(tr, "column", i, width=widths[i], 
       stretch=TRUE, anchor="w")
 }
\end{Sinput}
\end{Schunk}
%
The treeview widget does not do filtering internally.~\footnote{The
  model-view-controller architecture of \GTK, say, makes this task
  much easier, as it allows for an intermediate proxy model.} As such
we will replace all the values when filtering.  This following helper
function is used to fill in the widget with values from a data frame.
\begin{Schunk}
\begin{Sinput}
 fillTable <- function(tr, df) {
   children <- as.character(tcl(tr, "children", ""))
   for(i in children) tcl(tr, "delete", i)    # out with old
   shade <- c("none", "gray")
   for(i in seq_len(nrow(df))) 
     tcl(tr, "insert", "", "end", tag=shade[i %% 2], 
         text="",  
         values=unlist(df[i,]))               # in with new
   tktag.configure(tr, "gray", background="gray95")
 }
\end{Sinput}
\end{Schunk}
The initial call populates the table from the entire data frame.
\begin{Schunk}
\begin{Sinput}
 fillTable(tr, df)
\end{Sinput}
\end{Schunk}

The filter works by grepping the user input against the host value. We
bind to \Event{KeyRelease} (and not \Event{KeyPress}) so we capture the last keystroke.
\begin{Schunk}
\begin{Sinput}
 curInd <- 1:nrow(df)
 tkbind(filterEntry, "<KeyRelease>", function(W, K) {
   val <- tclvalue(tkget(W))
   possVals <- apply(df, 1, function(...) 
                     paste(..., collapse=" "))
   ind<- grep(val, possVals)
   if(length(ind) == 0) ind <- 1:nrow(df)
   fillTable(tr, df[ind,])
 })
\end{Sinput}
\end{Schunk}
%
This binding is for capturing a users selection through a double-click
event. In the callback, we set the CRAN option then withdraw the window.
\begin{Schunk}
\begin{Sinput}
 tkbind(tr, "<Double-Button-1>", function(W, x, y) {
   sel <- as.character(tcl(W, "identify", "row", x, y))
   vals <- tcl(W, "item", sel, "-values")
   URL <- as.character(vals)[4]          # not tclvalue
   repos <- getOption("repos")
   repos["CRAN"] <- gsub("/$", "", URL[1L])
   options(repos = repos)
   tkwm.withdraw(tkwinfo("toplevel", W))
 })
\end{Sinput}
\end{Schunk}
\end{example}


%% Comment on tktable XXX Do I want more XXX
\paragraph{Editing cells of a table}
There is no native widget for editing the cells of tabular data, as is
provided by the \function{edit} method for data frames. The
\code{tktable} widget (\url{http://tktable.sourceforge.net/}) provides
such an add-on to the base \TK. We don't illustrate its usage here, as
we keep to the core set of functions provided by \TK.  An interface
for this \TCL\/ package is provided in the \pkg{tcltk2} package
(\function{tk2edit}).  The \code{gdf} function of \pkg{gWidgetstcltk}
is based on this.



\subsection{Hierarchical data}

Specifying tree-like or hierarchical data is nearly identical to
specifying rectangular data for the \code{ttktreeview} widget.  The
widget provides column \code{\#0} to display this extra structure. If
an item, except the root, has children, a trigger icon to expand the
tree is shown. This is in addition to any text and/or an icon that is
specified. Children are displayed in an indented manner to indicate
the level of ancestry they have relative to the root.  To insert
hierarchical data into the widget the same
\subcommand{insert}{ttktreeview} subcommand is used, only instead of
using the root item, \qcode{}, as the parent item, one uses the item
ID corresponding to the desired parent. If the option \code{open=TRUE}
is specified to the \code{insert} subcommand, the children of the item
will appear, if \code{FALSE}, the user can click the trigger icon to
see the children. The programmer can use the
\subcommand{item}{ttktreeview} to configure this state. When the
parent item is opened or closed, the virtual events
\VirtualEvent{TreeviewOpen} and \VirtualEvent{TreeviewClose} will be
signaled.

%% example?
%% tcl(tr, "insert","I001","end", text="child", open=FALSE)
%% tcl(tr, item, "I001", open=TRUE)

%% traversal 

\paragraph{Traversal}
Once a tree is constructed, the programmer can traverse
through the items using the subcommands
\subcommanda{parent}{ttktreeview}{item} to get the ID for the parent of the
item; \subcommanda{prev}{ttktreeview}{item} and
\subcommanda{next}{ttktreeview}{item} to get the immediate siblings of the
item; and \subcommanda{children}{ttktreeview}{item} to return the children of
the item. Again, the latter one will produce a character vector of  IDs for the
children when coerced to character with \code{as.character}.



%% tree example using XML
\begin{example}{Using the treeview widget to show an XML file}{ex-tcltk-tree}
This example shows how to display the hierarchical structure of an XML
document using the tree widget.

We use the \pkg{XML} library to parse a document from the
internet. This example uses just a few functions from this library:
The \function(htmlTreeParse) (similar to \function{xmlInternalTreeParse}) to parse the file, 
\function{xmlRoot} to find the base node,
\function{xmlName} to get the name of a node, 
\function{xmlValue} to get an associated value, and
\function{xmlChildren} to return any child nodes of a node.



\begin{figure}
  \centering
  \includegraphics[width=.7\textwidth]{fig-tcltk-xml-viewer.png}
  \caption{Illustration of using \code{ttktreeview} widget to show
    hierarchical data returned from parsing an HTML document with the
    \pkg{XML} package.}
  \label{fig:fig-tcltk-xml-viewer}
\end{figure}
\begin{Schunk}
\begin{Sinput}
 library(XML)
 fileName <- "http://www.omegahat.org/RSXML/shortIntro.html"
 QT <- function(...) {}  # quiet next call
 doc <- htmlTreeParse(fileName, useInternalNodes=TRUE, error=QT)
 root <- xmlRoot(doc)
\end{Sinput}
\end{Schunk}
Our GUI is primitive, with just a treeview instance added.
\begin{Schunk}
\begin{Sinput}
 tr <- ttktreeview(f, displaycolumns="#all", columns=1)
 addScrollbars(f, tr)                    
\end{Sinput}
\end{Schunk}

We configure our column headers and set a minimum
width below. Recall, the tree column is designated \qcode{\#0}.
\begin{Schunk}
\begin{Sinput}
 tcl(tr, "heading", "#0", text="Name")
 tcl(tr, "column", "#0", minwidth=20)
 tcl(tr, "heading", 1, text="value")
 tcl(tr, "column", 1, minwidth=20)
\end{Sinput}
\end{Schunk}

To map the tree-like structure of the XML document into the widget, we
define the following function to recursively add to the treeview
instance.  We only add to the \code{value} column (through the
\code{values} option) when the node does not have children. We use
\code{do.call}, as a convenience, to avoid constructing two different
calls to the \code{insert} subcommand. 
\begin{Schunk}
\begin{Sinput}
 insertChild <- function(tr, node, parent="") {
   l <- list(tr, "insert", parent, "end", text=xmlName(node))
   children <- xmlChildren(node)
   if(length(children) == 0) {           # add in values
     values <- paste(xmlValue(node), sep=" ", collapse=" ")
     l$values <- as.tclObj(values)        # avoid split on spaces
   }
   treePath <- do.call("tcl", l)
 
   if(length(children))                          # recurse
     for(i in children) insertChild(tr, i, treePath)
 }
 insertChild(tr, root)
\end{Sinput}
\end{Schunk}
%
At this point, the GUI will allow one to explore the markup structure of the
XML file. We continue this example to show two things of general
interest, but that are really artificial for this example.

%%\XXX{Use index parent to place at same level just below}

\paragraph{Drag and drop}
First, we show how one might introduce drag and drop to rearrange the
rows. We begin by defining two global variables that store the row
that is being dragged  and a flag to indicate if a drag event is ongoing.
\begin{Schunk}
\begin{Sinput}
 .selectedID <- ""                               # globals
 .dragging <- FALSE
\end{Sinput}
\end{Schunk}
We provide callbacks for three events: a mouse click, mouse motion and mouse release.
This first callback sets the selected row on a mouse click.
\begin{Schunk}
\begin{Sinput}
 tkbind(tr, "<Button-1>", function(W,x,y) {
   .selectedID <<- as.character(tcl(W, "identify","row", x, y))
 })  
\end{Sinput}
\end{Schunk}
The motion callback configures the cursor to indicate a drag event and sets
the dragging flag. One might also put in code to highlight
any drop areas.
\begin{Schunk}
\begin{Sinput}
 tkbind(tr, "<B1-Motion>", function(W, x, y, X, Y) {
   tkconfigure(W, cursor="diamond_cross")
   .dragging <<-TRUE
 })
\end{Sinput}
\end{Schunk}

When the mouse button is released we check that the widget we are over
is indeed the tree widget. If so, we then move the rows. One can't
move a parent to be a child of its own children, so we wrap the
\subcommand{move}{ttktreeview} sub command within \code{try}. The
\code{move} command places the new value as the first child of the
item it is being dropped on. If a different action is desired, the
\qcode{0} below would need to be modified.
\begin{Schunk}
\begin{Sinput}
 tkbind(tr, "<ButtonRelease-1>", function(W, x, y, X, Y) {
   if(.dragging && .selectedID != "") {
     w = tkwinfo("containing", X, Y)
     if(as.character(w) == as.character(W)) {
       dropID <- as.character(tcl(W, "identify","row", x, y))
       try(tkmove(W, .selectedID, dropID, "0"), silent=TRUE)
     }
   }
   .dragging <<- FALSE; .selectedID <<- "" # reset
 })
\end{Sinput}
\end{Schunk}

\paragraph{Walking the tree}
Our last item of general interest is a function that shows one way to
walk the structure of the treeview widget to generate a list
representing the structure of the data.  A potential use of this might
be to allow a user to rearrange an XML document through drag and drop.
The subcommand \subcommand{children}{ttktreeview} proves useful here,
as it is used to identify the hierarchical structure. When there are children a recursive call is made.



\begin{Schunk}
\begin{Sinput}
 treeToList <- function(tr) {
   l <- list()
   walkTree <- function(child, l) {
     l$name <- tclvalue(tcl(tr,"item", child, "-text"))
     l$value <- as.character(tcl(tr,"item", child, "-values"))
     children <- as.character(tcl(tr, "children", child)) 
     if(length(children)) {
       l$children <- list()
       for(i in children) 
         l$children[[i]] <- walkTree(i, list()) # recurse
     }
     return(l)
   }
   walkTree("", l)
 }
\end{Sinput}
\end{Schunk}
\end{example}



\section{Menus}
\label{sec:tcltk:menus}

Menu bars and popup menus in \Tk\/ are constructed with
\constructor{tkmenu}. The \code{parent} argument depends on what the menu is
to do. A toplevel menu bar, such as appears at the top of a window has
a toplevel window as its parent; a submenu of a menu bar uses the
parent menu; and a popup menu uses a widget.  The menu widget in \Tk\/
has an option to be ``torn off.'' This features was at one time common
in GUIs, but now is rarely seen so it is recommended that this option
be disabled. The \argument{tearoff}{tkmenu} option can be used at
construction time to override the default behavior. Otherwise, the
following command will do so globally:
\begin{Schunk}
\begin{Sinput}
 tcl("option","add","*tearOff", 0)    # disable tearoff menus
\end{Sinput}
\end{Schunk}

A toplevel menu bar is attached to a top-level window using \code{tkconfigure}
to set the \code{menu} option of the window. For the aqua \TK\/
libraries for Mac OS X, this menu will appear on the top menu bar when
the window has the focus. For other operating systems, it appears at
the top of the window. For Mac OS X, a default menu bar with no
relationship to your application will be shown if a menu is not
provided for a toplevel window. Testing for native Mac OS X may be done via
the following function:
\begin{Schunk}
\begin{Sinput}
 usingMac <- function()  
   as.character(tcl("tk", "windowingsystem")) == "aqua"
\end{Sinput}
\end{Schunk}

The \function{tkpopup} function facilitates the creation of a popup
menu.  This function has arguments for the menu bar, and the postion
where the menu should be popped up. For example, the following code
will bind a popup menu, \code{pmb} (yet to be defined), to the right click event for a
button \code{b}. As \OSX\/ may not have a third mouse button, and when
it does it refers to it differently, the callback is bound
conditionally to different events.


\begin{Schunk}
\begin{Sinput}
 doPopup <- function(X, Y) tkpopup(pmb, X, Y) # define callback
 if (usingMac()) {
   tkbind(b, "<Button-2>", doPopup)      # right click
   tkbind(b, "<Control-1>", doPopup)     # Control + click
 } else {
   tkbind(b, "<Button-3>", doPopup)
 }
\end{Sinput}
\end{Schunk}


\paragraph{Adding submenus and action items}
Menus show a hierarchical view of action items. Items are added to a
menu through the \subcommand{add}{tkmenu} subcommand.  The nested
structure of menus is achieved by specifying a \code{tkmenu} object as
an item, using the \subcommand{add cascade}{tkmenu} subcommand. The
option \code{label} is used to label the menu and the \code{menu}
option to specify the sub-menu.

Grouping of similar items can be done through nesting, or on occasion
through visual separation. The latter is implemented with the \subcommand{add
  separator}{tkmenu} subcommand.


There are a few different types of action items that can be added:
 
\begin{description}
\item[Commands] An action item is one associated with a command. The
  simplest proxy is a button in the menu that activates a command when
  selected with the mouse. The \subcommand{add command}{tkmenu} allows
  one to specify a \code{label}, a \code{command} and optionally an
  \code{image} with a value for \code{compound} to adjust its
  layout. (Images are not shown in Mac OS X.) Action commands may
  possibly be called for different widgets, so the use of percent
  substitution is difficult. One can also specify that a keyboard
  accelerator be displayed through the option \code{accelerator}, but
  a separate callback must listen for this combination.

\item[Check boxes] Action items may also be proxied by checkboxes. To
  create one, the subcommand \subcommand{add checkbutton}{tkmenu} is
  used. The available arguments include \code{label} to specify the
  text, \code{variable} to specify a tcl variable to store the state,
  \code{onvalue} and \code{offvalue} to specify the state to the tcl
  variable, and \code{command} to specify a call back when the checked
  state is toggled. The initial state is set by the value in the
  \TCL\/ variable.

\item[Radio buttons] Additionally, action items may be presented
  through radiobutton groups. These are specified with the subcommand
  \subcommand{add radiobutton}{tkmenu}. The \code{label} option is
  used to identify the entry, \code{variable} to set a text variable
  and to group the buttons that are added, and \code{command} to
  specify a command when that entry is selected.
\end{description}

Action items can also be placed after an item, rather than at the end
using the \subcommand{insert command index}{tkmenu} subcommand. The
index may be specified numerically with 0 being the first item for a
menu.  More conveniently the index can be determined by specifying a
pattern to match the menu's labels.


\paragraph{Set state}
The \code{state} option is used to retrieve and set the current state of the a menu item.
This value is typically \code{normal} or
\code{disabled}, the latter to indicate the item is not available. The
state can be set when the item is added or configured after that fact,
through the \subcommand{entryconfigure}{tkmenu} command. This function
needs the menu bar specified and the item specified as an index or
pattern to match the labels.

\begin{example}{Simple menu example}{ex-tcltk-menu}
This example shows how one might make a very simple code editor using a text-entry widget. We use the \pkg{svMisc} package, as it defines a few GUI helpers which we use.
\begin{Schunk}
\begin{Sinput}
 library(svMisc)                         # for some helpers
 showCmd <- function(cmd) writeLine(captureAll(Parse(cmd)))
\end{Sinput}
\end{Schunk}

We create a simple GUI with a top-level window containing the text entry widget.
\begin{Schunk}
\begin{Sinput}
 w <- tktoplevel()
 tkwm.title(w, "Simple code editor")
 f <- ttkframe(w, padding=c(3,3,3,12)) 
 tkpack(f, expand=TRUE, fill="both")
 tb <- tktext(f, undo=TRUE)
 addScrollbars(f, tb)
\end{Sinput}
\end{Schunk}

We create a toplevel menu bar, \code{mb}, and attach it to our
toplevel window, then a file and edit submenu:
\begin{Schunk}
\begin{Sinput}
 mb <- tkmenu(w); tkconfigure(w, menu=mb)
 fileMenu <- tkmenu(mb)
 tkadd(mb, "cascade", label="File", menu=fileMenu)
 #
 editMenu <- tkmenu(mb)
 tkadd(mb, "cascade", label="Edit", menu=editMenu)
\end{Sinput}
\end{Schunk}

To these sub menu bars, we add action items. First a command to evaluate the contents of the buffer.
\begin{Schunk}
\begin{Sinput}
 tkadd(fileMenu, "command", label="Evaluate buffer",
       command = function() {
         curVal <- tclvalue(tkget(tb, "1.0", "end"))
         showCmd(curVal)
       })
\end{Sinput}
\end{Schunk}

Then a command to evaluate just the current selection
\begin{Schunk}
\begin{Sinput}
 tkadd(fileMenu, "command", label="Evaluate selection",
       state="disabled",
       command = function() {
         curSel <- tclvalue(tkget(tb, "sel.first", "sel.last"))
         showCmd(curSel)
       })
\end{Sinput}
\end{Schunk}

Finally, we end the file menu with a quit action. 
\begin{Schunk}
\begin{Sinput}
 tkadd(fileMenu, "separator")
 tkadd(fileMenu, "command", label="Quit", 
       command=function() tkdestroy(w))
\end{Sinput}
\end{Schunk}

The edit menu has an undo and redo item. For illustration purposes we add an icon to the undo item.
\begin{Schunk}
\begin{Sinput}
 img <- system.file("images/up.gif", package="gWidgets")
 tkimage.create("photo", "::img::undo", 
                      file=img)
 tkadd(editMenu, "command", label="Undo",
       image="::img::undo", compound="left",
       command = function() tcl(tb, "edit", "undo"))
 tkadd(editMenu, "command", label="Redo",
       command = function() tcl(tb, "edit", "redo"))
\end{Sinput}
\end{Schunk}

We now define a function to update the user interface to reflect any changes.
\begin{Schunk}
\begin{Sinput}
 updateUI <- function() {
   states <- c("disabled","normal")
   ## selection
   hasSelection <- function(W) {
     ranges <- tclvalue(tcl(W, "tag", "ranges", "sel"))
     length(ranges) > 1 || ranges != ""
   }
   ## by index  
   tkentryconfigure(fileMenu,1,  
                    state=states[hasSelection(tb) + 1]) 
   ## undo -- if buffer modified, assume undo stack possible
   ## redo -- no good check for redo
   canUndo <- function(W) as.logical(tcl(W,"edit", "modified"))
   tkentryconfigure(editMenu,"Undo",     # by pattern
                    state=states[canUndo(tb) + 1])
   tkentryconfigure(editMenu,"Redo", 
                    state=states[canUndo(tb) + 1])
 }
\end{Sinput}
\end{Schunk}

We now add an accelerator entry to the menubar and a binding to the top-level window for the keyboard shortcut.
\begin{Schunk}
\begin{Sinput}
 if(usingMac()) {
   tkentryconfigure(editMenu, "Undo", accelerator="Cmd-z")
   tkbind(w,"<Option-z>", function() tcl(tb,"edit","undo"))
 } else {
   tkentryconfigure(editMenu, "Undo", accelerator="Control-u")
   tkbind(w,"<Control-u>", function() tcl(tb,"edit","undo"))
 }
\end{Sinput}
\end{Schunk}

To illustrate popup menus, we define one within our text widget that will grab all
functions that complete the current word, using the
\function{CompletePlus} function from the \pkg{svMisc} package to find
the completions.  The use of \code{current wordstart} and
\code{current wordend} to find the word at the insertion point isn't quite
right for \R, as it stops at periods.
\begin{Schunk}
\begin{Sinput}
 doPopup <- function(W, X, Y) {
   cur <- tclvalue(tkget(W, "current  wordstart", 
                            "current wordend"))
   tcl(W, "tag", "add", "popup", "current  wordstart", 
                                 "current wordend")
   posVals <- head(CompletePlus(cur)[,1, drop=TRUE], n=20)
   if(length(posVals) > 1) {
     popup <- tkmenu(tb)                # create menu for popup
     sapply(posVals, function(i) {         
       tkadd(popup, "command", label=i, command = function() {
         tcl(W,"replace", "popup.first", "popup.last", i)
       })
     })
     tkpopup(popup, X, Y)
  }}
\end{Sinput}
\end{Schunk}

For a popup, we set the appropriate binding for the underlying
windowing system. For the second mouse button binding in OS X, we
clear the clipboard. Otherwise the text  will be pasted in, as this mouse
action already has a default binding for the text widget.

\begin{Schunk}
\begin{Sinput}
 if (!usingMac()) {
   tkbind(tb, "<Button-3>", doPopup)
 } else {
   tkbind(tb, "<Button-2>", function(W,X,Y) {
     ## UNIX legacy re mouse-2 click for selection copy
     tcl("clipboard","clear",displayof=W) 
     doPopup(W,X,Y)
     })      # right click
   tkbind(tb, "<Control-1>", doPopup)     # Control + click
 }
\end{Sinput}
\end{Schunk}
\end{example}

\section{Canvas Widget}
\label{sec:tcltk:canvas-widget}

 
The canvas widget provides an area to display lines, shapes, images
and widgets. Methods exist to create, move and delete these objects,
allowing the canvas widget to be the basis for creating interactive
GUIs. The constructor \constructor{tkcanvas} for the widget, being a
non-themeable widget, has many arguments including these standard ones:
\argument{width}{tkcanvas}, \argument{height}{tkcanvas}, and
\argument{background}{tkcanvas}, \argument{xscrollcommand}{tkcanvas}
and \argument{yscrollcommand}{tkcanvas}.


\paragraph{The create command}
The subcommand \subcommanda{create}{tkcanvas}{type [options]} is used
to add new items to the canvas. The options vary with the type of the
item. The basic shape types that one can add are \qcode{line},
\qcode{arc}, \qcode{polygon}, \qcode{rectangle}, and
\qcode{oval}. Their options specify the size using $x$ and $y$
coordinates. Other options allow one to specify colors, etc. The
complete list is covered in the \code{canvas} manual page, which we
refer the reader to, as the description is lengthy.  In the examples,
we show how to use the \qcode{line} type to display a graph and how to
use the \qcode{oval} type to add a point to a canvas. Additionally,
one can add text items through the \qcode{text} type. The first
options are the $x$ and $y$ coordinates and the \code{text} option
specifies the text.  Other standard text options are possible (e.g.,
\code{font}, \code{justify}, \code{anchor}).

The type can also be an image object or a widget (a window object). Images are added by specifying an $x$ and $y$ position, possibly an anchor position, and a value for the \qcode{image} option and optionally, for state dependent display, specifying \qcode{activeimage} and \qcode{disabledimage} values. The \qcode{state} option is used to specify the current state. Window objects are added similarly in terms of their positioning, along with options for \qcode{width} and \qcode{height}. The window itself is added through the \qcode{window} option. An example shows how to add a frame widget.

% Once created, a screenshot of the canvas can be created through the \subcommand{postscript}{tkcanvas} subcommand, as in \code{tcl(canvas, "postscript", file="filename")}. To store the widget so that it can be recreated is not supported directly. \TCL\/ code to do so can be found at \url{http://wiki.tcl.tk/9168}.


\paragraph{Items and tags}
The \code{tkcanvas.create} function returns an item ID. This can be
used to refer to the item at a later stage. Optionally, tags can be
used to group items into common groups. The \qcode{tags} option can be
used with \code{tkcreate} when the item is created, or the
\subcommand{addtag}{tkcanvas} subcommand can be used. The call
\code{tkaddtag(canvas, tagName, "withtag", item)} would add the tag ``\code{tagName}''to
the \code{item} returned by \code{tkcreate}. (The \qcode{withtag} is
one of several search specifications.) As well, if one is
adding a tag through a mouse click, the call \code{tkaddtag(W,
  "tagName", "closest", x, y)} could be used with \code{W}, \code{x}
and \code{y} coming from percent substitutions. Tags can be deleted
through the \subcommanda{dtag}{tkcanvas}{tag} subcommand.

%% interaction with items
There are several subcommands that can be called on items as specified
by a tag or item ID. For example, the \subcommand{itemcget}{tkcanvas}
and \subcommand{itemconfigure}{tkcanvas} subcommands allow one to get
and set options for a given item. The
\subcommanda{delete}{tkcanvas}{tag\_or\_ID} subcommand can be used to
delete an item. Items can be moved and scaled but not rotated. The
\subcommanda{move}{tkcanvas}{tag\_or\_ID x y} subcommand implements
incremental moves (where $x$ and $y$ specify the horizontal and
vertical shift in pixels). The subcommand
\subcommanda{coords}{tkcanvas}{ tag\_or\_ID [coordinates]} allows one
to respecify the coordinates for the item. The
\subcommand{scale}{tkcanvas} command is used to rescale items. Except for
window objects, an item can be raised to be on top of the others
through the \subcommanda{raise}{tkcanvas}{item\_or\_ID} subcommand.



\paragraph{Bindings}
As usual, bindings can be specified overall for the canvas, through
\code{tkbind}. However, bindings can also be set on specific items
through the subcommand \subcommanda{bind}{tkcanvas}{tag\_or\_ID event
  function} (or with \code{tkitembind}). This allows
bindings to be placed on items sharing a tag name, without having the
binding on all items. Only mouse, keyboard or virtual events can have
such bindings.

%% do with fonts now
\begin{example}{Using a canvas to make a scrollable frame}{ex:tcltk-scrollable-frame}
%%
This example\footnote{This example is modified from an example found
  at \url{
    http://mail.python.org/pipermail/python-list/1999-June/005180.html}}
shows how to use a canvas widget to create a box container that
scrolls when more items are added than will fit in the display
area. The basic idea is that a frame is added to the canvas equipped
with scrollbars using the \subcommand{create window}{tkcanvas}
subcommand. 

There are two bindings to the \Event{Configure} event. The first
updates the scroll region of the canvas widget to include the entire
canvas, which grows as items are added to the frame. The second
binding ensures the child window is the appropriate width when the
canvas widget resizes. The height is not adjusted, as this is
controlled by the scrolling.

\begin{Schunk}
\begin{Sinput}
 scrollableFrame <- function(parent, width= 300, height=300) {
   canvasWidget <- 
     tkcanvas(parent,
              borderwidth=0, highlightthickness=0,
              width=width, height=height)
   addScrollbars(parent, canvasWidget)
   #
   gp <- ttkframe(canvasWidget, padding=c(0,0,0,0))
   gpID <- tkcreate(canvasWidget, "window", 0, 0, anchor="nw", 
                    window=gp)
   tkitemconfigure(canvasWidget, gpID, width=width)
   ## update scroll region
   tkbind(gp,"<Configure>",function() {  
     bbox <- tcl(canvasWidget, "bbox", "all")
     tcl(canvasWidget,"config", scrollregion=bbox)
   })
   ## adjust "window" width when canvas is resized.
   tkbind(canvasWidget, "<Configure>", function(W) {
     width <- as.numeric(tkwinfo("width", W))
     gpwidth <- as.numeric(tkwinfo("width", gp))
     if(gpwidth < width)
       tkitemconfigure(canvasWidget, gpID, width=width)
   })
   return(gp)
 }
\end{Sinput}
\end{Schunk}

To use this, we create a simple GUI as follows:
\begin{Schunk}
\begin{Sinput}
 w <- tktoplevel()
 tkwm.title(w,"Scrollable frame example")
 g <- ttkframe(w); tkpack(g, expand=TRUE, fill="both")
 gp <- scrollableFrame(g, 300, 300)
\end{Sinput}
\end{Schunk}

To display a collection of available fonts requires a widget or
container that could possibly show hundreds of similar values. The
scrollable frame serves this purpose well
(cf. Figure~\ref{fig:fig-tcltk-all-fonts}).  The following shows how a
label can be added to the frame whose font is the same as the label
text. The available fonts are found from \function{tkfont.families}
and the useful coercion to character by \function{as.character}.
\begin{Schunk}
\begin{Sinput}
 fFamilies <- as.character(tkfont.families())
 ## skip odd named ones
 fFamilies <- fFamilies[grepl("^[[:alpha:]]", fFamilies)] 
 for(i in seq_along(fFamilies)) {
   fName <- sprintf("::font::-%s", i)
   try(tkfont.create(fName, family=fFamilies[i], size=14), 
       silent=TRUE)
   l <- ttklabel(gp, text=fFamilies[i], font=fName)
   tkpack(l, side="top", anchor="w")
 }
\end{Sinput}
\end{Schunk}

\end{example}

%% example with lines objects
\begin{example}{Using canvas objects to show sparklines}{ex-tcltk-sparklines}
Edward Tufte, in his book \textit{Beautiful
  Evidence}~\cite{Tufte:Beautiful-Evidence}, advocates for the use of
\textit{sparklines} -- small, intense, simple datawords -- to show substantial
amounts of data in a small visual space. This example shows how to use
a \code{tkcanvas} object to display a sparkline graph using a \texttt{line} object. The example also uses \texttt{tkgrid}
to layout the information in a  table. We could have spent more time on the
formatting of the numeric values and factoring out the data download, but leave improvements as an exercise.


\begin{figure}
  \centering
  \includegraphics[width=0.66\textwidth]{fig-tcltk-sparklines.png}
  \caption{Example of embedding sparklines in a display organized
    using \code{tkgrid}. A \code{tkcanvas} widget is used to display
    the graph.}
  \label{fig:fig-tcltk-sparklines}
\end{figure}


This function simply shortens our call to \texttt{ttklabel}. We use
the global \texttt{f} (a \code{ttkframe}) as the parent.
\begin{Schunk}
\begin{Sinput}
 mL <- function(label) { # save some typing
   if(is.numeric(label))
     label <- sprintf("%.2f", label)
   ttklabel(f, text=label, justify="right") 
 }
\end{Sinput}
\end{Schunk}
%
We begin by making the table header along with a toprule.
\begin{Schunk}
\begin{Sinput}
 tkgrid(mL(""), mL("2000-01-01"), mL("-- until --"), 
        mL("today"), mL("low"), mL("high"))
 tkgrid(ttkseparator(f), row=1, column=1, columnspan=5, 
        sticky="we")
\end{Sinput}
\end{Schunk}
%
This function adds a sparkline to the table. A sparkline here is just
a line item, but there is some work to do, in order to scale the
values to fit the allocated space. This example uses stock values, as
we can conveniently employ the \function{get.hist.quote} function from
the \pkg{tseries} package to get interesting data.
\begin{Schunk}
\begin{Sinput}
 addSparkLine <- function(label, symbol="MSFT") {
   width <- 100; height=15               # fix width, height
   y <- get.hist.quote(instrument=symbol, start="2000-01-01",
                       quote="C", provider="yahoo", 
                       retclass="zoo")$Close
   min <- min(y); max <- max(y)
   ##
   start <- y[1]; end <- tail(y,n=1)
   rng <- range(y)
   ##
   sparkLineCanvas <- tkcanvas(f, width=width, height=height)
   x <- 0:(length(y)-1) * width/length(y)
   if(diff(rng) != 0) {
     y1 <- (y - rng[1])/diff(rng) * height
     y1 <- height - y1   # adjust to canvas coordinates
   } else {
     y1 <- height/2 + 0 * y
   }
   ## make line with: pathName create line x1 y1... xn yn 
   l <- list(sparkLineCanvas,"create","line")
   sapply(1:length(x), function(i) {
     l[[2*i + 2]] <<- x[i]
     l[[2*i + 3]] <<- y1[i]
   })
   do.call("tcl",l)
 
   tkgrid(mL(label),mL(start), sparkLineCanvas, 
          mL(end), mL(min), mL(max), pady=2, sticky="e")
 }
\end{Sinput}
\end{Schunk}

We can then add some rows to the table as follows:
\begin{Schunk}
\begin{Sinput}
 addSparkLine("Microsoft","MSFT")
 addSparkLine("General Electric", "GE")
 addSparkLine("Starbucks","SBUX")
\end{Sinput}
\end{Schunk}
\end{example}

%% Moving an object
\begin{example}{Capturing mouse movements}{ex-tcltk-canvas}
This example is a stripped-down version of the \code{tkcanvas.R} demo
that accompanies the \pkg{tcltk} package. That example shows a
scatterplot with regression line. The user can move the points around
and see the effect this has on the scatterplot. Here we focus on the
moving of an object on a canvas widget. We assume we have such a
widget in the variable \code{canvas}.


This following adds a single point to the canvas using an
\code{oval} object. We add the \qcode{point} tag to this item, for
later use. Clearly, this code could be modified to add more points.
\begin{Schunk}
\begin{Sinput}
 x <- 200; y <- 150; r <- 6
 item <- tkcreate(canvas, "oval", x - r, y - r, x + r, y + r,
                  width=1, outline="black",
                  fill="SkyBlue2")
 tkaddtag(canvas, "point", "withtag", item)
\end{Sinput}
\end{Schunk}

In order to indicate to the user that a point is active, in some
sense, the following changes the fill color of the point when the
mouse hovers over. We add this binding using \code{tkitembind}
so that is will apply to all point items and only the point items.
\begin{Schunk}
\begin{Sinput}
 tkitembind(canvas, "point", "<Any-Enter>", function()
            tkitemconfigure(canvas, "current", fill="red"))
 tkitembind(canvas, "point", "<Any-Leave>", function()
            tkitemconfigure(canvas, "current", fill="SkyBlue2"))
\end{Sinput}
\end{Schunk}

There are two key bindings needed for movement of an object. First, we
tag the point item that gets selected when a mouse clicks on a point
and update the last position of the currently selected point.
\begin{Schunk}
\begin{Sinput}
 lastPos <- numeric(2)            # global to track position
 tagSelected <- function(W, x, y) {
   tkaddtag(W,  "selected",  "withtag",  "current")
   tkitemraise(W, "current")
   lastPos <<- as.numeric(c(x, y))
 }
 tkitembind(canvas, "point", "<Button-1>",  tagSelected)
\end{Sinput}
\end{Schunk}

When the mouse moves, we use \code{tkmove} to have the currently
selected point move too. As \code{tkmove} is parameterized by
differences, we track the differences
between the last position recorded and the current position.
\begin{Schunk}
\begin{Sinput}
 moveSelected <- function(W, x, y) {
   pos <- as.numeric(c(x,y))
   tkmove(W, "selected", pos[1] - lastPos[1], 
                         pos[2] - lastPos[2])
   lastPos <<- pos
 }
 tkbind(canvas, "<B1-Motion>", moveSelected)
\end{Sinput}
\end{Schunk}
%
A further binding, for the \Event{ButtonRelease-1} event, would be
added to do something after the point is released. In the original
example, the old regression line is deleted, and a new one drawn. Here
we simply delete the \qcode{selected} tag.
\begin{Schunk}
\begin{Sinput}
 tkbind(canvas, "<ButtonRelease-1>", 
        function(W) tkdtag(W,"selected"))
\end{Sinput}
\end{Schunk}


\end{example}



% \section{Dialogs}
% \label{sec:tcltk:dialogs}
% \SweaveInput{Dialogs}





%% Web based programming
\chapter{Web-based GUIs}
\label{chap:WebGUIS}
%% Web based GUIs -- Rpad, RApache, Rwui, ..



The internet affords one the opportunity to distribute their work in a convenient,
standardized way that allows people from around the globe to
share. Indeed, the \R\/ project has benefited greatly from the  web technologies
that enable user participation from disparate points.

This chapter shows some of the means to produce interactive interfaces
between the user and R through web technologies, at the time of
writing. Web interfaces to expose some resource have many obvious advantages
over the desktop interfaces discussed in previous chapters: no
installation issues for \R\/ and the toolkit libraries, user
familiarity with the browser interface, operating system independence, etc. This
makes it much easier to share ones work, but also puts an added burden
on the GUI writer, who must have some familiarity with new
technologies and the security implications contained therein. 

The web programmer coming to \R\/ will find relatively simple tools as
compared say to some open-source tools available for the python
programmer (Django \url{djangoproject.com}, pyjamas \url{pyjs.org},
...) or the ruby programmer (Ruby on Rails \url{rubyonrails.org}) or
even the web programmer used to one of the many available frameworks
built on PHP (Drupal \url{drupal.org}, Joomla! \url{joomla.org},
...). However, we will see that there are useful tools for \R\/ that
make it possible to develop \R-driven websites. Of course, web
technologies are changing quite rapidly, and \R\/ package writers are
hard at work, so one should check to see if newer, more powerful
resources, have been added to the mix.

This chapter does not even pretend to be comprehensive. It covers an
enormous array of technologies. Rather, its focus is to show how \R\/
can be used with these technologies. The interested reader will likely
need to seek additional help before implementation.


\section{Authoring Web Pages}
\label{sec:authoring-web-pages}
%% The request process

% COMMENT oN XHTML

% Form w3.org
% Uniform Resource Identifier (URI)

% A Uniform Resource Identifier (URI) is a string of characters which identifies an Internet Resource.

% The most common URI is the Uniform Resource Locator (URL) which identifies an Internet domain address. Another, not so common type of URI is the Universal Resource Name (URN).




The simplest web page is a static page that is returned when a user
makes a request. The basic architecture involves a browser (or some
other client) requesting a document from a web server. The request
must encode what document is desired so the web server can find
it. The request is specified in terms of a \defn{URI}, or uniform
resource identifier (a URL is technically a type
of URI). The web server in turn maps the URI request to a file on the
file system which the web server returns to the browser.

\begin{figure}
  \centering
\begin{verbatim}
browser -> request -> server -> page lookup -> return page to browser -> display
XXX -- REPLACE ME -- XXX
\end{verbatim}
  \caption{Basic flow of a how a static HTML file is displayed on a browser.}
  \label{fig:static-html-file}
\end{figure}


The type of HTML file just described is known as a static file, in contrast to a
dynamically generated file, as its contents do not reflect any
possible extra information in the request. The authoring of static HTML files
may involve three different techonologies described next.


\subsection{Markup languages}
\label{sec:markup-languages}

Typically a static web page is marked up in HTML. This now familiar
markup language allows the page author to indicate structure in
various parts of the document. Typical structures are paragraphs,
headers, images, etc. Additionally, markup can denote presentation, such
as color, font etc.

HTML is centered around the concept of a \dfn{tag} which is used to wrap a
portion of the text of a file. A tag has a name or keyword, in lower
case, and is enclosed in angle brackets. If the tag encloses some
text, it has a start and end style. The start tag for a tag \code{x}
would be \verb+<x>+ where the end tag would be \texttt{</x>} (an extra
slash). All text between theses tags would carry this tag. Some tags,
such as the image tag \tagger{img}, are used to define their attributes
only (a url of the file in this case) so do not come in pairs, in this case it is
common practice to end the tag with \texttt{/>}.~\footnote{There are
  two common variants of HTML one coming from SGML, another, XHTML, deriving
  from XML. Both are similar, but xhtml is stricter with its use of
  tags. Some basic rules (as opposed to conventions) include all tags
  are either ended with a closing tag, or with \texttt{/>}; tags are
  lower case; attributes must be enclosed in quotes and specified; the
  root element is different from that given in the examples. The
  Web Hypertext Application Technology Working Group
  (\url{http://whatwg.org}) has proposed specifications for the two
  that seem likely to become the standard for HTML5.}

A few typical tags are specified in
Table~\ref{tab:HTML-tags}.~\footnote{The site
  \url{http://www.w3schools.com/} provides a comprehensive, yet
  accessible, listing/}
Tags may indicate how text is supposed to be formatted (e.g. \texttt{b}),
others indicate what type of text it is (e.g. \texttt{code}), others
the document structure (e.g., \texttt{h1}, \texttt{p}, etc.).

\begin{table}
\centering
\label{tab:HTML-tags}
\caption{Table of common tags in HTML.}
\begin{tabular}{@{}lp{0.7\textwidth}@{}}
\toprule

Tag&Description\\
\midrule
\code{html}&Denotes an HTML file\\\code{head}&Marks header of file\\\code{body}&Marks off main body of file\\\code{script}&Used to include other types of files\\\code{p}&A paragraph. Also, \code{br} for a line break\\\code{h1}&First level header. Also \code{h2},...,\code{h6}\\\code{ul}&Unordered list. Also \code{ol}\\\code{li}&Denotes a list item\\\code{a}&An anchor for a hyperreference\\\code{img}&Denotes an image\\\code{div}&A text division, indicates a line break\\\code{span}&A text division, no implied break\\\code{b}&Denotes text to be set in bold\\\code{code}&Denotes text that is code\\\code{em}&Denotes text to be emphasized\\\code{table}&Creates a table element
\\ \bottomrule
\end{tabular}
\end{table}
A tag may have one or more \defn{attributes} specified. For example,
the anchor tag, \texttt{a}, has an attribute \tagattr{href}{a} to
specify the link that will open withn the user clicks on the
anchor. This attribute is indicated by name with an equals
sign. Quotes are optional for HTML, but recommended in general. They
are mandatory if there is white space involved. An example might be
\verb+<a href="http://www.r-project.org" />+.

All tags may have an \texttt{id} attribute specified, which is used to
give a unique ID to the part enclosed by the tag. This is used to
identify the tag within the document object model (DOM) described in
brief later. All tags may also have a \texttt{class} attribute to
indicate if the tagged content should be treated as a member of a
class. This provides a means to classify and treat similar objects as
a group. Some tags also allow one to specify style attributes, but a
more modern approach is to use a stylesheet to specify those. The
\texttt{span} and \texttt{div} tags are primarily used to specify
attributes for the tagged text.
\\

Some characters, such as angle brackets, have a reserved meaning.  As
such, to use an angle bracket in an HTML document requires the use an
\dfn{HTML entity reference}.  There are many such entities -- they are
also used for cahracter encodeings. Entities are
denoted by a leading ampersand \texttt{\&} and trailing semicolon, as
with \code{\&lt;} pr \code{\&gt;}. 


\begin{example}{Simple HTML file}{eg:sample-html-xhtml-header}
  A basic HTML file would include a structure similar to the following
  which shows the head and body. Within the head, a title is set.
  \begin{HTMLinput}
<!DOCTYPE HTML PUBLIC "-//IETF//DTD HTML//EN">
<html>
  <head>
    <title>Hello World Page</title>
  </head>
  <body>
    Hello world
  </body>
</html>
\end{HTMLinput}

A basic \code{xhtml} file has a different header, but otherwise
appears similar. For example the
following which specifies a version for the XML and a default name
space through the \tagattr{xmlns}{html} attribute.
\begin{HTMLinput}
<?xml version="1.0" ?>
<html xmlns="http://www.w3.org/1999/xhtml" xml:lang="en" 
  lang="en">
<head>
<meta http-equiv="content-type" 
  content="text/html; charset=UTF-8"/>
<title>Page title</title>
</head>
\end{HTMLinput}

\end{example}

\begin{example}{A basic table}{}
  Displaying tables is a common task for web pages. The \tagger{table}
  tag encloses a table. New rows are enclosed in a \tagger{tr} tag,
  and each cell can be a header cell, \tagger{th}, or a data cell
  \tagger{td}. The following shows one way alternate rows can be
  striped by hard coding a background color attribute (\code{bgcolor})
  to the rows.
  \begin{HTMLinput}
<table border="0" cellpadding="0">
  <tr>
    <th>Header 1</th><th>Header 2</th>
  </tr>
  <tr>
    <th>1</th> <th>2</th>
  </tr>
  <tr bgcolor="goldenrod">
    <th>3</th> <th>4</th>
  </tr>
</table>
 \end{HTMLinput}
\end{example}

\begin{example}{\R\/ helpers}{}
  Writing a tag and specifying a table can be tedious. Some helper
  functions are useful. The \pkg{hwriter} package includes a few, for
  now we mention \function{hmakeTag} which will produce a tag around
  some specified content along with attributes, that can be passed in
  through \R's \code{name=value} syntax. The first argument specifies
  the tag, and the second the values to be wrapped within the tag.
\begin{Schunk}
\begin{Sinput}
 require(hwriter)
 out <- hmakeTag("td",1:2, bgcolor="red")
 cat(out, sep="\n")
\end{Sinput}
\begin{Soutput}
<td bgcolor="red">1</td>
<td bgcolor="red">2</td>
\end{Soutput}
\end{Schunk}
The function is vectorized, as can be gathered from the output.
\end{example}


\subsection{Style sheets}
\label{sec:style-sheets}

%% http://www.w3.org/Style/LieBos2e/history/ for history

Casscading Style Sheets (\defn{CSS}) may be used to specify the
presentation of the text on a page. Common practice is to use the
markup language to specify document structure and a separate style sheet
to specify the layout of the first document. The advantage is a clean
separation of tasks so that one can make changes to the layout, say,
without affecting the text (and vice versa). A typical usage is to be
able to provide different layouts depending on the type of device.

Without going into detail, the style sheet syntax provides a means to
specify what type of tagged content the style will apply to (the
selector) and a means to specify what styles of markup will be
applied. For example, the specification below has
\code{h1, h2, h3, h4, h5, h6} as a \dfn{selector} to indicate that it
applies to all header tags. In the \dfn{declaration block} are style
specifications for the color of the text and the font
weight. Additionally, specifications for margins and padding are
given, along with a border on the bottom around the element.
\begin{HTMLinput}
  h1, h2, h3, h4, h5, h6 {
    color: Black;
    font-weight: normal;
    margin: 0;
    padding-top: 0.5em;
    border-bottom: 1px solid #aaaaaa;
}
\end{HTMLinput}
The full specification allows for much more complicated selections, be
they based on id of the tag (indicated with a prefix \code{\#}), class
of the tag (indicatd with a prefix \code{.}), or relation of tag to an
enclosing tag (left to right).  Style sheets can also refer to
positioning of the object within the page. Most modern web pages use
style sheets for layout, rather than tables, as it allows for greater
accessibility and offers advantages with search engines.


\begin{example}{A striped table using style sheets}{}
  Using the \tagattr{bgcolor}{table}  attribute of a table is deprecated in favor of
  style sheets for good reason. Here we illustrate a style sheet
  approach to striping a table. The style sheet may be defined in the
  HTML file itself with the \tagger{style} tag that appears within the
  document's \tagger{head}. 
  
  \begin{HTMLinput}
<style type="text/css">
table { border: 1px solid #8897be; border-spacing: 0px}
tr.head {background-color:#ababab;}
tr.even { background-color: #eeeeee;}
tr.odd { background-color: #ffffff;}
</style>
 \end{HTMLinput}
 A more common alternative, is to use the \tagger{link} tag to include
 the stylesheet through a url. For example,
\begin{HTMLinput}
<link rel="stylesheet" href="the.url.of.the.sheet" />
\end{HTMLinput}

 For the table itself, we need only replace the specification of the attribute with a
 class specification.
 \begin{HTMLinput}
<table>
  <tr class="head">
    <th>Header 1</th><th>Header 2</th>
  </tr>
  <tr class="odd">
    <th>1</th> <th>2</th>
  </tr>
  <tr class="even">
    <th>3</th> <th>4</th>
  </tr>
</table>
\end{HTMLinput}
  
\end{example}


\subsection{JavaScript}
\label{sec:javascript}

The third primary component of most modern web pages is JavaScript. This is a scripting
language that runs within the browser that allows manipulation of the
document. The document object model (DOM) specifies the elements of
the text that may be referenced from within JavaScript. For example,
individual elements can be found by unique ID, or common elements by class, or
elements sharing a tag, say \tagger{p}. JavaScript provides methods for
manipulating these elements. The simplest uses might be to change the
text when the mouse hovers over an element.

JavaScript allows web pages to be dynamic interfaces. The language
allows for callbacks to be defined for certain events, as with the
other GUI toolkits we've discussed. We don't pursue this, but note
that the \pkg{gWidgetsWWW} package uses JavaScript to make dynamic web
pages (cf. Figure~\ref{fig:gWidgets-three-oses}).

\begin{example}{Simple use of JavaScript to make a button have an action}{eg:javascript-action-button}
  The \tagger{button} tag produces a visual button. This tag has several
  event attributes, including \tagattr{onmouseover}{button} and
  \tagattr{onclick}{button}. When these occur, the specified
  JavaScript code is called. Here we show how to change the documents background
  color on a mouseover, and how to display a message on a mouse click.
  \begin{HTMLinput}
<button
  onMouseOver="document.bgColor='red'; return true;"
  onMouseOut="document.bgColor="; return true;"
  onClick="alert('clicked button'); return true;" >
  Click me...
</button>
 \end{HTMLinput}
\end{example}

There are several open source JavaScript libraries available that
offer convenient interfaces to JavaScript and UI widgets. We
mention ExtJS (\url{www.extjs.com}), jQuery (\url{jQuery.com}), YUI
(\url{developer.yahoo.com/yoi}) and Dojo
(\url{www.dojotoolkit.org}). 


\subsection{\R\/ tools to assist with authoring web pages}
\label{sec:r-tools-authoring}

There are quite a few packages for \R\/ to faciliate the authoring web
of pages from within
\R. We mention a couple.

\subsubsection{The hwriter package}
\label{sec:hwriter-package}


The \pkg{hwriter} package simplifies the task of creating HTML tables
for \R\/ objects, such as a matrix or vector. The package has a
self-generated example page in HTML which is created by its
\function{showExample} function (Or \code{example(hwriter)}). The main
function \function{hwrite} maps \R\/ objects into table objects (by
default) and has many options to modify the attributes involved. By
default, it writes its output to a file. The helper function
\function{openPage} takes a file name and returns a text
connection. The \function{closePage} function will close it. In the
examples below, so as the output will print, we use the
\function{stdout} function instead for the connection.

The package's examples show many different usages, we illustrate a few below.


A hyperlink can be generated through the \argument{link}{hwriter} argument.
\begin{Schunk}
\begin{Sinput}
 hwrite("R project", link="http://www.r-project.org",
        page=stdout())
\end{Sinput}
\begin{Soutput}
<a href="http://www.r-project.org">R project</a>
\end{Soutput}
\end{Schunk}
Although this usage doesn't save typing, a vectorized call could
easily do so.


To create a simple table, we need only call the constructor on a matrix or
data.frame object:
\begin{Schunk}
\begin{Sinput}
 m <- matrix(1:4, ncol=2)
 hwrite(m, page=stdout())
\end{Sinput}
\begin{Soutput}
<table border="1">
<tr>
<td>1</td><td>3</td></tr>
<tr>
<td>2</td><td>4</td></tr>
</table>
\end{Soutput}
\end{Schunk}


To get alternate rows to be striped we could have the following:
\begin{Schunk}
\begin{Sinput}
 styles <- c("odd","even")
 hwrite(m, page=stdout(), row.class=rep(styles, length=nrow(m)))
\end{Sinput}
\begin{Soutput}
<table border="1">
<tr>
<td class="odd">1</td><td class="odd">3</td></tr>
<tr>
<td class="even">2</td><td class="even">4</td></tr>
</table>
\end{Soutput}
\end{Schunk}
The \code{row.class} value is recycled for each entry in the row.


\subsubsection{The \pkg{R2HTML} package}
\label{sec:pkgr2html-package}

The \pkg{R2HTML} provides the generic function \code{HTML} for
creating HTML output from \R\/ objects based on their class.  As with
\function{hwrite}, this function writes its output to a connection for ease of
generating a file.

As \function{HTML} is a generic function, its usage is straightforward. For a
numeric vector we have:
\begin{Schunk}
\begin{Sinput}
 library(R2HTML)
 HTML(1:4, file=stdout())
\end{Sinput}
\begin{Soutput}
<p class='integer'>1&nbsp; 2&nbsp; 3&nbsp; 4</p>
\end{Soutput}
\end{Schunk}
The class is written using the \tagattr{class}{} attribute, so a style
sheet can be used:
\begin{Schunk}
\begin{Sinput}
 HTML(c(TRUE, FALSE), file=stdout())
\end{Sinput}
\begin{Soutput}
<p class='logical'>TRUE&nbsp; FALSE</p>
\end{Soutput}
\end{Schunk}

Functions may be formatted:
\begin{Schunk}
\begin{Sinput}
 HTML(mean, file=stdout())
\end{Sinput}
\begin{Soutput}
<br><xmp class=function>function (x, ...) 
UseMethod("mean")
<environment: namespace:base></xmp><br>
\end{Soutput}
\end{Schunk}

For more complicated objects, such as matrices and data frames, the
\function{HTML} function has other arguments. For example, a border
and inner border can be set (we omit the output).
\begin{Schunk}
\begin{Sinput}
 HTML(iris[1:3,1:2], Border=10, innerBorder=5, file=stdout())
\end{Sinput}
\end{Schunk}

The package also includes a number of functions to facilitate the
drafting of HTML files within \R, including \function{HTMLInitFile},
\function{HTMLCSS}, \function{HTMLInsertGraph} and
\function{HTMLEndFile}.


\subsubsection{The \pkg{brew} package}
\label{sec:pkgbrew-package}

\R\/ has the wonderful facility \pkg{Sweave} that passes through a
\LaTeX\/ file and can replace \R\/ code with the code and output
generated by evaluating the code. The \pkg{R2HTML} provides a means to
do the same with HTML files. Whereas, the \pkg{ascii} package provides
a means to do so for several ascii-based syntaxes for markup, many of
which have tools to create HTML pages.

The \pkg{brew} package does something similar, yet different. It
allows one to place a template within an HTML file that \R\/ will
eventually populate when called accordingly. In the next section, we
illustrate how this can be used to produce dynamically generated web pages. For now, we
mention how to make a template and how to process it.


A template is a file with parts of it marked by delimiters
(cf. Table~\ref{tab:brew-delimiters}). All text not within delimiters
is processed as is. Whereas, text within delimiters may be evaluated
by \R, and if evaluated the contents may be inserted into the output
or simply used to adjust the evaluation environment. When processed
with brew, the result may be stored in a file, or sent to
\code{stdout}.


\begin{table}
  \centering
  \begin{tabular}{lr@{\quad}c@{\quad}c}
    \toprule
    &&\multicolumn{2}{c}{Evaluate}\\
    && Yes & No \\
    \cmidrule{3-4}
   Print & Yes & \verb+<%= %>+ & no delimiters\\
%   \addlinespace[.5pt]\\
          & No  & \verb+<%  %>+  & \verb+<%# %>+\\
\bottomrule
 \end{tabular}
  \caption{The \pkg{brew} delimiters and how they are processed.}
 \label{tab:brew-delimiters}
\end{table}


\begin{example}{Differences in \pkg{brew} delimiters}{eg:brew-delimiters}

To illustrate the differences in the \pkg{brew} delimiters, the left
side has \pkg{brew} commands and the right side is their output.

\begin{minipage}{0.45\linewidth}
  \HTMLinputlisting{brew-basic.brew}
\end{minipage}
%%
\quad\quad
\begin{minipage}{0.45\linewidth}
  \HTMLinputlisting{brew-basic.brew.out}
\end{minipage}
\end{example}

\begin{example}{Dynamically formatted text}{eg:brew-dynamic-text}
  This example shows how brew can be used to insert dynamic text.

This template

\HTMLinputlisting{brew-fortunes.brew}

produces

\HTMLinputlisting{brew-fortunes.brew.out}

\end{example}

\begin{example}{Recursively calling \function{brew}}{eg:recursive-brew}
  Typically there will be more than one page on a web site with each
  sharing common features: a banner, a footer, navigation links, a
  side bar, ... Using templates for these pieces and then including
  the template in a file is one way to centralize these common
  pieces. The \function{brew} function can easily be used to do this.
  
  For example, here we define a header and footer and then call them
  in from a page. Our header is basic template, but includes a
  variable \code{title} to be defined in the page.
  
\HTMLinputlisting{brew-header.brew}  

  Our basic footer is
  
\HTMLinputlisting{brew-footer.brew}    

  And a typical page has this structure. We set the variable
  \code{title} in the scope of this page, but it is seen within the
  scope of the call to process the header page.
  
\HTMLinputlisting{brew-page.brew}    
  
  
\end{example}


\begin{example}{Creating a template within a template}{ex:brew-template-withing}
  This example shows how one can define a template within a template,
  as an alternative to a separate file. The basic idea is to use
  \function{paste} to bypass the issue of being unable to nest
  \pkg{brew} delimiters. We evaluate the template within a context, so
  that each time we get the values from different rows.

This template
 \HTMLinputlisting{brew-list.brew}

 produces
 
 \HTMLinputlisting{brew-list.brew.out}
\end{example}

\subsection{Graphics in web pages}
\label{sec:graphics-web-pages}
Web pages may be plain text, but most contain images or graphics. The
\tagger{img} tag allows one to display a graphics file in an HTML
page by specifying its \tagattr{src}{img} attribute. This is an image file,
often in \code{png}, \code{gif} or \code{jpeg} format. In this
section, we describe how \R\/ can be used to generate images by using
different device drivers. To list all the possible stock devices, see the
help page for \code{Devices}. The function \function{capabilities} lists
which devices are available for a given \R\/ installation.

\subsubsection{png}
\label{sec:images}
Typically when a plot command is issued, an interactive plot device is
opened or reused, however, the user can specify a device to save the
output to a file for further use. For example, the \function{pdf} and
\function{postscript} functions will turn \R\/ commands into files for
inclusion in written documents. For web pages, the \function{png} and
\function{jpeg} device drivers are available for many systems. These
may be used to insert a graphic into a web page.

The basic usage is like that of the \function{pdf} driver illustrated
below -- open the device, issue graphics commands, close the device:
\begin{Schunk}
\begin{Sinput}
 pdf(file="test.pdf", width=6, height=6) # in inches
 hist(rnorm(100), main="Some graphic")
 invisible(dev.off())                    # close device
\end{Sinput}
\end{Schunk}

To use the \code{png} driver on a linux server, the option \code{type}
should be set to \code{cairo} either through the constructor, or by
setting the option \code{bitmapType}. 

The \pkg{Cairo} device driver is an alternative which can also output
in png format.


\subsubsection{SVG graphics}
\label{sec:svg-graphics}
The web has other means to display graphics than an inclusion of an
image file. For example, Flash is a very popular method.~\footnote{The
  \pkg{FlashMXML} from \url{omegahat.org} provides a means to
  genearate flash files from within \R.} SVG (Scalable vector graphics)~\footnote{\url{http://www.w3.org/Graphics/SVG/}} is another way to specify graphical objects using XML. Many modern web browsers have support for
the display of SVG graphics. To insert the file, we have the
\tagger{object} tag and its attributes \tagattr{data}{object} and
\tagattr{type}{object}, as in
\begin{HTMLinput}
<object data="image-svg.svg" type="image/svg+xml"></object> 
\end{HTMLinput}
Not all browsers support svg, so one might also have a fall back
image, as in:
\begin{HTMLinput}
<object data="image-svg.svg" type="image/svg+xml">
<img src="image-png.png" alt="alternative file" /> 
</object> 
\end{HTMLinput}

There are a few drivers to create SVG files in \R, for example In the base
\pkg{grGraphics} package, the driver \code{svg} is available.  This non-interactive
driver is used as the \code{png} one illustrated above.


The \pkg{RSVGTipsDevice} package provides an alternate driver,
\function{devSVGTips}.  The ``Tips'' part of the package, is provided
by the function \function{setSVGShapeToolTip}, which allows one to
specify a tooltip to popup when the mouse hovers over an element. The
tooltip specified is placed over the next shape drawn, such as a
point.

For example, here we add a tip and a URL to each point in a
scatterplot. We initially call \function{plot} without plot characters to
set up the axes, etc.
\begin{Schunk}
\begin{Sinput}
 require(RSVGTipsDevice)
 f <- "image-svg.svg"
 devSVGTips(f, toolTipMode=2, toolTipOpacity=.8) 
 plot(mpg ~ wt, mtcars, pch=NA) 
 nms <- rownames(mtcars) 
 gurl <- function(x)                     # search google
   sprintf("http://www.google.com/search?q=%s", x)
 for(i in 1:nrow(mtcars)) { 
   ## need to add tooltip shape by shape 
   setSVGShapeToolTip(title=nms[i])    # add tooltip 
   setSVGShapeURL(gurl(nms[i]), target="_blank")
   with(mtcars, points(wt[i], mpg[i], cex=2, pch=16)) # add
 } 
 invisible(dev.off())
\end{Sinput}
\end{Schunk}




\subsubsection{The canvas tag}
\label{sec:canvas-tag}
HTML5 is a major extension to HTML that is being implemented in most
browsers at the time of the writing of this book. One of the new
features of HTML5 is the \tagger{canvas} element, which allows
JavaScript code to manipulate objects, similar to the \function{tkcanvas}
widget of \pkg{tcltk}. 

\R\/ has the \code{canvas} device driver, that can be used to generate
JavaScript code to produce the graphic in a canvas element. The basic
usage involves creating the JavaScript:
\begin{Schunk}
\begin{Sinput}
 require(canvas)
 f <- "canvas-commands.js"
 canvas(width=480, height=480, file=f)
 hist(rnorm(100), main="Some graphic")
 invisible(dev.off())
\end{Sinput}
\end{Schunk}

Then, within the HTML file, code along the lines of the following is
needed. 
\begin{HTMLinput}
<canvas id="canvas_id" width=480 height=480></canvas>
<script type="text/javascript" language="javascript">
var ctx = document.getElementById("canvas_id").getContext("2d");
</script> 
<script type="text/javascript" src="canvas-commands.js"></script>
\end{HTMLinput}
The first \tagger{script} tag is used to define the variable
\code{ctx} to hold the canvas object, as this is assumed by the canvas
package.

The \pkg{RGraphicsDevice} device from \url{omegahat.org} provides a
possible alternative to the \pkg{canvas} package.


\section{The rapache package}
\label{sec:calling-an-r}
While websites can consist of just static files, many webpages viewed
are dynamically generated in response to user input. In order to
implement this, the process of returning a page for a user request is
more complicated. Rather than simply look up a file, the web server
may call an external program that prepares the text to return. This
text may be HTML for a web page, or in the case of web services, may
be XML or some other form of data markup. For \R\/ users, there have
been a few projects in the past that allow an \R\/ process to be used
to generate the response. At this point, the best one is the
\pkg{rapache} package. The package web page lists a few projects that
use this technology to create web pages, including some highly
interactive web pages by Jeroen Ooms. The \pkg{gWidgetsWWW} package ports the
\pkg{gWidgets} API to the web using \pkg{rapache}.


The \defn{Apache} web server is one of several open-source projects
supported by the apache Apache Software Foundation. It is extremely
successful -- its website (\url{http://www.apache.org}) boasts it has
been the most popular web server on the internet since 1996. Like \R,
Apache's open source nature allows developers to customize its
standard behaviours, in this case using modules. The \pkg{RApache}
package (\url{http://biostat.mc.vanderbilt.edu/rapache/}) provides
such a module that inserts \R\/ in the processing phase of a request
to the web server.

The \pkg{rapache} package works under linux but not directly under
windows. However, one can use a virtual machine to run a linux version
of Apache under windows or Mac OS X. A ``virtual machine'' containing
a pre-built linux system is available from the \pkg{rapache} website.

\subsection{Configuration}
\label{sec:configuration}

The \pkg{rapache} package requires the Apache web server to be
properly configured. There are a number of steps in the process.
The \pkg{rapache} homepage has detailed instructions, we mention just
the steps here.

First, a module for Apache must be created by running \pkg{rapache}'s
configure script. For Debian users, the package can be installed
through the usual mechanism.
Afterwards, Apache must be configured.

Next, the module must be loaded into Apache. This is done in the
standard way for Apache, through its \code{LoadModule} directive. This
is done before any other \R-centric directives are given in Apache's configuration.

Finally, Apache must be configured for use with \pkg{rapache}. 
The \code{REvalOnStartup} directive is used to specify any packages
that should be loaded whenever the web server starts. The web server
embeds a copy of \R\/ in itself and spawns copies of this as it spawn
copies of itself to handle requests. The startup can be slow, so this
offers a chance to pre-load common packages to speed things up at the
cost of a larger memory footprint. \code{RSourceOnStartup} is similar,
only it used to specify a file to be sourced on startup.

\paragraph{The Directory directive}
There are a few directives to configure \pkg{rapache} to process an
incoming request. A standard configuration for Apache, is to have the
URL specify a file on the file system after some mangling of the name,
exchanging the base part of the URL with a document root. One can have
\pkg{rapache} process the file prior to being returned by creating the
appropriate directive

\begin{figure}
  \centering
\begin{verbatim}
request url -> mangle file name -> lookup, return file

 to

request url -> mangle file name -> run function file through rapache handler
            -> return output
\end{verbatim}
  \caption{Inserting \pkg{rapache} in the request lookup}
  \label{sec:rapache-brew}
\end{figure}


The \pkg{rapache} manual demonstrates a typical usage calling
\function{brew} on a template to produce the HTML file.  That is, to
make a dynamic web page one only needs to write a brew template and
plac it into the appropriate directory.

To configure \pkg{rapache} for this, a directive along the lines of
the following may be added to Apache's configuration files.
\begin{HTMLinput}
<Directory /var/www/brew>
  SetHandler r-script
  RHandler brew::brew
</Directory>
\end{HTMLinput}

If the ``DocumentRoot'' of Apache is \code{/var/www}, then a request
such as \url{http://servername/brew/file.brew} will resolve first to
Apache finding \code{file.brew} in the \code{/var/www/brew/}
directory, and then that file will be processed by the \function{brew}
function in the \pkg{brew} package. The output will then be returned
to the client making the request. 

The \code{SetHandler} directive can be \code{r-script}, in which case
the function called has two arguments a file path and an
environment. The brew call uses these to find the template file, and
give a context for evaluation. Alternatively, this directive can be
\code{r-handler} in which case no arguments are passed to the call.

\paragraph{The Location Directive}


\begin{figure}
  \centering
\begin{verbatim}
request -> rapache calls function -> returns output to client
\end{verbatim}
  \caption{Creating a web page from a script and inputs}
  \label{fig:rapache-location-directive}
\end{figure}

Requests need not map to a file system, but can simply map to a
function call. For example, an application might be designed around
data stored in a data base and all pages are generated dynamically. To
have a URL call a script without reference to a file, the
\code{LOCATION} directive is used. For example,
\begin{HTMLinput}
<Location /myapp>
  SetHandler r-handler
  RFileHandler /path/to/R/scripts/myapp.R
</Location>
\end{HTMLinput}
A request to \url{http://servername/myapp/extra} will call the script
\code{myapp.R}. The \code{extra} part of the request can be found from
one of the \pkg{rapache} variables discussed in
Section~\ref{sec:pkgrapache-variables} and the script can adjust its
output based on this.


\subsection{Creating files}
\label{sec:creating-files}

The typical use of \pkg{rapache} is to return an HTML file, but it is
possible of much more. For example, the server may be asked to
dynamically generate a graphic, and the output would be an image
file. As well, web services are used to pass some resource, say some
data to a client requesting it. This data may be stored in XML format,
or JSON or YAML etc. As such, information about the file type is passed back to the
client along with the page.  

If the page is generated by a function call, as with the Location
directive example, \pkg{rapache} provides some convenience functions
for providing this information.  Response headers can be added
throught the \function{setHeader} function. The set of headers is long
and technical.~\footnote{The definitions can be found at
  \url{http://www.w3.org/Protocols/rfc2616/rfc2616-sec14.html}.}  The
\function{setContentType} function is used to set the MIME type of the
response. It must be called before any \function{print} or
\function{cat} statements in the file.  To send back binary data, the
function \function{sendBin} is available.


\paragraph{Return Codes}
The return value of the handler call indicates the failure or success
of the request.  The return value should be an integer, \pkg{rapache}
provides named variables instead. For success a return value of
\code{DONE} will indicate success, whereas a value such as
\code{HTTP\_BAD\_REQUEST} will signal an error.~\footnote{A list of
  the \pkg{rapache} variables appear in its manual. A list of status
  codes can be found at \url{http://www.w3.org/Protocols/rfc2616/rfc2616-sec10.html}}. 

The function \function{RApacheOutputErrors} can be used to direct what
happens to the error, in particular it can be used to have errors
print out to the browser rather than the log file. This is useful when
developing a program.

\subsection{\pkg{rapache} variables}
\label{sec:pkgrapache-variables}

When a script of function is being evaluated within \pkg{rapache}
certain variables holding information about the request and web server
are created. The variables are lists with named arguments, the names
matching \code{Apache} variables.

\paragraph{\code{SERVER}}

The \code{SERVER} variable holds a large amount of information on the
request. For example, the \code{status} componenent returns the status
code. Some of the most useful, decompose the URL requesting the page.



The response depends on the configuration. If the we use
\code{/var/www/brew} to process requests through \function{brew}, as
above, then a request like
\url{http://localhost/brew/test.brew?some=brew} results in values of

\begin{tabular}{r@{\quad}l}
\code{uri} &being \code{/brew/test.brew},\\
\code{filename} &being  \code{/var/www/brew/test.brew},\\
\code{path\_info} &being an empty string and \\
\code{args} &holding the string \code{some=brew}.
\end{tabular}


However, if we use the Location directive above, then the request
\url{http://localhost/myapp/detail?some=brew} has 

\begin{tabular}{l@{\quad}l}
\code{uri} &being \code{/myapp/detail},\\
\code{path\_info} &being \code{/detail} (the``virtual'' part of the request), and\\
\code{args} &again holding the string \code{some=brew}.
\end{tabular}



\paragraph{\code{GET}}
Both of the example urls above result in the variable \code{SERVER\$method} being
\code{GET}. HTTP has a few conventions that are not enforced, but
are associated with it providing RESTful web services. One being that
one uses a limited set of methods to interact with the service. A
\code{GET} request is meant to return data, a \code{POST} request is
meant to create new data, a \code{PUT} request is meant to update data
and a \code{DELETE} request to delete data.

The two example requests above, result in \code{GET} reqests and the
\code{GET} variable contains some useful information, namely the
arguments passed through the URL after the \code{?}. (URLs use a
\code{?} to pass arguments in the form \code{key=value} with multiple
arguments separated by an \code{\&}. So in the above, \code{GET} is a
list with component \code{some} whose value is \code{brew}.



\paragraph{\code{POST}}
A \code{POST} request usually comes from within a form. As with a
\code{GET} request, arguments can be passed in with the request,
although they do not appear in a URL. As with the
\code{GET} variable, the arguments appear as named components in the
\code{POST} variable. \code{POST} requests can contain more information
-- they are not limited in length the same way --
and must be used to upload files, say.



\paragraph{\code{COOKIES}}

By design HTTP is a stateless protocol. This means that between
requests the web server remembers nothing about the past
requests. For large web sites, this has an advantage when multiple
servers are used to process requests. However, it has disadvantages as
the request must relay the state of a web page. Several mechanisms have
been developed to deal with this issue. Sessions, where information is
kept server side and an ID kept with the client allow a state to be
maintained server side. 


Another solution is to store information on the client side. This is
implemented through \dfn{cookies}. Although cookies have privacy
issues, their use is widespread.  A basic cookie consists of a name
and a value (a character vector of length 1).  Cookies must satisfy
certain validity constraints which are specified through a time to
expire, a path to which the cookie pertains and a domain for which the
cookie is valid. The \pkg{rapache} function \function{setCookie} can
be used to set a cookie. The first argument is the cookie name, the
second the value, and others are available to set properties, such as
an expiry time. Cookies are placed in the outgoing header of a
document, so this call is done before the \code{head} tag.

When a page is loaded, the \code{COOKIES} variable contains cookie
information. Again, as a list. In this case, the names are the valid
cookie names and the component's value is the cookie.


\subsection{Forms}
\label{sec:forms}

User input can be passed to the server through the URL request or
through a form. Forms are specified with the \tagger{form} tag, which
has a few important attributes.~\footnote{See
  \url{http://www.w3.org/TR/html401/interact/forms.html} for a
  specification}. The \code{action} attribute specifies the URI that
will process the form information. In our example, this will match a
Location directive. The \tagattr{method}{form} attribute is used to specify a
\code{GET} request or a \code{POST} request. For a post request that
includes a file upload, the \tagattr{enctype}{form} attribute should contain
\code{"multipart/form-data"}. In addition to these, the
\tagattr{onsubmit}{form} attribute is often used to specify some JavaScript to
call as the form is submitted. For example, this may be used to specify code to
validate the form entries.

\paragraph{The input tag}
Within the \tagger{form} tags control elements may be placed. The
\tagger{input} tag is used to specify several types of controls, he
\tagattr{type}{input} attribute indicating which control. The
default is \code{text} for a single line text entry, but other values are
\code{password} for a password entry; \code{checkbox} and \code{radio}
for selection of items; \code{file} for a file upload control;
\code{image} for an image; \code{button} to make a button; and
\code{submit} for a submit button.

The usual attributes \code{class} and \code{id}
apply, as do many others that are type specific.
The \tagattr{name}{input} attribute specifes the name for the
element. This is processed as a key in the \code{POST} variable. The
\tagattr{value}{input} attribute is used to specify an initial value. For sizing, the
attributes \tagattr{size}{input} and \tagattr{maxlength}{input} are used to
specify the control size and length of text string. For images
\tagattr{src}{input} is used to specify the image source as a URL. For
the selection widgets, \tagattr{checked}{option} is used to specify if the
button is on. 

To illustrate, this HTML would produce a simple text entry area:
\begin{HTMLinput}
<input type="text" value="initial text" />
\end{HTMLinput}
This would be used to specify a submit button:
\begin{HTMLinput}
<input type="submit" value="submit" />
\end{HTMLinput}

A radio group is created by having multiple inputs sharing a common name
\begin{HTMLinput}
<input type="radio" name="key" value="TRUE" checked="TRUE">
<input type="radio" name="key" value="FALSE">
\end{HTMLinput}


\paragraph{The select tag}
The \tagger{select} tag is used to create a control to select one or more
values from a list of options. This control may be a combobox or a
table display. The attribute
\tagattr{multiple}{select} is used to specify if the user can select
one or more values. When specified, the \code{POST} or \code{GET} variables
have multiple components of the same name. The \tagattr{size}{select}
attribute specifes the number of entries to make initially visible. 

The possible values for selection are given within \code{option}
tags. The attribute \tagattr{selected}{option} is used to specify if
the value is initially selected. The attribute
\tagattr{value}{option} can be used to specify a different value than
that displayed.

For example, 
\begin{HTMLinput}
<select name="id">
  <option value="1">one</option>
  <option value="2" selected="true">two</option>
</select>
\end{HTMLinput}

\paragraph{A textarea tag}
Single line text entries are created by the \tagger{input} tag by
default, but multiple line entries are formed by the \tagger{textarea}
tag. The attributes \tagattr{cols}{textarea} and
\tagattr{rows}{textarea} specify the size.



\subsection{Security}

Forms allow users to specify values, which may then be processed
by the underlying \R\/ process within \pkg{rapache}. As such a
malicious user may try to have code run that could compromise the web
server. More benignly, the user may specify responses that include
malformed HTML. If these are simply printed back when the web page is
created, a rendering error may occur. Regardless of the user base for
a web application, one should assume that user input for web sites should never be trusted.

\paragraph{Unclosed or malicious tags}
To avoid malformed HTML one should encode any user input that is
echoed back to a web page. The following function will replace certain
characters with their HTML entity for safe inclusion within a page.

\begin{Schunk}
\begin{Sinput}
 HTMLencode <- function(str) {
   str <- as.character(str)
   vals <- list(c('&','&amp;'),
                c('"','&quot;'),
                c('<','&lt;'),
                c('>','&gt;')
                )
   for(i in vals)
     str <- gsub(i[1],i[2],str)
   str
 }
\end{Sinput}
\end{Schunk}

\paragraph{Whitelists, Blacklists}
Even in the event of a fixed list of values for a user to choose from,
user input should always be checked. It is very easy to fabricate a
request without going through the web form, for example the \R\/ package
\code{Rcurl} can do this.

When checking values, one can use a whitelist -- a list of acceptable
values, or a blacklist -- a list of unacceptable values. The use of a
whitelist is better if possible, as it is very easy to miss somthing in
a blacklist.

In either case, it is a good idea to never evaluate directly a users input.


\paragraph{SQL injection}
Many web sites are built around queries to a data base. Websites
powered by \pkg{rapache} can take this approach, as the \pkg{Rdbi}
package allows an interface within the \R\/ process between a data
base and \R. The basic use is to create a query within \R\/ and then
call one of \pkg{Rdbi}'s functions to get the results from the
query. The technique of SQL injection, takes advantage of carelessly
constructed SQL queries that are made by pasting together SQL commands
with user-given input. 


\begin{example}{Using \pkg{rapache} to explore a data store}{eg:rapache}

  
  
This example shows how one can use \pkg{rapache} to allow a user to
explore a data set. This basic application is simple, but the
structure of it is typical and very extendible. There are three pages
to display: a page to greet the user, a page to select one of many
items, and a page to display detail on an item.

We use a \code{Location} directive for this application which allows
us to specify which page to display using the \code{path\_info}
variable.
\begin{HTMLinput}
<Location /simpleapp>
   SetHandler r-handler
   RFileHandler /var/www/GUI/simpleapp/app.R
</Location>
 \end{HTMLinput}
  
 The script \code{app.R} is responsible for processing the request and
 dispatching to the appropriate page. Our script contains the
 following to load packages and set the current working directory to
 match that of the script. This is needed for our calls to
 \code{brew}.
\begin{Schunk}
\begin{Sinput}
 require(brew, quietly=TRUE)
 require(hwriter, quietly=TRUE)
 dir <- "/var/www/GUI/simpleapp"
 setwd(dir)
\end{Sinput}
\end{Schunk}

We have four main pages, one for any errors, and the three
mentioned. The dispatch to the page will call these functions which
are responsible for setting the context for the \pkg{brew}
templates. Each template has a \code{title} variable that we set
within the function. This then will be within the scope of the call to
\function{brew}. The variable \code{df} is assumed to contain a data frame
of interest. This could be retrieved by some call to a data base, for
example.

Our error page is called by
\begin{Schunk}
\begin{Sinput}
 processError <- function(e) {
   title <- "Error"
   with(e, brew("error.brew"))
 }
\end{Sinput}
\end{Schunk}
The \code{error.brew} template has
\begin{HTMLinput}
<% brew("brew-header.brew") %>
<h2>
  <%= message %>
</h2>
<% brew("brew-footer.brew") %>
\end{HTMLinput}
where the value for \code{message} is passed in through the error
call. The header and footer templates are straightforward, and are
used to give a consistent look to each page. In this case, as we use
xhtml, we have for the header:
\begin{HTMLinput}
<?xml version="1.0" ?>
<html xmlns="http://www.w3.org/1999/xhtml" xml:lang="en" 
  lang="en">
<head>
<meta http-equiv="content-type" 
  content="text/html; charset=UTF-8"/>
<title> <%= title %> </title>
</head>
<body>
<%= 
  if(exists("user_name") && nchar(user_name)) 
    sprintf("<h2>Welcome %s</h2>", HTMLencode(user_name))
%>
\end{HTMLinput}
The \code{user\_name} variable is set in the greeting page, so may
not be present. Note the call to \function{HTMLencode} to ensure that the
value for the name, which comes from the user, does not contain any
malformed HTML. 

The footer simply closes the \code{body} and \code{html} tags. In both
cases, these templates could be much more complicated.

Our greeting page illustrates how to use a form to gather user input, in this case a name, but in general this might be used for authentification etc.
\begin{Schunk}
\begin{Sinput}
 showLogon <- function() {
   title <- "Logon"
   brew("login-form.brew")
 }
\end{Sinput}
\end{Schunk}
The main part of the \code{login-form.brew} template is a basic form using the
\code{input} tag in two different ways.
\begin{HTMLinput}
<form method="POST" action="/simpleapp/select">
<label>Enter your name:</label>
<input type="text" name="name" />
<input type="submit" value="submit" />
</form>
\end{HTMLinput}
We use a \code{POST} call, as this may be used to modify a data
source. As well, the \code{action} specification uses \code{select} so
that the \code{path\_info} variable can be used to determine which
page to call.


After logging on, the user may be asked to narrow the search for
data. In this example, the user is asked to select one of the rows of
the data source. We generically refer to the row identifier as
\code{ID}. 
\begin{Schunk}
\begin{Sinput}
 selectID <- function() {
   title <- "Select an ID"
   context <- list(nms=rownames(df))   
   with(context, brew("select-id.brew"))
 }
\end{Sinput}
\end{Schunk}
The \code{context} variable is used to pass in
different contexts to the \pkg{brew} template. Of course this could
also appear directly in the template, but it is better to separate the
logic from the presentation. In this case, the template for ID
selection includes this
\begin{HTMLinput}
<form method="GET" action="/simpleapp/id">
<select name="id">
<%= 
  hmakeTag("option", nms)
%>
</select>
<input type="submit" value="submit" />
</form>
\end{HTMLinput}
We use \code{GET} for the method, as we assume this is merely a
request to narrow the display of data, not modify the data store. The
useful \function{hmakeTag} function is employed to vectorize the
creation of the HTML \code{option} tags.

Finally, our call to show detail on the selected identifier includes
matching the user specified ID against a list of possible values (a
whitelist). If no match occurs, an error message is printed.
\begin{Schunk}
\begin{Sinput}
 showID <- function() {
   title <- "Show an ID"
   id <- GET$id
   if(! id %in% rownames(df)) {
     processError(list(message="id does not match"))
   } else {
     context <- list(d=df[id,], id=id)
     with(context, brew("show-id.brew"))
   }
 }
\end{Sinput}
\end{Schunk}
For the display, we have this basic template which uses
\function{hwrite} to put the output into a table.
\begin{HTMLinput}
<h3> Detail on <%= id %> </h3>
<% 
  hwrite(unlist(d), page=stdout()) 
%>  
\end{HTMLinput}


The main script must figure out the \code{user\_name} variable. This
may come from the greeting page through a POST request, or may be
stored using a cookie to make the name persistent. This leads to the
following (\function{get\_d} is used to provide a default, if the variable
is \code{NULL}):
\begin{Schunk}
\begin{Sinput}
 user_name <- ""
 if (!is.null(POST)) {
   user_name <- get_d(POST$name, "")
 }
 if(user_name == "" && !is.null(COOKIES)) {
   user_name <- get_d(COOKIES$name, "")
 } 
\end{Sinput}
\end{Schunk}

Finally, the script is used to dispatch to the proper page. We start
by setting the content type and a cookie to store the
\code{user\_name} variable.
\begin{Schunk}
\begin{Sinput}
 setContentType("text/html")
 if(user_name != "")
   setCookie("name",user_name)
\end{Sinput}
\end{Schunk}

Following how django processes URLs we set up a list of regular
expressions to check against \code{path\_info} and function names to handle the dispatch.
\begin{Schunk}
\begin{Sinput}
 urls <- list(select=list(regexp = "^/select", call="selectID"),
              id =   list(regexp = "^/id",     call="showID" )
              )
 default_call <- "showLogon"
\end{Sinput}
\end{Schunk}

With this, we then process the request as follows.
\begin{Schunk}
\begin{Sinput}
 path_info <- SERVER$path_info
 flag <- FALSE
 for(i in urls) {
   if(!flag && grepl(i$regexp, path_info)) {
     flag <- TRUE
     tryCatch(do.call(i$call, list()), error=processError)
   }
 }
 if(!flag)
   tryCatch(do.call(default_call, list()), error=processError)
\end{Sinput}
\end{Schunk}
We wrap the call inside \code{tryCatch} in case the page creation
throws an error.

The last line of the script is simply \code{DONE} to indicate to the
client that the request is finished.
\end{example}



\section{Web 2.0}
\label{sec:web-2.0}
%% web 2.0

%% Using R as a web service

The term web 2.0 is used to describe highly interactive web sites. A
key feature of many of these is the use of \defn{Ajax
  technologies}. The packages \pkg{Rpad} and \pkg{gWidgetsWWW} use
Ajax technologies for interactive web sites. ``Ajax'' comes from
\textbf{a}synchronous \textbf{J}avascript and \textbf{X}ML. The term
asynchronous refers to pieces of a web page being updated
independently of others, unlike in the previous section where each
request creates a new page. The JavaScript term is a substitute for a
browser side language to manipulate the web pages DOM, and XML simply
a means to encode data, and shouldn't be taken literally, as other
common text-based encodings are used, such as JSON.

Several JavaScript libraries are built around Ajax technologies, such
as the \code{extjs} library and \code{jQuery}. These provide a means to query a server
asychronously through an \defn{XMLHttpRequest}. This section discusses
briefly how to use \pkg{rapache} to provide the data for such a
request. 

\begin{example}{Creating a web service using \pkg{rapache}}{eg:web-service}
  This example will illustrate how to make a web service with
  \pkg{rapache}. There are two pieces, the JavaScript code in the web
  page, and the server code. For the JavaScript piece, we use the
  \code{jQuery} library, as the use is somewhat straightforward.
  
  We illustrate how to return content in either HTML, JSON or XML format. 
  
  First, the HTML. In the header of our web page, we must call in the
  \code{jQuery} JavaScript library. These files may be on local
  webserver, or called in with the following HTML code:
  \begin{HTMLinput}
<script 
  src="http://ajax.googleapis.com/ajax/libs/jquery/1.3/jquery.min.js" 
  type="text/javascript">
</script>    
  \end{HTMLinput}

  Inside the same HTML page, we have a place holder to put the text from
  the web service. We use \tagger{div} tag, with an unique id.
  \begin{HTMLinput}
<div id="htmlTarget"> [HTML target] </div>    
  \end{HTMLinput}
  There are also similar areas for JSON and XML. If things are working
  properly, the bit \code{[HTML target]} won't be seen, as our web
  service will provide its content.
  
  We want the request for data to happen when the page loads. The
  \code{jQuery} library provides a means to have a function called as
  the page loads (before any images are downloaded, say). We place the
  commands within this snippet of JavaScript.
  \begin{HTMLinput}
<script type="text/javascript">
  $(document).ready(function(){  
    // JavaScript commands go here
  })
</script>
  \end{HTMLinput}

  As for the JavaScript commands, the following jQuery code will
  produce the Ajax request. This assumes the webserver is running
  locally. One would replace \code{localhost} with the appropriate
  site.~\footnote{For security purposes, the server providing the web
    service content must have the same domain as that providing the
    web page.}
  \begin{HTMLinput}
$.ajax({
   type: "GET",
   url:"http://localhost/ajaxapp/html",
   dataType: "html",
   success: function(data) {
     $("htmlTarget").html(data);
   },
   error: function(e) {
     $("#htmlTarget").html("<em>Service is unavailable</em>");
   }
   });    
  \end{HTMLinput}
  To explain, the \code{\$} is a \code{jQuery} variable, the first
  occurence is a call to its \code{ajax} method. The \code{.}
  indicates that. Whereas, the \code{\$("htmlTarget")} is a data
  selection call. The arguments to the \code{ajax} method
  specify  a \code{GET} request to a certain url. The return
  data will be HTML. The request, if a success, will replace the HTML
  code within the node with id \code{htmlTarget} with that returned by
  the Ajax request. If an error is returned, an error message is placed
  there instead.
  
  Within the \R\/ script run by \pkg{rapache}, we have a call like
  this to produce the content.
\begin{Schunk}
\begin{Sinput}
 show_html <- function() {
   require(hwriter, quietly=TRUE)
   setContentType("text/html")
   hwrite(d[1:5,], page=stdout())
 }
\end{Sinput}
\end{Schunk}

This specifies the content type and some HTML text. No headers are
needed here.  The \code{d} variable refers to some data frame. If
there were an error, we would return an error code, say \code{404L}
for file not found. In this case the error handler is called.
  


Using \code{JSON} is not much different, although the JSON is just the
data, so there will need to be some formatting within JavaScript. This
illustration will use the package \pkg{rjson} to create encode the data
into json markup, but \pkg{RJSONIO} can be used instead (from
\url{www.omegahat.org}) or one could create the JSON within \R\/
directly. Here is the server side code (not written with any
generality):
\begin{Schunk}
\begin{Sinput}
 show_json <- function() {
   require(rjson, quietly=TRUE)
   n <- as.integer(GET$n)
   n <- min(max(n,1), 32)                # check
   out <- toJSON(list(mpg=mtcars$mpg[1:n],
                      car=rownames(mtcars)[1:n]))
   setContentType("application/json")
   cat(out)
 }
\end{Sinput}
\end{Schunk}
We allow a variable \code{n} to be passed in through the Ajax
call. The function \code{toJSON} prefers lists to data frames, so we
make a list with our data, in this case we have two named variables
\code{mpg} and \code{car}.

Within the HTML file we have this JavaScript code.
\begin{HTMLinput}
$.getJSON(
   "http://localhost/ajaxapp/json",
   {n:"5"},
   function(data) {
     $("#jsonTarget").html("");  // clear out
     for(i=0; i < data.mpg.length; i=i+1) {  
       $("#jsonTarget").append(data.car[i] + " gets " +
         data.mpg[i] + " miles per gallon" + "<br />");
    }
   });
\end{HTMLinput}
The \code{getJSON} method is a convenience for the \code{ajax}
method. The second argument is how we pass in the parameter
\code{n}. Finally, the last function is called on a success, and
simply loops over the vector and pieces together some HTML, appending
it to the target. (The last bit is much easier in \R, but not too hard
in JavaScript.)

Finally, we illustrate doing a similar task only using XML. The server
side code might look like
\begin{Schunk}
\begin{Sinput}
 show_xml <- function() {
   require(XML, quietly=TRUE)
   n <- as.integer(GET$n)
   n <- min(max(n,1), 32)                # check  
   children <- sapply(1:n, function(i) 
                      newXMLNode("car", 
                                 newXMLNode("make",d[i,1]), 
                                 newXMLNode("mpg", d[i,2])
                                 ))
   out <- saveXML(newXMLNode("data", .children=children))
   setContentType("text/xml")
   cat(out)
 }
\end{Sinput}
\end{Schunk}
We use the \code{XML} library to piece together our response. In this
case we make several car nodes, each with a make and mpg value.

The JavaScript to parse this response can look like this:
\begin{HTMLinput}
$.ajax({
  type: "GET",
  url:"http://localhost/ajaxapp/xml",
  data: {n:"4"},
  dataType: "xml",
  success: function(data) {
    $("#xmlTarget").html("");
    $(data).find("car").each(function() {
     $("#xmlTarget").append($(this).find("make").text()  + 
       " gets " + $(this).find("mpg").text() + 
       " miles per gallon" + "<br />")
    })
 }
})
\end{HTMLinput}
The \code{data} argument is again used to pass in a parameter. As for the
\code{success} callback, as before we append text to the target after
clearing it out. To find the text, is a bit tricky, as it uses
\code{jQuery}'s selector methods. Within the method call, the variable \code{this}
stands for each \code{car} node, and the \code{find} method gets the
child node for that variable. The \code{text} method converts the
object to text that can be appended to the target.
\end{example}












\backmatter
%% Appendix
%% Bibiligraphy
\bibliography{guis}
%% Index



\end{document}

%%% Local Variables: 
%%% mode: latex
%%% TeX-master: t
%%% End: 
